
\section{Theory for thick slab}
\subsection{Summary}
Had the concept that a ``quenched slab'' model could forecast the convergence temperture for layers below the annual skin-depth. Developed and tested this (early Jan 2016) but found that the existing KRC asymptotic exponential prediction (EPRED) algorithm did a better job.

\subsection{Introduction}

For real planets with eccentric orbits and obliquity, multi-year runs are
required to address the lower boundary. No practical way to set the lower
boundary exactly, and the numerical approach can be slow.

 However, ignoring both the diurnal and annual variation, the approach of the
 lower (insulating) boundary to equilibrium is similar to the problem of
 quenching a thick slab, that is, start with a slab of thickness 2l of uniform
 temperature $T_0$ (the KRC initial T) and force both boundaries to a
 different (and unknown) ultimate lower boundary average. Without loss of
 generality; the ultimate temperature can be taken as 0.

There is an analytic solution for slab quenching which I will call $
S(\eta,\tau)$ where the dimensionless variables are $\eta=x/L$ where x is
distance from the ``surface'' and L is the slab [half] thickness , and $\tau =
\kappa t/L^2$ where $\kappa$ is the diffusivity and $t$ is time. $S$ predicts
the normalized temperature at a location $X$ from the center of the slab that
started a $T=1$ and is boundary-quenched to $T=0$ at $t=0$. The slab center is
the location of no heat flow and thus corresponds to the bottom of a IB=0 KRC
model. For a KRC run with NRSET $<$ N3 and IB=0, the time of quenching is
effectively the last layer reset, which will happen one or more times during the
first season of a run; for normal KRC runs quenching happens after sol 3.

The average [annual] of the lower layers will approach the surface temperature
average only if the thermal properties are temperature independent and there is
no heat flow; for a fixed geothermal heat flow $H_g$, the gradient of the average
temperature will be $H_g/k$.

If T-constant properties and no geothermal heat flow, all layers and the surface
should have the same annual-average temperature. In this case; after each few
years one could offset the layer temperatures by $\langle T_S \rangle - \langle
T_i \rangle_a$ (this is done in KRCSIMPLE). Otherwise, need a forecast of the
$U_i$, then ultimate value of $ \langle T_i \rangle_a$ if the model were to run
forever. This forecast may be made though matching curve-shape of the S model
to determine what fraction of the approach to $U_i$ was accomplished in the prior
N years of run.

Convenient units for using S are $x$ as layer center depth from the bottom of
the model in diurnal scale-heights ($D$), which requires that the $\kappa t$
input be in unitsof SI$/D^2$, DIFFU * 86400.*PERIOD * J5*DELJUL / SCALE**2 . Then S
is the fraction of the way that T has changed between the starting (or last reset
??) time and the ultimate value:

\qb T_t=T_0+(1-S_t)(T_U-T_0) \mc{or}  T_U= T_0 + (T_t-T_0)/(1-S_t)  \mc{or} \qe 
\qbn \Delta T \equiv T_U-T_t = (T_t-T_0)(\frac{1}{1-S_t}-1) \ql{S1}
where $\Delta T$ is the perturbation to be applied to a layer.

If unsure of what to use for $T_0$, can try a two-point solution of: \  $1-S_i = (T_i-T_0)/(T_U-T_0)$. Change variable to $Q=T_U-T_0$ yielding 
\qbn Q= (S_2-S_s)/(T_2-T_1) \mc{and} T_0 = T_1-(1-S_1))Q \ql{S2} 
\qbn T_U=T_0+Q \mc{and}   \Delta T \equiv T_U-T_2 \ql{S3} 

May need to estimate the annual mean temperature corresponding to the first
season of the run. Do this by assuming that the offset (for each layer) between
the annual mean and the first season remains virtually constant over the
years. Test this by ... ?? NOPE




\subsection{Approach}

KRC saves the maximum and minimum diurnal temperature for all physical layers at
the end of each season, as well as the midnight temperature. From these can
easily derive a good estimate of the diurnal average temperature $P_{i,k}$ where
$i$ is the layer index and $k$ is the sol count (error negligable well below the
diurnal skin depth $D$); and the annual average $Y_{i,y}$ where $y$ is the year
count.



For the moment ignoring the jump perturbation KRC normally makes during the
first season, the annual layer temperatures after a few years and a few more
years.

Input values to the S model (effectively $\eta$ and $\tau$) can be computed
from the KRC parameters and the season index.

 
There is no point in adjusting the temperature for layers shallower than a few annual skin-depths 

Using diurnal (Tmin+Tmax)/2 as an average temperature may be poor until below
several (5 ?) skin-depths. Because temperature near Tmax are brief compared to
those near Tmin, this ``average'' is expected to decrease with increasing depth
until deep enough into the soil that the diurnal curve is virtually sinusoidal.


The 60S latitude without atmosphere is an extreme test in that the T range is about 100 to 240.



\subsection{The S model}

\vspace{-3.mm} 
\begin{verbatim}
Solutions found 2014mar9 in: 
 http://www.ewp.rpi.edu/hartford/~wallj2/CHT/Notes/ch05.pdf
   saved as /work2/Reprints/TherMod/Wallj2ch05.pdf

Similar treatment at found 2016jan12
http://www.ewp.rpi.edu/hartford/~ernesto/C_Su2003/MMHCD/Notes/Notes_pdf/s02.pdf

CJ:2.4-3 refers to Carslaw and Jaeger, Conduction of Heat in Solids, 2nd Edition: Section 2.4, equation 6.
\end{verbatim}

Here assume constant thermal properties.

\subsubsection{The semi-infinite case} Mathmatically:
\qb \frac{\partial T}{\partial t}= \kappa \frac{\partial^2 T}{\partial x} \qe
With boundary conditions 
\qb T(x,0)=T_0 \mc{and} T(0,t)=0 \qe
Solution is \qbn \frac{T(x,t)}{T_0}= \mathrm{erf} \left( \frac{x}{2 \sqrt{\kappa t}} \right)\mc{CJ:2.4-3}  \ql{sic}  where the error function erf, available as a function in IDL and Fortran, is: 
\qb \mathrm{erf}(z) =\frac{2}{\sqrt{\pi}}\int_0^z \exp \left( -\xi^2\right) d \xi \qe

\subsubsection{The finite slab case}

\qb \frac{\partial T(x,t)}{\partial t}= \kappa \frac{\partial^2 T(x,t)}{\partial x} \qe
Slab thickness $L$ with boundary conditions 
\qb  T(0,t)=T(L,t)=0  \mc{and} T(x,0)=f(x) \qe
Solution (\S 4.1, p.7) is
\qb T(x,t)=\sum_{n=1}^\infty \left[B_n \sin \left(\frac{n \pi x}{L} \right) \right] \exp \left( -\left( \frac{n \pi}{L} \right)^2 \kappa t \right) \qe

where \qb B_n =\frac{2}{L} \int_0^L f(x) \sin \left( \frac{n \pi x}{L} \right) dx\qe

For the special case of $f(x)=T_i =$ constant, $ B_n=-T_i\frac{ 2 (-1+(-1)^n)}{n \pi} $. Alternate terms are 4 and 0, which yields

\qb \frac{T(x,t)}{T_i}=\frac{4}{\pi} \sum_{n=0}^\infty \underbrace{ \frac{1}{2n+1} \sin \left( \frac{ (2n+1)\pi x}{L}\right)}_{fx} \underbrace{  \exp \left( -\left( \frac{n \pi}{L} \right)^2 \kappa t \right)}_{ft} \qe
 Coded as IDL \np{slabdiffu} of $L$, vector of $x$ and vector of $\kappa t$.


CJ 3.3-6 is for  slab from 0 to l, initial temperature $v$= constant $V_0$, and boundaries held at 0. Solution must be symmetric around $x=l/2$


\qb \frac{T(x,t)}{T_i}=\frac{4}{\pi} \sum_{n=0}^\infty \frac{1}{2n+1} \sin \left( \frac{ (2n+1)\pi x}{2l}\right) \exp \left( -\left( \frac{(2n+1) \pi}{2l} \right)^2 \kappa t \right)  \mc{CJ:3.3-6} \qe

CJ 3.3-8 is for slab from -l to +l Solution must be symmetric around $x=0$. CJ 3.4-2 is basically the same relation. 
 
\qbn \frac{T(x,t)}{T_i} \equiv S =\frac{4}{\pi} \sum_{n=0}^\infty  \underbrace{\frac{-1^n}{2n+1} \cos \left( \frac{ (2n+1)\pi x}{2l}\right)}_{fx}  \underbrace{  \exp \left( -\left( \frac{(2n+1) \pi}{2l} \right)^2 \kappa t \right)}_{ft}   \mc{CJ:3.3-8} \ql{Smod}

Using dimensionless parameters:

\qb S =\frac{4}{\pi} \sum_{n=0}^\infty  \underbrace{\frac{-1^n}{2n+1} \cos \left( \frac{ (2n+1)\pi}{2} \eta \right)}_{fx}  \underbrace{  \exp \left( -\left( \frac{(2n+1) \pi}{2} \right)^2 \tau \right)}_{ft}  \qe


Within the summation, the initial factor and the sine or cosine term are of
order unity; and the exp term decreases in magnitude with increaseing $n$. The
summation error will be of order the last term in the exp factor;
$E \sim \exp \left( - \left( \frac{(2n+2) \pi}{L} \right)^2 (\kappa t)_\mathrm{min} \right) $.  
Treating $E$ as a fractional tolerance; need 
$ n \geq \frac{L}{\pi} \sqrt{-\ln E / (\kappa t)_\mathrm{min} } -\frac{1}{2}$


\subsubsection{Beware}
Slab are commonly defined as extending from -l to +l, or ) to L; which yield different equations. 

\vspace{-3.mm} 
\begin{verbatim}
http://www-unix.ecs.umass.edu/~rlaurenc/Courses/che333/lectures/Heat%20Transfer/Lecture9.pdf 
\end{verbatim} 
 gets a similar results ( using dimensionless variables), but then exp term has $(2n+1)^2$ rather than $n^2$
 
\begin{verbatim}
http://www-unix.ecs.umass.edu/~rlaurenc/Courses/che333/lectures/Heat%20Transfer/Lecture9.pdf
\end{verbatim}
use a slab thickness of 2H. Page 14 also has in effect $(2n+1)$ rather than $n$ in the exp term.


\subsection{Extreme case: 6\qd S with no atmosphere}
 Several characteristics are shown in Figures  \ref{tm644s}, \ref{tm644l}, \ref{tm644a}, \ref{tm644h}and \ref{tm644y}.

The Slab model seems not as successful as the asymptotic exponential prediction
algoritm used by KRC for prediction to the end of each season; see
Fig. \ref{tm652}.
 
\begin{figure}[!ht] \igq{tm652}
\caption[Layer predictions]{Forward predictions for layer temperatures for KRC
  run D, 60S no atm.. In legend: T1= annual mean temperature at time 1, year 2;
  T2= annual mean temperature at time 2, year 10; T0= initial temperature
  derived using 2-point S-model,; TU1= ultimate temperature using 1-point
  S-model; TU= ultimate temperature using 2-point S-model; Tep= asymptotic
  exponential prediction to the last year using time 2 and the two years prior;
  Tfin= annual mean temperature for the last year of the KRC run, year 16.
\label{tm652} tm652.png  }
\end{figure} 
% how made:  kv3 ... 6   pari[21:22]=2,10  tttmod 644 65 652

\subsection{Asymptotic exponential prediction: EPRED } %-----------------------

tthmod @651
case and latitude set by pari[7] and 17
pari[21:22] set the first and last years for the pivot point, ; code will limit to safe values.

