\documentclass{article}  % See Skeleton.tex for examples of many things

% epstopdf fig.eps       TO convert .eps fig to pdf NOPE
% latex  file[.tex]      TO generate .dvi
% pdflatex  file[.tex]   TO generate .pdf   NOPE losses  figs 
% then  dvipdfm Vtest.dvi To generate .pdf

% see definc.sty for other page format settings
%\usepackage{epsfig}
\usepackage[dvips]{graphicx}  % unComment this line for eps and latex
%\usepackage[dvipdf]{graphicx}  % unComment this line for non-eps pdflatex 

\usepackage{../definc}  % Hughs conventions
%\textheight=10.00in  \topmargin=-0.5in  %  hobo normal=final
%\textheight=10.38in \topmargin=-1.33in  %  hulk printer extreme limits centos 5.3
%\textheight=10.38in \topmargin=-1.0in to much for pdf
\textheight=10.00in \topmargin=-0.5in
%                                         bottom right  top  left
\newcommand{\igc}[1]{\includegraphics[trim=2.5mm -20.mm 6.5mm 1.mm,clip,angle=90,width=170.mm]{#1.eps}}  % was 175

% \newcommand{\igd}[1]{\includegraphics[trim=2.5mm -20.mm 6.5mm 1.mm,clip,angle=90,width=150.mm]{#1.eps}} % narrower

\title{KRC version 2 and 3: Thin/deep layers and long runs.}
\author{Hugh H. Kieffer  \ \ File=-/krc/Doc/thin.tex 2014may21  minor edit 2015dec22 2017feb25}
\begin{document}
\maketitle
\tableofcontents
\listoffigures
%\listoftables

\section{Introduction / Purpose}
 This started circa early 2014 with Robin Fergason's deceptively simple question about Mars surface temperatures: ``What is the truth?''. 

 KRC can not answer that, nor can any model.  It is possible and essential to
 compare numerical model results with observations, recognizing that numerical
 models omit some physics, have numerical approximation errors, and may contain
 blunders. Observations can and always do involve calibration errors,
 measurement noise, and natural variations.

What can be done with a model alone is to assess the numerical approximation
errors, and that is the purpose here. In particular, what is the effect of
``spin-up'', season-spacing, layer thickness and model depth for KRC. 

In the middle of this, the double-precision version of KRC was developed and I
have decided to use only that for this study to avoid any issue of build-up of
numerical roundoff.


One must be careful if results near polar frosts are desired. I encourage
individual studies for the particular circumstances of interest.  Many of the
studies here were done only at the equator where the diurnal effects are
maximum.


\subsubsection{Version 3 master} %....
Differs from version 2 in that:
\qi N2, step/days increased by factor of 4 to 1536, \ \ later to 384*4*256=393216
\qi N1, number of layers, increased by 2 to 22, \ \ later to 50
\qi Uses the feature to specify starting output Ls
\qi Includes after the end-of-run  7 lines, which if inserted before the first ``0/'' line will set the model to run every sol and output the 3rd year. These are called ``\textbf{soly}'' (rhymes with Holy) runs.

\section{Major runs} %___________________________________

All runs used KRC version 3.1.1==311 except the two noted as 233==2.3.3 in the appendix. However, as of 2016jun22, neither of these is archived on H3
\qi v241 should be the same as 233 except when using the Viking pressure curve.
\qi v321 should be the same as 311 

Mars sol is 1.0274910 days and its year is 668.61207sol ; Mars year for KRC is defined in the geometry block.

For run-length testing, must make the year an integral number of sols and need an integral number of reasonable-length seasons in a year. Choose to have 672=$2^5 \cdot 3 \cdot 7 $ seasons; this sets sol as 1.0223109 , 0.5\% shorter than real.
Can leave DELJUL as 1/42 of a MarsYear; 16.356974 days.
 
In order to concentrate on the effect of model parameters, some idealized objects used:
\qi M0: Same as version 3 master.inp, except only the equator with elevation 0.
\qi M1: As M0, but sol decreased by 0.5\% to be 1/672 of a year
\qi M2: As M1, but no atmosphere
\qi M6: sol $\equiv$ 1 day and year $\equiv$ 720 days, no atmosphere, skip DelJul=16. 
\qi M7: as M6, and  eccentricity =0
\\ and several of the tests were done only at the equator.

All output in /work1/krc/test *.t52

\subsection{Double precision versus single \label{dp} } %---------------------

Run with the version 2 master and 5 latitudes. Sol = 1/672 of a year. Exercise most of the KRC capabilities.  Basin input file in \nf{thin4.inp} Cases:
\begin{enumerate}    % numbered items 
 \item deep: N2=1538 time steps/sol, normal FLAY=.18 and RLAY=1.2, N1=26 layers (most layers must be first case). Total depth 102 diurnal skin depths (D) or about 3.9 annual skin depths.   No atmosphere
\item thin: Flay=0.08664, a little less than 1/2 normal, total depth 49.1 D
\item 1536: normal FLAY, 22 layers, total depth 48.6D or 1.85 annual skin depths.
\qii  Holds for all the following cases. 
\item  384: N2=384, traditional for KRC
\item  dual: As above, IC2=7  Ice as the lower material
\item KofT:  As above, with T-dependent properties
\item Kcon:  T-dependent code, but constant properites
\item Atmos: Mars atmosphere with constant pressure, homogenous, T-constant properties
\item Viking: as above, with Viking pressure curve and variable Frost albedo and temperature
\end{enumerate}

The primary metric is the mean value of the difference in Tsur through a sol,
R*8 - R*4, $< \Delta T_s> $, here called \textbf{DATs} (ADTs is
unpronounceable).  The ratio of this average to the average of the absolute
difference, $< \Delta T_s> /< | \Delta T_s | >$, is 0.96 so the preponderance of
$\Delta T_s$ are of the same sign through a day.

\subsection{Skip versus soly  \label{skso}} %---------------------
KRC normally uses extrapolation to advance between seasons, see \S 3.2.7. in
\cite{Kieffer12} however KRC can be run with no extrapolation; computing the
insolation geometry every sol and effectively running continuously. Both the
former ,''skip'' and the latter, every-sol or ``soly'' methods were used for
long runs and the results compared. The incentive for ``skip'' runs is execution
speed and generating manageable files for mission model sets. For the two runs
described here, both skip was 12 times faster than soly.

Done using M2 object. Skip run and saved for 6 years. Soly has 2 year spin-up then save for 4 years.  Eight cases with I=200 run:
\qi 1: 1536deep; 1538 time steps/sol, normal FLAY and RLAY, 26 layers. Total depth 102 diurnal skin depths (D) or about 3.9 annual skin depths.
\qi 2: 1536thin; Flay=0.08664, a little less than 1/2 normal, total depth 49.1 D
\qi 3: 1536; normal FLAY, 22 layers, total depth 48.6 or 1.85 annual skin depths.
\qii  Holds for all the following cases. 
\qi 4: 384; traditional KRC 16 steps/hour.  This and the remainder have 22 layers
\qi 5: 768; 384*2 steps/hour
\qi 6: 3072; 384*8 steps/hour
\qi 7: 6144; 384*16 steps/hour
\qi 8: 1536atm; 384*4 steps/hour and default Mars atmosphere.

Skip-soly is nearly the same every year, Figure \ref{tm41}, with the exception
of case 1, which has a secular trend. Differences are largely under 0.1K, shown
in more detail in Figure \ref{tm42}. Jumps in the difference are associated with
changes in the number of convergence days (NDJ4) before the asymptotic
extrapolation; the size of these jumps, typically under 50~mK may be taken as an
indication of the numerical error in the extrapolation.

There is little variation with the number of time steps, as seen in Figure
\ref{tm43} where the outliers are the atmosphere and deep cases.

\begin{figure}[!ht] \igc{img/tm41}
\caption[Skip-soly]{Skip-soly for every hour at the skip seasons for years 3
  through 6.  The smooth blue curve is variation of the diurnal average
  attenuated by a factor of -80 and phase offset of -20\qd.
  \ tm41.eps \label{tm41} } \end{figure}

\begin{figure}[!ht] \igc{img/tm42}
\caption[Skip-soly last year]{Skip-soly for every hour at all skip seasons in
  the last year.  The diamonds indicate the number of convergences day (NDJ4,
  offset and scaled) for the 1536atm case; the boxes for the 1536deep
  case. \ tm42.eps \label{tm42} } \end{figure}

\begin{figure}[!ht] \igc{img/tm43}
\caption[Skip-soly spaced seasons]{Skip-soly for every hour at 8 seasons spaced
  through the last year.  \ tm43.eps \label{tm43} } \end{figure}

Secular trends in convergence were visualized by forming the diurnal averages
and subtracting the values for the final year from each of the earlier years.
By the start of year 3, the difference is less than 2 mK for the skip models,
Figure \ref{tm52}; the asymptotic extrapolation handles the deep model
particularly well. For the every-sol models, Figure \ref{tm54}, the difference
is less than 4 mK after a 2-year spin except for 30 mK for the deep case, which
only slowly diminishes and would be larger if the run extended beyond 6 years.

\begin{figure}[!ht] \igc{img/tm52}
\caption[Skip secular]{Residuals in diurnal-average surface temperature for each
  year made by subtracting the values at the corresponding season in the last
  year of the run. Shown are the absolute values; all values were negative. The
  model years are indicated with vertical dotted lines at the end of each
  year. \ tm52.eps \label{tm52} } \end{figure}

\begin{figure}[!ht] \igc{img/tm54}
\caption[Soly secular]{As in Fig. \ref{tm52} but for the every-sol model; all
  values were positive.  \ tm54.eps \label{tm54} } \end{figure}

The effect of a deeper run or using thinner layers is less than 0.1K, Figure
\ref {tm622}. A double-deep model has only a 0.2K effect, similar to that for a
difference in iteration days.
 
\begin{figure}[!ht] \igc{img/tm622}
\caption[Deep or thin]{Skip models in year 3; the normal model (case 3) compared
  to a double-deep model (case 1) and to a model with thinner layers (case
  2). The two spikes in each curve are due to changes in the number of iteration
  days, scaled by 1/100 and plotted as colored diamonds; multiple diamonds at
  one season indicate the convergenge days differed.  \ tm622.eps \label{tm622}
} \end{figure}

DATs are less than 0.1 K except near the pole, Figure \ref{tm71}; the source of
these differences has not been identified.
 
\begin{figure}[!ht] \igc{img/tm71}
\caption[All DATs]{DATs for every season and year for all latitudes and cases.
  \ tm71.eps \label{tm71} } \end{figure}


\begin{figure}[!ht]  \igc{img/tm722}
\caption[DATs at 30\qd S, year 3]{DATs for 30\qd S and year 3. Diamonds are the
  scaled change in number of convergence days. \ tm722.eps \label{tm722}
} \end{figure}


\begin{figure}[!ht] \igc{img/tm73}
\caption[DATs year 3 each Lat.]{DATs for all seasons in year 3; sections show
  each latitude.  \ tm73.eps \label{tm73} } \end{figure}


\begin{figure}[!ht] \igc{img/tm74}
\caption[$\Delta T_s$ vrs hour]{$\Delta T_s$ as a function of hour for seasons
  spaced by 5 for -30\qd. The jump betweeen the 3rd and 4th seasons is also
  shown in Figure \ref{tm722} between indices 10 and 15.
  \ tm74.eps \label{tm74} } \end{figure}


The number of days to convergence is key to the Tsur differences. For instances
(season, latitude, case) where NDJ4 was the same, $<|T_s|>$ is 23 mK, where
different, 86 mK. NDJ4 is least changed for Latitude 0

tttmod @757; columns are each latitude; rows are cases.
\vspace{-3.mm} 
\begin{verbatim}
Number of NDJ4 differences
Lat -30.     0.    30.    60.   87.5
      5      0      0      7      6  deep
     17      1      0     12      1  thin
     11      0      1      5     13  1536
     10      5      0      0      1  384
     17      0      3     17      8  dual
      6      0      0     23     33  KofT
     15      0      1     18      7  Kcon
      3      0      1      4      2  Atmos
      4      0      2      2      0  Viking

Mean Abs Delta Diurnal Mean Tsur: mK
Lat -30.     0.    30.    60.   87.5
    7.6    4.0    5.2   24.6  118.9  deep
    7.1    3.2    4.7   21.5   82.0  thin
    7.4    3.4    4.6   22.3   77.9  1536
    4.3    1.9    4.6   14.0   48.6  384
   25.3    7.8   12.9   55.2   69.7  dual
   18.9   19.2   27.5   36.1  121.5  KofT
   24.8    8.0   11.0   53.7   80.3  Kcon
    5.0    2.1    4.0    4.6    3.1  Atmos
    4.9    2.3    3.5    5.5    2.6  Viking
\end{verbatim}  

\subsubsection{Timing and data volume}
For the ``thin4'' set of runs. Skip runs with 5 latitudes, soly (an ``e'' in the file name) runs with only -30\qd. -8 in name means R*8 version.
\vspace{-3.mm} 
\begin{verbatim}
thin- .t52   Total     Allowed
       size  time, s.  cases
4     13.6M    5.996    RASE=23.43
48     30M    8.300     ``
4e     29M    5.433    RASE= 10.89
4e8    57M   12.017     ``        
\end{verbatim}
 
\clearpage

\section{May 5 Email, modified to include figure} %_______________________

Gentlemen and Ladies:

Following lasts weeks telecon, I have set the double-precision version of
KRC to allow fairly thin layers. Version 3 will accommodate a factor 256 more
time steps than version 2, for a maximum of 393216; hopefully thin enough for
virtually any need. Also, the maximum number of time doublings has been
increased from 6 to 14.

For Inertia of 50., the minimum top soil layer of 32 micrometers. I have run an
initial study of layer thickness, the figure is attached and may take some time
to sink in. The first case is a "throw-away" of extreme total depth required
because the number of layers in a KRC run is limited by the first case [TDISK
must set some array sizes]. 

I have also attached a draft version of white-paper to myself about version
3; of interest here is only Section 2, which has some useful layer thickness
relations.

\textsc{Summary of the cases:} Z= is the thickness of the upper material in
meters. The number of layers was adjusted to keep the total depth about the
same.

\vspace{-3.mm} 
\begin{verbatim}
1: Start with normal master, Inertia=200    
  Set the sol  PERIOD = 1.0 day
  Set the year  = 720 days [inside the geometry matrix]
  Set the orbital eccentricity = 0.  [inside the geometry matrix]
  Set the season step   DELJUL=16.0
  Set the lower material to rock: DENS2=2500., SpHeat2=647. [COND2=2.77]
  Set to spin-up for two years, then output the 3rd year
  Start at Ls=0.01
    The above are each minor changes; just to get a clean situation.
  Set to no atmosphere: PTOTAL=0.5
  Set the times/day: N2=1536  this is 4 times normal
  Set the first layer of lower material: IC2=5   Three real layers.
  Set number of layer N1=44  Needed because N1 not allowed larger than first case
2: Set N1=22, roughly normal                          Z=0.0252
3: IC2=3  I.e., one real layer of upper material       Z=.0069

4:  Inertia=50 and reset IC2=7   Z=0.0129
5,6:  Same as 2,3 above  Z=0.0063 and 0.0017

7: Set first layer thin, FLAY=0.0533 thinnest allowed in theory
    and reset IC2=7     Z=0.0038
8,9: Same as 2,3 above  Z= 0.0019 and  0.000512

10:  FLAY=  0.2665E-01   Factor of 2 smaller
     N2= 6144  Factor of 4 larger               Z= 0.000256

11:  FLAY= 0.1333E-01 Factor of 2 smaller
     N2=  24576  Factor of 4 larger        Z= 0.000128 

12:  FLAY= 0.6663E-02 Factor of 2 smaller
     N2= 98304  Factor of 4 larger         Z=0.000064

13:  FLAY=0.3331E-02  Factor of 2 smaller
     N2= 393216   Factor of 4 larger        Z=0.000032

14: Only the lower material, so Z==0.
\end{verbatim}

Diurnal curves for case Ls=0 for this fictional Mars-like planet for all these
are shown in Figure \ref{thin5}; note that case 14 is the asymptotic
limit for thin layers.

\begin{figure}[!ht] \igc{img/thin5}
\caption[Thin surface layer]{Diurnal surface temperature variation for models
  fluff (I=50) over rock (I=2117). See text for details. The first three cases
  have a surface material of I=200 for context. Greatest temperature variation
  is case 4, with fluff thickness of 1.3 cm. Case 13 has a fluff thickness of 32
  \um.  \ thin5.eps \label{thin5} } \end{figure}
%qq

All cases 5 or more had secular Tsurf changes less than 0.1 K over the last year.

A study could be done of other changes that mimic extremely thin layers.

\newpage
FYI, below is a composite layer table of the last entry for each case. After
case 1, an objective was to maintain the total dept in meters, the sixth column
below, at nearly the same value. The KRC layer table will need some
modification, as it has not enough significant figures in the upper layers for
such use.

\begin{verbatim}
Case       ___THICKNESS____    __CENTER_DEPTH__    Total  Converg.
   LAYER    scale     meter     scale     meter   kg/m^2  factor
 1   44 3096.5500   99.212417019.1701  545.2881******************
 2   22   56.0907    1.7971  296.6440    9.5044 25984.656  32.742
 3   22   56.0907    1.7971  299.9363    9.6099 26264.807  32.742
 4   22  224.3627    1.7971 1162.7904    9.3138 25519.438  32.742
 5   22  224.3627    1.7971 1184.2174    9.4855 25954.427  32.742
 6   22  224.3627    1.7971 1199.0972    9.6047 26256.502  32.742
 7   29  238.0532    1.9068 1288.2084   10.3184 28176.136  36.859
 8   29  238.0532    1.9068 1294.5531   10.3693 28304.941  36.859
 9   29  280.5393    2.2471 1530.7769   12.2614 33461.873  51.190
10   32  242.3860    1.9415 1327.0281   10.6294 29000.074  38.213
11   36  251.3058    2.0129 1379.1344   11.0467 30132.914  41.077
12   40  260.5538    2.0870 1431.5223   11.4664 31274.624 176.625
13   44  270.1422    2.1638 1485.0203   11.8949 32441.939 759.455
14   22    8.2809    1.7971   44.4651    9.6498 26370.884  32.742
\end{verbatim}

\subsection{Timing} %------------------------
Case 13, N2=393216, 44 layers, one lat., took 25.1 sec on my computer (H3)

Note that this is a step time of 0.220 sec. At this resolution, it takes 390
time steps for the Sun to rise on Mars; a nicety that is not in KRC, which does
it in 1 step, although the incidence angle to the center of the Sun is correct.

[Attach:  thin5.jpg  krcv3.pdf ]
\\ END May 5 EMAIL

Additional runs with related cases:
\vspace{-3.mm} 
\begin{verbatim}
  1806848 May  5 06:21 thin5 N1=22:44  IC2=3:7  N2 to the max
      N2 adjusted to keep bottom similar meters
  1935872 May  6 10:17 thin6  skip; N1=22 IC=99  I=200 with N2 384 to 24576 
                          and N2=1536 with I 50 to 400 and 1800+
 62899712 May  6 10:25 thin7  soly for 6 yr. N1=22 IC=99  N2 factor 2 I=200
          Total time [s]=   75.010590
thin7b ==thin7 except deljul=16  Total time [s]=   5.5571561
\end{verbatim} 

%\clearpage
\section{Early runs}
\vspace{-3.mm} 
\begin{verbatim}
thin1.t52 krc run R*4

thin2.t52 krcd run. R*8  with N2max=393216
END KRC   Total time [s]=   1.0418410 

thin3  krcd, with N2max=393216 and MAXBOT=14E
END KRC   Total time [s]=   1.0278440 

thin4  krcd    all cases 22 layers
flay=  0.0533000    0.0266500    0.0133250   0.00666250   0.00333125
N2=    1536         6144         24576       98304         393216
 Case 13  DTIME: total, user, system=   26.9289   26.9269    0.0020
      END KRC   Total time [s]=   36.377472 
------------- the above runs deleted -------------- 

thin5
FLAY:   0.180000    0.0533000    0.0266500    0.0133250   0.00666250   0.00333125
N2  :      1536        1536        6144        24576        98304       393216
N1  :    22.0000      28.6751      32.4769      36.2787      40.0805      43.8823
 Case 13  DTIME: total, user, system=   25.0892   25.0872    0.0020
      END KRC   Total time [s]=    34.324783    no music 

 Case  7  DTIME: total, user, system=   36.3625   36.3625    0.0000
      END KRC   Total time [s]=   73.485825    

\end{verbatim} 
 
The logic in \np{kv3.pro} is getting complicated:
\\ KRC cases are 1-based, combine and deltas are 0-based. Some actions:
\qi 136, 123 , read two files, load tth and ttt
\qii 48 clot ttt
\qii 482 plot secular in ttt [uses tdd]
\qii 483 CLOT tth
\qii 484 print all tth and ttt cases
\qii 485 edit  and make combine ttc
\qii 486  plot combine
\qii 487  edit deltas
\qii 488  do deltas tdd
\qii 489  plot deltas

The change over years of the diurnal Temperature at Ls=0, is virtually
independent of N2; relative to the last date (the first day of year 7) it
reaches 4.4, 0.4, 0.05, 0.004 mK at the start of year 3, 4, 5, and 6
respectively.

Changing only the density of the upper material does not change the
temperatures, only the scale and hence the layer thicknesses and depth.

\clearpage
\subsection{Changing only N2 and Long runs} %___________________
Run every sol (of 1.0 days) for spinup of 2 years (of 720 sols) for total of 6
years. Typical Mars homogeneous inertia of 200 with 22 layers; vary the number
of time-steps per day from 384 to 384*64 in factors of 2. Results are in Figure
\ref{thin6c}

\begin{figure}[!ht] \igc{img/thin6c}
\caption[N2 only]{Effect of the number of time-steps per day; homogeneous I=200
  with steps/day from 384 to 384*64 \ thin6c.eps \label{thin6c} } \end{figure}

\clearpage
\subsection{Non-uniqueness}

Very thin layers will be hard to indentify as unique. Fig. \ref{thin6b} shows 
the difference in diurnal curves for some cases close to the minimum layer allowed.
\qi 1: with 32 \um layer of I=50.
\qi 2: 10\% I=50 and the rest bare rock
\qi 3: Entire surface I=1950
\qi 4: Entire surface I=2000
\\ The entire range is -1:+1.5 K. Only at night do models 2 to 4 differ from the thinnest layer by more than 1K.

\begin{figure}[!ht] \igc{img/thin6b}
\caption[Similar to thinnest]{Temperatures relative to those for bare rock of I=2116.7; the minimum fluff layer thickness allowed in Version 3 and some similar cases.  Skip model.  \ thin6b.eps \label{thin6b} } \end{figure}

\appendix  
\section{Files retained 2015dec22}
 All runs used version 3.1.1==311 except the two noted as 233==2.3.3
in time order:\vspace{-3.mm} 
\begin{verbatim}
 3555 May  6  2014 thin6.inp
 4025 May  6  2014 thin5.inp
 3801 May  8  2014 thin7.inp
 4093 May  8  2014 thin8.inp
 4329 May 21  2014 thin4.inp

 48912 May  4  2014 thin1.prt        
 49232 May  4  2014 thin3.prt     
 76130 May  5  2014 thin5.prt     1806848 May  5  2014 thin5.t52
 71460 May  6  2014 thin6.prt     1935872 May  6  2014 thin6.t52      
 76132 May  6  2014 thin5b.prt    1806848 May  6  2014 thin5b.t52      
 33758 May  6  2014 thin7b.prt    5193728 May  8  2014 thin7b.t52     
 34068 May  8  2014 thin7.prt    62899712 May  8  2014 thin7.t52  
 33758 May  8  2014 thin7a.prt     
 10521 May  8  2014 thin7c.prt    1484288 May  8  2014 thin7c.t52
 38991 May  8  2014 thin8.prt                                        
 39305 May  8  2014 thin8e.prt    5935616 May  8  2014 thin8.t52      
 43678 May 10  2014 thin4e8.prt  58749440 May  8  2014 thin8e.t52      
 42976 May 10  2014 thin4e.prt   29374976 May 21  2014 thin4e.t52  233 
 47873 May 10  2014 thin4.prt    22531328 May 10  2014 thin4.t52q  233
 48597 May 10  2014 thin48.prt   30724352 May 10  2014 thin48.t52  
                                 58749440 May 10  2014 thin4e8.t52 

 2013 Jul 24 11:28:09=RUNTIME.  IPLAN AND TC= 104.0 0.10000 Mars:Mars
   104.0000      0.1000000      0.8644665      0.3226901E-01  -1.281586     
  0.9340198E-01   1.523712      0.4090926       0.000000      0.9229373    
   5.544402       0.000000       0.000000       720.0000       3397.977    
   24.62296       0.000000      -1.240317      0.4397025       0.000000    
   0.000000      0.3244965      0.8559126      0.4026359     -0.9458869    
  0.2936298      0.1381285       0.000000     -0.4256703      0.9048783 
\end{verbatim}      
4 and 8 have standard Mars geometry matrix. 42 seasons/year
\\ 5,6 and 7 have idealized orbit with 16 days per season

4: Described in \S \ref{dp} 

5: N2=1536 flay=.18 rlay=1.2 INertia=200
\qi 1 single material
\qi IC2=5
\qi IC2=3
\qi Inertia=50  IC2=7
\qi IC2=5
\qi IC2=3
\qi flay=0.0533 N1=29  IC2=7
\qi IC2=5
\qi IC2=3
\qi flay=0.02665 N1=32  n2=6144
\qi flay=0.02665 N1=32  n2=6144

6: NLAY=22 N2=384
\qi cases 2:7, N2 increase by factor of 2 to  24576
\qi case 8, N2=1536 Inertia=50
\qi cases 9:11, inertia increase by factor of 2 to 400
\qi case 12:15, inertia=: 1800 1900 1950 2000

7: described in \S \ref{skso}
\\ N1=26 N2=1536  PTOTAL=0.5=no atmosphere
\qi 2: FLAY=.0866434
\qi FLAY=.18 N1=22
\qi N2=384
\qi N2=768
\qi N2=3072
\qi N2=6144
\qi N2=1536 PTOTAL=546
\\ 7.  1.000=N3  99.00=GGT  9999=NRSET  1.000=DELJUL  4321=N5  1441=Jdisk  720=Notif 
\qi 7b. no initial changes   720=$2^4*3^2*5$ sols/year, 0 eccentricity
\qi 7c N1=26
\\ check by: fgrep -n Changed thin7*.prt
\qi 7b differs from 7a only in  DAU 1.55765 becomes 1.52371 
\qi 7 and 7c had DAU=1.465
\\ 7==7a==7b  for cases.   7 is soly, 71 and 7b are??
\qi 7c has N1=26, and 2nd case is  FLAY=0.08664 and failed; no other cases

8: Similar to 7, but 
\vspace{-3.mm} 
\begin{verbatim}
2  3    1  'N3'         /| no daily iteration
1 39   99. 'GGT'        /| avoid iteration for convergence
2  9 9999  'NRSET'      /| avoid reset of layers
1 42 1.0223109   'DELJUL'     /| every sol
2  5  4033 'N5' / last of 6 years +1
2 12  1343 'Jdisk' / 2 year spinup
2 19   672 'Notif' / every year
\end{verbatim} 

\bibliography{mars}   %>>>> bibliography data
\bibliographystyle{plain}   % alpha  abbrev

\end{document} %===============================================================
% ===================== stuff beyond here ignored =============================
  

\begin{figure}[!ht] \igc{img/qq}
\caption[]{  \ qq.eps \label{qq} } \end{figure}  %qq
