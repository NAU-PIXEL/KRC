
\subsection{Insolation \ \ \ Implimented 2002jul13} %............................
Define one-layer atmosphere, grey in the solar and thermal regions.

The current local solar wavelength region atmospheric opacity of dust, $\tau$
can vary with atmospheric pressure: $\tau = \tau_0 \cdot P/P_0$

Direct and diffuse illumination are computed using a double-precision
Delta-Eddington model, with single scattering albedo $\varpi$ and
Henyey-Greenstein asymmetry parameter $G_H$ .

The incidence angle from zenith onto a horizontal surface is $i$. Direct
(collimated) insolation is computed for the local surface, which may be sloped
in any direction and has incidence angle $i_2$; direct insolation is zero when
either $i$ or $i_2$ is $>90$\qd.  Diffuse illumination is based on $i$, with the
optional extension into twilight (see Section \ref{sec:twi}). For a sloped
surface, the solid angle of skylight is reduced and light reflected off the
regional surface (presumed Lambert) is added; the Delta-Eddington downward
diffuse radiance is multiplied by $\cf{DIFAC}=(1-\alpha) + \alpha A $, where
$\alpha$ is fraction of the upper hemisphere obscured by ground.

The diurnal variation of insolation onto the surface at the bottom of the
atmosphere is computed for the current season and latitude: 
Incidence angles are computed by:
\qbn \cos i = \sin \delta \sin \theta - \cos \delta \cos \theta \cos \phi \qen
\qbn \cos i_2 = \sin \delta \sin \theta_2 - \cos \delta \cos \theta_2 \cos \phi_2 \qen
where 
\qi $\delta$ = the solar declination ,
\qi $\theta$ = latitude, 
\qi $\theta_2$ = latitude + slope north,
\qi $\phi$ = hour angle from midnight, 
\qi $\phi_2$ = hour angle + slope east.

If a sloping surface is used, the regional surface is assumed to be at the same
temperature (may be a poor approximation) and of the same albedo.

The incident flux at the top of atmosphere: $ I = S_M  \cos i$, \  
where  $S_M \equiv \frac{S_o}{U^2} $, $S_o$ is the solar constant and $U$ is
heliocentric range of Mars.

Use Delta-Eddington model for atmosphere scattering and fluxes (\ct{DEDING2.f});
output parameters are normalized to unit solar irradiance along the incident
direction at the top of atmosphere; so they must be multiplied by $S_M$ to get flux.
\qi  BOND      Planetary (atm plus surface system) albedo
\qi  COLL   Direct beam at bottom = collimated + aureole
\qi  RI(2,2)   Diffuse irradiances: 
\qii             (1,=  $I_0$ = isotropic  \ \  (2, =  $I_1$ = asymmetric
\qii             ,1)= at top of atmosphere  \ \  ,2) = at bottom of atm

 The net diffuse flux is $ F = \pi* [I_0 \pm(2/3)*I_1]$ where  + is
 down,$F^\downarrow$;   - is up, $F^\uparrow$ .  [Shettle \& Weinman eq. 8]

Solar heating of the atmosphere, by simple conservation of energy, is 
 $ H_V = S_M*\left( \mu_0-F^\uparrow(0)-(1-A_s) 
\left[ \mu_0 \mathrm{COLL} +F^\downarrow(\tau) \right] \right) $

\subsubsection{Twilight \label{sec:twi}}
Twilight is allowed to account for having a turbid atmosphere. It is implimented
as having the diffuse downward illumination depend upon an incidence angle
scaled to go to 90\qd ~when the Sun is \ct{TWILI} below the geometric horizon. 

Because of the twilight extension, there can be a small negative energy balance
near twilight. Physically, this is lateral scattering and does not strictly fit
a one-layer model. There is no solar heating of the atmosphere during twilight.

\subsubsection{Atmospheric IR radiation}

Assume a gray IR spectrum with opacity  $\tau_R =P/P_0 \cdot (C_1+C_2\tau)$
where $C_1$ represents the opacity of the ``clear'' atmosphere, primarily due to
the 15\um ~band, and $C_2$ is the IR/visual opacity ratio for dust.

The fractional thermal transmission of the atmosphere at zenith is
$e^{-\tau_R}$. Define the fractional absorption $\beta \equiv 1. -e^{-\tau_R}$.


The fractional transmission of planetary (thermal) radiation in a hemisphere is:
\qbn  e^{-\tau_e} \equiv\
  \int_0^{90} e^{-\tau/\cos \theta} \cos \theta \sin \theta  \ d \theta \qen

Numerical integration in \pname{hemi\_int.pro} shows that the effective
hemispheric opacity is, within about 0.05 in the factor,  
\qbn  \tau_e \sim \left[ 1.0 < 
( 1.50307  -0.121687 * \ln \tau_R )  < 2.0 \right] \tau_R \qen

 Define the effective absorption $\beta_e \equiv 1. -e^{-\tau_e}$.

The hemispheric downward (and upward) emission from a gray slab atmosphere is:
$ R_{\Downarrow t} =\sigma T_a^4 \beta_e$.

The IR heating of the atmosphere is: 
$H_R = \epsilon \sigma T_s^4 (1.-e^{-\tau_e}) -2R_{\Downarrow t}
  \ \  = \ \  \sigma \beta_e (\epsilon T_s^4-2 T_a^4)$

\subsubsection{Atmospheric temperature}

Compute atmospheric temperature by simple perturbation:$ \frac{ \partial
  T_a}{\partial t} = \frac{H_R+H_V}{c_p M_a} $ where $M_a=P/G$ is the mass of the
atmosphere and $c_p$ is its specific heat at constant pressure.  

Because the atmospheric temperature variation has significant time lag relative
to the surface, one can use the surface temperature from the prior time step
(typically 1/384 of a sol) with little error.

If the computed atmospheric temperature drops below the \qcc ~saturation temperature
for one scale height above the local surface, it is bounded at this value. This
strictly does not conserve energy. SHOULD BE RECODED

The nadir planetary temperature is given by 
\qbn \sigma T_P^4 = \epsilon \sigma
T_S^4 (e^{-\tau_R}) + \sigma T_a^4 (1-e^{-\tau_R}) \ \mathrm{or}  \
T_P = \left[ \epsilon (1.-\beta) T_S^4 +  \beta T_a^4 \right]^{1/4} \qen 

\subsection{Starting Conditions: Day-average equilibrium}

Atmosphere temperature balance, $H_V$ is solar (Visible) heating of the atmosphere:
\qbn <H_V>+\epsilon \sigma < T_s^4 > (1.-e^{-\tau_e}) = 2 (1.-e^{-\tau_e})
  \sigma < T_a^4 > \ \mathrm{or} \   <H_V>+\epsilon \sigma < T_s^4 > \beta_e = 2 <  R_{\Downarrow t} > \ \ [Eq:abal] \label{Eq:abal} \qen

Surface energy balance, where $I$ is the insolation onto the surface (ignoring
diurnal-average sub-surface heat flow)
\qbn \epsilon \sigma < T_s^4 > -  \epsilon \sigma (1.-e^{-\tau_e}) < T_a^4 >=
(1-A)<I>  \label{Eq:sbal} \qen

Solve: replace $+\epsilon \sigma <T_s^4> $ in Eq. \ref{Eq:abal}, and use $
\beta_e$ :
\qbn <H_V>+ \left[ \epsilon \sigma \beta_e < T_a^4 >
+(1-A)<I>\right]\beta_e  = 2\beta_e  \sigma < T_a^4 > \qen

\qbn < T_a^4 > = \frac{<H_V>/\beta_e + (1-A)<I>}{ \sigma (2- \epsilon \beta_e)} \qen

Substitute into Eq. \ref{Eq:sbal} to get $<T_s>$:
\qbn < T_s^4 > =  \beta_e < T_a^4 > + \frac{(1-A)<I>}{  \epsilon \sigma}  \qen

 
The planetary heating values can be based on the average surface temperatures
from the prior season; this is similiar to allowing some long-term lag in
atmospheric temperature response. They could be based on the prior day, but
there is then some concern over numerical stability.

\subsection{Liens}

Note: as of 2004jul19  $\beta$ is 1.- transmission

 Atm\_Cp.  should be 860. in krcin master.inp
%%% Local Variables: 
%%% mode: latex
%%% TeX-master: "krc"
%%% End