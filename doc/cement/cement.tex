\documentclass{article}  
\usepackage{epsfig}
\usepackage{../definc}  % Hughs conventions
% see definc.sty for other page format settings
\textheight=9.30in  \topmargin=-0.6in
\textwidth=7.0in  \oddsidemargin=-0.3in \evensidemargin=-0.3in 

% Used in place of  \qen for development to identify equation labels 
\newcommand{\ql}[1]{\label{eq:#1} \hspace{1cm} \mathrm{eq:#1} \end{equation}}
%\newcommand{\ql}[1]{\label{eq:#1} \end{equation} } % for final

\title{Particulate thermal conduction}
\author{Hugh H. Kieffer  \ \ File=~/xtex/themis/cement.tex 2010jul16}

% local definitions

\begin{document}
\maketitle
\tableofcontents
%\pagebreak
\subsubsection{History}
First version 2009feb09, modified 2010mar15
\qi 2010jul16 Add dependence on Knudsen Number and revise the numerical intergration section

 \section{Introduction}

This is an unsolicited attempt to look at the form of the variation of thermal
conductivity of unconsolidated and consolidated ``soils'' (spherical grains in a
lattice network) as a function of cement volume. It was instigated by a review
of the manuscripts by Sylvain Piqueux, \cite{Piqueux09a} [Paper 1] and
particularly \cite{Piqueux09b} [Paper 2], and was done largely in
2008nov-2009feb.

Many of the relations developed are coded in:
\np{idl/themis/partcond.pro}. Reference to '@n' is to part n of the large
case statement therein.

The equation numbers are the normal \LaTeX~ dynamic way of doing things; the
``eq:xxx'' to the right of some equations are additional static labels I use for
convenience and keying to the code.

\S \ref{sec:rad} develops the form for radiative conductivity, including fractional emissivity

\S \ref{sec:vol} relates the volume of cement disk (cylindrical walls) to the central angle between the host sphere contact point and the edge of the cement disk

\S \ref{sec:crd} Is a simple model of the cement as a flat plate of uniform
thickness, but same volume as a meniscus

\S \ref{sec:ana} Is the analytic solution [?] for a meniscus model. Although \S
\ref{sec:sol} show a solution, the corresponding calculation has wild results,
so there must be an error someplace. There are a couple changes of variables,
which complicates things a bit. \S \ref{sec:num} shows the form I used for
numerical integration, which is the basis for most of the Figures. \S
\ref{sec:smx} shows small-angle-approximations (SMX) used for some testing
(comparisons of full versus the much simpler SMX integrations); it is not used
in the results shown.

\S \ref{sec:code} Has a few random comments related to the IDL code.

\S \ref{sec:results} Has some results for input values similar to those of \cite{Piqueux09b}

2010jul14 Separate gas conductivity into two parts: between spheres $_g$ and across open cell $_G$, and each is dependant upon Knudsen number.

\section{Radiative conductivity in uniform particulate medium \label{sec:rad}}

Model as consistent heat flow across stack of radiative intervals. 

For packed spheres, the normalized density $F$ ranges roughly from $\pi/6=
0.5236$ to $\frac{\pi}{3\sqrt{2}} = .7405$ . A 1-m cube packed with spheres will
contain $F/ (\frac{4}{3} \pi r^3) $ spheres, so the number of ``layers'' along
one axis is $N= \sqrt{F/ (\frac{4}{3} \pi)}^3  \frac{1}{r} $

 For each interval \qb H/A=\epsilon \sigma_B T_1^4 -\epsilon \sigma_B T_2^4=
\epsilon \sigma_B\left(T_1^4-(T_1+\delta)^4 \right) \sim -\epsilon \sigma_B
4\delta T^3 \qe where $A$ is the radiating area, $\sigma_B$ the Stephan-Boltzmann
constant, and $\epsilon$ the emissivity (assumed to be near unity).

Basic definition of $k$ is $-H= A k \frac{dT}{dz}$. Without loss of generality,
assume a bulk volume of 1 m cube, with $\Delta T$ between upper and lower
surface. Across the stack, $\Delta T = N \delta $.

\qbn k=- \frac{H/A}{\Delta T} = \epsilon \sigma_B 4\delta T^3 \ / \ \Delta T = 4
\sqrt{ (\frac{4}{3} \pi) / F)}^3 r \epsilon \sigma_B  T^3  \mc{thus} \hcm1  k= 
 [ 3.564 : 4.0 ] D  \frac{\epsilon}{2-\epsilon} \sigma_B  T^3\qen
 Where the values in brackets is $f_P$, the packing factor, $D$ is the grain diameter in meters and the values with brackets represent the range for various packed spheres, the result for non-unit emissivity from the next section is included.

\subsection{Non-unit emissivity}

Between two plane surfaces of the same emissivity,  heat transfer is $ H/A = F_\downarrow - F_\uparrow $ where 
\qb      F_\downarrow = \epsilon \sigma T_2^4 + (1-\epsilon) F_\uparrow 
\mc{and}   F_\uparrow = \epsilon \sigma T_1^4 + (1-\epsilon) F_\downarrow \qe
Solve:  \qb  F_\downarrow = \epsilon \sigma T_2^4 + (1-\epsilon) \left(
 \epsilon \sigma T_1^4 + (1-\epsilon) F_\downarrow \right)
=\frac{(1-\epsilon)\epsilon \sigma T_1^4+\epsilon \sigma T_2^4}{1.-(1-\epsilon)^2} \qe

similarly \qb F_\uparrow  = \frac{(1-\epsilon)\epsilon \sigma T_2^4+\epsilon \sigma T_1^4}{1.-(1-\epsilon)^2} \qe

\qb H/A= \frac{\epsilon \sigma}{2\epsilon-\epsilon^2} \left[ (1-\epsilon)T_1^4+T_2^4 -\left( (1-\epsilon)T_2^4+T_1^4) \right) \right] =  \frac{\epsilon}{2-\epsilon} \sigma \left[ T_2^4-T_1^4  \right]  \simeq  \frac{\epsilon}{2-\epsilon} 4 \sigma T^3  \Delta T \qe 

\section {Volume of cement versus contact angle \label{sec:vol}}
Symbols 
\qi $R =$ radius of conductive sphere
\qi $x =$ central angle from contact point.  Unit cell has side of $2R$. 
\qi $B =$ central angle to edge of cement cylinder
\qi $V_1 =$ volume of one cement ring between two spheres
\qi $V_c=$ volume of 3 orthogonal cylinders as fraction of unit cell  $= \frac{3V_1}{(2R)^3}$
\qi $V_s=$ volume of 3 orthogonal cylinders as fraction of host sphere  $= \frac{3V_1}{\frac{4}{3} \pi R^3}$

Model a face-centered lattice, wherein spheres are stacked directly on top of each other. Initial pore proportion is $1-\frac{4}{3*8} \pi = 0.4764$

\qbn V_1 = \int_0^B 2 \pi R \sin x \cdot 2 R(1-\cos x) \cdot d \, (R \sin x) \qen

for $B$=90\qd, this formulation should yield 
 $ V_1 = \pi R^2 \cdot 2 R - \frac{4}{3} \pi R^3  \equiv \frac{2}{3} \pi R^3 $,
although at 45\qd, cylinders begin to intersect and the formula is no longer correct.

\qbn V_1 = 4 \pi R^3\int_0^B  \sin x \cdot (1-\cos x) \cdot d \sin x \ql{V_1} 

\qbn Q \equiv \frac{V_1}{4 \pi R^3} =  \left( \int_0^B  \sin x  \cos x  \, dx 
  - \int_0^B  \sin x  \cos^2 x  \, dx \right) \qen

\qbn  Q = \left[ _0^B -\frac{\cos^2 x}{2}\right] -  \left[ _0^B 
 - \frac{\cos^3 x}{3} \right] \qen

\qbn Q = \left( -\frac{1}{2} (\cos^2 B -1) \right) - \left( -\frac{1}{3} (\cos^3 B -1) \right)  
 \equiv  \frac{1}{2} (1- \cos^2 B ) -\frac{1}{3} (1- \cos^3 B ) \ql{B}

For $B= \pi/2$, this yields $Q=1/2 -1/3 =1/6 $ , as expected.  Volume of 3-rings of cement, relative to the volume of sphere, is $V_s \equiv 3 V_1 / \frac{4}{3} \pi R^3 =9Q$ is only valid for $B<45^\circ$
\subsection{ Small angle approximation} 
For small angles, these involve small differences of large numbers, so use small-angle approximations: 
\\ $\cos x \simeq 1 - \frac{x^2}{2} + \frac{x^4}{4!}  $
 \  and $\left( 1.-\delta\right) ^ n \sim 1-n\delta.$ 

Then  $\cos^2 x \sim 1.- x^2$ and  $ \cos^3 x \sim 1.- \frac{3}{2}x^2$
 \ \ so \ \ $ 1- \cos^2 x=  \sim x^2$ and $1- \cos^3 x \sim  \frac{3}{2} x^2 $. \qr{B} becomes: 

\qbn Q = \sim \left( \frac{B^2}{2} - \frac{B^2}{2} \right)
\equiv 0 \qen 

Thus, need to include higher-order terms. Including higher-order
terms in the trig. functions after integration generates a lot of terms. Rather,
try small-angle approximations before integration, and \qr{V_1} becomes:

\qbn  Q  \simeq \int_0^B x \cdot \frac{x^2}{2} \cdot  \, dx 
\equiv   \left[ _0^B \frac{x^4}{8} \right]
  \mc{or}  V_1 \simeq \frac{\pi}{2}R^3 B^4  \ql{Vb}
where the ignored terms are two or more degrees higher than those included.

With this approximation, $ V_c \simeq \frac{3 \pi}{16} B^4 $ and  $ V_s \simeq \frac{9}{8} B^4 $ ; cement volume as fraction of solids is $V_s/(1+V_s)$ 
 
\section {Gas conductivity \label{sec:gas}} % -----------------------
Follow the development of \cite{Piqueux09a}, where the conductivity of a gas in
a pore $k_\mathrm{gas}$ is a function of the bulk gas conductivity at the
appropriate temperature, $k_0$ and the Knudsen number $Kn$.

 $k_0$ is assumed to be known; Paper 1 Eq. 14 gives an emperical relation for \qcc.   Paper 1 Equations 30 to 32 give
\qb  k_\mathrm{gas} = \frac{k_0}{1+e^\frac{2.15 -\mathrm{Log}_{10} \left(Kn^{-1}\right) }{0.55}} \qe

The Knudsen number is the ratio of the mean free path to the pore size $L$, or (Paper 1, Eq. 7) 
\qbn Kn = \frac{k_B}{\sqrt{2}\pi d_m^2} \cdot \frac{T}{LP} \qen
where $k_B$ is the Boltzmann constant, and $d_m$ is the molecule collisional diameter;  the first term is a constant for a given molecule.

Here, $L$ is taken to be $2R$ in the open corner region and the average
separation of the spheres, $L_S$, in the region between the outer edge of the
cement and the projected edge of the grain.

\qb L_S = \mathrm{Volume / Area } = \int_B^{\pi/2} 2R \left( 1- \cos \theta \right) 2 \pi R \sin \theta  \: R \cos  \theta \: d \theta \ / \ \int_B^{\pi/2} 2 \pi R \sin \theta \: R \cos  \theta \: d \theta \qe

\qb = 2R\left(  \int_B^{\pi/2} \left( \sin\theta \cos\theta -\sin\theta \cos^2\theta \right) d \theta
  \ / \  \int_B^{\pi/2} \sin\theta \cos\theta d \theta \right) \qe

\qb = 2R\left(  \bigg[_B^{\pi/2}  \frac{\sin^2 \theta}{2}  +  \frac{\cos^3 \theta}{3}  \bigg]
  \ / \ \bigg[_B^{\pi/2} \frac{\sin^2 \theta}{2}   \right) \qe

\qbn L_S= 2R\left( 1 - \sin^2 B -  \frac{2 \cos^3 B}{3}  \right) 
 \ / \ \left( 1 - \sin^2 B  \right)  \qen

Check limits: $B \rightarrow \frac{\pi}{2}, L_S \rightarrow 2R $, correct; 
$B \rightarrow 0, L_S \rightarrow  \frac{2}{3}R $, correct. 

To avoid complications in attempt at analytic solutions, the $B \rightarrow 0$
limit has been used for all $B$ . This is within 10\% for $B<0.3$
radian. However, the full $L_S$ will be used in the numerical integrations.
%IDL> b=.01*findgen(157)
%IDL> qq=(1.-sin(b)^2 -(2./3.)*cos(b)^3)/(1.-sin(b)^2)
%IDL> plot,qq

\section{Crude estimate of net effect of small amounts of cement \label{sec:crd}} %_____________

Model is a stack of ``cement'' plate of radius $r_c$ that is $Y$ thick, in
series with a cylinder of the sphere material of cross-section $\sigma_h$ that
is $2R-Y$ thick. Treat heat flow as strictly parallel to axis, except for
``magic'' lateral dispersal at the cement-host interface.

 For this crude approximation, include only conduction through the solids,
 ignoring the effect of gas conduction and radiation.

 Let $r_c$ be the radius of the cement cylinder ($\sim BR$ above).
 Let $Y$ be the average thickness of the cement in the heat-flow direction: 
$Y \equiv V_1/\sigma_c $.

 Let the thermal conductivity of the sphere (the ``host'' material) be $k_h$ and
 that of the cement be $k_c$.

Because of the relatively larger physical cross-sections of the host sphere than
the cement ring, assume the effective cross-section of the host sphere
$\sigma_h$ to be $G \sigma_c$; the minimum for G would be unity. One possibility
for $\sigma_h$ would be the geometric mean between the cross-section of the
cement ring and central cross-section of the sphere, 
$ G\sigma_c=\sqrt{\sigma_c \cdot \pi R^2 }$ or $G=1/B$. 
An alternate would be to assume the average: 
$G \sigma_c=\frac{ \sigma_c +\pi R^2 }{ 2 }$ or $G=\frac{1/B^2+1}{2}$.

Because cement and host are in series, heat-flow is the same in the cement and host portions. 

\qbn H= \Delta T_c  k_c \sigma_c /Y = \Delta T_h k_h \sigma_h/(2R-Y) \qen

where $\Delta T =  \Delta T_c + \Delta T_h$. Effective net conductivity is $k= H/(2R \Delta T) $

\qbn  k= \frac{H / \, 2R}{ \frac{YH}{k_c\sigma_c} +  \frac{(2R-Y)H}{k_h G \sigma_c} } \qen


With some manipulation; get effective thermal conductivity of the cell:
\qbn \frac{k}{k_h}=\frac{ \sigma_c / ( 2RY)}{ \frac{k_h}{k_c}+ \frac{2R/Y \  -1}{G}  } \qen

 Using \qr{Vb} and 
with the small-angle approximation: $Y=\frac{1}{2}B^2R$ and $\sigma_c= \pi (BR)^2 $ and $\frac{\sigma_c}{2RY} =\pi $.


\qbn \frac{k}{k_h}=\frac{\pi}{ \frac{k_h}{k_c} +  \frac{1}{G} \left( \frac{4}{B^2} -1 \right)  } \mc{and} B=\sqrt {\frac{16}{3 \pi} V_c}^4 \ql{kkh}

Implemented @2. The results for several options for $G$ are shown in Figure \ref{fig:plate}. 
 For very small cement fractions, the gas conductivity, and possibly radiation,
 would be important.

\begin{figure}[!h] 
% \resizebox{!}{12cm}{\rotatebox{90}{\includegraphics*[0.86in,0.5in][7.75in,10.5in] {plate.eps}}}
\caption{Normalized conduction through circular flat plate of cement between cylinders of host material for several options of $G=\sigma_h /\sigma_c$. Largest $B$ is 40\qd. }
\label{fig:plate}
\end{figure}

\section {Attempt at analytic model for both non-cemented and cemented \label{sec:ana}}

Try for face-centered (Paper 1, Fig. 5 left). Start with assumption that all
heat flow is parallel to vertical axis. Thus, for any radial distance from
contact point, up to end of cylindrical cement, $B'$, which may be zero, heat
flow in host is equal to that in cement. Outside this and up to particle radius
$R$, heat flow in solid is equal to sum of gas and radiation heat flow; for an
infinitesimal area \qbn H_s = H_g+H_r \ql{H} Outside $R$, there is only gas and
radiation heat flow across the full unit cell.

% Consider a cubic unit cell of side length 2$R$. 

Let temperature of lower side of cell (through center of particle) be $T_1$ and
that of upper side be $T_4$, temperature at upper surface of lower particle is
$T_2$, temperature at mid-line is $T_m$, by symmetry (ignoring the small change
in $T^3$, $T_m=(T_1+T_2)/2$. For notation convenience, define $U \equiv T_m-T_1$
and temperature change across 1/2 the gap is $\delta \equiv T_m-T_2$.  $R$ is
the sphere radius (and 1/2 the unit cell side), $r=R \sin x$ is the radial
coordinate, and $y$ is the vertical distance between the sphere surface and the
mid-line; $ y=R(1-\cos x)$; for small angles $y\simeq R x^2/2 $.

General form for heat flow: $\frac{dH}{dA} = k \bigtriangledown T $.  For
notation simplicity, represent $\frac{dH}{dA}$ by $H$ for each component of heat
flow.

Total heat flow across the unit cell is $ \mathbf{H_t}=\int_0^R H_{(r)} 2 \pi r
dr + \mathbf{H_o} $ where $ \mathbf{H_o}$ is the heat flow across the open
corners of the cell where there is no particle. Beware: $H_{(r)}$ is total heat
flow for infinitesimal area and $H_r$ is only the radiative portion.

Effective net conductivity:
\qbn k_t=\frac{\mathbf{H_t}}{A}\frac{\Delta z}{\Delta T} = \frac{\mathbf{H_t}}{4R^2} \frac{2R}{2 U}=  \frac{\mathbf{H_t}} {U 2 \pi R^2} \cdot \frac{ \pi R}{2} \qen

Total heat flow:
 \qbn \mathbf{H_t}= \int_0^{B'} H_c 2 \pi r  d r + \int_{B'}^R (H_g+H_r) 2 \pi r  d r  +\int_R^\mathrm{corner}  (H_G+H_r) dA \qen

Sticking with $r$ as the independent variable leads to forms such as $\int \frac{r}{b+\sqrt{a^2-r^2}} dr$, for which I could not find a solution. Hence the formulation in terms of the central angle $x$ where $r=R \sin x $.

\qb \int_A^B F_{(r)} r dr = R^2\int_{\arcsin \frac{A}{R}}^{\arcsin \frac{B}{R}} F_{(r\rightarrow x)}  \sin x d \sin x  \mc{And}  d \sin x = \cos x \ dx \qe

Change coordinates from radial distance from the contact point to angle 
from the contact point:
\\ $ \mathbf{H_t-H_o}= 2 \pi R^2 \int_0^{\pi/2} H_{(x)} \sin x \ d \ \sin x $.
Central angle of the edge of cement is $B$.
 
For a possible cemented section, allow the effective cross-section in the host to be larger than strictly parallel heat flow by a factor $G_{(B)}$; if this causes significant complications, will drop this option.


\qbn  \mathbf{H_t}= \int_0^{B} H_c 2 \pi R \sin x  d x + \int_{B}^{\pi/2} (H_g+H_r) 2 \pi R \sin x  d x   +\int_R^\mathrm{corner}  (H_G+H_r) dA \qen

$H_c= k_c \frac{\delta_c}{y} $ and $H_g=k_g\frac{\delta_c}{y} $  and $H_r= \frac{\epsilon }{2-\epsilon }4 \sigma_B T^3 \cdot 2 \delta  \equiv C \delta $. Units of $C$ are  W m$^{-2}$ K$^{-1}$. 

For $0 \leq x \leq B $, heat flow in the host is same as in the cement
\qbn k_h G \frac{U-\delta_c}{R-y} = k_c \frac{\delta_c}{y} \mc{and}  \delta_c /U = \frac{G k_h}{R-y}  / \left[ \frac{ G k_h }{R-y} +\frac{k_c}{y} \right] 
= \frac{1}{1+\frac{k_c}{k_h} \frac{R-y}{Gy} }
= \frac{1}{1+\frac{k_c}{Gk_h} \frac{\cos x}{1-\cos x} }\ql{delcu}

\qbn \frac{\delta_c}{Uy}=\frac{1}{y+ \frac{k_c}{k_h}\frac{R-y}{G} } 
= \frac{1}{R(1-\cos x)+ \frac{k_c}{Gk_h} R \cos x }
= \frac{1/R}{ 1+\left( \frac{k_c}{Gk_h} -1 \right)  \cos x }  \ql{delc}

For $B < x \leq \pi/2 $, expand terms in \qr{H} (all units are W m$^{-2}$).
\qbn k_h\frac{U-\delta}{R-y} = k_g\frac{\delta}{y}+C \delta  \mc{and} 
\delta /U =   \frac{ k_h}{R-y} / \left[ \frac{k_h }{R-y} +\frac{k_g}{y} + C \right]  \ql{del} 

%\pagebreak

\qb \delta /U =  \frac{1} {1+\frac{k_g}{y} \frac{R-y}{k_h} + C\frac{R-y}{k_h} } 
= \frac{1}{1+\frac{k_g}{k_h} \frac{\cos x}{ 1.-\cos x}  + \frac{CR}{k_h} \cos x}
=\frac{1-\cos x}{1-\cos x + \frac{k_g}{k_h}\cos x +(1-\cos x) \frac{CR}{k_h} \cos x} \qe

\qbn = \frac{1-\cos x}{1 +\left( \frac{k_g+CR}{k_h} -1 \right) \cos x -\frac{CR}{k_h} \cos^2 x} \ql{delU}

\qbn \frac{\delta}{yU}=  \frac{\delta}{U} \frac{1}{R \left( 1-\cos x \right) } 
= \frac{1/R}{1 +\left( \frac{k_g+CR}{k_h} -1 \right) \cos x -\frac{CR}{k_h} \cos^2 x} \ql{yU}

\subsection{Solved integrals \label{sec:sol}} % -----------------------------

\qbn \mathbf{H_t} =  2 \pi R^2 \int_0^B  k_c \frac{\delta_c}{y}  \sin x \ \ d \sin x 
+ 2 \pi R^2 \int_B^\frac{\pi}{2} \left( k_g\frac{\delta}{y}  + C \delta \right) \sin x \ d \sin x + (4-\pi)R^2 \left[k_G \frac{U}{R} + C U \right] \ql{Ht}
Note that $k_g \frac{\delta}{y} +C\delta = \left( k_g+CR \right)\frac{\delta}{y} -CR \frac{\delta}{y} \cos x$ in order to separate integrals of similar form.

\qb \frac{\mathbf{H_t}} {U 2 \pi R^2} 
=  k_c\int_0^B  \frac{\delta_c}{Uy} \sin x  \ d \sin x  
+ \left( k_g+CR \right) \int_B^\frac{\pi}{2} \frac{\delta}{Uy}\sin x \ d \sin x
- CR \int_B^\frac{\pi}{2} \frac{\delta \cos x}{Uy} \sin x \ d \sin x 
+ \frac{4-\pi}{2 \pi} \left[ \frac{ k_G}{R} + C \right] \qe

\qb =  \frac{ k_c}{R} \underbrace{\int_0^B  \frac{1}{ 1+ \left(\frac{k_c}{Gk_h} -1 \right) \cos x }    \sin x  \ d \sin x }_{I1} 
+\left(\frac{k_g}{R}+C \right)  \underbrace{ \int_B^\frac{\pi}{2} \frac{1}{1+\left( \frac{k_g+CR}{k_h}-1 \right) \cos x - \frac{CR}{k_h} \cos^2 x } \sin x \ d  \sin x }_{I2}\qe

\qbn -C \underbrace{\int_B^\frac{\pi}{2}  \frac{\cos x}{1+\left( \frac{k_g+CR}{k_h}-1 \right) \cos x - \frac{CR}{k_h} \cos^2 x }  \sin x \ d \sin x}_{I3} +  \left[ \frac{ k_G}{R} + C \right]  \underbrace{\frac{4-\pi}{2 \pi}}_{I4} \ql{HU}

Integral I1 has the form: $ \int \frac{ \cos x \sin x}{a+b \cos x } dx $.  Use variable transform $z = \cos x$ to get  $ \int - \frac{ z }{a+b z } dz$, and the integration limits become 1. to $\cos B$.  Then use \cite{Dwight61} 91.1 for $X \equiv a+bx$ to get 
 \qbn \int \frac{x \ dx}{X} = \frac{1}{b^2}\left[ X -a \ln \left| X \right| \ \right]\ql{i1} 


Integral I2  has the form:
$ \int \frac{ \cos x \sin x}{1+b \cos x + a \cos^2 x} dx $. Use variable transform $z = \cos x$ to get  $ \int - \frac{ z }{1+b z + a z^2} dz$, and the integration limits become $\cos B$ to 0.   Then use \cite{Dwight61} 160.11 for $X \equiv ax^2+bx+c$ to get 
\qbn \int \frac{x \ dx}{X} = \frac{1}{2a} \ln \mid X \mid -\frac{b}{2a} \int \frac{dx}{X}\ql{i2} 
The RHS integral has solution that depends upon sign of $b^2-4ac$, see \cite{Dwight61} 160.01. This looks promising, so get rough order of terms using typical values.  Using $\epsilon=0.9$ and $T=240.$ yields $ C= 5.13$;  using  $R=$100.E-6 and  $k_h=$ 2, yields  $a = -\frac{CR}{k_h}$=- 2.6E-4; using $k_g=$ 0.012, get $b= \frac{CR+k_g}{k_h}-1=$-.988 . $c=1$  and $b^2-4ac$=+.979 

For $b^2 > 4ac $,  roots of $X=0$ are real; $p,q =\frac{-b \pm \sqrt{b^2-4ac} }{2a}$ and use  \cite{Dwight61} 160.01; \qbn \int \frac{dx}{X}=\frac{1}{a \left( p-q \right)} \ln \left| \frac{x-p}{x-q} \right| \ql{dxox}
Integral I3; can be written: 
\qbn I3 =   \int_{\cos B}^0 \frac{z^2}{a z^2 + b z +c} dz, \ql{i3} 
where the three constants have the same values as in Integral I2; use  \cite{Dwight61} 160.21 
\qbn \int \frac{x^2dx}{X} = \frac{x}{a}-\frac{b}{2a^2} \ln \left| X \right| + \frac{b^2-2ac}{2a^2} \int \frac{dx}{X} \ql{i32} 
The last integral has the same form as \qr{dxox}. 

Implemented @4.

\subsection{Numerical Integration \label{sec:num}} %---------------

The forms in \qr{HU} can be integrated by any of several means. Because this
simple model does not have lateral heat flow, the cement part can be integrated
outward once (for any set of fixed parameters excluding $B$) saving the results
for increasing $B$ and the non-cement parts integrated inward once, also saving
the results as a function of $B$. Then the results for any $B$ found by simply
adding the results.
   
Numerical integration alternate to I2 and I3 is to retain form in \qr{Ht} and
compute $ \int_B^\frac{\pi}{2} \left( k_g\frac{\delta}{Uy} + C \delta/U \right)
\sin x \ d \sin x $ where the () term is $\frac{\left( \frac{k_g}{y} +C \right)
\cdot \frac{k_h}{R-y}} {\left( \frac{k_g}{y} +C \right) + \frac{k_h}{R-y}}$. This is a check against some algebra or coding errors, and was found to yield the identical result.

Expand \qr{Ht} in the fashion of \qr{HU}, but keeping gas and radiation separate:
\qb \frac{\mathbf{H_t}} {U 2 \pi R^2} 
=  k_c\int_0^B  \frac{\delta_c}{Uy} \sin x  \ d \sin x  
+\int_B^\frac{\pi}{2} \frac{ k_g \delta}{Uy}\sin x \ d \sin x
+ C \int_B^\frac{\pi}{2} \frac{\delta }{U} \sin x \ d \sin x 
+ \frac{4-\pi}{2 \pi} \left[ \frac{ k_G}{R} + C \right] \qe

\qb =  \frac{ k_c \Delta x}{R}\sum_0^B   \underbrace{\frac{1}{ 1+ \left(\frac{k_c}{Gk_h} -1 \right) \cos x }    \sin x  \cos x }_{J1} 
+\frac{ \Delta x}{R} \sum_B^\frac{\pi}{2}  \underbrace{\frac{k_g}{1+\left( \frac{k_g+CR}{k_h}-1 \right) \cos x - \frac{CR}{k_h} \cos^2 x } \sin x  \cos x }_{J2}\qe

\qbn +C \Delta x \sum_B^\frac{\pi}{2} \underbrace{\frac{1-\cos x}{1 +\left( \frac{k_g+CR}{k_h} -1 \right) \cos x -\frac{CR}{k_h} \cos^2 x} \sin x \cos x}_{Jr} 
+ \left[ \frac{ k_G}{R} +C \right] \underbrace{\frac{4-\pi}{2 \pi}}_{I4} \ql{HN}

Implemented @43, with Small-angle-approximation (SMX) for $1-\cos x$ if single precision and $x<0.001$.
\subsection{Small-angle-approximation integrals \label{sec:smx}} %---------------
For each of the integrals in \qr{HU}, evaluate for $x \ll 1$: then
 $\sin x \simeq x$ \ and \ $\cos x \simeq 1-x^2/2 $   

Make $B$ small and replace $\pi/2$ with $q=2B$ as a test of coding the solved integrals.

 I1, using $b= \frac{k_c}{Gk_h} -1  $ and  $X=g^2 \pm x^2$, \cite{Dwight61} 121.1 and 141.1 :  
 
\qb \int \frac{x}{1 + b(1-x^2/2)} dx 
= \int \frac{x \ dx}{1 + b - \frac{b}{2}x^2} \qe
\qbn  \mathrm{I1} \stackrel{b>0}{\longrightarrow} \frac{2}{b}\int \frac{x \ dx}{g^2 - x^2} 
\stackrel{141.1}{\longrightarrow}  \frac{2}{b} \left[ \frac{-1}{2} \ln \mid g^2-x^2 \mid \right] 
\Longrightarrow \frac{-1}{b} \left[ \ln ( \mid g^2-\epsilon^2) \mid - \ln g^2 \right]  \ql{S1}

 where $g^2=2\frac{1+b}{b}$. \ If $b<-1 $ then: \qbn  \mathrm{I1} 
\stackrel{b<-1}{\longrightarrow} \frac{2}{b}\int \frac{x \ dx}{g^2 + x^2} 
\stackrel{121.1}{\longrightarrow} \frac{2}{b} \left[ \frac{-1}{2} \ln  (g^2+x^2) \right] \Longrightarrow  \frac{-1}{b} \left[ \ln (  g^2+\epsilon^2) - \ln g^2 \right] \qen

If $-1<b <0 $ need to define  $g^2=-2\frac{1+b}{b}$, then: 
\qbn \stackrel{-1<b<0}{\longrightarrow} \frac{-2}{b}\int \frac{x \ dx}{g^2 + x^2} 
\stackrel{121.1}{\longrightarrow}  \frac{-2}{b} \left[ \frac{-1}{2} \ln g^2+x^2 \right] 
\Longrightarrow \frac{1}{b} \left[ \ln ( g^2+\epsilon^2) - \ln g^2  \right] \qen

\subsubsection {Treating cos x as 1.}

I1 becomes  $ \int_0^B  \frac{ \sin x }{ 1+ \left(\frac{k_c}{Gk_h} -1 \right) \cos x }    \ d \sin x  \rightarrow \int_0^B  \frac{ \sin x }{c_1 }  d \sin x  $ where $c_1=\frac{k_c}{Gk_h}$  yielding  $ = \left[_0^B  \ \frac{\sin^2 x}{2 c_1} \right] = \frac{\sin^2 B}{2c_1} $.

I2 becomes $ \int_B^p \frac{1}{1+\left( \frac{k_g+CR}{k_h}-1 \right) \cos x - \frac{CR}{k_h} \cos^2 x } \sin x \ d  \sin x  \rightarrow  \int_B^p\frac{\sin x}{1+\left( \frac{k_g+CR}{k_h}-1 \right) - \frac{CR}{k_h} }  \ d  \sin x  $ where the denominator is now constant $c_2 =  \frac{k_g+CR}{k_h} - \frac{CR}{k_h} = \frac{k_g}{k_h} $ yielding $ = \left[_B^p  \ \frac{\sin^2 x}{2 c_2} \right] = \frac{\sin^2 p -\sin^2 B}{2c_2} $.

Similarly, I3 becomes $ \int_B^p \frac{\cos x \sin x}{c_2 }  \ d  \sin x  = \frac{1}{c_2} \int_B^p \cos^2 x \sin x dx  = \frac{-1}{c_2} \left[ _B^p \  \frac{ \cos^3 x}{3} \right] $ and $\cos^3 \epsilon  \simeq 1.-\frac{3}{2} \epsilon^2$. 
\\ Thus I3 $ = \frac{\cos^3 p - Cos^3 B}{3 c_2} 
\simeq \frac{3 (p^2-B^2) /2}{3 c_2} = \frac{p^2-B^2}{2 c_2}  $
\section {Comments on the IDL code \label{sec:code}} % -----------------------

Code model in \np{partcond.pro}. Include both analytic and numerical
integration. Also include test section with small-angle-approximation where
replace $\pi/2$ limit with $2B$. Coded as a large case statement (as is my
habit) so it is not very easy to follow.

\subsection{Arguments} %-----------------------------------------
\subsubsection{Parameters} %...................................
Default values for input parameters:
\begin{verbatim} 
@16: Inputs: Float values
       0      0.00000  edit: flag -=stop >1=help 
       1      100.000  Rg: grain radius, mu m
       2      250.000  T: temperature, K
       3     0.937000  kh: grain thermal conductivity
       4   0.00300000  k0: bulk gas cond.  J/m.s.K
       5      2.00000  kc: cement cond.  =SI
       6     0.980000  emis: grain emissivity
       7     0.900000  G: host grain cond. factor -1=ave -2=geomean
       8      500.000  P: Pressure in Pascal
       9    0.0100000  B: Cement contact angle, radian
      10  4.65000e-10  Dm: molecule collision diameter, m
      11      77.7700  vvvv:--numerical parameters
      12      0.00000  flag: test integrals
      13     0.100000  beta: SMX limit radian
      14      30.0000  nLo: steps along cement [Int]
      15      100.000  nHi: steps along gap [Int]
      16      0.00000  DP: Flag, double precision
      17   0.00100000  B1: Initial B
      18      1.10000  Brat: ratio for loop
\end{verbatim}
 For these values, the contributions to thermal conductivity from the corner
 areas are: gas = 0.000643805, radiation = 0.000146148, and their sum =
 0.000789953.

\subsubsection{Actions} %...................................
Targets in the large case statement:
\qi \at -4... return 43 [nB,2] 0]=volume fraction cement  1]=net conductivity
\qi \at -2... 2: return,knet
\qi \at 0.... Stop
\qi \at 123.. start auto-script 
\qi \at 125.. kons=[20,4,42,44,126,-3] Loop over set of Bval
\qi \at 126.. loop sequence
\qi \at 127.. kons=[20,43,128,-3] Loop kc  REQ 43 
\qi \at 128.. loop sequence
\qi \at 131.. kons=[20,2,43,-1,431,-1,432,-1,433,-1,434,-1,435,-1,436] std
\qi \at 132.. kons=[4,42,44,51,-1,57,-1,577] Details for one Bval
\qi \at 16... Modify parr
\qi \at 20... Set limits and constants. Auto after 16
\qi \at 2.... Crude cement only
\qi \at 4.... Solved integrals 
\qi \at 42... Simplest numerical integration
\qi \at 425.. Simplest numerical integration 2010jul
\qi \at 43... Numerical integration for all B
\qi \at 4302. Test of 42 \& 43
\qi \at 431.. CHART: vvv
\qi \at 4312. vrs vol.
\qi \at 432.. CHART www
\qi \at 4322. vrs vol.
\qi \at 433.. Plot k\_t
\qi \at 434.. vrs vol.
\qi \at 435.. Plot 5A1
\qi \at 436.. 5B1
\qi \at 437.. Plot 5A
\qi \at 4372. 2nd set
\qi \at 438.. 5B
\qi \at 44... Sum parts REQ 4,42
\qi \at 51... Plot del/U vrs B
\qi \at 52... CHART,qqq
\qi \at 53... CLOT,qqq
\qi \at 54... CHART knet
\qi \at 56... CHART qab
\qi \at 561.. `` qb/qa
\qi \at 562.. `` qb*qa
\qi \at 57... CLOT qab
\qi \at 577..  `` Log10
\qi \at 58... Add parr 12:14 Subtitle 
\qi \at 77... oldstuff:

Standard actions of \np{kon91}: \begin{verbatim}
0=Stop  121=kons=-3  122=Edit Kons      801/803 output to LP/.jpg
Plots : 8=new  80=restart  85[x]=SETCOLOR  87=close  88=subtitle  9=plot
MAKE99: 991=Expand current kons   992/995=1-line each   994=expand all \end{verbatim}
\subsubsection{Arrays} %...................

Contents of the \nv{vvv} [*,8] array (Figure \ref{fig:431}): 
\qi 0] $B$ as independent variable
\qi 1] $G$, the Grain conductivity factor
\qi 2] Knudsen number
\qi 3] gas conductivity 
\qi 4] denominator in I2 and Ir
\qi 5] J1 function; Normalized heat flow through cement
\qi 6] J2 function; Normalized gas heat flow
\qi 7] Jr function; Normalized radiative heat flow

 Contents of the \nv{www} [*,4] array (Figure \ref{fig:432}): 
\qi 0] outward summation of J1 = $q \frac{k_c}{R} \cdot \sum $ vvv[*,5] 
\qi 1] inward summation of J2 = $q \frac{1}{R} \cdot \sum $ vvv[*,6]
\qi 2] inward summation of Jr = $q C \cdot \sum $ vvv[*,7]
\qi 3] total of above 3, plus the open corner conduction 
\qii where $q= \frac{\Delta x}{ \frac{\pi}{2} R}$ 

\subsubsection{Common Sequences} %...................................
Input for $G$ = parr[7]. 
\qi if $\leq -2$, uses  $G=1/B$
\qi else if $\leq -1$, uses $G=\frac{1/B^2+1}{2}$.
\qi else if $<1$ uses $G=1/B^x$
\qi else if $\geq 1$ uses that fixed value

@42 does one B in detail

@43 does a larger range of B values at once.

To generate results for 6 values of $k_c$ for 130 values of $B$ takes 8 millisec.

Make Fig \ref{fig:5A} by defaults, 127,437, 16,: 7=G=.5, 127, 4372, 16: 7=G=.9, 127, 4372
\qi Make  Fig \ref{fig:5B} as above; change only 437 to 438, leave 4372

\section {Results \label{sec:results}} % -----------------------------

NOTE: All in this sub-section used $k_g = k_G=$ constant. The routine is
archived as partcond.2010may16

The small value of $a$ in the analytic integrals leads to large values for some
terms in I2 and I3, and I have not been able to get realistic results; must be
an error in the algebra or code someplace.

Analytic solution to integrals involves small differences of large numbers, but
single versus double precision yields virtually same result, so numerical
precision is not the source of the unrealistic results.

Numerical integration of alternate form (last sentence of \S \ref{sec:sol}) had
initially been run in double precision to avoid zero divide, so I added a
small-angle approximation, single precision now agrees with numerical
integration of \qr{HU} to six decimal places. The ratio of $\frac{\left(
\frac{k_g}{y} +C \right)}{\frac{k_h}{R-y}}$ ranges over 6 orders of magnitude,
whereas their product ranges over 3.5 orders of magnitude.

Coded test cases for small-angle approximation and numeric integrations agree
well.

\begin{figure}[!h] 
% \resizebox{!}{12cm}{\rotatebox{90}{\includegraphics*[0.86in,0.5in][7.75in,10.5in] {431gp9.eps}}}
\caption{Chart of radial behavior, each scaled to the ranges listed, for the
independent heat-flow elements; conditions are shown in the upper label. The top
three curves are the heat flow due to cement conduction, gas conduction, and
radiation, as a function of radius. del/U is the fractional temperature jump
across the open gap; delc/U the corresponding temperature difference with
cement in place; there would be a substantial lateral temperature discontinuity
at the edge of the host sphere with this model when B is $~0.2$ radian.} 
\label{fig:431} \end{figure}

\begin{figure}[!h] 
% \resizebox{!}{12cm}{\rotatebox{90}{\includegraphics*[0.86in,0.5in][7.75in,10.5in] {432gp9.eps}}}
\caption{Chart of heat-flow magnitudes and net conductivity as a function of
central angle to the edge of cement, same conditions as Figure \ref{fig:431}.
See \S \ref{sec:code} for full list of curves for www}
\label{fig:432}\end{figure}

The ratio of Gas/radiation conduction ranges from 150 for small B to 20 for
large.  The cement has greater conduction than gas for $B>0.085$ radian, or
cement volume $>$ 5\xEm5, for the parameters shown in Figures \ref{fig:431} and
\ref{fig:432}.

\begin{figure}[!h] 
% \resizebox{!}{12cm}{\rotatebox{90}{\includegraphics*[0.86in,0.5in][7.75in,10.5in] {5A.eps}}}
\caption{Net thermal conductivity as function of cement fraction for several
cement thermal conductivities and 3 options for $G=\sigma_h / \sigma_C$; similar
to Fig 5A in Paper 2.  Values of $k_c$ shown in the legend. The lower set of 6
curves is for $G=1$, then middle set for $G=1/\sqrt{B}$ and the upper set for
$G=1/B^{0.9}$. }
\label{fig:5A} \end{figure}
% Make by defaults, 127,437, 16,: 7=G=.5, 127, 4372, 16: 7=G=.9, 127, 4372

\begin{figure}[!h] 
% \resizebox{!}{12cm}{\rotatebox{90}{\includegraphics*[0.86in,0.5in][7.75in,10.5in] {5B.eps}}}
\caption{Same data as Figure \ref{fig:5A}, with logarithmic abscissa; similar to
Fig 5B in Paper 2.}
\label{fig:5B}
\end{figure}
% Make as 5A; change only 437 to 438, leave 4372
%\pagebreak
 The differences between the results here and in \cite{Piqueux09b} are probably from two main differences in the models: \begin{enumerate}
\item Lateral expansion of heat flow away from the contact point. Crudely estimated here by the $G$ factor
\item Heat flow through the four cement disks around the axes perpendicular to the main heat flow. Omitted entirely here. This should be significant only for large cement volume fractions.
 \end{enumerate}

\begin{verbatim}
generate Fig 5A       to generate 5B, change only 437 to 438
43  127  8  437   16,7 -2   127 4372  16,7 -1  127  4372  88 9

@42 produces the numeric integration for one B value ggg =[I1,I2,I3] 
IDL>  print,ggg
   0.00238796      2.24765      1.80089
\end{verbatim}
\subsection{Using variable $k_g$: 2010jul14} %---------------------
Figures \ref{fig:431b} and \ref{fig:432b} are the revised versions of Figs.  \ref{fig:431} and \ref{fig:432} respectively.

\begin{figure}[!h] 
% \resizebox{!}{12cm}{\rotatebox{90}{\includegraphics*[0.86in,0.5in][7.75in,10.5in] {431b.eps}}}
\caption{Chart of radial behavior, each scaled to the ranges listed, for the
independent heat-flow elements; conditions are shown in the upper label. The top
three curves are the heat flow due to cement conduction, gas conduction, and
radiation, as a function of radius. Third and 4th curve show the Knudsen Number
and the resulting gas conductivity.  See \S \ref{sec:code} for full list of
curves for vvv.}
\label{fig:431b} \end{figure}

\begin{figure}[!h] 
% \resizebox{!}{12cm}{\rotatebox{90}{\includegraphics*[0.86in,0.5in][7.75in,10.5in] {432b.eps}}}
\caption{Chart of heat-flow magnitudes and net conductivity as a function of
central angle to the edge of cement, same conditions as Figure \ref{fig:431}.
See \S \ref{sec:code} for full list of curves for www}
\label{fig:432b} \end{figure}

\bibliography{mars}   %>>>> bibliography data
\bibliographystyle{plain}   % alpha  abbrev 
\end{document} %===============================================================
% ===================== stuff beyond here ignored =============================
