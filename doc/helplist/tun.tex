
\section{Special tabular output; routine tun8.f}    % 2017oct01
Beginning with version 3.2.1, the user can request tabular output of items not
in the type 52 files. This output is triggered by setting I15 to a value of
100+N where N is an option coded in the subroutine TUN8 (T unique). For each
output case, there is a header line of 8 integers:
\qi 1: the number of dimensions to the data, Ndim, currently always 4
\qi 2: total number of columns in the table, 3 larger than the data columns
\qi 3: number of times per day output
\qi 4: number of latitudes
\qi 5: number of seasons output
\qi 6: IDL-style type of data for conversion, currently always 4 (single-precision floating point)
\qi 7: The run number in the KRC run
\qi 8: the case number
\\ Thus, the first Ndim+2 values can be used to construct a storage array

This is followed by a line for every hour, latitude and season: the first 3
columns are those values as integers. The rest of the columns are [normally]
floating-point values for that instant on the final convergence day of each
season.

The three options coded as of 2017 Mar 2 are:
\qi N=1: Temperatures for each layer at each hour, layer number increases to the right
\qii the virtual layer is omitted
\qi N=2: Down-going solar fluxes, atmosphere and apparent sky temperatures
\qii floating-point columns are: 
\qiii 1: collimated insolation
\qiii 2: diffuse and 'Bounce' insolation 
\qiii 3: Atmosphere kinetic temperature (redundant with Type 52)
\qiii 4: Apparent zenith sky temperature: $^4 \sqrt{ \beta} \ T_a$, see [40] in the JGR KRC paper.
\qiii 5: Effective hemispheric sky temperature:  $^4 \sqrt{1-e^{-\tau_r}} \ T_a$, see [42] 
\qi N=3: Photmetric albedoes
\qii floating-point columns are: 
\qiii 1: time index 
\qiii 2: collimated insolation onto slope surface
\qiii 3: all downward insolation except collimated 
\qiii 4: solar heating of atm. $H_v$ 
\qiii 5: cos of incidence angle on horizontal 
\qiii 6: current hemispheric albedo

Output is always to a file 'fort.77' in the directory where KRC was invoked, so for preservation it must be renamed before the next KRC run. 

Output will be for the same set of seasons as written by TDISK. Output can be
for any subset of cases in a run (single TDISK output file) or multiple runs in
one input file.  Output is turned on by a change line, e.g., 
\qi 2 15 101 'I15' / values to fort77
\\ and it is turned for by, e.g., 
\qi 2 15 75 'I15' / RESET to no fort77
\\  However, the current IDL routine to read a TUN8 output file does
not accomodate cases with different sized tables.

Additional options can be coded into tun8.f .
