\documentclass{article}
\usepackage{definc}  % Hughs conventions
%\textheight=9.80in  \topmargin=-0.5in           %  hobo normal=final
%\textwidth=7.5in  \oddsidemargin=-0.3in \evensidemargin=-0.3in  % hobo final
%\parindent=0.em \parskip=1.ex %  no indent & paragraph spacing
\title{THermal model of dust layer on ice}
\author{Hugh H. Kieffer \ \ \ file=~/hkieffer/xtex/tes/krc/onelay.tex 2006jan06}
% individual variables
\def\qF{\Delta M}       % delta frost
\def\qH{\mathcal{H}} % scale height
%                with parameters

\begin{document}

\maketitle

History of this \LaTeX ~file:


Try to stay with KRC adn vjet paper terminology and symbols

\section{Surface boundary condition} %===============================
In \pname{TDAY}, the frost-free surface condition is (see Eq. \ref{eq:pit} for
more detail):
\qbn W=(1.-A)S_{(t)}'  + \ql 1.-\alpha \qr \epsilon R_{\Downarrow t}
+\frac{k}{X_2}(T_2-T) - \ql 1.-\alpha \qr \epsilon\sigma T^4 \qen

where $X_2$ is the depth to the center of the first soil layer, $S_{(t)}'$ is
the total solar radiation onto the surface, and $ R_{\Downarrow t}$ is the
downwelling thermal radiation (assumed isotropic).

\section{Slopes and Conical Holes } %======================================

Surface condition ( from krc)
 The surface condition for a planar sloped surface or a flat-bottomed pit can be
 written as follows, using the crude assumption that the surfaces visible to the
 point of computation are at the same temperature and have the same brightness
 where illuminated.

\qbn W=(1-A)S_M \left[D_1 \cos i_2 + \Omega D_2 + (1-\Omega) A (G_1 D_1 +
  \Omega D_2) \right]  +  \Omega \epsilon R_{\Downarrow t}
+\frac{k}{X_2}(T_2-T) -  \Omega \epsilon\sigma T^4 \label{eq:pit}\qen

Where $\Omega$ is visible fraction of the sky, $D_1$ is the collimated beam in
the Delta-Eddington model and $D_2$ is the diffuse beam. $G_1$ is a geometric
term for the solid angle of illuminated surface seen by the target surface of
the pit. Within the [ ] for $W$, 
\qi the first term is \ct{DIRECT} = the direct collimated beam,
\qi the second is \ct{DIFFUSE} = the diffuse skylight directly onto the target
surface,
\qi the third term is \ct{BOUNCE} = light that has scattered once off the
surrounding surface.

As a first approximation, for pits $G_1=$ min$( 1, (90-i)/z')$ where $z'$ is the
slope of the pit walls. For a sloped surface, $G_1$ is unity. For a
flat-bottomed pit, $i_2 = i$ when the sun is above the pit slope, and $ \cos i_2
=0$ when the sun is below the pit slope.

\subsection{} %-----------------------------------------------
Simplest conditions: 
\qi Equilibrium.

Net radiation and heat lfow budget, above

Conductive heat lfow thorugh the dust layer
\qbn H_b=k\frac{T_s-T_c}{\Delta z} \qen

Net sublimation  $ \dot{m} = H_b/L$

Heat exchange of the gas $\Delta W = \left( T_S-T_c \right) \dot{m} C_p $




\end{document} %===============================================================


%==============================================================================
\section{} %==================================================
\subsection{} %-----------------------------------------------
\subsubsection{} %............................................

\begin{verbatim}
\end{verbatim}

\begin{enumerate}
\item
\end{enumerate}
