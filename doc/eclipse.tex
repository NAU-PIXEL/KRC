\documentclass{article} 
\usepackage{underscore} % accepts  _ in text mode
\usepackage{ifpdf} % detects if processing is by pdflatex
\usepackage{/home/hkieffer/xtex/newcom}  % Hughs conventions
% \newcommand{\qj}{\\ \hspace*{-2.em}}      % outdent 1

\textheight=10.00in \topmargin=-1.1in % bot need 0.1 more        %OK
\textwidth=7.20in  \oddsidemargin=-0.4in \evensidemargin=-0.4in  %OK

% 1 of next 2 used in place of  \qen for development to identify equation labels
 
\newcommand{\ql}[1]{\label{eq:#1} \hspace{1cm} \mathrm{eq:#1} \end{equation}}
%\newcommand{\ql}[1]{\label{eq:#1} \end{equation} } % for final

\newcommand{\bq}{$ < \! > \!   \! >$ } %  begin quote
\newcommand{\eq}{ $< \! \! < \! > $ } %  end quote

% absorb is the verb, absorption is the process

\title{KRC Version 3.5 with eclipses and planetary fluxes}
\author{Hugh H. Kieffer  \ \ File=-/krc/Doc/v35/eclipse.tex  2017mar12:Apr07}
\begin{document} %==========================================================
\maketitle
\tableofcontents
%\listoffigures
%\listoftables
%\hrulefill .\hrulefill
% \pagebreak 

\begin{abstract}

KRC has been expanded to handle two kinds of eclipses. Rather than wait for
Version 4 with full longitude support, Version 3.5 has been generated with
one-longitude-at-a-time support for two types of eclipses
\qi 1) Daily: as for Jovian satellites.
\qi 2) Rare, in that the lead-up days did not have eclipse, as for Earth-lunar or Phobos shadow on Mars. \\
The insolation profile through an eclipse has been modeled in consideable
detail. However, PORB has not been changed, so the user will have to do some
work to calculate the eclipse ``bias'' from perfect alignment. 
For Jovian (and similar) satellites, reflected and thermal radiation from the
planet can be significant, especially during eclipse; a sinusoidal approximation
for these in the form $F=c_1 + c_2 \cos (\psi -c_3)$ has been included for each. 
Added the  capability to write binary files of surface temperatures at every computed time-step on the last day of the last season for any set of cases.  
Version 3.5 is backward compatible with earlier versions 3.x, so that
non-eclipse use is unchanged.

% This development has addressed many of the numerical issues associated with the rapid insolation changes of an eclipse. 
\end{abstract}


% \subsection{Remaining issues}

\section{Introduction}


Terminology: 
\\ \textbf{Occulting body: OB} The body casting the shadow. For daily eclipses, this is typically a planet. For rare eclipses, this is commonly a satellite. 
\\ \textbf{Eclipsed body: EB} The body in shadow.  For daily eclipses, this is typically a satellite.  For rare eclipses, this is commonly a planet.
\\ \textbf{Eclipse body Surface Point: ESP} The location on the EB for which calculations are done. 
\\ \textbf{Bias} The Sun:ESP line closest approach to the OB center, as a fraction of the OB radius.
\\ \textbf{Central hour} The KRC hour at ESP

Eclipse insolation profile includes the full geometry for round body occulting a round Sun. Simplifying assumptions for eclipse insolation profile.
\qi 1. Assume circular, uniform irradiance source (Sun)
\qi 2. Effect of planet atmosphere treats Sun as a point source. ? [atmosphere not implimented]
\qii  Convolve this with the geometric extinction.

NOTE: The atmosphere effects became too messy, and are currently omitted!

KRC 3.5 assumes synchronous rotating satellites, so longitudes are not all the
same as in earlier versions of KRC. For simplicity, specify surface longitudes
as Hours from the sub-solar point at inferior conjunction (from the Sun) and
increasing eastward (right-hand about the North pole).

The finite size of the EB is included in computing distances; in the Solar
System, this is important only for Phobos shadow on Mars.

The symbols \bq and \eq are used here to bound direct quotes from articles or
prior documents.
\section{Users Guide}

This guide is a supplement to prior KRC User Guides; it repeats virtually
nothing.

Version 3.5 is backward compatible with earlier versions 3.x, so that
non-eclipse use is unchanged.

Suggest starting with an input file from the distribution, e.g.,\nf{eurD.inp},
and modifying as you wish.

\begin{enumerate}    % numbered items  
\item Generate the geometry matrix for the planet:satellite of interest, and
  cut-and-paste it into the input file. A matrix for Europa is in
  \nf{PORBCM.mat} and in the suggested input file.

\item To invoke an eclipse, insert a change line 14 ; see \S \ref{eline}. This
  eclipse will be in following cases until a change line '14 0 /' is used.  If a
  Rare eclipse is specified, a binary file named \nf{tfinexx.bin5} will appear
  in the running directory, where \nf{xx} is the case number.

\item To invoke planetary fluxes, insert a change line 15; see \S \ref{pline}.
  This will apply to following cases until a change line '15 0 /' is used.
  \\ It may be helpful to look at the discussion in \S \ref{nomp} calculating
  the flux values.

\item To output a binary file containing the detailed surface temperature versus
  hour for one latitude, insert a change line 15; see \S \ref{tline}. This will
  apply to following cases until a change line '16 0 /' is used.

\end{enumerate}

 If both eclipse and planetary heating are invoked in a case; the longitudes
 (expressed in Hours) should be the same. KRC does not check for this
 consistency.

\subsection{New routines} 

There are two major new routines:
\begin{description}  % labeled items  
 \item [ECLIPSE] eclipse.f  \ Calculates the detailed insolation history of a circular body occulting a uniform round source.  

\item [TFINE] tfine8.f  \ Increases the depth (layer) and time resolution beyond that of TDAY to follow the details of a Rare eclipse.  

\item [EVMONO3D] evmonod.f \ Evaluation of 3rd-degree polynomial with scaling. This is a
  modification of EVMON03 that has the scaling coefficients firm-coded, thus two
  less arguments, and is 9\% faster.
\end{description}

 Also utility routines \textbf{STRUMI} and \textbf{STRUMR8}, and two routines uses only in testing that can modify parameters: \textbf{GETPI4} and \textbf{GETPR8}.

\section{Liens}
1).  Type 52 output for Rare eclipses in version 3.5.1 contains un-eclipsed
values until a discontinutity at the end of the eclipse, with the proper details
in a seperate binary file; these are merged in a post-run IDL routine. The TFINE
algorithm could be moved into an optional (LRARE only) loop entirely within TDAY
so that the Type 52 file had the eclipse results.

The layer TMIN and TMAX do not consider temperatures during a Rare eclipse; they
do consider the remainder of the eclipse day. However, only the near-surface
TMAX would likely be affected by rare eclipse.

Use of a special change line to toggle TOUT binary file is crude. Should be moved to an integer when KRCCOM ID is increased.

\section{Eclipse design} 

Length input paramters are in physical units of km, but within the ECLIPSE
routine, all distances are arbitrarily scaled to a characteristic length taken
as the semi-major axis (radius, in this simplified case) of the mutual orbit of
the occulting and eclipsed bodies.

\subsection{Notation}
``time-step'' means as used in TDAY unless specifically called a fine time-step (or f-time) as used in TFINE.


Define a few angles and variables:
\\ $M$: mutual orbital radius between the centers-of-mass of the OB and EB. 
\qi This is the normalization scale for all distances
\\ $R_O, r_O$: radius (km) and normalized radius of the OB 
\\ $R_E, r_E $: radius (km) and normalized radius of the EB
\\ $\psi$: orbital longitude; zero when EB is at inferior conjunction as seen from the Sun 
\\ $\phi$: orbital angle from the center of the eclipse $\phi=\psi -\pi$
\\ $H_c$: KRC hour at the center-of-eclipse for the satellite surface point of interest.
\\ $K_L$: fine layer factor for rare eclipse
\\ $K_T \equiv K_L^2$; fine-time factor for rare eclipse
\\ $\beta$: orbital angle from center to edge of eclipse shadow
\\ $ \alpha $: Angular radius of the Sun from OB:EB system, radians 
\\ $r_S$: Normalized radius of the Sun in the working plane.
\\ $ b$: Closest approach of the sun-line to the center of OB, as a fraction of OB radius.
\\ $N_2$=N2: KRC number of time-steps per sol; = N2
\\ $J_7 $=J7: Last 1-based time step before the start of eclipse phenomona
\\ $J_8$=J8: First 1-based time step after the end of eclipse phenomona
% \\ $ $: 
% \\ $ a_B$: Angular radius of the OB from the EB; identical to the EB orbital angle from center to edge of eclipse when bias is zero.


\subsection{Basic eclipse phenomenon equation}
 Basic assumption is that a round source (the Sun) is being blocked by a round
 Occulting Body (OB). The formula from the intersection of two circles is taken from
 http://mathworld.wolfram.com/Circle-CircleIntersection.html

\qb A= r^2 \underbrace{ \arccos \left( \frac{d^2+r^2-R^2}{2dr} \right)}_{ANG2} 
      + R^2\underbrace{ \arccos \left( \frac{d^2+R^2-r^2}{2dR} \right)}_{ANG1} \qe
\qbn -\frac{1}{2}\underbrace{ \sqrt{(-d+r-R)(-d-r+R)(-d+r+R)(d+r+R)} }_{SQP} \ql{e14}
where $r$ and $R$ are the radii of the two circles and $d$ is the separation of their centers.

Implement using $B$ for the radius of the Bigger circle (which might be either
$r_O$ or $r_S$) and $R$ for the other, with many intermediate variables and
tests for speed and avoiding faults.

If $d \geq (B+R)$ then $A=0$ ; if  $d \leq (B-R)$ then $A=\pi R^2$ . 

The fraction of sun-light reaching the surface of EB is $F=1-A/(\pi r_S^2)$ . 

If B is the Sun, then have an annular eclipse and  $ F_{min}= 1.-(R/B)^2$.  If R is the sun, then have total eclipse with $F=0$ for some time.

\subsubsection{Relations in eclipse} 

Define the ``eclipse surface point'' (ESP) as the point of interest on the
surface of the EB at local hour $H_C$, and assummed to be on the EB equator (relax
this assumption later?).

For solar eclipses by Phobos on Mars, need to consider the radius of Mars as it
is a significant fraction of the radius of Phobos' orbit. So, include these
geometric relations in the code, they will be trivial for most objects.
However, do not include the small variation through an eclipse of the relative
angular size of Phobos and the Sun as seen from the surface of Mars, just use
the sizes at the center of an eclipse.
  
The angular radius of the Sun is $\alpha=R_S/(H_UU)$ where $R_s$ is the radius
of the Sun (km), $H_U$ is the heliocentric range in Astronomical Units, and U is
the Astronomical Unit in km.

Define a geometry ``working'' plane through the center of OB normal to the
direction to the Sun. +Z is away from the Sun.The X direction is parallel to the
EC orbital plane

The effective orbital radius of the ESP is $Q= M-R_E \cos(z)$, where $z$ is the
zenith angle of the Sun from ESP at the mid-time of eclipse. Assuming the
eclipse is on the equator, $z=\frac{ \pi}{2}(H_c-12)/6 = \pi(H_c/12 -1)$. In the
normalized system, $Q=1-r_E \cos(z)$

At any time, the position of the ESP is $X=Q \sin \phi $ and $\phi=2 \pi t/P $
where $t$ is time from mid-eclipse and $P$ is the EB orbital period.  Also,
$\phi=2 \pi (J-J_C)/N2 $ where J is the c-time count and $J_C$ is the c-time of
mid-eclipse.

The apparent size of the Sun in this plane, as seen from the surface of the EB,
is $r_S= \alpha Q/M $
There will some eclipse effect if the bias $b<(1+r_S/r_O)$

At first contact, $X_c=\sqrt{(R+B)^2-(br_O)^2} $ , this sets the half-duration
 of the eclipse $t_h$ in the same units as $P$.

At any time, the center separation is 
\qb d=\sqrt{X^2+\underbrace{(br_O)^2}_{Y2} } \qe

In the KRC diurnal system, if surface point of interest is at hour $H_c$ when
the middle of eclipse occurs, and there are N2 time steps in a sol, with the
last at midnight, then fractional (1-based) indices at the beginning and end of
eclipse are $ \frac{H_c}{24} N_2 \pm \frac{t_h}{\mathrm{sol}} N_2$.

V 3.5 does not handle a sol different from P; this case may never be of interest.

Eclipse is symmetric about orbital angle of $\pi$ but the eclipse function must
be centered about ESP at the satellite surface hour requested.

To allow any resolution in the satellite surface position, KRC uses fractional
orbit angles that are on the time grid, but shifts application in TLATS or TDAY
by integral time steps.

For an eclipse with zero bias, the transition to totality, as a fraction of a
sol, is $\alpha /\pi$. In these units, the full eclipse lasts $(r_o
+\alpha)/\pi$. For Europa, the values are roughly 1/3500 and 1/30. Thus, to begin
to resolve the penumbral phase, would require N2$>7000$.


\subsection{Implementation}
In general, for any eclipse, set N24 as large as allowed (MAXNH).

 To deal with rapid insolation changes, shorten the time-steps by some
 factor. Do not need to change the layering for stability, but should change it
 for responsiveness. To keep same stability factor, divide each layer by
 integral factor f and increase the number of times/day by $f^2$.

To resolve the penumbra stage, there should be several time steps within it; if
this is impractical, there should be many (more than a dozen?) time steps within
the entire eclipse.

 Ideally $f^2$ would be roughly length-of-a-sol / 2t, where t is the time for
 the satellite orbital phase to change by the angular diameter of the Sun
 $\theta_S$ . E.g., for Europa, t is ???

Daily: Handle entirely in Tlats, with consideration for the large N2 required to
see shape of insolation through an eclipse.

.
\\ Planet thermal load into new array: PLANH, compute in TLATS 
\\ Planet reflected solar load into new array: PLANV, compute in TLATS.
\qi For daily eclipses, these are incorporated in TDAY, see \qr{wbe}
\qi For rare eclipses, TFINE combines the two and does linear interpolation to fine time.


Satellites are assumed synchronus. Yet, both PERIOD and PARC(4) must be
specified and should be the same.

\subsubsection{Vector geometry}
 The coordinate system used by TLATS is the ``Day'' system
\qi +Z toward body right-hand spin axis (north pole),
\qi +X in the true solar midnight meridian,
\qi +Y is Z cross X, and is in the equatorial plane
\\ Sun at declination $\delta$ at midnight: $M=[ \cos \delta, 0., \sin \delta]$
\qi Sun diurnal progress is left-hand   $\phi=- \frac{2 \pi}{24} t$ where $t$ is in hours, rotating around +Z
\\ Local surface normal at latitude $\alpha$, in the noon meridian: $F =[ -\cos \alpha, 0, \sin \alpha]$
\\ To get the normal to a surface with slope (dip) $\beta$ facing toward azimuth $\psi$ measured east from North:
 \qi rotate $F$ around +Y by $\beta$ , generates temporary vector $Q$
\qi  then rotate $Q$ around the original $F$ by $-\phi$ to get tilted surface normal  $T$

For an ESP at hour $H_C$, $\omega=\pi H_c/12$ a planetary heat source above the
equator at noon would have a unit vector $P=[\cos \omega, \sin \omega, 0]$

\subsubsection{Time indices for eclipses}

Time indices passed between routines are always in TDAY units, in some places
called coarse time or ctime. Where they are converted to/from fine-time (or
ftime), they are treated as refering to the start of a time interval, before the
diffusion calculation.

ECLIPSE and TFINE are the only FORTRAN routines that deal with fine-time.

ECLIPSE calculates the time of first and last contact in floating-point time;
first contact is rounded down to JBE(1) and last contact is rounded up to
JBE(2); JBE is passed between routines; the rounding ensures that the indices in
JBE capture the full optical eclipse. ECLIPSE returns an array for the
insolation factor FINSOL; the fraction of insolation that makes it past the
occulting body to the Eclipse Surface point (ESP).  FINSOL is 1 outside the
eclipse.

\subsection{Eclipse Specification, \label{eline}}

Eclipse specification: 
\qi 1:  Style: 0=none  1=Daily  1.3+=rare, round of value is layer factor
\qii time factor is square of layer factor to retain stability
\qi 2:  Distance to sun,  AU (used to get  Sun angular diameter)
\qi 3:  Occulting body (OB) radius, km
\qi 4:  Mutual center-of-mass orbit radius, km $=M$
\qi 5:  Eclipsed body (EB) surface radius, km
\qi 6:  [ Mutual orbit period, days ] Assumed same as diurnal PERIOD
\qi 7:  Eclipse Bias
\qi 8:  [ J2000 date of Rare eclipse ] Assumed to be on the last ``season''
\qii KRC 3.5 uses the sign as a flag for base treatment. + is maintain heat-flow
 - is maintain temperature. 
\qi 9:  Eclipse central hour
\qi 10:  Debug code.  ne.0 prints constants and >1 prints one point,
\qii Negative runs a `` null eclipse'' test mode in which the OB is considered transparent.
\qi x:  Extinction scale height of  OB's atmosphere, km. \  NOT implimented 

These will be input as a change line 14: first real value being non-positive means turn off.  Typical input line: 
\qi  14  1 5.2026 71492. 0.6711D6 3121.6 3.551 0.01 6000. 12.  2 / Europa

\subsection{Eclipses: Daily}

For ``Daily'' (typically long) eclipses, fine-time is never used; the user can
set N2 as large as they want to get the eclipse details. FINSOL covers the
entire day in ctime steps; it is unity outside ot the JBE range.

Binary output files are the same as earlier versions of KRC.

\subsubsection{Details} 

TLATS: Handles only daily eclipse: Sets LECL flag if PARC(1) $ 0.8 < x< 1.2 $. If set, then
\qi Calls ECLIPSE once per season, which generates FINSOL insolation factor for each time-step
\qii and duplicates as SOLAU the variable for solar flux at current AU
\qi Each time step, multiplies the solar insolation by FINSOL(JJ)

TDAY: No change for daily eclipses. 

\subsection{Eclipses: Rare}
For ``Rare'' (typically short), the uses specifies a fine-layer factor $K_L$
(rounding the eclipse ``style'' parameter); the fine-time factor $K_T$ is the
square of the layer-factor. FINSOL covers in fine-time steps from the beginning
of JBE(1) to the end of JBE(2).

In TLATS and TDAY, the Hour of a time step J(1-based) is in the middle:  H= (J-.5)24./N2. 
\\ The hour of FINSOL element I is in middle: (J7-1)*24./N2 + (I-.5) 24./($K_T$*N2)
\qi where J7 is the index of the coarse time-step  when fine-time starts 

Because of the rapid changes that can follow the return of sunlight at the end
of eclipse, the detailed calculations of TFINE are continued for the number of
ctime steps of the optical eclipse (at least one) KRC output interval after last
contact.

TFINE always outputs additional binary file \nf{tfinexx.bin5} with ASOL, FINSOL
and all layer temperatures at every fine time-step within eclipse for each Rare
eclipse case. Header contains N2,J7,J8,J9.  Array is [2 + fine layer, fine-time] 

\subsubsection{Details} 

To ensure catching all the eclipse effects, expand the fine-time range by one
earlier TDAY timestep and later by the duration of the eclipse dJ=
JBE(2)-JBE(1). Thus the ``trigger'' values of the time-steps JJ in TDAY are:
J7=JBE(1)-1 and J9=JBE(2)+[dJ$>1$] , and TDAY switches to and from TFINE before
the diffusion loop. TFINE will run over ctime JJ: J7 through J9-1, with linear
interpolation of upper boundary conditions to fine time-steps.

In TFINE, as in TDAY, the surface temperature is stored in layer index 1, but
that layer is reconstructed as the virtual layer in each time-step.

Number of fine layers: virtual layer + (number of physical TDAY layers * layerFactor)
\qi  N1F=1+(N1-1)*KFL   and must store one more for the base

. LRARE is normally False. and TDAY(1 normally sets J7=-1
\\ - If PARC(1) is $\geq$1.3 then TDAY(1 sets flag LRARE True and will do a RARE
eclipse. It calls ECLIPSE to get JBE which contains the normal time steps before
and after the eclipse. It calls TFINE(1 to do all that can be done without
having layer temperatures.
\\- The eclipse is entirely within ctime JBE(1) to JBE(2), which could be the same for a very short eclipse.
\\ - In TDAY(2 day loop, if LRARE, then on the last season, at the start of the
last day, TDAY(2 sets J7 to JBE(1)-1 .
\\ - In TDAY(2 time loop, when JJ equals J7, if J7 $\leq$  JBE(1), TDAY transfers
the layer temperatures to TFINE.
\\- After TFINE returns, TDAY sets J7 equal to the end of eclipse followon J9+1
and proceeds normally until the (JJ equal J7) test is again satisfied, when
(before diffusion) it sets the temperature profile to the final from TFINE, and
finishes the last day normally, leaving a discontinuity in temperature at the
end of the eclipse.

TFINE is a modified copy of TDAY, with a single ``day''. It increases the layer
and time resolution, interpolating in time and depth as
needed, then steps through the eclipse. It calls ECLIPSE to get the detailed
insolation profile.
 
Because solar eclipse must occur before dusk, except for the pathologic case of
an eclipse near the summer pole, there would be many output ``hours'' after
the start of eclipse into which the detailed time results could be loaded.
\\ Might not be enough, and is hokey. So, output a separate file

TFINE has a large storage buffer, and stuffs the eclipse factor and temperature
profile (up to 99 fine layers) into this for writing to a .bin5 file for each
rare eclipse case.  This file covers the hours H(I,J)= (I/KFT +J-.5)* 24./N2  from I,J=1,J7  to I,J= KFT,J9 . 

The output file is always named ``tfinexx.bin5'' where xx is the
case number. Thus, user should \textbf{rename tfinexx.bin5 before any other  eclipse run}. 

\subsubsection{Extra printout}

Rare eclipse cases put additional material in the print file, apart from debug
options. Below, left-adjusted lines are example printout and inset lines are
explanations.

. 
\\ TFINE IQ,J4=           1           2
\qi IQ is the TDAY action requested: 1=setup, 2=do the time and layer loops
\qi J4 is latitude index; not reliable or relevant for IQ=1  
\\ TFINE layers: Num,lowest center[m]   82    0.8995
\qi TFINE(1: the number of fine layers, including virtual. Depth to the center of the lowest (not base) layer
\\  Min safety: layer,factor= 1       0.000       0.000
\qi  TFINE(1: Minimum convergence safety factor: layer, factor, 
\\  TFINE low lay of time doubli:  12 20 27 34 42 49 57 64 72 82
\qi Deepest layer for each fine-time doubling
\\    -777      3     10      2      4    760    776    792     82
\qi  TFINE(1: tag, NCASE, J5, J4(+1), J3, JBE(1:2), J9, N1F
\\  TFINE exit \ \ \ notification of exiting TFINE
\\  TFINE IQ,J4=           2           1 
\qi
\\  TTJ(1)...   260.19471582799264        258.02291589626202       -0.0000000000000000 
\qi  Virtual layer T for TDAY and fine layer system. Delta T at fine base   
\\ LZONE,LALCON,J5, IK1:4=  F  T    10     0     0     0     0
\qi In TDAY(2: last 4 are the T-dep. layer specifications.
\\ TFINE: Case= 3  JJJ=    3   83  306    1    0    0    0    5   80    0
\qii JJJ is the set of 10 integers given to BIN5F
\\  TFINE wrote tfine03.bin5  iret=            0
\\  TFINE exit
\\ End eclipse: JJ,KG,delT,delE  793  28    0.63383E-01     8067.1 
\qi KG is the lowest  TDAY layer represented in the fine layer system   
\qi delT is the T discontinuity of the lowest TDAY layer at end of eclipse
\qi delE is the delta thermal energy in the lowest TDAY layer at that time.

\section{TFINE}
 TFINE interacts primarily through the many KRC common's. Subroutine arguments are used to transfer in:
\qi the stage index: with value 1 or 2
\qi the physical properties of each layer.
\\ and transfer out:
\qi the coarse layer temperatures after eclipse
\qi the number of reliable output temperatures, or a negative values indicating an error.

The initial call to ECLIPSE returns the last original time step before the
eclipse starts and the first after it ends. When TDAY on the last day of
convergence reaches the starting time setp, it calls TFINE which proceeds
through all of its time steps and returns the temperature depth profile that it
gets, TTF, which is remembered by TDAY. TDAY proceeds with its normal
calculations until it reaches the time step after those covered by TFINE, when
the temperature profile is reset to TTF, and TDAY runs through the rest of the
last day.

TFINE ignores any atmosphere, except for any effect TLATS may have included on
collimated insolation. It does handle far-field radiation.

There can be a small non-physical effect if the number of finer layers
1+(N2-1)*PARC(1) would exceed the firm-code size MAXFL. The bottom of the fine
system is considered insolating during the eclipse, whereas the normal interface
at that depth would be conducting. The non-physical change in system energy is
roughly $B_j \rho_j C_j \Delta T_j$ where the last term is the amount that the
temperature of the deepest original layer treated by the fine system $j$ is
changed at the end of the eclipse.

Note that continuation to another season using the asymptotic predictor would be
inappropriate after a ``rare'' eclipse.

The temperature gradient that existed in the TDAY system at the depth of the
bottom of the TFINE layers at the start of the eclipse is held constant at the
bottom of TFINE through the eclipse.  

\qbn \nabla T \ = \ \frac {T_{j+1}-T_j}{ ( B_{j+1}-B_j)/2}= \frac{ T_{i+1}-T_i}{ ( B_{i+1}-B_i )/2} \qen 
where the i subscripts are for the TFINE values at it lowest two layers and the 
j subscripts are for the TDAY layer values at the corresponding depth. Thus  
\qbn T_{i+1}=T_i+ \underbrace{(T_{j+1}-T_j)\frac{B_{i+1}-B_i} {B_{j+1}-B_j}}_{delbot}  \ql{tbot}

When TFINE starts an eclipse period , it uses cubic spline interpolation with
natural boundary conditions (zero 2nd derivative at top and bottom). To avoid
interpolation failure, all fine layer centers must be within range of the TDAY
layer centers; thus maximum I is $K(N_1-1)+1+K/2$ .

% \clearpage
\subsection{Details}

Design with two sections, similar to TDAY. TFINE(1 does everthing that does not
require the starting temperature profile of the time-dependent radiation
field. TFINE(2 does the timesteps.

Has access to commons
\\ Creates finer layers and has an inner timeloop for the finer time steps.

Needs the original center depths of each layer, and must generate the center
depths of the new layers for interpolation and the bottom depths for the
diffusion equations.

Anything that is defined in TDAY (vrs defined in commons) is not available in
TFINE unless an argument.

Zone table logic is complex, could duplicate them in TFINE but better to pass in
as arguments the ultimate products: KTT, DENN, CTT as arguments; TTJ, XCEN and
BLAY are in common

Each TDAY layer of thickness $B_j$ is divided into K layers with thickness $B_i
\equiv B_{j_k}=f_iB_j$

K fine layers must have a geometric ratio $r$ that yields the same full-layer
ratio as $R\equiv$ RLAY.

\qbn r^K= R \mc{or} r=\exp \frac{\ln R}{K} \ql{rk}
and the sum of $f_k$ must be 1.

The sum of a geometric series of ratio $r$, first term 1 and $n$ terms is 
\qbn S \equiv \sum_{j=0}^{n-1} r^j = \frac{1-r^n}{1-r}  \ql{sumr}

\qbn f_1 =  1/S=  \frac {r-1}{r^K-1}  \ \Rightarrow \ \frac{r-1}{R-1}   \ql{f1}  
and $f_i=rf_{i-1}$

Each time-dependent input is linearly interpolated to the fine-time steps:
\qi ALBJ \ Surface albedo, which may vary with incidence angle
\qi ASOL \ Collimated flux onto (sloped) surface
\qi FARAD \ Far-field radiance
\qi SOLDIF \ Diffuse solar flux
\qi $\epsilon$ PLANH + $(1-A_s)$PLANV \ absorbed Planetary flux 

TFINE always writes to print file 
\qi  Number of fine layers and depth[m] to center of deepest layer
\qi If any T-dependent layers, the first and number of the A and B layers
\qi Low layers for fine-time doubling
\qi -777, Indices for: case,season,latitude,converg.day, eclipse hour range ...

Layers temperatures $j$ returned by TFINE, after TSUR as the layer (1) the rest of the layers are from the fine layers $i$. This could be based on:
\qi if K odd, $i= jK - 3(K-1)/2$
\qi if K even, average of layers $i= jK - 3K/2$ and $i+1$
\\ However, simpler (and better for even K) to use spline interpolation.

\section{Planetary fluxes}

For synchronous satellites, the temperatures depend strongly on longitude but
KRC version 3.5 does not treat longitude explicity.  For the side facing the
planet (the ``near-side''), the peak reflected radiation comes at midnight and
eclipses come near noon. For the side away from the planet, there is no
reflected or thermal planetary heat load and no eclipses.

Since there are two bodies in addition to the Sun, must treat the effect of
solar reflection and thermal emission from the ``planet''.  Version 3 will
assume these are sinusoidal with orbital phase. 
E.g., in the form $F=c_1 = c_2 \cos (\psi -c_3)$ 
and in units of W m$^{-2}$. Phase = $\psi$ is zero when the satellite is at
inferior conjunction as seen from the Sun (is this the convention?). 

\subsection{Planetary Flux specification \label{pline}}

Must specify 7 values:
\qi 1: Average Solar flux from OB (planet) at the EB (satellite)
\qi 2: \ \ half-amplitude of variation with phase
\qi 3: \  \ Phase lag, in degrees
\qi 4: Average thermal emission from OB at the EB
\qi 5:  \ \ half-amplitude of variation with phase
\qi 6:  \ \ Phase lag, in degrees
\qi 7: The longiitude (in Hours)  of the surface point.
 
These will be input as a change line 15: the first real value being non-positive
means turn off.  

Example: \vspace{-3.mm} 
\begin{verbatim}
15  0.62 0. 0.     0.21 0.21  0.    12. / Jupiter heat load on Europa at Sub-J
\end{verbatim} 

\subsection{Planetary load away from zenith}
Assume the absorption surface is Lambertian. For a point source, the effect varies
with zenith angle as $\cos \phi$. For a modest source of radius R radians, the
effect is
\qbn W(\phi_1) =\frac{\int_y^\pi \cos \phi \cdot  2 R \sin \theta \ R \sin \theta \ d \theta }{ \pi R^2} 
 = \frac{2}{\pi} \int_y^\pi \cos (\phi_1 + R \cos \theta ) \cdot  \sin^2 \theta \ \ d \theta \qen

where  $y $ is the horizon limit of $\arccos \left( (\frac{\pi}{2}-\phi_1)/R \right)$ if $\phi > (\frac{\pi}{2}-R)$ and 0 otherwise.

As $R \rightarrow 0$, 
\qb W  \rightarrow  \frac{2}{\pi} \cos \phi_1 \int_0^\pi \sin^2 \theta \ d \theta
 = \frac{2}{\pi} \cos \phi_1   \left. \coprod_0^\pi \frac{x}{2}-\frac{\sin 2x}{4}  \right] 
= \frac{2}{\pi} \cos \phi_1 \cdot \frac{\pi}{2}  = \cos \phi_1  \qe
as expected. Here $ \coprod _l^u \cdots \ ] $ stands for evaluation at the upper
  limit minus evaluation at the lower limit

\vspace{2mm}

Expanding $ \cos (\phi_1 + R \cos \theta ) \Rightarrow  \cos \phi_1 \cos( R \cos \theta ) -   \sin \phi_1 \sin( R \cos \theta ) $ The form of the integral becomes 

\qb c \int \cos ( a\cos x) \sin^2x \ dx \ + \ s \int  \sin ( a \cos x) \sin^2x \ dx \qe
 for which I could not find analytic solution.

Numerical solution coded in IDL planheat.pro; see Fig. \ref{planheatb}. For
Europa, finite size makes at most 2\% difference, and greater than 1\% only when
within 1.5\qd~ of the horizon.  Until the lower edge of the planet nears the
horizon of the satellite, the normalized factor is virtually constant; the
effect is about 0.13\% for Europa and barely 1\% for a 0.3 radian source

Because the effect of a finite angular size is small, I elected to omit it for
version 3.5; the influence follows the cosine of the incidence angle for the OB
center onto the [tilted] surface, $\mu_P$.
 
\begin{figure}[!ht] \igq{planheatb}
\caption[Effect of extended size of planet]{Normalized heat-load for
  finite-sized round sources as a function of zenith angle. Lambertian surface
  assumed. Legend shows the radius of the source in radians; for Europa the
  value is 0.1067. Infinitesimal source follows a cosine relation. Larger source
  are slightly less than cosine except near the horizon.
\label{planheatb}  planheatb.png }
\end{figure} 
% how made: planheat  b
\subsection{Static geometry for synchronous rotation}
 Only synchronous satellites are considered.

In TLATS, a planet source is assumed to be in the equatorial plane and above
``noon'' so that it has a fixed relation to the target (tilted) surface with
cosine of angle onto tilted surface $\mu_P$=COSP.  / IR and visual
fluxes are initially set to zero for each latitude. If the logical flag LPH is
True, then Planetary fluxes computed for orbital phase at each time-step and
multiplied by $\mu_P \geq 0$.

In TDAY, if LPH is True, then both the plantary fluxes are multiplied by their
absorption coefficents and added to the surface enery budget.

\subsection{How it works}
TLATS: Sets the flag LPH True if PARW(1) positive. At each time step, if LPH
True, \qi sets PLANH(JJ)= $ w_1 +w_2 \cos( \theta - w_3/\mathrm{RADC})$ where
RADC is degrees/radian.  
\qi add to the diffuse light SOLDIF(JJ) $ w_4 +w_5\cos( \theta - w_5/\mathrm{RADC})$

TDAY:  Sets the flag LPH in the same way as TLATS.  At each time step, if LPH True
\qi adds to the absorbed hemispheric downwelling IR flux ABRAD the amount FAC6*PLANH(JJ) where FAC6 is fraction of the sky visible times surface emissivity.

\section{Detailed temperature output \label{tline}}
KRC has traditionally output temperatures and other values every Hour (1/24th of
the bodies day) or sub-multiple thereof, down to 1/4 Hour, the firm-code limit
for N24 being 96. To track surface temperatures through an eclipse requires
higher resolution. While a Rare eclipse will output a file of temperatures at
high resolution (every fine-time step) for times around the using the special
routine TFINE, the Daily eclipse calculations are done with a small modification
of TDAY with no special output.

To address this in part a generic way, the existing array \nv{TOUT} that is
already in a COMMON can now be written to a binary file on the last day of the
last season for one latitude.  This is invoked by a change line 16, containing 
 the 1-based index of the latitude desired and the leading part of the file
name. Added to that part will be \nf{cxx.bin5} where \nf{xx} is the case number,
generated automatically by KRC.  The recommended leading name is
\qi <input file stem> 
\qi + 'lat'
\qi + <latitude in degrees with a following N or S as appropriate>

Example: \vspace{-3.mm} 
\begin{verbatim}
16 1 'eurDlat0'  / output file for Tsurf every time-step
\end{verbatim} 

This will generate an output file in the running directory for each case untill
stopped by a change-line: ``16 0 / ``


\section{Summary of code changes}

. \\
KRC:
\qi  In the case Loop, after TCARD and TPRINT, if any eclipse or planetary heat 
  is active, will print:  'Eclipse or PlanHeat on',PARC(1),PARW(1)
\qi   Calls TDAY(1
\\ \\
TSEAS:  none
\\ \\
TLATS:
\qi   [un]set LPH and LECL=Daily flags
\qi   Before the latitude loop, if Daily and first season, call ECLIPSE to get FINSOL
\qi   In the time loop:
\qii      If LECL,  multiply Sun by insolation factor; can affect TEQUIL
\qii      If LPH, calc PLANH(JJ) and PLANV(JJ).
\qi   After time Loop
\qii      If LPH, incorporate the average absorbed planetary heating into TEQUIL
\qi after TDAY(2 call: If at NLAD latitude and last season. Write TOUT to binary file
\\ \\
TDAY(1:
\qi    [un]Set the LPH flag, [un]Set the LRARE flag
\qi    If LRARE, 
\qii    call ECLIPSE to get the time-step range of eclipse, JBE
\qii    Set full eclipse range to start 1 time step earlier and end after 2nd duration
\qii    call TFINE(1 to do what can be done without temperatures
\\ \\
TDAY(2:
 \qi  If the last day and LRARE and the last season, set J7 to start of eclipse
 \qi  In the time loop, when reach J7, then 
 \qii  if at start of eclipse call TFINE(2, before layer loop, then set J7 to end of eclipse followon J9
    else, transfer TFINE results into layer temperatures, set J7 negative
\qi After the layer loops: if LPH, add in the planetary heating at each time step
\qi After last day, exit even if daily convergence tests fail (as they should) 
\\ \\
TFINE(1: [called only for LRARE and only at start of eclipse on last day of last season] 
\qi  call ECLIPSE to get both time range JBE and FINSOL
\qi  set J7=JBE(1)-1 and J8=JBE(2)
\qi  set J9= J8+ (J8-J1 $>$1) 
\\ \\
TFINE(2:  Diffusion calculations cover ctime J7:J9 
\qii Thus has J9-J7 +1 ctime-steps and K$^2$ times this ftime-steps.
\qi  Interpolates current T/depth profile.
\qi Uses FINSOL and steps forward in time until end of eclipse J8
\qi  Throughout the follow-on (J8 to J9) treat FINSOL as unity. 
\\ \\
Layer relations: 1-based
\qi  fine, first in set = I = (J-1)*KFL +1 where J is TDAY layer. KFL is layer factor
\\ 
Time relations: 1-based.  KFT=KFL$^2$
\qi  Fine, first in set = I = (JJ-t0)KFT where JJ is TDAY time-step. \ H(JJ)=JJ*24./N2
\qii    and 1=(JBE(1)-1-t0)KFT, thus:   I=(JJ-(JBE(1)-1))KFT +1
\qi  Since H=24*JJ/N2, \ H(I)= ((I-1)/KFT)+JBE(1)-1)* 24./N2

\subsection{Other}

TCARD: reads and prints a 14 or 15 line, loads the values into PARC or PARW in
HATCOM

Because the first real value is used as a test for activation, either effect can
be turned off by a single negative value. e.g.  
\qi 14 0. / turn off eclipses

Hour-dependant values computed in TLATS. Constant factors applied in TDAY.
\\ In some cases, rather than logic tests for eclipse or Planetary loads, it is easier to always add them, but ensure they are zero when not invoked.

FINSOL in common used differently for daily and rare eclipses, which cannot be
invoked at once.  ECLIPSE calculates values only through the eclipse, so FINSOL
must be replaced with 1.0 during the follow-on.

In TDAY, the insolation is evaluated at the instant of the middle of each time
interval and the upper boundary condition evaluated after the diffusion $\Delta
T$ is applied. Thus the assessment in ECLIPSE should also be at the middle of a
TDAY interval. Strictly, the interpolation in TFINE should use the same instant,
which can be done with no extra logic because eclipses cannot occur near the end
s of the day (except at the poles)

As with TDAY, the upper boundary condition is applied after the layer loop for
each timestep. In TDAY(2, TFINE(2 is called ???

A change 15 lien, 


\section{Formulation}

Starting with Equation wb=27 and some associated text of V34UG: \bq \ 

\qb \underbrace{W}_{POWER}=  \underbrace{(1.- \overbrace{A_{h(i_2)}}^{ALBJ} )}_{FAC3}
 \overbrace{ S_M  F_\parallel \cos i_2}^{ASOL}
+  \underbrace{(1-\overbrace{A_s}^{SALB} )}_{FAC3S} \underbrace{ S_M 
  \left( \overbrace{ \Omega F_\ominus^\downarrow  }^{DIFFUSE}
+ \overbrace{ \alpha A_s (G_1 \cos i F_\parallel
+ \Omega F_\ominus^\downarrow ) }^{BOUNCE}  \right)  }_{SOLDIF}
\qe

\qbn
+ \underbrace{\Omega \epsilon}_{FAC6} \underbrace{ R_{\Downarrow}^0}_{ATMRAD}
+ \underbrace{k \frac{\partial T}{ \partial z}_{(z=0)}}_{SHEATF}  
- \underbrace{\epsilon \sigma}_{FAC5} T^4  
+ \overbrace{\underbrace{(1-\Omega) \epsilon \sigma \epsilon_x }_{FAC5X} T_x^4}^{FARAD}   \ql{wb} 
where the overbrace items are computed in TLATS and
transfered in COMMON. All terms up to and including ATMRAD make up the total
absorbed radiation ABRAD.  When frost is present, its albedo replaces $A_h$ and
$A_s$ on a time-step basis except the $A_s$ in SOLDIF (from TLATS) is on a
season basis; however, the $A_S$ term includes the far-ground fraction $\alpha$
which is small except for steep slopes.

Assumes that normal albedo is the same for the sloped and the flat surfaces.

The fraction of solar flux reflected ALBJ$\equiv A_h =$ALB*AHF is composed of
two factors, ALB$\equiv A_0$ and AHF$=A_h(i)/A_h(0)$, a hemispherical
reflectance function.  Likewise, the spherical albedo is $A_s=$ ALB*PUS where
the second factor is $P_s$.

The floor of a ``pit'' does not see the flat terrain, but rather the same slope
at all azimuths, and therefor different temperatures. The most practical
assumption is that the average radiation temperature of the pit walls is the
same as flat terrain. This will be an under-approximation. In a later version of
KRC with more input parameters, a radiation scale factor could be included; if
practical, code to include a constant factor, initally unity for v 3.4.

Because \nv{FARAD} is not dependent upon the calculation of $T$, it can
pre-computed for a given day. $T_x$ is interpolated to the proper season in
\np{TSEAS}; \np{TLATS} selects the proper latitude, multiplies by \nv{FAC5X} for
each of its stored hours, and interpolates to each time-step to form
\nv{FARAD}$_t$ transfered to \nv{TDAY}. However, to then accomodate variable
frost emission, need to multiply by $\epsilon_f/\epsilon$ for the frost case
(relatively rare).
\eq

Version 3.5, add eclipse attenuation of solar insolation FINSOL$=F_X$ and add
visual and IR planetary fluxes, PLANV$=P_V$ and PLANH$=P_H$.  For daily
eclipses, TLATS includes $F_X$ into $S_M$, so that TDAY needs be no different. TLATS does nothing for rare eclipses and  $F_X$ in handled
entirely within TFINE.


\qb \underbrace{W}_{POWER}=  \underbrace{(1.- \overbrace{A_{h(i_2)}}^{ALBJ} )}_{FAC3}
 \overbrace{ S_M  F_\parallel \cos i_2}^{ASOL}
 \ + \ \underbrace{(1-\overbrace{A_s}^{SALB} )}_{FAC3S} \left[  \underbrace{ S_M 
  \left( \overbrace{ \Omega F_\ominus^\downarrow  }^{DIFFUSE}
+ \overbrace{ \alpha A_s (G_1 \cos i F_\parallel
+ \Omega F_\ominus^\downarrow ) }^{BOUNCE}  \right)  }_{SOLDIF}  + \ \mu_P P_V e^{-\tau_v/ \mu_P}
 \right]  \qe

\qbn
+ \epsilon \mu_P P_H e^{-\tau_R/ \mu_P} \ +  \underbrace{\Omega \epsilon}_{FAC6} \underbrace{ R_{\Downarrow}^0}_{ATMRAD}
 + \ \underbrace{k \frac{\partial T}{ \partial z}_{(z=0)}}_{SHEATF}  
- \underbrace{\epsilon \sigma}_{FAC5} T^4  
+ \ \overbrace{\underbrace{(1-\Omega) \epsilon \sigma \epsilon_x }_{FAC5X} T_x^4}^{FARAD}   \ql{wbe} 

However, in version 3.5, the atmosphere effects on planetary fluxes are ignored,
and the $\mu_P$ term is handled in TLATS.

ABRAD accumulates terms until SHEATF.

%\clearpage
\subsubsection{Synopsis of TLATS radiation calculations}
%\vspace{-3.mm} 
\begin{verbatim}
TLATS
      LATM=PTOTAL.GT.1.0        ! atmosphere present flag
      LPH = PARW(1).GT.0.      ! doing planetary heat loads
      LECL= (ABS(PARC(1)-1.).LT. 0.2)       ! doing daily eclipses
      IF (LATM) allow twilight, else TWILFAC = 1. and LTW is False
      IF (LOPN3) setup TFAR8 and set LINT iff will need to interpolate in time
      SOLAU=SOLR=SOLCON/(DAU*DAU)! solar flux at this heliocentric range
      SALB=PUS*ALB              ! spherical albedo, for diffuse irradiance
        CALL ECLIPSE(PARC,PARI JBE, FINSOL) only if DailyEclipse and first season
in Lat. loop
 in time loop
  calc PUH= PhotFunc for horizontal surface using COSI
  calc AVEA=ALB*PUH and ensure 1-A cannot be negative
  If LATM do delta-Eddington, else  
             TOPUP=COSI*AVEA         ! upward solar 
             BOTDOWN=0.         ! no atm scattering
             ATMHEAT=0.         ! no atm absorbtion
             COLL=1.D0          ! no atm attenuation of beam
             DIRFLAT=COSI ! incident intensity on horizontal unit area 
  if day or twilight 
           DIFFUSE=SKYFAC*BOTDOWN ! diffuse flux onto surface
             G1=1.0D0
           BOUNCE=(1.D0-SKYFAC)*SALB*(G1*DIRFLAT+DIFFUSE) 
  else 
           DIFFUSE=0.
           BOUNCE=0.

  if  target is directly illuminated
   calc PUH=PhotFunc for (sloped) surface using COS2, HALB=ALB*PUH
           DIRECT=COS2*COLL    
         IF (LECL) SOLR=SOLAU*FINSOL(JJ) ! eclipse factor     Daily only
         QI=DIRECT*SOLR         ! collimated solar onto slope surface

         ASOL(JJ)=QI            ! collimated insolation onto slope surface
         ALBJ(JJ)=MAX(MIN(HALB,1.D0),0.D0) ! current hemispheric albedo
         SOLDIF(JJ)=(DIFFUSE+BOUNCE)*SOLR ! all diffuse, = all but the direct.

         IF (LPH) THEN ! calc planetary heat loads and add to day sum
           PLANH(JJ) and PLANV(JJ)

         ADGR(JJ)=HUV=ATMHEAT*SOLR ! solar flux available for heating of atm. H_v
 end of time loop 
        IF (LPH)  add in absorbed planetary heating
        IF (LATM)  set BETA and TEQUIL and other equilbrium temperatures 
          else BETA=0. and set TEQUIL
         If first season TATMJ=77.7.  If no atm, no routine changes this
         CALL TDAY8 (2,IRL)      ! execute day loop
         Predict and store results
End of latitude loop

\end{verbatim}
 
%\clearpage
\subsubsection{Synopsis of TDAY radiation calculations}
%\vspace{-3.mm} 
\begin{verbatim}
TDAY(2
      FAC9=SIGSB*BETA           ! factor for downwelling hemispheric flux
      if no atm, ATMRAD=0.
Top of day loop
        IF (LDAY) THEN if LRARE and last season then J7=JBE(1)-1 
IN time loop:
       IF (JJ.EQ.J7) and  J7 .LE. JBE(1)  CALL TFINE8  the reset J7
                     else Transfer layer T's and set J7=1 
after layer loops: when no frost
            ABRAD=FAC3*ASOL(JJ)+FAC3S*SOLDIF(JJ) ! surface absorbed radiation
            IF (LATM) THEN 
              ATMRAD=FAC9*TATMJ**4 ! hemispheric downwelling IR flux
              ABRAD=ABRAD+FAC6*ATMRAD ! add absorbed amount
            ENDIF 
            IF (LPH) ABRAD=ABRAD+EMIS*PLANH(JJ)+FAC3S*PLANV(JJ)
            SHEATF= FAC7*(TTJ(2)-TSUR) ! upward heat flow to surface
            POWER = ABRAD + SHEATF - FAC5*TSUR*TS3 ! unbalanced flux
            IF (LOPN3) POWER=POWER+FARAD(JJ) ! fff only

          IF (LATM .AND. LSELF) THEN  !v-v-v-v-v  Adjust atmosphere temperature
            TATM4=TATMJ**4
C  ADGR is solar heating of atm
            HEATA=ADGR(JJ)+FAC9*(EMIS*TSUR4-2.*TATM4) ! net atm. heating flux
            TATMJ=TATMJ+HEATA*DTAFAC ! delta Atm Temp in 1 time step
          ENDIF                 !^-^-^-^-^

  IF (LATM) THEN  DOWNIR(IH,J4)=ATMRAD ! save downward IR flux  ELSE left as was!

            DOWNVIS(IH,J4)=ASOL(JJ)+SOLDIF(JJ) ! downward coll.+diffu. solar flux
\end{verbatim}

\section{Test results}
\subsection {Validation}
 Against 344.  Minimal edit of 342/run/342v3t.inp to 344/run/344v3t.inp
\qi difference negligable away from cap edges.
 

351: edit krc/Eur/351v3t.inp

\subsection {New capabililties}

During development, many runs were done with all the eclipse debug options enabled, which generates 6 ``fort.xx'' text files; the results viewed  using the IDL routine krv35.pro.  

Some test were done with largely realistic conditions. E.g., for the sub-Jovian longitude on Europa, using the nominal values listed in \S \ref{nomp} with thermal inertia 200 in MKS and 22 layers to a total depth of 11.8 diurnal skin depths. Cases are 
\qi 0 \ Insolation only, no eclipse
\qi 1 \ add geothermal  heatflow= 100mW/m$^2$
\qi 2 \ Daily eclipse at local noon, with heatflow  
\qi 3 \ Daily eclipse at local noon, no heatflow   
\qi 4 \ Daily eclipse at local noon, with Planetary fluxes
\qi 5 \ Rare eclipse at local noon, with base heatflow at start of eclipse maintained
\qi 6 \ Rare eclipse at local noon, with base layer temperature at start of eclipse maintained
\qi 7 \ Daily eclipse at local noon, 30\qd~ East slope,with heatflow and planetary fluxes
\qi 8 \ As above, but no planetary fluxes

Results were displayed using the IDL routine krv36.pro and some are shown in
Figures \ref{eurCTs} and Figure \ref{eurClats}.  With the values used, at the
equator, basal heat-flow of 100 mW/m$^2$ increases T about 0.5 K, and radiation
from Jupiter increases T about 3.2 K. The two methods of handling the lower
boundary condition for Rare eclipses differ by $<$ 1 nK.

\begin{figure}[!ht] \igq{eurCTs}
\caption[Europa surface T]{Europa surface temperatures for several conditions
  for the equator, with eclipse at local noon. Legend has an abbreviation for
  the conditions; see Table ??? for description
\label{eurCTs} eurCTs.png }
\end{figure} 
% how made:

\begin{figure}[!ht] \igq{eurClats}
\caption[Europa 3 latitudes]{Europa surface temperatures for several conditions for latitudes 0, 30 and 60. See Table ??? 
\label{eurClats}  eurClats.png }
\end{figure} 
% how made:

\begin{figure}[!ht] \igq{teaser}
\caption[Europa test]{Results from the eurB.inp run. Nominal Europa,
  albedo=0.67, Inertia=200, obliquity and inclination assumed=0. Results for
  sub-Jupiter equator.
\label{teaser}  teaser.png }
\end{figure} 
% how made:

\section{Other version 3.5 changes}
Replace EVMONO38 with EVMONO3D, which has the scaling factors firm-coded in the
routine, eliminating 2 arguments. Latter routine is 9\% faster

Change line: ``  16 N 'ffff' `` will toggle output of a binary file named ffffxx.bin5 for
each case; xx will be the case number. This file will contain the surface
temperature for every time step for the last season for the N'th latitude. A
non-positive value of N turns this off.

Because all the KRCCOM arrays are full, add storage of N to HATCOM and use FMOON in FILCOM for the file name stem.

\appendix %%%%%%%%%%%%%%%%%%%%%%%%%%%%%%%%%%%%%%%%%%%%%%%%%%%%%%%%%%%%%%%%%%

%\clearpage

\section{Debug options new with v 3.5}

TFINE always outputs \nf{tfinexx.bin5}: ASOL, FINSOL
and all layer temperatures at every fine time-step. 
\qi The file header contains N2,J7,J8,J9 
\qi array is [2+fine layer, fine-time]  \hfill IDL krc35.pro reads as bbb

Optional files:  Each may have more than one case

Table below: columns are:
\qi 1: Minimum IDB5 value to trigger output
\qi 2: fort.X file. \  P means it goes to print file. \ M means to Monitor
\qi 3: Routine that writes this. D=TDAY, F=TFINE. And which stage: 1 or 2


%\begin{table} 
%\caption{Optional output} \label{optout}
%\begin{center}
\begin{tabular}{|| l  c  c  l | l ||} \hline 
ID &     & St- &             & IDL    \\ 
B5 & out & age & Description & code     \\  \hline
1 & P & F & IQ,JJ upon entry, print exit & \\
1 & P & F1 &  Least stable layer and T-dep. layer set & \\
1 & M & F1 &   QB.. key values & \\
1 & P & D2 &  LZONE... T-dep layer ranges & \\
2 & P & F2 &   Layer stability table & \\
2 & 42 & F1 & J,BLAF,SCONVF,QA for each fine layer  & \\ 
3 & P & F  &  Starting Tsurf, delbot & \\ 
4 & 43 & F1 & for N1 layers at start: TDAY: depth,T,splineY,c-thick,f-thick,f-depth & fff[layer,item]\\
  & `` & F1 & for fine layers: depth, T, FA1, FA3  & uuu[layer,item] \\
? & 44 & F2 &  JFI,FINSJ, TSUR,ABRAD,SHEATF,POWER,FAC7,KN  & ddd[ctime,item] \\
 & & & \ \ \ each fine time near edge  & \\
4 & 47 & F2 &   T for fine layers and for coarse layers at end of eclipse, followed by :  & vvv[item*case, layer] \\
4 &47 & D2 & T for layers, just before being replace by eclipse results. & ? \\
5 & P & F &  Index, center depth and initial temperature for each fine layer & \\
6 & M & F &  I,J, fine-layer factor for each layer & \\
7 & 44 & F2 & values for each fine time step near eclipse ends & \\
7 & 46 & D2 &  JJ ,ATMRAD,TSUR,ABRAD,SHEATF,POWER,FAC7,KN  &  aaa[time,item]\\ 
 & & &  \ \ \ each coarse time. Rare only. & \\ \hline
\end{tabular} % \end{center}  \end{table}

Notes: 1) Radiation fields do not show eclipse because they are normal for Rare eclipse 
\qii 2) TSUR ( and SHEATF) will show discontinuity at end of followon.

\section{Some values for Solar system satellites \label{nomp}}

\textbf{Earth:} 
\\ Eclipse card for Earth lunar eclipse might be
\qi 14 3 1. 6378.2 384.4d3 1737. 29.53 0.345 6000 12. 7 / Moon
\\ Eclipse card for Earth solar eclipse might be
\qi 14 3 1. 1737.4 384.4d3 6315. 29.53 0.345 6000 12. 7 / Earth solar
\qi but need to account for sol not the same as lunar synodic month
\qii a test of the routine is that with bias=0, mid-eclipse would be about 6\% short of total.

\textbf{Mars:} eq. radius = 3396.2 km
\qi Orbit SMA= 1.523679 AU
\qi Period 1.8808 yr or 686.971 day
\\ Satellites =['Phobos','Deimos']
\\ Satellite orbit radius = 9376.,  23463.2 km
\\ Satellite radius: 11.2667, 6.2   km 
\\ Mutual period  0.3189, 1.263   day
\\ For Phobos solar eclipse, the surface radius of Mars is a significant term.

\textbf{Jupiter:} eq. radius =71492. km
\qi Orbit SMA= 5.2026 AU
\qi Period 11.8618 yr or 4332.59 day
\\ Satellites =['Io','Europa','Ganymede','Callisto']
\\ Satellite orbit radius =[.4218,.6711,1.0704, 1.8827]*1.e6
\\ Satellite radius: =[3640.,3121.6,5268,2,4820.6]
\\ Mutual period =[1.77,3.55,7.15,16.69] days
\\ Angular radius of Jupiter from satellite: $\arcsin(d/R)$
\qi 0.1703  \   0.1067 \   0.0668  \  0.0380

Europa heat flow: 30 to 130 mW/m2: J. Ruiz, 'The heat flow of Europa', Icarus v. 177, p438:446 (2005) 

Jupiter: emission temperature about 134 K, or 18.3 W/m$^2$/steradian. 
\qi At Europa, Jupiter is about 0.0356 steradian. 
\qi Thus thermal flux onto Europa about 0.62 W/m2

Jovian bolometric albedo about 0.73(?),
so reflected radiance about  0.73 * scon/(5.2026$^2*\pi$) = 11.7 W/ster
\qi or 0.42 W/m2 onto Europa at inferior conjunction. 
\\ Thus, planHeatcard for Europa might be
\qi 15  0.62 0. 0.  0.21 0.21  0.  12. / Jupiter heat load on Europa, nearside center

These can be compared to the solar irradiance at Jupiter of 50.53 W/m2

\textbf{Saturn:} Eq. radius.  60268  km
\qi Orbit SMA= 9.5549  AU
\qi Period 29.4571 yr or 10759.22  day 
\\ Satellites =['Enceledus','Titan','Iapetus']
\\ Satellite orbit radius =[0.237950,1.22193,3.56082]*1.e6
\\ Satellite radius: =[504.2,5149.,1468.6]/2.  km 
\\ Mutual period =[1.370, 15.945 ,79.3215] days
\qi Titan has atm: Psurf=147 Pa N$_2$+ 1.4\% CH$_4$ 
\qii Lakes and varied surface geology
\qi Iapetus has inclination 15.5\qd

\textbf{Neptune:} $r_m$=24622. SMA=30.33 AU
\\ Triton, r=1353.4, sma=354759.  incl.=157 (to nep)
\qi Psurf=1.4:1.9 Pa N$_2$2 , ``geysers''


\subsection{Test input files}
Chronologic; several run many times. Any run older than 2017 Apr 5 13:15 should be abandoned. Many .inp files deleted.
\vspace{-3.mm} 
\begin{verbatim}
0=no eclipse, D=Daily, R=Rare H=PlanetaryHeating, n=nill 
cirMars = circular orbit at Mars distance, zero obliquity

 3874 Dec  9 06:46 thin9.inp Mars  5 lats, 120 days, 9 cases: vary layers
 3448 Mar 20 16:50 V35a.inp Europa 5 lats, 20 days, 4 cases: 0,D,R,0H
 3536 Mar 30 12:49 eur6.inp Europa 1 lat, 10 days, 3 cases: 0, D and R
 3168 Mar 30 14:21 phob.inp Mars, real phobos, 1 lat, 10 days, 3 cases: 0, D and R
 3195 Mar 30 14:25 phon.inp Mars, no atm,  1 lat, 10 days, 3 cases: 0, D and R
 3942 Mar 30 15:58 351v3t.inp  Mars 5-lats, 6 cases for standard V3 validation
 3255 Mar 31 06:04 phoz.inp Mars,   1 lat, 10 days, 4 cases: all 0, vary PTOTAL
 3339 Apr  2 16:22 phoc.inp cirMars 600 km Phobos 1 lat 4 cases: 0,D,R,Rn
 4127 Apr  4 23:26 eurA.inp 1lat, 20 days 
 3916 Apr  5 15:38 eurB.inp Europa 1 lat,
 3916 Apr  5 15:39 eurC.inp
\end{verbatim}

Analysis of each run using IDL kv3 calling krc35

FORTRAN routines are tested individually using testrou.f, executable testr

\subsection {Notes on the need to separate radiation fields}
Want to allow planetary loads when have an atmosphere.

Must separate radiation fields:
\qi Solar incident top-of-atm. all and only these influenced by eclipse
\qii  abs in atm,  
\qii  collimated at surface,  
\qii  diffuse at surface,
\qii  [lost]
\qi Atm down-going IR
\qii  Assume existing treatment included multiple reflections, messy to rederive
\qi Planetary visible top-of-atm:  PLANV
\qii  abs in atm, 
\qii  abs at surface
\qii  [lost]
\qi Planetary thermal top-of-atm:  PLANH
\qii  abs in atm,  
\qii  abs at surface,
\qii  [lost]
\qi Hemispherical albedo: ALBJ

\section{Integer to:from real conversion}
Let a real value $x$ run from 0 to $V$, e.g., 0 to $2\pi$ or  0. to 24.
\\ Let the integral indices $I$ representing this interval run from 1 to $N$; the 1-based system
\qi For notation convenience, define $R \equiv \frac{V}{N}$
\vspace{-3.mm} 
\begin{verbatim}
x:  0=|++^++|++^++|++ ... ++|++^++|=V   Real representation
I:    |  1  |  2  |   ...   |  N  |     Integral representation, 1-based
M:    |  0  |  1  |   ...   | N-1 |     Integral representation, 0-based
\end{verbatim}    

Integer to real: $x=(I-\frac{1}{2})\frac{V}{N}  \mc{or}  x=(I-0.5)R $
\\ Real to integer:  $I=$ NINT $( x/R +.5 )$ \ 
\qi BEWARE, the default real:integer conversion in many languages is to truncate magnitude. 
\qii This results in a relationship discontinuity (no change in I) at $x=0$.
\qiii If $y$ is always positive: NINT(y-.5) and INT(y) are identical.
 
or \  $I=$ $ x/R + 1$ if $x$ is non-negative and the default real:interger conversion is to truncate magnitude.

FORTRAN intrinsics for real to integer (all tested in testrou.f @28 )
\qi CEILING - Integer ceiling function
\qi FLOOR - Integer floor function
\qi INT - Convert to integer type    \ \ identical results to I=x
\qii INT2 - Convert to 16-bit integer type
\qii INT8 - Convert to 64-bit integer type
\qii LONG - Convert to integer type
\qi NINT - Nearest whole number
\qi FLOAT - Convert integer to default real
\\ Only CEILING, FLOOR,and NINT are consistent across 0. 
\\ The use of NINT and FLOAT maintains integrity. 

% \section{Debug printout}
\subsubsection {version testing}
 
\begin{verbatim}
 against 344.  minimal edit of 342/run/342v3t.inp to 344/run/344v3t.inp
 
351: edit krc/Eur/351v3t.inp

kv3.pro 

File names
  0 VerA=new DIR    200 = ~/krc/Eur/out/
  1  " case file   202  = 351v3tb
  5 VerB=prior DIR 201  = /work2/KRC/344/run/out/
  6  " case file   202  = 344v3tb


@115 123 116 123
kv3 Enter selection: 99=help 0=stop 123=auto> 550
Num lat*seas*case with NDJ4 same/diff=        1197           3

@116 makes kons=233      56     561     562     563     564     565      61     622      -1      63


@56: t
@561: 0
.
% ARRSUB: some index error, see above comment
ARRSUB error        2
SOME ERROR CONDITION at kon=     561.  Any key to Go
@12  11=0 12=4 17=0 18=-1
@561
 help,qy
QY              DOUBLE    = Array[48, 5, 40, 6]

Tsurf caseRange=all LatRange=0:4  SeasonRange=all  hour   lat  seas  case
 quilt before any other display
                   Mean       StdDev      Minimum      Maximum
         1    -0.00101340    0.0169005    -0.800859    0.0731013  signed
N=   57600     0.00268893    0.0167159      0.00000     0.800859  absolute

kv3 Enter selection: 99=help 0=stop 123=auto>  562
351v3tb - 344v3tb:  Tsurf. caseRange=all LatRange=0:4  SeasonRange=all
         -60.         -30.           0.          30.          60.
% Compiled module: MEAN_STD2.
Mean= (each case)
    0.0453352      0.00000      0.00000      0.00000      0.00000
    0.0280654      0.00000      0.00000      0.00000      0.00000
   0.00726742      0.00000      0.00000      0.00000      0.00000
      0.00000      0.00000      0.00000      0.00000      0.00000
      0.00000      0.00000      0.00000      0.00000      0.00000
      0.00000      0.00000      0.00000      0.00000      0.00000
StDev=
    0.0616917      0.00000      0.00000      0.00000      0.00000
    0.0299598      0.00000      0.00000      0.00000      0.00000
    0.0100296      0.00000      0.00000      0.00000      0.00000
      0.00000      0.00000      0.00000      0.00000      0.00000
      0.00000      0.00000      0.00000      0.00000      0.00000
      0.00000      0.00000      0.00000      0.00000      0.00000

kv3 Enter selection: 99=help 0=stop 123=auto> 563
    Item       Mean     StdDev        Min        Max    MeanAbs     MaxAbs  0]=NDJ4
    NDJ4   -0.00167    0.07072   -2.00000    1.00000    0.00333    2.00000
    DTM4   -0.00000    0.00059   -0.00399    0.01548    0.00008    0.01548
    TTA4   -0.00036    0.00710   -0.10882    0.02495    0.00126    0.10882

QUILT3 displayed value range is      -0.10881889     0.024951097
sample is: latitude(5)  * 8 planes of case
line is: season(40)  * 1 groups of case.  Lines increase upward
SOuthern lats show the changes
  FROST4    0.05272    0.34532   -0.23976    3.60657    0.05312    3.60657
QUILT3 displayed value range is      -0.23975942       3.6065696
sample is: latitude(5)  * 8 planes of case
line is: season(40)  * 1 groups of case.  Lines increase upward
Any key to go
   AFRO4    0.00000    0.00000    0.00000    0.00000    0.00000    0.00000
  HEATMM   -0.00174    0.02303   -0.20097    0.11920    0.00585    0.20097
QUILT3 displayed value range is      -0.20096924      0.11919618
sample is: latitude(5)  * 8 planes of case
line is: season(40)  * 1 groups of case.  Lines increase upward

kv3 Enter selection: 99=help 0=stop 123=auto> 564
    Item       Mean     StdDev        Min        Max    MeanAbs     MaxAbs  0]=Lat
    Lat.    0.00000    0.00000    0.00000    0.00000    0.00000    0.00000
    elev    0.00000    0.00000    0.00000    0.00000    0.00000    0.00000

kv3 Enter selection: 99=help 0=stop 123=auto> 565
    Item       Mean     StdDev        Min        Max    MeanAbs     MaxAbs  0]=DJU5
    DJU5    0.00000    0.00000    0.00000    0.00000    0.00000    0.00000
    SUBS    0.00000    0.00000    0.00000    0.00000    0.00000    0.00000
   PZREF    0.00000    0.00000    0.00000    0.00000    0.00000    0.00000
    TAUD    0.00000    0.00000    0.00000    0.00000    0.00000    0.00000
    SUMF    0.00000    0.00000    0.00000    0.00000    0.00000    0.00000
kv3 Enter selection: 99=help 0=stop 123=auto> 61
Maximum difference in Ls is:       0.0000000
kv3 Enter selection: 99=help 0=stop 123=auto> 62

RESULT, negligable differences away from frost edge.

\end{verbatim}   


\end{document} %==========================================================
Figure \ref{}  
\begin{figure}[!ht] \igq{}
\caption[]{
\label{}  .png }
\end{figure} 
% how made:
