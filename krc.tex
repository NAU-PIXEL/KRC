\documentclass{article}

\usepackage{definc}  % Hughs conventions

\textheight=9.80in  \topmargin=-0.5in           %  hobo normal=final
\textwidth=7.5in  \oddsidemargin=-0.3in \evensidemargin=-0.3in  % hobo final
\parindent=0.em \parskip=1.ex %  no indent & paragraph spacing

\title{KRC Thermal Model}
\author{Hugh H. Kieffer \ \ \ file=~/hkieffer/xtex/tes/krc.tex 2004jul21}
% individual variables
\def\qF{\Delta M}       % delta frost
\def\qH{\mathcal{H}} % scale height
%                with parameters

\begin{document}

\maketitle

History of this \LaTeX ~file:
\qi 2002jul12-31  Write new atmosphere section
\qi 2002sep17
\qi 2004Jul18-22 upgrade description of atmosphere

\section{Symbols}
Code variable names are shown in text as \ct{VARIAB} 
and in formulas as $\cf{VARIAB}$. 
\\ Program and routine names are shown as \pname{PROGRM}. 

\section{Background}

The TES Thermal Inertia routines can run with thermal models generated either by
Mike Mellon or Hugh Kieffer.  The thermal model described here is Hugh's. For
the MGS production thermal inertia software, see MGSmodel.tex.

Hugh's MGS thermal model is derived from the Viking thermal model
(Kieffer et al., 1976, JGR 82,p4249-4291, appendix), and later
modifications to apply to comets. All the comet-related code has been
removed, and on 2002jul13 the atmospheric interaction was substantially modified.

The KRC model development began when computing a single case for 19 latitudes at
40 seasons took an hour on a large university main frame computer. For this
reason, the code was highly optomized for speed and uses layer thickness
increasing exponentially downward and time steps that increase by factors of two
deeper into the subsurface where stability criteria are met. The code is
modularized based on time scale and function, and there is extensive use of
Commons. For THEMIS, a ``one-point'' capability was included that allows input
of a set of points defined by season, latitude, hour and a few major physical
parameters; KRC will produces the surface kinetic temperature and planetary
brightness temperature for these points.

In response to an oft-asked question, the acromyn KRC is simply K for
conductivity, R for ``rho'' $( \rho)$ for density, and C for specific heat; the
three terms in thermal inertia. 

The numerous input parameters that control the time-depth grid and convergence
are based upon extensive testing done during the development of the TSEAS, TLATS
and TDAY routines; be carefull changing them!

A guide to running KRC is in the file \pname{helplist.txt}

\begin{table} 
\caption{Fortran Code set}
\label{tab:fort}
\begin{center}
\begin{tabular}{ l l} \hline 
Name & description \\ 
\hline 
 & \hcm3 Primary routines \hcm3 \\
KRC &  Planet surface thermal model; top routine,  MGS-TES version \\
TSEAS & Advance one "season" along planets orbit for KRC system \\
TLATS  & Latitude computations for the  KRC thermal model system \\
TDAY  &  KRC day and layer computations \\
 & \hcm3 Input / output routines \hcm3 \\
TCARD &  Data input routine for  KRC system \\
TDISK  &  Save/read results at the end of a season;  Version with BINF5 \\
TPRINT &  Printed output routine \\
 & \hcm3  Specific task routines \hcm3 \\
ALBVAR & Compute frost albedo as linear function of insolation \\
ALSUBS  &  Convert between L-sub-s and days into a Martian year \\
AVEDAY &  Average daily exposure of surface to sunlight. \\
CO2PT  &   CO2 pressure/temperature relation \\
DEDING2 &  Delta-Eeddington solution for single homogeneous layer \\
EPRED &   Exponetial Prediction of numerical iteration \\
TINT  &   Spherical integrals over globe \\
VLPRES  &  Viking lander pressure curves \\
 & \hcm3 Orbit geometry routines \hcm3\\
PORB &  Computes planetary angles and location for specific time. \\
PORB0 & Planetary orbit. Read pre-computed matrices and do rotation; minimal for
KRC \\
ECCANOM &  Iterative solution of Keplers equations for eccentric orbit \\
ORBIT &  Compute radius and coordinates for elliptical orbit \\
 & \hcm3 Utility routines listed in Makefile \hcm3\\
Fortran & catime.o datime.o idarch.o sigma.o vaddsp.o xtreme.o binf5.o white1.o \\
C &  b2b.o r2r.o u\_move1.o u\_move4.o u\_swapn.o primio.o pio\_bind\_c.o \\
C &  binf5\_bind.o b\_alloc.o b\_c2fstr.o b\_f2cstr.o b\_free.o \\
 & \hcm3 Other routines \hcm3\\
IDLKRC  &  Interface to IDL. Planet surface thermal model  MGS-TES version \\
\end{tabular} \end{center}
\end{table}


\section{Surface boundary condition} %-------------------------------
In \pname{TDAY}, the frost-free surface condition is
\qb W=(1.-A)S_{(t)}'  + \ql 1.-\alpha \qr \epsilon R_{\Downarrow t}
+\frac{k}{X_2}(T_2-T) - \ql 1.-\alpha \qr \epsilon\sigma T^4 \qe

where $X_2$ is the depth to the center of the first soil layer, $S_{(t)}'$ is
the total solar radiation onto the surface, and $ R_{\Downarrow t}$ is the
downwelling thermal radiation (assumed isotropic).

Most constant terms are pre-computed, see Table \ref{tab:sym}.

The boundary condition is satisfied when W=0. A Newton iteration is done:
\qb \frac{\partial W}{\partial T} = -F_7 -4F_5 T^3 \qe
thus estimate the change in surface temperature as 
\qbn \Delta T = W / \ql F_7 +4F_5 T^3\qr \label{Eq:delt} \qen


Subscript $F$ indicates the values when frost is present. and the values in
Eqn.~\ref{Eq:delt} are replaced with $\epsilon_F$, $A_F$, and $T_F$, and no
iteration is done; leaving $W$ as a non-zero quantity. See Section
\ref{sec:frost}.

%\pagebreak %<<<<<<<<<<<<<<<<<<<<<<<<<<<<<<<<<<<<<<<<<<<<<<<<<<
\cinput{newatm} %<<<<<<<<<<<<<<<<<<<<<<<<<<<<<<<<<<<<<<<<<<<<<<<<<

\section{Pressure variation}
$P_0$ = annual mean surface pressure at the reference elevation (input as
\ct{ PTOTAL } ). 
\\$P_g$ = the current global pressure \ct{= PZREF}, can be any of the following: 
\qi 1) constant at $P_0$
\qi  2) $P_0$ times the normalized Viking Lander pressure curve \ct{VLPRES}
\qi  3) based on depletion of atmospheric CO$_2$ by growth of frost caps; =$P_0-$cap .
\\$P=P_g e^{-z/\qH}$ \ct{= PRES} is the current local total pressure at a specific elevation. The exponential term is \ct{PFACTOR} 

The initial partial pressure of CO$_2$ at zero
elevation is $P_{c0} = P_0 \cdot (1.- $noncondensing fraction). \ct{= PCO2M}

The current CO$_2$ partial pressure at zero elevation is $P_{cg}=P_{c0} +
(P_g-P_0)$.  \ct{= PCO2G}

The nominal scale height is: $\qH=T_a \mathcal{R} / \mathcal{M} G $;
where $T_a$ is the mean atmospheric temperature over
prior day (or season), $\mathcal{R}$ is the
universal gas constant, $\mathcal{M}$ is the mean molecular weight of
the atmosphere (43.5, firm-coded), and G is the martian gravity.

Local current dust opacity scales with total pressure: $ \tau = \tau_0 P/P_0$

\subsection {CO$_2$ Frost condensation and Sublimation \label{sec:frost}}

The local frost condensation temperature \ct{TFNOW} may be either fixed at an
input value \ct{TFROST}, or derived from $P_c= e^{-z/\qH} P_{cg}$.

The relation between condensation/sublimation temperature and partial pressure
is taken to be the Clausius-Clapeyron relation: $ \ln P_c=a-b/T$ , in
\pname{CO2PT} with a=27.9546 [Pascal] and b=3182.48 [1/Kelvin], derived from
MARS page 959.

The code logic is: after subsurface layer calculations:

If frost is present $E=W \cdot \Delta t$ energy is used to modify the amount of
frost; $\Delta M = -E/L$ , where $L$ is the latent heat of sublimation. And, the
frost albedo may be variable, and there may be an exponential attenuation of the
underlying ground albedo. E.g.,

If Frost then
\qi calc the change in frost amount, $\Delta M$ , apply it
\qii if $M \le 0$, then
\qiii set Frost false \& set $M=0$
\qi else
\qii calc unbalanced Power W based on prior T \& determine $\Delta T$ to balance
\qiii if $T < T_F$ then
\qiii set Frost true \& set $T=T_F$.

\section {Subsurface} %------------------------------------------------------
\subsection{Diffusion theory for layered materials}
Symbols used:
\qi $i=$ layer index, layer 1 is above the physical surface
\qii subscript + is shorthand for $i+1$ and subscript - is shorthand for $i-1$
\qi $I=$ thermal inertia $\equiv \sqrt{k \rho C}$
\qi $k=$ thermal conductivity
\qi $\rho=$ bulk density
\qi $C_p$ or $C=$ specific heat of the material
\qi$\kappa =$  Thermal diffusivity $ \equiv \frac{k}{\rho C_p}$
\qi $B_i=$ thickness of layer i, or $\Delta z$
\qi $t=$ time
\qi $T=$ temperature
\qi $H=$ heat flow=$-k \frac{dT}{dz}$

Basic 2nd difference equation of heat flow:
\qbn \frac{\partial T}{\partial t} = \frac{-1}{\rho C} \frac{\partial}{\partial z} \left( -k \frac{\partial}{\partial z} \right) \\ \mathrm{or}\frac{k}{\rho C} \frac{\partial
  ^2T}{\partial z^2} \\ \mathrm{or} \\ \frac{\Delta T_i}{\Delta t} = -\frac{ H_{i+1/2} -H_{i-1/2}}{B_i \rho_i C_i} \qen
Use steady-state relations to find heat flow at interface between two layers:
$H=-k \nabla T$
\qb H_{i+.5} = -\frac{T'-T_i}{B_i/2}k_i \ \ \mathrm{or} \ \ 
T'-T_i = -\frac{ H_{i+.5} B_i}{2 k_i} \qe
where $T'$ is the temperature at the interface.
\qb  \mathrm{similarly  \ \ } T_{i+1}-T' = -\frac{ H_{i+.5} B_{i+1}}{2 k_{i+1}} \qe
\qb  \mathrm{Thus \ \  } T_{i+1}-T_i = -\frac{H_{i+.5}}{2} \left( \frac{B_i}{k_i} +
    \frac{B_{i+1}}{k_{i+1}} \right) \qe
\qb  \mathrm{or  \ \  } H_{i+.5} = -\frac{2 (T_{i+1} -T_i) }{  \frac{B_i}{k_i} +
    \frac{B_{i+1}}{k_{i+1}} } \qe

\qbn \frac{\Delta T_i}{\Delta t} = \frac{2}{B_i \rho_i C_i} \left[ 
  \frac{T_+ -T_i}{ \frac{B_i}{k_i} + \frac{B_+}{k_+} }
- \frac{T_i -T_-}{ \frac{B_i}{k_i} + \frac{B_-}{k_-} } \right] \qen
Put into the form $\Delta T = F_{1_i} \left[ T_+ +F_{2_i} T + F_{3_i} T_+
\right] $ where $F_2 = -(1+F_3)$.

\qbn F_{1_i} = \frac{k_i}{\rho_i C_i} \frac{ \Delta t}{ B_i^2} \frac{2}{
  1+\frac{B_+k_i}{B_i k_+}} \qen

\qbn F_{3_i}=\frac{ B_i/k_i \ \ + \ \ B_+/k_+}{B_-/k_- \ \ + \ \ B_i+k_i } 
 = \frac{1+ \frac{B_+}{B_i} \frac{k_i}{k_+} } { 1+ \frac{B_-}{B_i}
   \frac{k_i}{k_-} }\qen

 
 The classic convergence stability criterion is $\frac {\Delta t}{(\Delta Z)^2}
 \kappa < \frac{1}{2}$ or $ (\Delta z)^2 \equiv B^2 > 2 \Delta t \kappa $.  
[ To be safe, at each  layer should use the largest diffusivity of the 3 layers 
involved in the 2nd difference scheme ]

If a discontinuity of physical properties is invoked, need to reset the layer
thickness scheme at that point to maintain stability.


\subsection{KRC code}

The user inputs the thermal inertia $I$, the bulk density $\rho$, and the
specific heat of the material $C_p$. Thermal conductivity $k$ is computed from
  $I^2 / [\rho C_p]$. The thermal diffusivity is $\kappa = \frac{k}{\rho C_p}$.
While $k$, $\rho$, and $C_p$ do not independantly influence the surface
temperature for a homogeneous material, they set the spatial scale of the
subsurface results; SCALE $= \sqrt{kP /\pi \rho C_p}$. 

KRC uses layers that increase geometrically in thickness by a factor RLAY.

In order to simplify the innermost code loops, KRC places the surface between
the first and second model layers. The input parameter FLAY specifies the
thickness of this ``virtual'' layer is dimensionless units (in which the diurnal
skin depth is 1.0), so thatthe thickness of the uppermost layer in the soil is
FLAY*RLAY, and the depth of its center in meters is 0.5*FLAY*RLAY*SCALE.
Normally (LP2 set true) a table of layer thickness, depth, (both scaled and in
meters), overlying mass, and numerical convergence factor is printed out at the
start of a run.
\qi $R=$ RLAY = ratio of thickness of succeeding layers
\qi $X=$ X = scaled depth to middle of each layer
\qi $B=$ TLAY = scaled thickness of each layer. $l_1=$ FLAY
\qi $P=$ PERSEC = diurnal period in seconds
\qbn \mathrm{F4} = 1 + 1/R \qen
\qbn B_i= B_1*R^{i-1} \sqrt{\frac{k_i}{\rho_iC_p} \frac{P}{\pi}} \qen 
\qbn X_1 = -B_1/2 \ \ \ \ X_i = X_{i-1}+(B_{i-1}+B_i)/2 \qen
\qbn \mathrm{FA}_i = \frac{ 2 \cdot k_i \cdot \Delta t}{\rho_iC_p \cdot B_i^2
  \left( 1 + \frac{B_{i+1}k_i}{B_ik_{i+1}} \right) } \qen
\qbn \mathrm{FA3}_i=\frac{B_i/k_i \ + \ B_{i+1}/k_{i+1} }{B_i/k_i \ + \
  B_{i-1}/k_{i-1} } \qen

The convergence safety factor is $\Delta Z / \sqrt{ 2. \Delta t \cdot
  \kappa}$. If this is less than about 0.8, the process is numerically
unstable. If possible, the routine will keep this larger than 2.

\subsection{Layered Material}
 Beginning with layer IC, all lower layers can have their conductivity, density
 and volume specific heat reset to COND2, DENS2, and SPHT2 respectively. If
 LOCAL is set true, then the physical thickness of these layes scales with the
 local thermal diffusivity; otherwise, geometric increase continues unaltered.

\section{One-point Model 2002aug04}
To support the THEMIS team, an interface to the KRC system was built that
computes the temperature for a single condition. The user generates a file
'one.inp' that contains lines of specific times and conditions.

The input file \ct{krcone\_master.inp} is set to do one latitude for 2 seasons.
It contains a change-card 10 which points to 'one.inp' as the file of specific
points.

\subsection{Code to accomodate One-point mode}
krc.f: 
\qi at the start of each case, sets IQ=1, which tells TSEAS to restart.
\qi after each season, calls TPRINT(9)

tseas.f
\qi does not report the time elapsed

tcard:
\qi Has a section for first item on change line being 11, which decodes the line into
\qii L\_s,ALAT(1),HOURO,ELEV(1),ALB,SKRC,TAUD,SLOPE,SLOAZI
\qii computes DJUL for the desired L\_s using \pname{ALSUBS}
\qi does not call TPRINT to print changes

tprint(9):
\qi Interpolates in hour,  Prints one line of: 
\qii SUBS,ALAT(1),HOURO,ELEV(1),ALB,SKRC,TAUD,SLOPE,SLOAZI,touto,q4
\qiii touto is the surface kinetic temperature at the requested hour
\qiii q4 is planetary bolometric 

\section{Evolution}
\subsection{2004jul20} 
Allow scale height to depend upon diurnal-average atmosphere temperature

Modify output type 52 to include atmospheric temperature, but only up to 5
cases. It now reduces file size to actual number of latitudes and seasons.
\end{document}
