%%% intend for includion in V34UG.tex

\subsubsection{Notation} %..........................
Some notation here follows \qcite{Hapke93}, sections therein are indicated as
H[x.x] and equations by H(x.x) 
\qi $i$ is the incidence angle from the surface normal. $\mu_0 \equiv \cos i$
\qi $e$ is the emergence (viewing) angle from the surface normal. $\mu \equiv \cos e$
\qi $g$ is phase angle
\qi $\psi$ is the azimuthal angle between the planes of incidence and emergence 
\\ Other symbols
\qi $I$ radiance, usually in the emergence direction
\qi $J$ irradiance, usually the incident power per unit are normal to the incidence
\qi $A_K$ the traditional KRC input albedo ``ALB''.

 ``TOI'' means ``Table of Integrals'' \qcite{Dwight61}

\subsubsection{KRC needs} %..........................

KRC needs two types of albedos; the fraction of collimated light hitting a
surface at local incidence angle $i$ that is reflected, and the fraction of
diffuse (presumed isotropic) light relected by a surface. KRC deals with energy,
so all photometric terms are assumed to represent the bolometric value, as
weighted by the solar spectrum.

Compute the absorbed direct, diffuse and bounce insolation in TLATS where it has
been done.  For thermal models, need an \textbf{absorption} photometric
function; equivalent to $1-A_h$ H[10.D].  KRC does not need the full BRDF.


\subsubsection{Thrashing with Hapke} %..........................

BRDF == Bidirectional reflectance distribution function, H[10.B]. Commonly a
function of the incidence angle relative to the surface normal, $i$, $e$ and
$g$.

$A_h$ == Directional-hemispherical reflectance (or hemispherical reflectance) is
the integral of the BRDF over all viewing directions. Equivalent to the ratio of
the total power scattered into the upper hemisphere to the collimated power
incident on the same area, H[10.D.2]. This can be a function of incidence
angle.

$A_s$ == Bi-hemispherical reflectance (or spherical reflectance) is the
reflectance of a surface under diffuse illumination, H[10.D.4]. It has no
geometric dependence.

Prior to KRC v3.3, all albedoes were considered to be Lambertian.  



In general, Hapke uses $r$ for ``reflectance'', however, there are many of them.

A potential source of confusion is the KRC treats ``albedo'' as the reflected
fraction of power incident on a unit area; this incident power is $J \cos i$
where $i$ is the incident angle relative to the local surface normal. Thus,
``albedo'' has no units. However, Hapke generally uses reflectance as the
fraction of incidence irradiance that is reflected, which is different by a
factor of $\cos i$ .

He defines $r$, the bidirectional reflectance, at the bottom of page 261 by:
``The incident radiant power [collimated] per unit area of surface is $J \mu_0$,
and the scattered radiance is $Jr(i,e,g)$, where $r(i,e,g)$ is the bidirectional
reflectance of the surface.'' [10.B]

The ratio of the reflectance of a surface to that of a perfectly diffuse surface
under the same conditions is $\frac{\pi}{\mu_0}r(i,e,g)$ H(10.3)

Also: H[8.E.1] starts with ``the scattered radiance $I(i,e,\psi)
=Jr(i,e,\psi)$.
The scattering of light from a (planar) surface is in general described by its
bidirectional-reflectance distribution function(\textbf{BRDF}) or
$r(i,e,g)/\mu_0$ H(10.1).

Whatever the photometric function, it should obey reciprocity.

The incident power (no atmosphere) on a surface is $\mu_0 J = S_m \cos i $
\\ The reflected fraction is $A_h(i)$ and the absorbed power is $S_m \cos i \left(1-A_h(i) \right)$


In Hapke terminology of his Table 8.1, the two albedos KRC needs are the:
\qi  $r_h \equiv r_{dh}$ or ``hemispherical albedo'' or hemispherical reflectance, =$\mathbf{A_h}(i)$ H(10.33) 
\qi spherical reflectance, his $r_s$, my $\mathbf{A_S}$. 

I found Hapke to have many similiar reflectance terms that are not needed 
here, and the normalization is obscure.  Here, I will simply treat the
reflectance $r(i,e,\psi \ \mathrm{or} \ g)$ as the relative radiance into
direction $e,\psi$ for an irradiance from direction $i$; and heuristically
determine the normalization required by KRC.

H[10.D.2] has ``The general expression for hemispherical reflectance is
$r_h(i)= \frac{1}{\mu_0} \int_{2 \pi} r(i,e,g) \mu \ d\Omega_e $ H(10.10) where
$d\Omega_e = \sin e \ d e \ d \psi$ .

 H[10.D.4] has ``the spherical reflectance is $r_s= \frac{1}{\pi} \int_{2 \pi}  \int_{2 \pi}  r(i,e,g) \mu \ d\Omega_e  \ d\Omega_i $ H(10.21)

\subsubsection{Procedure here} %.........................................
We shall (must) assume that the planar surface has no preferred orientation;
that is, its reflectance is invarient under rotation of the surface abouts its
normal relative to the scattering plane.

Apparently what Hapke means by $\int_{2 \pi} d\Omega_e$ includes a normalization
of $1/2\pi$, although I did not find where he stated this.  Here the
hemipherical and spherical albedo will be used to establish any normalization
factors needed by KRC.

If $ r(i,e,\psi)$ does not involve $\psi$ (or $g$), which is true for the simple
photometric functions considered here, then the integration over $\phi$ or $\xi$
simply results in a factor of $2 \pi$

Desire that the ALB used in KRC retain its historic meaning of the fraction of
energy reflected by the surface. By conservation of energy, this can never
exceed unity, so must scale some of the results here for $A_h(0)$ and $A_s$. See
table in \S \ref{all},

% \pagebreak
% \large
\subsubsection{Lambert}
  A Lambertian surface by definition has the same radiance when viewed from any
  direction, and that radiance scales with $\cos i$. Thus $A_h$ must be
  independent of $i$. However, here develop the mathmatical formalities to be
  used for other photometric function.

$ r(i,e,\psi)=A_L \frac{\mu_0}{\pi}$ \ H(8.13)

Hemispherical:
\qbn A_h(i)= \frac{1}{\mu_0}  \int_{\psi=0}^{2\pi} \int_{e=0}^{\pi/2} r(i,e,\psi) \cdot \cos e \ \sin e \ de \ d\psi \ql{Ahr}

\qbn A_h(i) = \frac{2 \pi}{\mu_0} \int_{e=0}^{\pi/2}  r(i,e,\psi) \cdot \cos e \ \sin e \ de \ql {gah}

\qb  = \frac{2 \pi}{\mu_0} \int_{e=0}^{\pi/2} A_L \frac{\mu_0}{\pi} \cdot  \cos e \ \sin e \ de \qe

\qb =  \frac{2 \pi}{\mu_0} A_L \frac{\mu_0}{\pi} \int_{e=0}^{\pi/2} \cos e \ \sin e \ de  \qe

\qb = 2 A_L \left[_{e=0}^{\pi/2} \  \frac{\sin^2 e}{2} \right| 
   = 2 A_L \left[ 1/2 -0 \right| = A_L \ \ \qe 
% \mathrm{using \ TOI \ 450.11}

Spherical:

\qb A_s= \frac{1}{\pi} \int_{2 \pi}  \int_{2 \pi}  r(i,e,g)  \cdot \cos e  \ d \Omega_e  \ d \Omega_i \qe

\qbn A_s= \frac{1}{\pi} \int_{i=0}^{\pi/2}  \int_{\psi=0}^{2\pi}  \int_{e=0}^{\pi/2} \int_{\xi=0}^{2\pi} r(i,e,\psi) \cdot \cos e   \sin i  \sin e  \  d\xi  \ de \ d\psi \ di \ql{Asr}
 
\qbn = 4 \pi^2\frac{1}{\pi} \int_{i=0}^{\pi/2} \int_{e=0}^{\pi/2} r(i,e,\psi) \cdot \sin i \ \cos e \sin e \ de \ di \ql {gas}
 
\qb = 4 \pi  \int_{i=0}^{\pi/2}  \int_{e=0}^{\pi/2} \frac{ A_L}{\pi} \cos i \cdot \sin i \ \cos e \ \sin e \ de \ di  \qe

\qb = 4 \pi  \frac{ A_L}{\pi} \int_{i=0}^{\pi/2} \left[_{e=0} ^{\pi/2}  \ \frac{sin^2 e}{2} \right]  \cos i\sin i \ di  \qe

\qb = 4 \frac{ A_L}{2} \left[_{i=0}^{\pi/2} \  \frac{sin^2 i}{2} \right] = A_L 
\mc{;}  \frac{A_S}{A_h(0)}=1.  \qe

Can use the forms of \qr{gah} and \qr{gas} for other photometric functions.

Comparing \qr{gah} and \qr{gas}, find
\qbn A_s= 4 \pi \int_{i=0}^{\pi/2}   \frac{\mu_0}{2 \pi} Ah(i) \cdot \sin i  \ di =  2 \int_{i=0}^{\pi/2}  Ah(i) \cdot  \cos i \ \sin i  \ di \ql{Ahs} 

Confirm by substitution for Lambert, Minnaert, Lommel-Seliger. 

\subsubsection{Minnaert}  
 $ r(i,e,\psi) =A_M \mu_0^\nu \mu^{(\nu-1)}$  \  where: $0<\nu \le 1$ H(8.14)
\qi When $\nu=1$ , $r=A_M \mu_0$ is Lambert behaviour and $A_M=A_L/\pi$ 
\qi Minnaert breaks down at limb, where $\mu=0$

\qb A_h(i) = \frac{2 \pi}{\mu_0} \int_{e=0}^{\pi/2} A_M \cos^\nu i \cos^{\nu-1} e \cdot  \cos e \ \sin e \ de \qe

\qb A_h(i) =  A_M \frac{2 \pi}{\mu_0} \left[_{e=0}^{\pi/2} \ - \frac{\cos^{\nu+1}e}{\nu+1} \right| \cos^\nu i \qe

\qb A_h(i) =  A_M \frac{2 \pi}{\mu_0} [ \frac{1}{\nu+1}] \mu_0^{\nu}  
=  A_M \frac{2 \pi}{(\nu+1) } \mu_0^{\nu-1}  \qe

Spherical:

\qb A_s = 4 \pi  \int_{i=0}^{\pi/2}  \int_{e=0}^{\pi/2}  A_M \cos^\nu i  \cos^{\nu-1} e \cdot \sin i \ \cos e \ \sin e \ de \ di  \qe

\qb = 4 \pi  A_M  \int_{i=0}^{\pi/2} \int_{e=0}^{\pi/2} \cos^\nu i \sin i \ di \ \ \cos^\nu e  \sin e \ de  \qe

\qb = 4 \pi  A_M  \int_{i=0}^{\pi/2} \left[_{e=0}^{\pi/2} \ - \frac{\cos^{\nu+1}e}{\nu+1} \right| \ \cos^\nu i  \sin i \ di  \qe

\qb = 4 \pi  A_M   \left[ \frac{1}{\nu+1} \right]  \left[_{i=0}^{\pi/2} \ - \frac{\cos^{\nu+1} i}{\nu+1} \right| \qe

\qb = 4 \pi  A_M   \left[ \frac{1}{\nu+1} \right]  [\frac{1}{\nu+1}] = A_M \frac{ 4 \pi }{(\nu+1)^2} \qe

\qbn P_f \equiv A_h(0)/A_M  = \frac{2 \pi}{(\nu+1) } \mc{;}  P_s  \equiv \frac{A_s}{A_h(0)}= \frac{2}{\nu+1} \ql{aam}

\begin{figure}[!ht] \igq{minnHemi}
\caption [Minnaert $A_s$]{Spherical albedo for Minnaert reflectance.
\label{minnHemi}  minnHemi.png }
\end{figure} 
% how made: hemialb 46 48

\subsubsection{Lommel-Seeliger} 

 $r= $ Lambert $/ 4(\cos i + \cos e) $, H(6.11) and H(6.12), 
thus $r= \frac{A}{4 \pi} \frac{\cos i}{\cos i+ \cos e} $

Hemispherical:
\qb A_h(i) = \frac{2 \pi}{\mu_0} \int_{e=0}^{\pi/2} \frac{A}{4 \pi} \frac{\cos i}{\cos i+ \cos e}  \cdot  \cos e \ \sin e \ de \qe

\qb  = \frac{2 \pi}{\mu_0} \frac{A}{4 \pi} \cos i \int_{e=0}^{\pi/2} \frac{ \cos e}{\cos i+ \cos e} \ \sin e \ de \qe

\qbn  = \frac{A}{2 \mu_0} \cos i \int_{e=0}^{\pi/2} \frac{ \cos e \ \sin e  }{\cos i+ \cos e} \ de \ql{LSAha}

\qb  = \frac{A}{2} \left[_{e=0}^{\pi/2} \  \cos i \ln ( \cos i + \cos e )- \cos e  \right| \qe

\qbn A_h(i) = \frac{A}{2} \left( \cos i \ln\frac { \cos i}{\cos i+1} +1) \right)  
 \ = \frac{A}{2} \left(\mu_0 \ \ln \frac{\mu_0}{ 1+ \mu_0} +1 \right) \ql{LSAh}

Spherical:

\qb A_S= 4 \pi  \int_{i=0}^{\pi/2}  \int_{e=0}^{\pi/2} \frac{A}{4 \pi} \frac{\cos i}{\cos i+ \cos e} \cdot \sin i \ \cos e \ \sin e \ de \ di  \qe

\qb = 4 \pi \frac{A}{4 \pi}  \int_{i=0}^{\pi/2} \cos i \sin i \ \int_{e=0}^{\pi/2} \frac{\cos e \ \sin e }{\cos i+ \cos e}\ de \ di  \qe

\qb =A  \int_{i=0}^{\pi/2} \cos i \sin i  \left[_{e=0}^{\pi/2} \  \cos i \ln ( \cos i + \cos e )- \cos e  \right| \ di  \qe

\qbn =A  \int_{i=0}^{\pi/2} \cos i \sin i   \left[  \underbrace{1}_1 +  \underbrace{\cos i \ln \cos i}_2 - \underbrace{\cos i \ln (1+ \cos i) }_3 \right] \ di  \ql{LSAs}
Numerical integration yields $A$ 0.20483 . Using Wolfram integral solver for each part, many opportunities for a blunder: 
% \pagebreak

\qb A_s=A \left[_{i=0}^{\pi/2} \ \underbrace{\frac{\sin^2 i}{2} }_1 
- \underbrace{\frac{1}{9} \cos^3 i \ (3 \ln(\cos i) -1 ) }_2   \right. \qe
\qb \left. - \underbrace{\frac{1}{36} \left( -3 \cos 2i + \cos 3i -6 \ln \cos \frac{i}{2} + \cos i ( 15-9 \ln(1+\cos i) )  -3 \cos 3i \ln( 1+\cos i) -9 \ln(1+\cos i) \right) }_3  \right| \qe
 \qbn A_s=  0.20457 A \ql{LSAa}
\qb A_s=A \left( \underbrace{ \frac{1}{2} -0 }_1 
- \underbrace{\frac{1}{9} [ 0-(-1 (3 \cdot 0 -1) ]  }_2  \right. \qe
\qb \left.  - \underbrace{\frac{1}{36} [ -3(-1-1) +(0-1)-6(\ln \sqrt{2} -0) +(0-1(15-9 \ln2)) -3(0- \ln 2)-9(0 - \ln2) }_3 \right)  \qe
   Yields 0.542315 A, some blunder in part 3  above.

\qbn  P_f \equiv A_h(0)/A  =\frac{\ln (1/2) +1 }{2}=0.153426  \mc{;}  P_s  \equiv \frac{A_s}{A_h(0)}=1.3333333 =4/3  \ql{aal}

\begin{figure}[!ht] \igq{LomSee}
\caption[Lommel-Seeliger $A_h$ ]{Hemispherical albedo for Lommel-Seeliger reflectance, and the integrand for $A_s$ 
\label{LomSee} LomSee.png }
\end{figure} 
% how made: hemialb 48 49   

\subsubsection{Lunar-like}
\qcite{Keihm84} does not list a BRDF, but has hemispherical albedo (his equation A5): 
\qb A(\theta)= 0.12+0.03 \left( \theta/45\right)^3 + 0.14 \left( \theta/90\right)^8 \qe

\qcite{Vasavada12} measurements with Diviner led to the form: 
\qb A(\theta)= A_0 +0.045 \left( \theta/45\right)^3 + 0.14 \left( \theta/90\right)^8 \qe

In both cases $\theta$ in degrees is equivalent $i$ in radians. See Appendix \S
\ref{LA} for background.

Implimentation in KRC 34 is a slightly more generalized form:

 \qbn A_h(i)= A \left( 1. + x \left( \theta /45\right)^3 + (0.14/0.12) \left( \theta /90 \right)^8 \right) = A \left( 1. + x \underbrace{ (\frac{4}{\pi})^3}_{f3} i^3
+  \underbrace{ \frac{0.14}{0.12}( \frac{2}{\pi})^8}_{f8}  i^8 \right)   \qen

Thus $x=0.25$ yields Keihm and  $x=0.375$ yields Vasavada for $A=0.12$.

\qb A_s=  2 A  \int_{i=0}^{\pi/2}\left( 1. + x f_3 i^3 + f_8 i^8 \right) \cos i \sin i \ di 
 = A \left[ 1. + 2f_3 D_3 \ x \ + 2f_8 D_8 \right] \qe 

where  (Wolframalpha.com)

$\int x^3 \cos x \sin x \  dx = \frac{3}{8} (2 x^2-1) \sin x \cos x -\frac{1}{8}(2 x^2-3) \cos(2 x)+constant$ and 
$ [_0^{\pi/2} = \frac{1}{32} \pi ( \pi^2-6) \sim 0.37989752  \equiv  D_3 $

$ \int x^8 \cos x \sin x \ dx = \frac{1}{4}(4 x^6 -42 x^4 +210 x^2 -315) \sin(2 x)-\frac{1}{8} (2 x^8 -28 x^6 +210 x^4 -630 x^2 +315) \cos (2 x)+constant$ 
and $ [_0^{\pi/2} = \frac{80640-20160 \pi^2+1680 \pi^4-56 \pi^6+\pi^8}{1024} \sim 0.944125 \equiv  D_8 $

\qbn A_s=  A \left(1.0 + 1.5683 \ x \ + 0.05944\right) \ql{KAs}

\qbn P_f \equiv A_h(0)/A = 1  \mc{;} P_s  \equiv  \frac{A_s}{A_h(0)}=  1.05944 + 1.5683 \ x \ql{aak}

% 1.5682916  + 0.059435709
 See Figure \ref{KeihmAh} 
 
\begin{figure}[!ht] \igq{KeihmAh}
\caption[Lunar-like albedo]{Lunar-like hemispherical albedo.
\label{KeihmAh} KeihmAh.png }
\end{figure} 
% how made: hemialb 48 50 

\small

The following coded in IDL but not needed; eventually good agreement between
analytic and numeric integration.

\qbn A_s=  2 A \left[_{i=0}^{\pi/2} \ \underbrace{\frac{\sin^2 i}{2}}_{p1} 
+  x \underbrace{ \overbrace{\frac{64 }{\pi^3}}^{f3} \cdot  
\left[ \overbrace{ \frac{3}{8}(2 i^2-1) \sin i \cos i }^{b21}
- \overbrace{\frac{i}{8} (2i^2-3) \cos 2i }^{b22} \right] }_{p2}  \right. \ql{LLAs}
\qb  \left. + \ \underbrace{ \overbrace{\frac{0.14}{0.12} \frac{ 256 }{ \pi^8}}^{f8}
 \cdot \left[  \overbrace{\frac{i}{4} ( 4i^6-42i^4+210i^2-315) \sin 2i}^{b31} 
- \overbrace{\frac{1}{8}( 2i^8 -28i^6 +210i^4-630i^2+315) \cos 2i}^{b32} 
 \right]  }_{p3} \right| \qe 

\vspace{-3.mm} 
\begin{verbatim}
in hemialb.pro @ 46,48
f3,f8:       2.0640982       0.031476598
sum3,8=      0.38127627      0.95209557
D3,D8=       0.37989752      0.94412538
qsum=        0.72721736      0.82559132
        i          b21        b22          b31      b32        2A factor
   1.5707963   0.0000000  -0.3798975   0.0000000  38.4308746  -0.5136367
   0.0000000  -0.0000000  -0.0000000  -0.0000000  39.3750000  -1.2393910
   9.0351430e-17     -0.37989752   3.6271733e-16     -0.94412538      0.72575430
del P2,3      0.37989752      0.94412538
fun3,8=      0.38127627      0.95209557
P1              FLOAT     =      0.500000
P2              DOUBLE    =       0.78414580
P3              DOUBLE    =      0.029717855
final terms:       1.0000000       1.5682916     0.059435709
Using x=     0.250000     0.375000
total factor       1.4515086       1.6475451
\end{verbatim} 

If wish to try a lunar-like in the form $A ( 1+ b i^a)$,   
integral:  $x^a \sin x \cos x dx = -2^{-a-3} x^a (x^2)^{-a} ((-i x)^a \Gamma(a+1, 2 i x)+(i x)^a \Gamma(a+1, -2 i x))$
where $\Gamma(a,x)$ is the incomplete gamma function.

But using $(\cos i)^a$ would be easy:  $\int  \cos^a x \sin x \cos x dx = \int \cos^{a+1} x \sin x dx =  -\frac{\cos^{a+2} x}{a+2} +constant $

Possible form \qbn A \equiv r_h = \frac{a}{1+\frac{1-b}{b} \mu^c} \ql{hemi}
$a$ is the albedo for grazing incidence, $b$ the backscatter ratio (Albedo at zenith / albedo at horizon) and $c$ is a sharpness factor;
\qi $c=1$ is linear from zenith to horizon 
\qi $c=$ small drops off quickly at the horizon
\qi $c=$ large rises quickly at zenith
\\ All these have the non-physical propery of discontinuous 2nd derivative if the Sun gets to the zenith.

\normalsize

\subsubsection{All \label{all}} 
 The following table lists $A_h$ and $A_s$ factors for several reflectance types.
\qi yf is $A_h(0)$
\qi ys is the spherical albedo based on $A_h(i)$
\qi ys/yf is the spherical albedo based on $A_h(i)$ with $A_h(0)$ scaled to unity.
\qii values greater than 1.0 are non-physical.
\vspace{-3.mm} 
\begin{verbatim}
 yid=  Lambert Lomm-See    Kheim     Vasa Minnk0.3 Minnk0.5 Minnk0.7 Minnk0.9
  yf=  1.00000  0.15343  1.00000  1.00000  4.83322  4.18879  3.69599  3.30694
  ys=  1.00000  0.50000  1.92259  2.28435  7.43572  5.58505  4.34823  3.48099
ys/yf  1.00000  3.25889  1.92259  2.28435  1.53846  1.33333  1.17647  1.05263
\end{verbatim} 

Each of these relations has hemispherical albedo decreasing as insolation
becomes more oblique; Lommell-Seeliger is close to $cos^{0.3} i$ from 0 to
45\qd; see Fig. \ref{hem55n} and \ref{hem55c}. $A_h(i)$ increases with $i$ for
all models here. All but Lambert and high-k Minnaert have the property of
becoming larger than 1 at high incidence angles, so that the absorped power can
become negative for reasonable values of ALB, see Figure \ref{hem55a}. The only
practical way to prevent such a creation of energy for all photometric models is
to invoke a lower limit of 0 on adsorbed insolation; TLATS restricts $0 \le A_H \le 1$ .

\begin{figure}[!ht] \igq{hem55n}
\caption[Normalized Hemispherical Albedo]{Hemispherical albedo as a function of incidence 
angle for several photometric models; see legend. Each curve normalized to the 
value at normal incidence. % Values above unity are non-physical
\label{hem55n} hem55n.png }
\end{figure} 
% how made: q.pro @ 46 48 55
\begin{figure}[!ht] \igq{hem55c}
\caption[Hemispherical Albedo]{Same as Fig. \ref{hem55n} but with abscissa being $\cos i$. 
\label{hem55c} hem55c.png }
\end{figure} 
% how made: q.pro @ 46 48 55

\begin{figure}[!ht] \igq{hem55a}
\caption[Absorbed power]{Absorbed power for ALB=.2 as a function of incidence
 angle for several photometric models; see legend. 
Values are $\cos i \ (1.-0.2 A_h(i)/A_h(0)$ .
\label{hem55a}  hem55a.png }
\end{figure} 
% how made: q.pro @ 46 48 55

\subsection{Code in KRC} %...................
Version 3.4 does not allow photometric functions when there is an atmosphere
because of input parameter overloading.

The use of $A_h$ and $A_s$ is shown in \S \ref{fffd}

SALB is $A_s$ and is a constant for a case.

Frost is always considered Lambertian

Each photometric function is normalized to unity at normal incidence, so that
the midday temperatures for low thermal inertia would be the same. The same
scaling is applied to the spherical albedos. Thus, the absorbed flux, using the
traditional KRC input parameter \nv{ALB}$\equiv A$ :
\qi Collimated: $(1-A P_f) \cos i$ where $P_f = A_h(i)/A_h(0)$ and 
$A P_f$=\nv{HALB=ALBJ(JJ)} for the sloped surface.
\qi Diffuse and bounce: $(1-A P_s)$ where $P_s = A_S/A_h(0)$ = \nv{ASF} and 
$A P_s$=\nv{SALB}

Items in COMMON (see also the comments in TLATS)
\vspace{-3.mm} 
\begin{verbatim}
krcccom ALB               Input albedo
hatccom SALB            ! spherical albedo of the soil
hatccom ALBJ(MAXN2)     ! hemispherical albedo at each time of day
hatccom SOLDIF(MAXN2)   ! Solar diffuse (with bounce) insolation at each time W/m^2 
dayccom ASOL(MAXN2)     ! Direct solar flux on sloped surface at each time of day
dayccom ADGR(MAXN2)     ! Atm. solar heating at each time of day 
\end{verbatim} 

Access to photometric models; in version 3.4.2 only when no atmosphere.
\qi Set PTOTAL to 0.1:  1 12 .1 'PTOTAL'   /
\qi Set photometric model by ARC2:  1 21 x 'ARC2/PHT' /  where x is: 
\qii 0. is Lambert ( the default with an atmosphere)
\qii -1. is Lommel-Seeliger
\qii $-1<x<0$ is Minneart with exponent  $|x|$
\qii   $0<x<1.$ is Lunar-like, with x being the coefficent of $i^3$

Hemispheric albedos for the implimented photometric functions are shown Figure \ref{kv651}  
\begin{figure}[!ht] \igq{kv651}
\caption[Test of Photometric functions]{Hemispheric albedo computed in KRC
  versus cosine of the incidence angle, solid lines, at every time step. Dashed
  lines (largely invisible) are values at every degree computed in IDL
  (hemialb.pro @55). In both cases ALB=0.12
\label{kv651}  kv651.png }
\end{figure} 
% how made: kv3 65 66 651 662

 The effect on surface temperature, relative to a Lambertian surface, is shown in 
Figure \ref{kv572}  
\begin{figure}[!ht] \igq{kv572}
\caption[Effect of Photometric functions]{Effect of various photometric
  functions; low thermal inertia, I=50, and at 60\qd S for Mars orbit but no
  atmosphere. Ordinate is Tsur relative to the values for a Lambertian
  model. Solid lines are Ls=0, dashes Ls=93, dotted Ls=272.
\label{kv572}  kv572.png }
\end{figure} 
% how made: kv3@ 572 for 

The effect at all hours, latitudes, seasons and cases is shown in
Figure \ref{quilt}  
\begin{figure}[!ht] \igq{quilt}
\caption[Effect of Photometric functions]{Effect of various photometric functions;
 low thermal inertia, I=50,  and at the equator for Mars orbit but no atmosphere. 
QUILT3 image of delta temperature from a Lambertian model.  Displayed value
range is: 0.0 to 36.5 K.  Sample is: hour(48) * 5 planes of seas*case.  Line is:
lat(5) * 40 groups of seas*case; Lines increase upward..  Latitudes: -60 -30 0
30 60 .  Season range: 0.1 to 351.4 . From the bottom upward, models are:
Lommel-Seeliger, Kheim, Vasavada, Minnaert 0.3, Minnaert 0.7 .
\label{quilt}  quilt.png }
\end{figure} 
% how made: Beaming @45


\subsection{TLATS: Sequence within the hour-angle loop }
.
\\ If not atm., twilight forced to zero
\\ Compute $\mu_0$ (angle onto flat terrain)
\qi If twilight, adjust $\mu_0$
\\ If slope, compute $i_2$; no consideration of twilight
\qi compute $\alpha$ and $G_1$
\\ Compute hemispheric albedo of the surface, based in [adjusted]  $\mu_0$
\qi If Far, frost albedo is thick-deposit value based on Tfar, as have no frost-amount.
\\ Compute C=Collimated beam and boundary fluxs; use DE if an Atm.
\\ Compute D=Diffuse, which does not depend on slope
\\ Compute B=Bounce flux.
\\ Sum C+D+B at each time

If twilight:
\qi flux onto top of atmosphere unchanged, so atm heating should not be changed
\qi diffuse flux out bottom of atm is extended by cos3
\qi total energy should not change, so need to scale [C+D] by their diurnal sum.


Delta-Eddington always uses the Lambert albedo, otherwise becomes complex
(undefined) to consider the twilight region. Thus, surface photometric fuction
considered only for Collimated beam. However, frost is always treated as
Lambertian.


 Sky factor for a pit with wall slope $s$
\\ Solid angle is 
\qb \int_0^{2\pi} \int_0^\theta \sin x \  dx \ d\phi =2 \pi \left[ -\cos x \right]_0^\theta =2 \pi (1-\cos \theta) \qe where $\theta = 90 -s$ and $x$ is the angle from zenith

But projection onto a horizontal surface has an additional term $\cos i $ in the integrand.  
\qb \int_0^{2\pi} \int_0^\theta \sin x \cos x\  dx \ d\phi =  = 2 \pi \left[ \frac{\sin^2 x}{2} \right]_0^\theta = 2 \pi \frac{\sin^2 \theta}{2} \qe
.  

Absorbed Direct insolation: C = Albedo * PhotoFunc * AtmTrans * SunAtMars 
\qi Albedo: constant for soil, may be variable for frosts;   TDAY
\qi PhotoFunc: cos i for Lambert, several others available;  HALBF and BND2 in TLATS
\qi AtmTrans: compute with DEDING2 for atmospheres, otherwise 1. [or 0 at night]; COLL in TLATS
\qi SunAtMars: $1/AU^2$; TSEAS 


Trying to put all the incidence angle calculations into one DownVIs in TLATS may be asking too much.
\vspace{-3.mm} 
\begin{verbatim} 
ASOL[jj]                    
     &, ASOL(MAXN2)     ! Insolation at each time of day, direct + diffuse
     &, ADGR(MAXN2)     ! Atm. solar heating at each time of day 
Both use LFROST at the start of the season and do not treat a change during a season.
\end{verbatim} 


% See detailed treatment of kv3@7846, still get 0.8 W/m2. 
