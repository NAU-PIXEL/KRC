\documentclass{article} 
\usepackage{underscore} % accepts  _ in text mode
\usepackage{ifpdf} % detects if processing is by pdflatex
\usepackage{/home/hkieffer/xtex/newcom}  % Hughs conventions
% \newcommand{\qj}{\\ \hspace*{-2.em}}      % outdent 1

% 1 of next 2 used in place of  \qen for development to identify equation labels 
\newcommand{\ql}[1]{\label{eq:#1} \hspace{1cm} \mathrm{eq:#1} \end{equation}}
%\newcommand{\ql}[1]{\label{eq:#1} \end{equation} } % for final

\title{Guide to KRC version 3.4, and background }
\author{Hugh H. Kieffer  \ \ File=-/krc/Doc/V34UG.tex  2016sep05}
\begin{document} %==========================================================
\maketitle
\tableofcontents
\listoffigures
%\listoftables
\hrulefill .\hrulefill
% \pagebreak 

\section{Preamble}
This accompanies an alpha release of KRC version 3.4.2 . ASU latest formal
release is version 2.4.1, latest version in the ASU repository is 3.2.1 .

The prior version 3.3 has been abandoned; that alpha release should be ignored.

This document includes both a Users Guide and a detailed description of the development.

Sections \ref{UG} to \ref{invis} are a Users Guide to changes since the prior formal release.
\\ The Users Guide ends with  \S \ref{zform}; most users should not need to read beyond that, where the tone of this document becomes closer to lab notes.  

Sections \ref{devd} to \ref{issu} have some development details.
 
Sections \ref{t1} to \ref{t2} cover testing.

Sections \ref{plans} discusses briefly plans for additional capabilities

The appendix includes a short discussion of tuning KRC for efficiency. 

One intention of including the far-flat and photometric function capabilities was
to address asteroid models, including ``thermal beaming'' . This can be done by
running a large grid of sloped models and post-processing the output files; see
\nf{Beaming.tex}
 

\subsection{Distribution files}
 Organization into directories is the same as other recent KRC distributions. 
Files in the V3.4 alpha distribution: 
\qi All source code and a Makefile
\qi Revised helplist
\qi Updated KRC evolution 
\qi Several example input files for V3.4
\qii Some annotated sample zone tables.
\qii One lengthy file for generating a grid of slopes 
\qi Print file and binary .t52 file generated by input files
\qi Ash.pdf; 2016 January comparison of KRC and Ashwin Vasavada thermal models for Mars
\qi simple.pdf; 2016 Feb. study of convergence and performance of KRC versus a simple, but slow, thermal model
\qi V34UG.pdf; this document, and its source files: \nf{V34UG.tex, farg.tex, fard.tex, v34p.tex}.

\section{Introduction \label{UG}}

This includes a users guide for version 3.4.2 of KRC, which is in alpha release. It covers ONLY THE CHANGES from version 2.4.1, which are:
\qi May have a condensing gas other than CO$_2$. \S \ref{gas} 
 \qi May invoke some Surface Photometric Models when there is no atmosphere. \S \ref{pm}
 \qi May specify Tables of material properties versus depth. \S \ref{zone}
 \qi May include Geothermal heat flow. \S \ref{ghf} 
\qi Optional user-specified ``unique'' output; two specific sets are coded thus far. \S \ref{tun}
\qi Can now write type 52 and a direct-access file at the same time
\qi For sloped surfaces, the far-field solar and thermal radiance can be from a flat surface or the traditional``self-heating''.

Some new default values are: 
\qi TAURAT=0.25: closer to modern estimates of dust properties.
\qi RLAY=1.12, FLAY=0.1, N2=1536, CONVG=3.
\qii these improve the numerical accuracy at a modest cost in speed
\qi N24=48: binary output every 1/2 hour, rather than once per hour.

With the incorporation of zone tables, the two materials specified in the
'ConUp0 ...' and 'SphUp0 ...' lines are no longer necessarily the ``upper'' and
``lower'' materials; they are here referred to as material A (normally Above,
was 'upper') and material B (normally Below, was 'lower')

\subsection{Notation use here}
The following fonts styles have been partially implemented: 
\qi File names are shown as \nf{filename}. 
\qi Program and routine names are shown as \np{PROGRM [,N]} , or simple UPPERCASE
\qii where \np{N} indicates a major control index.
\qii The last character''8'', representing the double precision (8-byte) version , is often omitted. 
\qi Code variable names are shown as \nv{variab} or  \nv{VARIAB} and within equations as $\nvf{variab}$.  
\qi Input parameters are shown as \nj{INPUT} and within equations as $\njf{INPUT}$


\subsection{New routines}
FORTRAN routines:
\\ TFAR \ Open and read KRC type -n direct-access file
\\ CUBUTERP \ Cubic interpolation of uniformly spaced points to higher density
\\  TDISK \ Revised extensively to store/write type 52 and type -n files simultaneously 
\\ FILLMV a set of 12 routines to fill an array or any contiguous part thereof
with a constant or MoVe (copy) a part of one array to another. Arrays are
dimensioned with the size of the move, so the compiler can check
out-of-bounds. Individual routines FILLx and MVx for each data type where x is:
B=byte, I=integer*2, L=integer*4, R=real*4, D=real*8.  Also:
\qi MVDF multiplies each value by a constant factor
\qi MVDA multiplies each value by a constant factor and adds it to the current destination
\qii these last two are handy for linear interpolation, such as in season.

IDL routines:
\\ KRCLAYER \ Compute and print KRC layer depth table from KRCCOM values
\\  NUMGEOMLAY \ Compute number of geometrically thick layers for total depth
\\  FLAYER \ Replicate Fortran KRC TDAY statement function
\\  KV3 \ Check consistency of KRC within and between versions
\qii this has grown almost unmanageable, 2650 lines
\qiii @535 can generate zone table for lunar-like compaction profiles.
\\ HEMIALB Evaluate various photometric functions, especially for hemispheric albedo

\section {User changes needed even if none of the new capabilities are used \label{need}}
\begin{enumerate}    % numbered items 

\item Specification of the saturation point. Real parameters 11 and 32 were
  unused; they now specify the saturation temperature relation for the
  condensing gas. Values for CO$_2$ are 3182.48 and 27.9546. These values are in
  the latest KRC master input file.

\item The 22'nd real parameter, ARC3, was unused. It is now the minimum
  allowable numerical convergence factor, which in theory is 1. The default
  value is 0.801 as KRC has been found to be generally stable to this level. If
  you have any hint of instability, set this to 1.0 (and let me know).

 \item The 7'th integer parameter IIB (was IB). This controls the lower boundary
   condition. Positive values are heat-flow, see \S \ref{ghf}. The flag values
   are now the negative of the earlier meanings. e.g.:
\qi  0 = insulating (as before)
\qi -1 = constant temperature 
\qi -2 = start all layers =TDEEP and keep boundary at a constant temperature 

\item The 14'th logical value, LZONE, is now a flag for use of a zone table. It
  should initially be 'F'; it will be set/reset automatically by specification
  of a zone table file name.

\end{enumerate}

\subsection{ALERT: Layer depths may vary with albedo or emissivity}

In V342, if Temperature-dependent properties are invoked, the scaling to
layer thicknesses in meters is done using the diffusivity computed from the input parameter DENS and the specific
heat and conductivity computed for the global equilibrium temperature, unless
zones defined the layers.  I.e., if LKOFT is True and there is not a zone table,
the layer thicknesses in meters will depend upon Albedo. In V321 and earlier, the
scaling is always based on the input parameters DENS, SPHT and conductivity
derived from INERTIA; and corresponding parameters for the lower material if
used.

\subsection{OnePoint mode}
All the new capabilities have not been thoroughly tested in onePoint mode. The
master onePoint file \nf{Mone.inp} works. The geothermal-flow, non-CO$_2$ gas
and photometric functions are expected to be compatible with onePoint mode, but
use of optional files for zone tables or far-field temperatures is questionable
and has not been tested; see the notes in \S \ref{fffd}

\section{Any condensible gas \label{gas}}

The coefficients for the saturation temperature relation for the condensing gas
are now input parameters. KRC uses the Clausius-Clapeyron relation $\ln P = a -
b/T$ where $P$ is pressure in Pascal and $T$ is temperature in Kelvin. This
should be useful for Titan, Pluto, ...

Real parameter 11, ABRPHA, was unused; it is now SatPrA, the Clausius-Clapeyron
coefficient 'a'. Real parameter 32, fd32, was unused; it is now SatPrB, the
Clausius-Clapeyron coefficient 'b' .  Values for CO$_2$ are a=3182.48 and
b=27.9546.

The proper molecular weight should be input as real parameter 10, AMW. The
Mass-fraction of mean atmosphere that is non-condensing, real parameter 13,
FANON, should also be specified.

\section{Surface Photometric Models \label{pm} }
This is only active if there is not an atmosphere; the Delta-Eddington model
used to handle atmospheric opacity includes the assumption that the surface is
Lambertian. In some later version when more input parameters are available, it
is intended to be allowed when there is an atmosphere

The 21'th real parameter, ARC2, otherwise meaningless when no atmosphere, is defined as follows:
\qi 0 = Lambertian: $\frac{I}{\pi F} = A \cos i$ 
\qi $\leq -1$  = Lommel-Seeliger: $\frac{I}{\pi F} =A \cos i /(\cos i+ \cos e) $
\qiii This is the result for isotropic single scattering.
\qi  $-1< x< 0$ = Minnaert exponent, k=-x: $\frac{I}{\pi F}= A \mu_0^k \mu^{(k-1)}, \ \mu \equiv \cos e, \  \mu_0 \equiv \cos i $
\qiii $k=1$ is same as Lambert
\qi $ 0<x $ = Lunar-like. : $A_h= A \left( 1.+ x (\theta/45)^3+1.17(\theta/90)^8 \right)$
\qiii x= +0.25 [and A=0.12] is \qcite{Keihm84} \ \ \ x=0.375 is  \qcite{Vasavada12}

With incorporation of non-Lambertian albedos; must be thorough in defining just
what is meant by the ``albedo'' input parameter in KRC.  As of version 3.4.1, it
is the hemispheric albedo (solar weighted) for incidence angle of zero; which
simplifies comparisons between KRC runs but places a burden of scaling on the
user of non-Lambertian photometric functions. The  KRC equilibrium temperature
(normal incidence, zero thermal inertia) is thus not affected by the
photometric function.

[Another option might be to normalized to the apparent brightness at some
  standard non-opposition geometry, such as $i=0^\circ,  e=30^\circ$ ]
 
The bolometric hemispherical albedo is computed as a function of incidence
angle.  See \S \ref{Ah} for details.

Curves for various hemispherical albedos are shown in Figure 
\ref{hem46n}  
\begin{figure}[!ht] \igq{hem46n}
\caption[Various hemispherical albedos]{Dependence of hemispherical albedo on
  incidence angles for various models, indicated in legend. All have been
  normalized to unity at normal incidence.
\label{hem46n}  hem46n.png }
\end{figure} 
% how made: 

Development details are in \S \ref{photo}.  

\section{Tables of material properties \label{zone}}
A zone file name is defined by a change card: 8 25 x 'fileName' / where the 3rd column (x) is
ignored. The zone file may have up to 20 lines of comments before a ``C_END''
line. Thereafter each row defines a zone and must contain 4 columns (additional
columns are allowed but not read):
\qi Col 1: thickness, m
\qi Col 2: density, kg/m$^3$ \hspace{1.0 cm} However
\qiii -1 = use DENS [real parameter 8] the material A density
\qiii -2 = use DENS2 [real parameter 5] the material B density
\qi Col 3: Conductivity, SI Units. \ $\lceil$   If negative then col 4 is a pointer
\qi Col 4: Specific heat, SI Units \ \  $\lfloor$  1=material A , 2=material B 
\\ There must be at least 3 zones [otherwise the two materials within the current input parameters would be adequate]
\\ The thickness of the last zone is ignored; KRC will fill out to the number of geometric layers specified by N1 [integer parameter 1] 
\\ A line with column 1 non-positive will stop reading. E.g., `` 0 0 0 0 '' . This may be followed by any number of comment lines.

KRC will be expand the zones into an appropriate number of layers to approximate
the RLAY and FLAY relation [real parameters 32 and 33]. If the table contains
zones that are near the implied geometric progression; that exact thickness
value will be used.  If a zone definition violates the convergence stability
requirements or the number of layers specified by N1, the case will be skipped
with a warning.

If the second debug parameter is set to 3, a detailed list of layer generation will be printed. 

To cancel a zone table, specify a file path-name containing a single character.

Development details are in \S \ref{devd}.

\section{Geothermal heat flow \label{ghf}}

The 7'th integer parameter IIB (was IB): positive values are the geothermal heat
flow in milli-Watts/m$^2$. Zero and negative integers set other lower-boundary
conditions ( see \S \ref{need})  

Development details are in \S \ref{hfd}.

\section{Multiple output files}
KRC can now write type 52 files and one of the direct-access file types at the same time. Either or both can be changed before any case.

A type 52 file is initiated by defining its name: 
\vspace{-3.mm} 
\begin{verbatim}
8  5 0 '<file>.t52' / Initiate multi-case output file of temperatures and other values 
\end{verbatim} 

A type 52 file can be terminated by setting the file path-name to less than 4
characters; e.g., 'off'. A direct-access file opened for write is automatically
closed after one case.

\section{Far-field for sloped surfaces} 



Prior to version 3, KRC with a sloped surface assumes that the far ``ground''
was at the same temperature as the slope, a model called in the
literature``self-heating''. This assumption becomes increasingly non-realistic
with increasing slope, and with slopes oriented East or West. Version 3.4 allows
the far-field ground temperature to come from any prior model for which an
appropriate direct-access file was saved, normally a zero-slope model of the
same thermal properties; this is termed the ``far-field file'' or fff.

To create a type -3 model, set K4OUT=-3 and include the input line:
\vspace{-3.mm}
\begin{verbatim}
8 21 0 '<file>.tm3' / Direct-access output file for far-field
\end{verbatim}
This may be the first case of a run. This can be in addition to saving a type 52
file. If there is no atmosphere (PTOTAL less than 1.), may use type -2 or -1,
with the corresponding changes to K4OUT and the file extension.

To invoke use of a ``far-flat'' model, include the input line:
\vspace{-3.mm}
\begin{verbatim}
8 3 0  '<file>.tm3' / Direct-access file to read for far-field
\end{verbatim}
Because the type of file (-1,-2 or -3) is stored in the file (as K4OUT in
KRCCOM), the current value of K4OUT is ignored. Thus, it is possible to save a
different type of direct-access by setting K4OUT as desired for output.


To revert to ``self-heating'', include an input line with full path name less than 4 characters; e.g. 'off'
\vspace{-3.mm}
\begin{verbatim}
8 3 0  'off' / Revert to self-heating for sloped cases
\end{verbatim}

Note: For change type 8, the third argument, 0 in the above examples, is ignored.

The latitudes in a fff must be within 0.1\qd of those of the slope run. The
seasons of fff must include the range of the slope run. The number of hours,
N24, need not match, TLATS will do cubic interpolation if needed.


\subsubsection{How things work:  short file names as 'off'}
Zone table: TCARD detects name length, sets LZONE true if length is 4 characters or more, else sets it false.

 Far-field file: TCARD does not look at the name length. Any write should be closed after each case.
\qi  KRC calls TFAR(1  inside the case loop if the name length is $>3$
\qi  If Far-field invoked, N2 will be limited to the values of MAXFF (firm-code as 6144)

Although TFAR can handle type -1 to -3 files, the far-field algorithm requires
type -3 if there is an atmosphere; TSEAS will check for this, and an error will
return code 41.

To terminate writing a type 52 file or reading a fff, define a new name of less
than 4 characters lengths, such as 'off'; to change to a new file, enter its
name (4 characters or more).

fff's are written as type -1,-2, or -3 by TDISK; they are read by
TFAR. TDISK can read type -n files, but it is not used for that.

KRC will close any direct-access file open for write at the end of a case
and turn off further writes to such files until a new direct-access file name is
read.

A fff may be left open for read for multiple cases

All type -n files contain KRCCOM with the values when the file being written was
closed. Because such files are closed at the end of each case, the value of
K4OUT must be the same as when the file was opened for write, and thus proper
for the file.

Development details are in \S \ref{fffd}.


 %<<<<<<<<<<<<<<<<<<<<<<<<<<<<<<<<<<<<<<<<<<<<<<<<<<

\section{Special output \label{tun}}
The ``unique'' routine TUN8 is designed to address potential special requests
for data output. It writes lines of text to 'fort.77', which must be renamed
after a run to avoid being overwritten. There are currently two uses coded into
the KRC system; more can be added. 

TUN8 has two required arguments:
\qi Arg. 1 (integer) is a code that determines what to print  
\qii existing: 101 writes temperatures for each layer.
\qii existing: 102 writes a number of radiation items. 
\qi Arg. 2 (integer) is a stage:
\qii 1 writes case count and expected sizes 
\qii Else: writes a line of data
\qi Two addition arguments are available to the user.
\qii Arg. 3, Integer
\qii Arg. 4, Real*8

Sections of code within TUN8 could write anything that is in the COMMONs.

Activity is controlled by integer parameter 15, I15. To avoid conflict/confusion 
with older uses of this parameter, only values greater than 100 activate TUN8.

Existing calls to TUN8:
\\ For any value I15 of 100 or more: KRC will call: TUN8 (I15,1) for each case. Existing code:
\\ 101: TDAY will write call TUN8(I15,2,IH,(ignored)) at each output 'hour' on last day of every season of JDISK or more 
\\ 102: TLATS will call TUN8(I15,2,I,SOLDIF) after TDAY done for every latitude for every season of JDISK or more.    

\section{Invisible changes \label{invis}} 
\begin{itemize}      % ticked items 

 \item Layer stability when using T-dependent materials: need stability from
   midday equator to winter poles. For low-obliquity or airless bodies this
   could be large range in temperature.

Because KRC design has the same layer structure for all latitude and seasons, a
practical compromise is to use the global equilibrium temperature: $ S_o
(1-A)/(2 \pi U^2)= \epsilon \sigma T_g^4$ .

Layer thickness prior to V3.3 was based on the T-constant properties even if
using T-dependent materials.  V 3.4 uses the appropriate materials for each
layer at $T_g$

Note: Basing layer on $T_g$ properties means that layering will change with ALB
and EMISS for T-dependent materials.

Layer table generation was moved into TDAY to accomplish this even with
intermixed T-con and T-dep materials from zone tables.

\item  A third value of the TDAY input argument is used to print a layer table.

\item  The logic for page-header lines in the print file changed slightly.

\end{itemize}

\section{Zone table Details \label{devd}} %--------------
\subsubsection{Format \label{zform}}
Zone table format: 
\\ Up to 19 lines of comments are permitted before a required lines starting exactly as ``C_END''
\\ Zones specified from surface down. Columns white separated.
\\ Each line specifies a zone of uniform material properties.
\qi Col 1: zone thickness, m. \  Value for last row ignored; KRC fills out layers
\qi Col 2: density, kg/m$^3$.   \  -1=A material DENS,   -2=B material DENS2
\qi Col 3: Conductivity, SI Units. $\lceil$ \ If negative, then col 4 is a pointer
\qi Col 4: Specific heat, SI Units. $|$ \ 1,2=Tcon: 1=A material, 2=B material
 \\. \hspace{4.cm} $\lfloor$ \ 3,4=Tdep: 3=A material, 4=B material
\\ 3 or more valid lines are required. Comments may follow the 4 numeric columns
\\ The thickness in the last valid line is ignored. KRC will fill out layers up to N1. 
\\ A non-positive first column (of 4 numeric columns) stops the input; Required.
\qi Any number of zone lines or comments may follow the stop line

KRC will convert a zone to multiple layers as allowed by stable convergence.

Numerical stability: $ (\Delta t / (\Delta Z)^2) \kappa < 1/2  \mc{or} C_s \equiv B / \sqrt{2  \kappa \Delta t}$
\qii If definition of a zone has convergence safety factor $C_s <$ ARC3, KRC will print a layer table and quit that case.
\qii $C_s$ is an estimate of how many layers could be used in this zone.

\subsubsection{Modifications to TDAY: Logic flow }
The potential to have more than two types of material required a major change in the organization of TDAY. 

\vspace{2.mm}
\large  V3.4 logic flow for the TDAY \textbf{setup} call: \normalsize
\\ - Compute global equilibrium temperature $T_g$.
\\ - Compute properties of T-dep materials at $T_g$.
\\ - Calculate a table of the sum of normalize layer thicknesses using RLAY and FLAY.
\\ - If zone table active, LZONE true, read a zone table.
\\ - If LZONE is true, process the zone table from top to bottom, expanding each
zone into the number of layers which is the closest fit to maintaining the RLAY
sequence. Generate layer-arrays for conductivity $k$, density $\rho$, and
specific heat $C_P$.  Write each layer to the print file. Generate the virtual
layer using the properties of the first physical layer.
\qi - If number of layers exceeds N1, print message
\\ - If LZONE is false, generate the same 3 layer-arrays as above.
\\ - Set a logical flag LALCON true if and only if all layers are T-con.
\\ - If LALCON false, compute first and number of layers in the T-dep zones.
\\ - If LALCON false, print a table of T-dep material properties versus temperature.
\\ - If using 2 materials, check the convergence factor of the top layer of the lower material; increase its thickness if required for stability.
\\ - Compute the center depth and the convergence safety factor for each layer.
\\ - Compute the minimum safety factor at each layer for itself and all lower layers.
\qi - If minimum factor less than allowed, Set return flag to skip this case and print message.
\\ - Compute time-doubling allowed and the resulting factors: $F_{1_i}, \ F_{2_i}, \ F_{3_i}, \ F_{B_i}, \ F_{C_i}$. See \S 3.2 of \qcite{Kieffer12}
\\ - Build the doubling-depth comb.
\\ - If using geothermal heat-flow, set the flag LGHF true and compute the lower-boundary temperature offset.
\\ - Print the layer table.

\vspace{2.mm}
\large V3.4 logic flow for the TDAY \textbf{season} call: \normalsize
\\ - Compute constant factors; set atmosphere and frost flags.
\\ - Loop on days
\qi - Loop on time-steps
\qii - Set the lower boundary condition
\qii - Loop on layers
\qiii - If LALCON true, $\Delta T_i$ for all layers in the comb  
\qiii - If LALCON false, compute $k_i$ and $C_{Pi}$ for any layers in the first and second T-dep zones. 
\qiiii Then compute  $F_{1_i}, \ F_{3_i}$ and $\Delta T_i$ for all layers in the comb   
\qii - Apply  $\Delta T_i$ to all layers in the comb 
\qii - Satisfy the upper boundary condition. Adjust atmosphere temperature and frost amounts as needed.
\qii - If on an output hour, save values.
\qi - Check if next day could be the last of the season
\\ - If this is the last day; save values and return


\vspace{2.mm}
\large  Detailed printouts: \normalsize
 \\ to monitor: 
\qi As zones are read: e.g.,  K,IH,LALCON=           1           0 T
\qii K is zone (row in table) count
\qii IH is code set by column 3. 0 means was an actual conductivity
\qiii + 1,2,3,4 are codes for A/B material and Tcon/Tdep
\qii LALCON is current value of ``all layers thus far are Tcon''
\qi As layers are created: e.g.,  Zone,I,II,ZBOT,DBOT= 22  1  2     0.79880    35.60257
\qii Zone is zone (row in table) count
\qii I is is count of new layers within one zone
\qii II is upper index of layers planned in this zone
\qii ZBOT is depth in meters of the bottom of PRIOR layer
\qii DBOT is depth in diurnal scale of the bottom of PRIOR layer
 \\ to print file:
 \vspace{-3.mm} 
\begin{verbatim}   
 Zon I Lay   D___bottom___m thick_m Conductiv Density SphHeat  Diffusive  Inertia
  1  1  2    0.134   0.0029  0.0029 0.1004E-01 1002.8  600.00 0.1669E-07    77.72
Zon: is 1-based index of zone in file
  I: is 1-based layer count within a zone
Lay: is KRC layer index
\end{verbatim} .
\qi J,BLAY,SCONVG,QA   2 0.29042E-02 0.52716E-10     1.0
\qii J is KRC layer 
\qii BLAY is  layer thickness in m
\qii SCONVG is convergence safety factor before any time-doubling
\qii QA is the time doubling factor for that layer
  
If Layer depths are increasing too rapidly, must decrease FLAY by appropriate factor.

\subsubsection{Generated layer thickness}
As layer are generated from zones, maintain cumulative depth in local-scale
units. For each new zone, determine the ideal first-layer thickness from the
depth/thickness relation generated by FLAY and RLAY; do this by fractional
interpolation in a normalized depth/thickness table. Because the summation
formula is degenerate for RLAY=1, need some logic branches. Maintain total depth
in D units based on N1, FLAY and RLAY. Logic flow:
\\ - get $\Delta z$ in m from zone table
\qi -  If this is last zone, compute depth-to-go using a sum based on N1, FLAY, RLAY
\\ - Compute desired first layer of zone based on depth thus far in D units
\qi - If RLAY $\leq 1$ then $n=\Delta z /$ first in m 
\qi - else: use formula in normalized units
\\ - Round n to nearest integer
\\ - Recompute first layer based on integral number of layers
\\ - Loop on layers, increasing each by RLAY, keep track of $\sum$ in m and D units

$D=\sqrt{\kappa}\sqrt{P/\pi} = I/(\rho C_P)\sqrt{P/\pi}$.
\qi For typical geologic materials, $\rho C_P \sim $1.e6, and for a diurnal
period of 1 day, $D \sim I \cdot $ 1.7e-4 m

\qsd{Grott 2007 profile} I wrote a small IDL function that generates the
lunar-like conductivity and density profiles of \qcite{Grott07}, which were then
output at geometric series spacing close to KRC v3.4 normal; making the zone
table \nf{Grott07.tab}.

Using last layer as 15 yields a little nicer result than 14, which puts 2
layers in the last zone and has less efficient time doubling.

\subsubsection{Ensure stability}%.............. 

If the first layer would be unstable, then can adjust RLAY to generate a stable
BLAY and while not changing the total depth or the number of layers. No [easy]
analytic solution; must iterate to find RLAY.

Minimum safe first layer: BSAFE= $\sqrt{2 \kappa \Delta t }$ where $\Delta t = $ PeriodDays*86400./N2   If BLAY lt BSAFE, then can use IDL 
routine NUMGEOMLAY to find RLAY that will allow FLAY to yield the needed BSAFE:
\qi Compute the diffusivity $\kappa = k/( \rho C_P)$
\qi Compute the diurnal scale: $D=\sqrt{ \kappa P/\pi}$ where P is the spin period in seconds, = days*86400.
\qi Set the desired total depth in D units, Z \ =Z(meters)/D
\qi Set the desired number of physical layers N ( N1-1)
\qii One way: 
\qi Set the desired FLAY in D-units or compute it as B/D where B is the first physical layer thickness in m.
\qi In IDL: yy=NUMGEOMLAY(Z,jint=N,flay=FLAY)  ; yy will be the needed RLAY
\qiii Z must be real, not integer.

Can add the following to .inp file, then will run very quickly and produce a layer table. 
\vspace{-3.mm} 
\begin{verbatim}
2 5 2 'N5' / few seasons THIS AND BELOW USED WHEN TESTING layer structure
2 4 1 'N4' / single latitude (beware if more than one latitude row)
\end{verbatim}  

\section{Photometric Function Details \label{photo} \label{Ah}}

%%% intend for includion in V34UG.tex

\subsubsection{Notation} %..........................
Some notation here follows \qcite{Hapke93}, sections therein are indicated as
H[x.x] and equations by H(x.x) 
\qi $i$ is the incidence angle from the surface normal. $\mu_0 \equiv \cos i$
\qi $e$ is the emergence (viewing) angle from the surface normal. $\mu \equiv \cos e$
\qi $g$ is phase angle
\qi $\psi$ is the azimuthal angle between the planes of incidence and emergence 
\\ Other symbols
\qi $I$ radiance, usually in the emergence direction
\qi $J$ irradiance, usually the incident power per unit are normal to the incidence
\qi $A_K$ the traditional KRC input albedo ``ALB''.

 ``TOI'' means ``Table of Integrals'' \qcite{Dwight61}

\subsubsection{KRC needs} %..........................

KRC needs two types of albedos; the fraction of collimated light hitting a
surface at local incidence angle $i$ that is reflected, and the fraction of
diffuse (presumed isotropic) light relected by a surface. KRC deals with energy,
so all photometric terms are assumed to represent the bolometric value, as
weighted by the solar spectrum.

Compute the absorbed direct, diffuse and bounce insolation in TLATS where it has
been done.  For thermal models, need an \textbf{absorption} photometric
function; equivalent to $1-A_h$ H[10.D].  KRC does not need the full BRDF.


\subsubsection{Thrashing with Hapke} %..........................

BRDF == Bidirectional reflectance distribution function, H[10.B]. Commonly a
function of the incidence angle relative to the surface normal, $i$, $e$ and
$g$.

$A_h$ == Directional-hemispherical reflectance (or hemispherical reflectance) is
the integral of the BRDF over all viewing directions. Equivalent to the ratio of
the total power scattered into the upper hemisphere to the collimated power
incident on the same area, H[10.D.2]. This can be a function of incidence
angle.

$A_s$ == Bi-hemispherical reflectance (or spherical reflectance) is the
reflectance of a surface under diffuse illumination, H[10.D.4]. It has no
geometric dependence.

Prior to KRC v3.3, all albedoes were considered to be Lambertian.  



In general, Hapke uses $r$ for ``reflectance'', however, there are many of them.

A potential source of confusion is the KRC treats ``albedo'' as the reflected
fraction of power incident on a unit area; this incident power is $J \cos i$
where $i$ is the incident angle relative to the local surface normal. Thus,
``albedo'' has no units. However, Hapke generally uses reflectance as the
fraction of incidence irradiance that is reflected, which is different by a
factor of $\cos i$ .

He defines $r$, the bidirectional reflectance, at the bottom of page 261 by:
``The incident radiant power [collimated] per unit area of surface is $J \mu_0$,
and the scattered radiance is $Jr(i,e,g)$, where $r(i,e,g)$ is the bidirectional
reflectance of the surface.'' [10.B]

The ratio of the reflectance of a surface to that of a perfectly diffuse surface
under the same conditions is $\frac{\pi}{\mu_0}r(i,e,g)$ H(10.3)

Also: H[8.E.1] starts with ``the scattered radiance $I(i,e,\psi)
=Jr(i,e,\psi)$.
The scattering of light from a (planar) surface is in general described by its
bidirectional-reflectance distribution function(\textbf{BRDF}) or
$r(i,e,g)/\mu_0$ H(10.1).

Whatever the photometric function, it should obey reciprocity.

The incident power (no atmosphere) on a surface is $\mu_0 J = S_m \cos i $
\\ The reflected fraction is $A_h(i)$ and the absorbed power is $S_m \cos i \left(1-A_h(i) \right)$


In Hapke terminology of his Table 8.1, the two albedos KRC needs are the:
\qi  $r_h \equiv r_{dh}$ or ``hemispherical albedo'' or hemispherical reflectance, =$\mathbf{A_h}(i)$ H(10.33) 
\qi spherical reflectance, his $r_s$, my $\mathbf{A_S}$. 

I found Hapke to have many similiar reflectance terms that are not needed 
here, and the normalization is obscure.  Here, I will simply treat the
reflectance $r(i,e,\psi \ \mathrm{or} \ g)$ as the relative radiance into
direction $e,\psi$ for an irradiance from direction $i$; and heuristically
determine the normalization required by KRC.

H[10.D.2] has ``The general expression for hemispherical reflectance is
$r_h(i)= \frac{1}{\mu_0} \int_{2 \pi} r(i,e,g) \mu \ d\Omega_e $ H(10.10) where
$d\Omega_e = \sin e \ d e \ d \psi$ .

 H[10.D.4] has ``the spherical reflectance is $r_s= \frac{1}{\pi} \int_{2 \pi}  \int_{2 \pi}  r(i,e,g) \mu \ d\Omega_e  \ d\Omega_i $ H(10.21)

\subsubsection{Procedure here} %.........................................
We shall (must) assume that the planar surface has no preferred orientation;
that is, its reflectance is invarient under rotation of the surface abouts its
normal relative to the scattering plane.

Apparently what Hapke means by $\int_{2 \pi} d\Omega_e$ includes a normalization
of $1/2\pi$, although I did not find where he stated this.  Here the
hemipherical and spherical albedo will be used to establish any normalization
factors needed by KRC.

If $ r(i,e,\psi)$ does not involve $\psi$ (or $g$), which is true for the simple
photometric functions considered here, then the integration over $\phi$ or $\xi$
simply results in a factor of $2 \pi$

Desire that the ALB used in KRC retain its historic meaning of the fraction of
energy reflected by the surface. By conservation of energy, this can never
exceed unity, so must scale some of the results here for $A_h(0)$ and $A_s$. See
table in \S \ref{all},

% \pagebreak
% \large
\subsubsection{Lambert}
  A Lambertian surface by definition has the same radiance when viewed from any
  direction, and that radiance scales with $\cos i$. Thus $A_h$ must be
  independent of $i$. However, here develop the mathmatical formalities to be
  used for other photometric function.

$ r(i,e,\psi)=A_L \frac{\mu_0}{\pi}$ \ H(8.13)

Hemispherical:
\qbn A_h(i)= \frac{1}{\mu_0}  \int_{\psi=0}^{2\pi} \int_{e=0}^{\pi/2} r(i,e,\psi) \cdot \cos e \ \sin e \ de \ d\psi \ql{Ahr}

\qbn A_h(i) = \frac{2 \pi}{\mu_0} \int_{e=0}^{\pi/2}  r(i,e,\psi) \cdot \cos e \ \sin e \ de \ql {gah}

\qb  = \frac{2 \pi}{\mu_0} \int_{e=0}^{\pi/2} A_L \frac{\mu_0}{\pi} \cdot  \cos e \ \sin e \ de \qe

\qb =  \frac{2 \pi}{\mu_0} A_L \frac{\mu_0}{\pi} \int_{e=0}^{\pi/2} \cos e \ \sin e \ de  \qe

\qb = 2 A_L \left[_{e=0}^{\pi/2} \  \frac{\sin^2 e}{2} \right| 
   = 2 A_L \left[ 1/2 -0 \right| = A_L \ \ \qe 
% \mathrm{using \ TOI \ 450.11}

Spherical:

\qb A_s= \frac{1}{\pi} \int_{2 \pi}  \int_{2 \pi}  r(i,e,g)  \cdot \cos e  \ d \Omega_e  \ d \Omega_i \qe

\qbn A_s= \frac{1}{\pi} \int_{i=0}^{\pi/2}  \int_{\psi=0}^{2\pi}  \int_{e=0}^{\pi/2} \int_{\xi=0}^{2\pi} r(i,e,\psi) \cdot \cos e   \sin i  \sin e  \  d\xi  \ de \ d\psi \ di \ql{Asr}
 
\qbn = 4 \pi^2\frac{1}{\pi} \int_{i=0}^{\pi/2} \int_{e=0}^{\pi/2} r(i,e,\psi) \cdot \sin i \ \cos e \sin e \ de \ di \ql {gas}
 
\qb = 4 \pi  \int_{i=0}^{\pi/2}  \int_{e=0}^{\pi/2} \frac{ A_L}{\pi} \cos i \cdot \sin i \ \cos e \ \sin e \ de \ di  \qe

\qb = 4 \pi  \frac{ A_L}{\pi} \int_{i=0}^{\pi/2} \left[_{e=0} ^{\pi/2}  \ \frac{sin^2 e}{2} \right]  \cos i\sin i \ di  \qe

\qb = 4 \frac{ A_L}{2} \left[_{i=0}^{\pi/2} \  \frac{sin^2 i}{2} \right] = A_L 
\mc{;}  \frac{A_S}{A_h(0)}=1.  \qe

Can use the forms of \qr{gah} and \qr{gas} for other photometric functions.

Comparing \qr{gah} and \qr{gas}, find
\qbn A_s= 4 \pi \int_{i=0}^{\pi/2}   \frac{\mu_0}{2 \pi} Ah(i) \cdot \sin i  \ di =  2 \int_{i=0}^{\pi/2}  Ah(i) \cdot  \cos i \ \sin i  \ di \ql{Ahs} 

Confirm by substitution for Lambert, Minnaert, Lommel-Seliger. 

\subsubsection{Minnaert}  
 $ r(i,e,\psi) =A_M \mu_0^\nu \mu^{(\nu-1)}$  \  where: $0<\nu \le 1$ H(8.14)
\qi When $\nu=1$ , $r=A_M \mu_0$ is Lambert behaviour and $A_M=A_L/\pi$ 
\qi Minnaert breaks down at limb, where $\mu=0$

\qb A_h(i) = \frac{2 \pi}{\mu_0} \int_{e=0}^{\pi/2} A_M \cos^\nu i \cos^{\nu-1} e \cdot  \cos e \ \sin e \ de \qe

\qb A_h(i) =  A_M \frac{2 \pi}{\mu_0} \left[_{e=0}^{\pi/2} \ - \frac{\cos^{\nu+1}e}{\nu+1} \right| \cos^\nu i \qe

\qb A_h(i) =  A_M \frac{2 \pi}{\mu_0} [ \frac{1}{\nu+1}] \mu_0^{\nu}  
=  A_M \frac{2 \pi}{(\nu+1) } \mu_0^{\nu-1}  \qe

Spherical:

\qb A_s = 4 \pi  \int_{i=0}^{\pi/2}  \int_{e=0}^{\pi/2}  A_M \cos^\nu i  \cos^{\nu-1} e \cdot \sin i \ \cos e \ \sin e \ de \ di  \qe

\qb = 4 \pi  A_M  \int_{i=0}^{\pi/2} \int_{e=0}^{\pi/2} \cos^\nu i \sin i \ di \ \ \cos^\nu e  \sin e \ de  \qe

\qb = 4 \pi  A_M  \int_{i=0}^{\pi/2} \left[_{e=0}^{\pi/2} \ - \frac{\cos^{\nu+1}e}{\nu+1} \right| \ \cos^\nu i  \sin i \ di  \qe

\qb = 4 \pi  A_M   \left[ \frac{1}{\nu+1} \right]  \left[_{i=0}^{\pi/2} \ - \frac{\cos^{\nu+1} i}{\nu+1} \right| \qe

\qb = 4 \pi  A_M   \left[ \frac{1}{\nu+1} \right]  [\frac{1}{\nu+1}] = A_M \frac{ 4 \pi }{(\nu+1)^2} \qe

\qbn P_f \equiv A_h(0)/A_M  = \frac{2 \pi}{(\nu+1) } \mc{;}  P_s  \equiv \frac{A_s}{A_h(0)}= \frac{2}{\nu+1} \ql{aam}

\begin{figure}[!ht] \igq{minnHemi}
\caption [Minnaert $A_s$]{Spherical albedo for Minnaert reflectance.
\label{minnHemi}  minnHemi.png }
\end{figure} 
% how made: hemialb 46 48

\subsubsection{Lommel-Seeliger} 

 $r= $ Lambert $/ 4(\cos i + \cos e) $, H(6.11) and H(6.12), 
thus $r= \frac{A}{4 \pi} \frac{\cos i}{\cos i+ \cos e} $

Hemispherical:
\qb A_h(i) = \frac{2 \pi}{\mu_0} \int_{e=0}^{\pi/2} \frac{A}{4 \pi} \frac{\cos i}{\cos i+ \cos e}  \cdot  \cos e \ \sin e \ de \qe

\qb  = \frac{2 \pi}{\mu_0} \frac{A}{4 \pi} \cos i \int_{e=0}^{\pi/2} \frac{ \cos e}{\cos i+ \cos e} \ \sin e \ de \qe

\qbn  = \frac{A}{2 \mu_0} \cos i \int_{e=0}^{\pi/2} \frac{ \cos e \ \sin e  }{\cos i+ \cos e} \ de \ql{LSAha}

\qb  = \frac{A}{2} \left[_{e=0}^{\pi/2} \  \cos i \ln ( \cos i + \cos e )- \cos e  \right| \qe

\qbn A_h(i) = \frac{A}{2} \left( \cos i \ln\frac { \cos i}{\cos i+1} +1) \right)  
 \ = \frac{A}{2} \left(\mu_0 \ \ln \frac{\mu_0}{ 1+ \mu_0} +1 \right) \ql{LSAh}

Spherical:

\qb A_S= 4 \pi  \int_{i=0}^{\pi/2}  \int_{e=0}^{\pi/2} \frac{A}{4 \pi} \frac{\cos i}{\cos i+ \cos e} \cdot \sin i \ \cos e \ \sin e \ de \ di  \qe

\qb = 4 \pi \frac{A}{4 \pi}  \int_{i=0}^{\pi/2} \cos i \sin i \ \int_{e=0}^{\pi/2} \frac{\cos e \ \sin e }{\cos i+ \cos e}\ de \ di  \qe

\qb =A  \int_{i=0}^{\pi/2} \cos i \sin i  \left[_{e=0}^{\pi/2} \  \cos i \ln ( \cos i + \cos e )- \cos e  \right| \ di  \qe

\qbn =A  \int_{i=0}^{\pi/2} \cos i \sin i   \left[  \underbrace{1}_1 +  \underbrace{\cos i \ln \cos i}_2 - \underbrace{\cos i \ln (1+ \cos i) }_3 \right] \ di  \ql{LSAs}
Numerical integration yields $A$ 0.20483 . Using Wolfram integral solver for each part, many opportunities for a blunder: 
% \pagebreak

\qb A_s=A \left[_{i=0}^{\pi/2} \ \underbrace{\frac{\sin^2 i}{2} }_1 
- \underbrace{\frac{1}{9} \cos^3 i \ (3 \ln(\cos i) -1 ) }_2   \right. \qe
\qb \left. - \underbrace{\frac{1}{36} \left( -3 \cos 2i + \cos 3i -6 \ln \cos \frac{i}{2} + \cos i ( 15-9 \ln(1+\cos i) )  -3 \cos 3i \ln( 1+\cos i) -9 \ln(1+\cos i) \right) }_3  \right| \qe
 \qbn A_s=  0.20457 A \ql{LSAa}
\qb A_s=A \left( \underbrace{ \frac{1}{2} -0 }_1 
- \underbrace{\frac{1}{9} [ 0-(-1 (3 \cdot 0 -1) ]  }_2  \right. \qe
\qb \left.  - \underbrace{\frac{1}{36} [ -3(-1-1) +(0-1)-6(\ln \sqrt{2} -0) +(0-1(15-9 \ln2)) -3(0- \ln 2)-9(0 - \ln2) }_3 \right)  \qe
   Yields 0.542315 A, some blunder in part 3  above.

\qbn  P_f \equiv A_h(0)/A  =\frac{\ln (1/2) +1 }{2}=0.153426  \mc{;}  P_s  \equiv \frac{A_s}{A_h(0)}=1.3333333 =4/3  \ql{aal}

\begin{figure}[!ht] \igq{LomSee}
\caption[Lommel-Seeliger $A_h$ ]{Hemispherical albedo for Lommel-Seeliger reflectance, and the integrand for $A_s$ 
\label{LomSee} LomSee.png }
\end{figure} 
% how made: hemialb 48 49   

\subsubsection{Lunar-like}
\qcite{Keihm84} does not list a BRDF, but has hemispherical albedo (his equation A5): 
\qb A(\theta)= 0.12+0.03 \left( \theta/45\right)^3 + 0.14 \left( \theta/90\right)^8 \qe

\qcite{Vasavada12} measurements with Diviner led to the form: 
\qb A(\theta)= A_0 +0.045 \left( \theta/45\right)^3 + 0.14 \left( \theta/90\right)^8 \qe

In both cases $\theta$ in degrees is equivalent $i$ in radians. See Appendix \S
\ref{LA} for background.

Implimentation in KRC 34 is a slightly more generalized form:

 \qbn A_h(i)= A \left( 1. + x \left( \theta /45\right)^3 + (0.14/0.12) \left( \theta /90 \right)^8 \right) = A \left( 1. + x \underbrace{ (\frac{4}{\pi})^3}_{f3} i^3
+  \underbrace{ \frac{0.14}{0.12}( \frac{2}{\pi})^8}_{f8}  i^8 \right)   \qen

Thus $x=0.25$ yields Keihm and  $x=0.375$ yields Vasavada for $A=0.12$.

\qb A_s=  2 A  \int_{i=0}^{\pi/2}\left( 1. + x f_3 i^3 + f_8 i^8 \right) \cos i \sin i \ di 
 = A \left[ 1. + 2f_3 D_3 \ x \ + 2f_8 D_8 \right] \qe 

where  (Wolframalpha.com)

$\int x^3 \cos x \sin x \  dx = \frac{3}{8} (2 x^2-1) \sin x \cos x -\frac{1}{8}(2 x^2-3) \cos(2 x)+constant$ and 
$ [_0^{\pi/2} = \frac{1}{32} \pi ( \pi^2-6) \sim 0.37989752  \equiv  D_3 $

$ \int x^8 \cos x \sin x \ dx = \frac{1}{4}(4 x^6 -42 x^4 +210 x^2 -315) \sin(2 x)-\frac{1}{8} (2 x^8 -28 x^6 +210 x^4 -630 x^2 +315) \cos (2 x)+constant$ 
and $ [_0^{\pi/2} = \frac{80640-20160 \pi^2+1680 \pi^4-56 \pi^6+\pi^8}{1024} \sim 0.944125 \equiv  D_8 $

\qbn A_s=  A \left(1.0 + 1.5683 \ x \ + 0.05944\right) \ql{KAs}

\qbn P_f \equiv A_h(0)/A = 1  \mc{;} P_s  \equiv  \frac{A_s}{A_h(0)}=  1.05944 + 1.5683 \ x \ql{aak}

% 1.5682916  + 0.059435709
 See Figure \ref{KeihmAh} 
 
\begin{figure}[!ht] \igq{KeihmAh}
\caption[Lunar-like albedo]{Lunar-like hemispherical albedo.
\label{KeihmAh} KeihmAh.png }
\end{figure} 
% how made: hemialb 48 50 

\small

The following coded in IDL but not needed; eventually good agreement between
analytic and numeric integration.

\qbn A_s=  2 A \left[_{i=0}^{\pi/2} \ \underbrace{\frac{\sin^2 i}{2}}_{p1} 
+  x \underbrace{ \overbrace{\frac{64 }{\pi^3}}^{f3} \cdot  
\left[ \overbrace{ \frac{3}{8}(2 i^2-1) \sin i \cos i }^{b21}
- \overbrace{\frac{i}{8} (2i^2-3) \cos 2i }^{b22} \right] }_{p2}  \right. \ql{LLAs}
\qb  \left. + \ \underbrace{ \overbrace{\frac{0.14}{0.12} \frac{ 256 }{ \pi^8}}^{f8}
 \cdot \left[  \overbrace{\frac{i}{4} ( 4i^6-42i^4+210i^2-315) \sin 2i}^{b31} 
- \overbrace{\frac{1}{8}( 2i^8 -28i^6 +210i^4-630i^2+315) \cos 2i}^{b32} 
 \right]  }_{p3} \right| \qe 

\vspace{-3.mm} 
\begin{verbatim}
in hemialb.pro @ 46,48
f3,f8:       2.0640982       0.031476598
sum3,8=      0.38127627      0.95209557
D3,D8=       0.37989752      0.94412538
qsum=        0.72721736      0.82559132
        i          b21        b22          b31      b32        2A factor
   1.5707963   0.0000000  -0.3798975   0.0000000  38.4308746  -0.5136367
   0.0000000  -0.0000000  -0.0000000  -0.0000000  39.3750000  -1.2393910
   9.0351430e-17     -0.37989752   3.6271733e-16     -0.94412538      0.72575430
del P2,3      0.37989752      0.94412538
fun3,8=      0.38127627      0.95209557
P1              FLOAT     =      0.500000
P2              DOUBLE    =       0.78414580
P3              DOUBLE    =      0.029717855
final terms:       1.0000000       1.5682916     0.059435709
Using x=     0.250000     0.375000
total factor       1.4515086       1.6475451
\end{verbatim} 

If wish to try a lunar-like in the form $A ( 1+ b i^a)$,   
integral:  $x^a \sin x \cos x dx = -2^{-a-3} x^a (x^2)^{-a} ((-i x)^a \Gamma(a+1, 2 i x)+(i x)^a \Gamma(a+1, -2 i x))$
where $\Gamma(a,x)$ is the incomplete gamma function.

But using $(\cos i)^a$ would be easy:  $\int  \cos^a x \sin x \cos x dx = \int \cos^{a+1} x \sin x dx =  -\frac{\cos^{a+2} x}{a+2} +constant $

Possible form \qbn A \equiv r_h = \frac{a}{1+\frac{1-b}{b} \mu^c} \ql{hemi}
$a$ is the albedo for grazing incidence, $b$ the backscatter ratio (Albedo at zenith / albedo at horizon) and $c$ is a sharpness factor;
\qi $c=1$ is linear from zenith to horizon 
\qi $c=$ small drops off quickly at the horizon
\qi $c=$ large rises quickly at zenith
\\ All these have the non-physical propery of discontinuous 2nd derivative if the Sun gets to the zenith.

\normalsize

\subsubsection{All \label{all}} 
 The following table lists $A_h$ and $A_s$ factors for several reflectance types.
\qi yf is $A_h(0)$
\qi ys is the spherical albedo based on $A_h(i)$
\qi ys/yf is the spherical albedo based on $A_h(i)$ with $A_h(0)$ scaled to unity.
\qii values greater than 1.0 are non-physical.
\vspace{-3.mm} 
\begin{verbatim}
 yid=  Lambert Lomm-See    Kheim     Vasa Minnk0.3 Minnk0.5 Minnk0.7 Minnk0.9
  yf=  1.00000  0.15343  1.00000  1.00000  4.83322  4.18879  3.69599  3.30694
  ys=  1.00000  0.50000  1.92259  2.28435  7.43572  5.58505  4.34823  3.48099
ys/yf  1.00000  3.25889  1.92259  2.28435  1.53846  1.33333  1.17647  1.05263
\end{verbatim} 

Each of these relations has hemispherical albedo decreasing as insolation
becomes more oblique; Lommell-Seeliger is close to $cos^{0.3} i$ from 0 to
45\qd; see Fig. \ref{hem55n} and \ref{hem55c}. $A_h(i)$ increases with $i$ for
all models here. All but Lambert and high-k Minnaert have the property of
becoming larger than 1 at high incidence angles, so that the absorped power can
become negative for reasonable values of ALB, see Figure \ref{hem55a}. The only
practical way to prevent such a creation of energy for all photometric models is
to invoke a lower limit of 0 on adsorbed insolation; TLATS restricts $0 \le A_H \le 1$ .

\begin{figure}[!ht] \igq{hem55n}
\caption[Normalized Hemispherical Albedo]{Hemispherical albedo as a function of incidence 
angle for several photometric models; see legend. Each curve normalized to the 
value at normal incidence. % Values above unity are non-physical
\label{hem55n} hem55n.png }
\end{figure} 
% how made: q.pro @ 46 48 55
\begin{figure}[!ht] \igq{hem55c}
\caption[Hemispherical Albedo]{Same as Fig. \ref{hem55n} but with abscissa being $\cos i$. 
\label{hem55c} hem55c.png }
\end{figure} 
% how made: q.pro @ 46 48 55

\begin{figure}[!ht] \igq{hem55a}
\caption[Absorbed power]{Absorbed power for ALB=.2 as a function of incidence
 angle for several photometric models; see legend. 
Values are $\cos i \ (1.-0.2 A_h(i)/A_h(0)$ .
\label{hem55a}  hem55a.png }
\end{figure} 
% how made: q.pro @ 46 48 55

\subsection{Code in KRC} %...................
Version 3.4 does not allow photometric functions when there is an atmosphere
because of input parameter overloading.

The use of $A_h$ and $A_s$ is shown in \S \ref{fffd}

SALB is $A_s$ and is a constant for a case.

Frost is always considered Lambertian

Each photometric function is normalized to unity at normal incidence, so that
the midday temperatures for low thermal inertia would be the same. The same
scaling is applied to the spherical albedos. Thus, the absorbed flux, using the
traditional KRC input parameter \nv{ALB}$\equiv A$ :
\qi Collimated: $(1-A P_f) \cos i$ where $P_f = A_h(i)/A_h(0)$ and 
$A P_f$=\nv{HALB=ALBJ(JJ)} for the sloped surface.
\qi Diffuse and bounce: $(1-A P_s)$ where $P_s = A_S/A_h(0)$ = \nv{ASF} and 
$A P_s$=\nv{SALB}

Items in COMMON (see also the comments in TLATS)
\vspace{-3.mm} 
\begin{verbatim}
krcccom ALB               Input albedo
hatccom SALB            ! spherical albedo of the soil
hatccom ALBJ(MAXN2)     ! hemispherical albedo at each time of day
hatccom SOLDIF(MAXN2)   ! Solar diffuse (with bounce) insolation at each time W/m^2 
dayccom ASOL(MAXN2)     ! Direct solar flux on sloped surface at each time of day
dayccom ADGR(MAXN2)     ! Atm. solar heating at each time of day 
\end{verbatim} 

Access to photometric models; in version 3.4.2 only when no atmosphere.
\qi Set PTOTAL to 0.1:  1 12 .1 'PTOTAL'   /
\qi Set photometric model by ARC2:  1 21 x 'ARC2/PHT' /  where x is: 
\qii 0. is Lambert ( the default with an atmosphere)
\qii -1. is Lommel-Seeliger
\qii $-1<x<0$ is Minneart with exponent  $|x|$
\qii   $0<x<1.$ is Lunar-like, with x being the coefficent of $i^3$

Hemispheric albedos for the implimented photometric functions are shown Figure \ref{kv651}  
\begin{figure}[!ht] \igq{kv651}
\caption[Test of Photometric functions]{Hemispheric albedo computed in KRC
  versus cosine of the incidence angle, solid lines, at every time step. Dashed
  lines (largely invisible) are values at every degree computed in IDL
  (hemialb.pro @55). In both cases ALB=0.12
\label{kv651}  kv651.png }
\end{figure} 
% how made: kv3 65 66 651 662

 The effect on surface temperature, relative to a Lambertian surface, is shown in 
Figure \ref{kv572}  
\begin{figure}[!ht] \igq{kv572}
\caption[Effect of Photometric functions]{Effect of various photometric
  functions; low thermal inertia, I=50, and at 60\qd S for Mars orbit but no
  atmosphere. Ordinate is Tsur relative to the values for a Lambertian
  model. Solid lines are Ls=0, dashes Ls=93, dotted Ls=272.
\label{kv572}  kv572.png }
\end{figure} 
% how made: kv3@ 572 for 

The effect at all hours, latitudes, seasons and cases is shown in
Figure \ref{quilt}  
\begin{figure}[!ht] \igq{quilt}
\caption[Effect of Photometric functions]{Effect of various photometric functions;
 low thermal inertia, I=50,  and at the equator for Mars orbit but no atmosphere. 
QUILT3 image of delta temperature from a Lambertian model.  Displayed value
range is: 0.0 to 36.5 K.  Sample is: hour(48) * 5 planes of seas*case.  Line is:
lat(5) * 40 groups of seas*case; Lines increase upward..  Latitudes: -60 -30 0
30 60 .  Season range: 0.1 to 351.4 . From the bottom upward, models are:
Lommel-Seeliger, Kheim, Vasavada, Minnaert 0.3, Minnaert 0.7 .
\label{quilt}  quilt.png }
\end{figure} 
% how made: Beaming @45


\subsection{TLATS: Sequence within the hour-angle loop }
.
\\ If not atm., twilight forced to zero
\\ Compute $\mu_0$ (angle onto flat terrain)
\qi If twilight, adjust $\mu_0$
\\ If slope, compute $i_2$; no consideration of twilight
\qi compute $\alpha$ and $G_1$
\\ Compute hemispheric albedo of the surface, based in [adjusted]  $\mu_0$
\qi If Far, frost albedo is thick-deposit value based on Tfar, as have no frost-amount.
\\ Compute C=Collimated beam and boundary fluxs; use DE if an Atm.
\\ Compute D=Diffuse, which does not depend on slope
\\ Compute B=Bounce flux.
\\ Sum C+D+B at each time

If twilight:
\qi flux onto top of atmosphere unchanged, so atm heating should not be changed
\qi diffuse flux out bottom of atm is extended by cos3
\qi total energy should not change, so need to scale [C+D] by their diurnal sum.


Delta-Eddington always uses the Lambert albedo, otherwise becomes complex
(undefined) to consider the twilight region. Thus, surface photometric fuction
considered only for Collimated beam. However, frost is always treated as
Lambertian.


 Sky factor for a pit with wall slope $s$
\\ Solid angle is 
\qb \int_0^{2\pi} \int_0^\theta \sin x \  dx \ d\phi =2 \pi \left[ -\cos x \right]_0^\theta =2 \pi (1-\cos \theta) \qe where $\theta = 90 -s$ and $x$ is the angle from zenith

But projection onto a horizontal surface has an additional term $\cos i $ in the integrand.  
\qb \int_0^{2\pi} \int_0^\theta \sin x \cos x\  dx \ d\phi =  = 2 \pi \left[ \frac{\sin^2 x}{2} \right]_0^\theta = 2 \pi \frac{\sin^2 \theta}{2} \qe
.  

Absorbed Direct insolation: C = Albedo * PhotoFunc * AtmTrans * SunAtMars 
\qi Albedo: constant for soil, may be variable for frosts;   TDAY
\qi PhotoFunc: cos i for Lambert, several others available;  HALBF and BND2 in TLATS
\qi AtmTrans: compute with DEDING2 for atmospheres, otherwise 1. [or 0 at night]; COLL in TLATS
\qi SunAtMars: $1/AU^2$; TSEAS 


Trying to put all the incidence angle calculations into one DownVIs in TLATS may be asking too much.
\vspace{-3.mm} 
\begin{verbatim} 
ASOL[jj]                    
     &, ASOL(MAXN2)     ! Insolation at each time of day, direct + diffuse
     &, ADGR(MAXN2)     ! Atm. solar heating at each time of day 
Both use LFROST at the start of the season and do not treat a change during a season.
\end{verbatim} 


% See detailed treatment of kv3@7846, still get 0.8 W/m2. 
 %<<<<<<<<<<<<<<<<<<<<<<<<<<<<<<<<<<<<<<<<<<<<<<<<<<<<

\clearpage
\section{Heat flow Details \label{hfd}}

\subsubsection{Lower boundary}
 To stay in the numerical scheme of KRC, impose upward basal heat flow of $H_g$
 by setting \qbn T_{n+1}=T_n+\frac{H_g (B_n+B_{n+1})}{2 k_n} \ql{H1} This value
 is constant for all latitudes and seasons of a single case.

\subsubsection{Upper boundary}
No change to numerical iteration, but starting conditions can include expected
effect.

Modified from KRC paper, equation number in [13] there has additional term on
the right, $+H_g$.

Surface radiation balance, from Eq.[13] for a flat surface with sub-surface heat
flow $H_g$; revised [12] :
\qbn \epsilon \sigma  \langle T_s^4 \rangle =(1.-A) \langle S_{(t)}' \rangle +H_g 
+ \epsilon \sigma  \beta_e \langle T_a^4 \rangle  \ql{Hs}

This will be further revised with the inclusion of photometric functions, see \S 
\ref{eqT} .
Expansion of $ \langle H_R \rangle$ using Eq. [5] and a combination of Eqs. [10]
and [13], yields revision of [12];

\qbn  \langle T_a^4 \rangle = \frac{ \langle H_V \rangle / \beta_e + (1-A) \langle 
S_{(t)}' \rangle + H_g} { \sigma (2- \epsilon \beta_e) }  \ql{Ha}


\subsubsection{General relations}

 Heat flow at any time between any two layers, Layer i of thickness $B_i$,
 conductivity $k_i$ and temperature$T_i$ and the next deeper layer with values
 $B_+, \ k_+$ and $T_+$. Assume the temperature holds at the middle of each
 layer. Equating the heat flow in the adjacent 1/2 layers so that no energy goes
 into the interface at temperature $T_p$: \qbn H_i=\frac{k_i}{B_i/2}(T_p-T_i)
=\frac{k_+}{B_+/2}(T_+-T_p)= H_+ \ql{H4} 
yields; \qbn T_p=\frac{ k_iB_+T_i \ + \ k_+B_iT_+ }{ k_iB_+ + k_+B_i} 
\mc{or} T_p= \frac{T_i+fT_+}{1+f} \mc{where} f=\frac{k_+B_i}{k_iB_+} \ql{H5} 
Use either of the first relation to get heat-flow; the first yields 
\qbn H_i=\frac{k_i}{B_i} \frac{2f}{1+f} (T_+-T_i) \ql{H6}

However, the values will be near the geothermal
 level only well below the annual skin-depth.

\subsubsection{General expectation}
Compared to a model without heat flow, for which the diurnal mean surface
temperature is $ \epsilon \sigma T_m^4= S_o f(1-A)/U^2 $ where $f$ is a
``garbage'' factor that accounts for the effects of illumination geometry and
the atmosphere; with steady geothermal heat-flow $H_g$ would expect a diurnal mean
surface temperature $ \epsilon \sigma T_H^4= f S_o (1-A)/U^2 + H_g $ , some
manipulation leads to a change in the average surface temperature: 
\qbn T_H-T_m \simeq H_g/ \left( 4 \epsilon \sigma T_m^3 \right) \qen   
The effect will be larger at night and less in the day. With depth, the average 
temperature increase for layer $j$ is, 
\qbn \overline{T_{Hj}}- \overline{T_j}\approx + \frac{H_g}{k} \left[ \sum_{i=2}^{j-1} B_i \ + B_j /2 \right] \qen 
where $B_i$ is the thickness of KRC layer $i$ in meters.

\subsubsection{Slow convergence}

KRC starts with an isothermal profile.  Heat-flow induces a thermal slope which grows from the lower boundary.  

Characteristic time for a slab of thickness $l$ is $l^2/\kappa$

\qcite{Carslaw59} [CJ] section 3.4 discusses a region $-l <x<l$ with zero
initial temperature and with the $x=\pm l$ kept at constant temperature $V$ for
$T>0$''. Apart from the imposition of a linear gradient, this region $0<x<l$ is
similar to starting KRC isothermal and imposing a constant heat-flow that in
infinite time would increase the lower boundary by $V$.  Introducing the
dimensionless parameters $T=\kappa t/ l^2$ and $\xi=x/l$ (all this is CJ
symbols) yields solution for the temperature increase $v_{(T,\xi)}$ in the form

\qbn \frac{v}{V}= 1-\frac{4}{\pi} \sum _{n=0}^\infty \underbrace{
  \overbrace{\frac{(-1)^n}{2n+1}}^{a}e^ {\overbrace{-(2n+1)^2\pi^2/4}^{b} \cdot
    T}}_{p1} \underbrace{ \cos \overbrace{\frac{(2n+1)\pi}{2}}^{c} \xi }_{p2}
\mc [3.4.4]\qen and the family of solutions is shown in CJ Fig 11 (p. 101),
which I have recalculated [IDL routine qkrcsimp.pro] and show in Fig
\ref{CJFig11}; the under- and over-braces indicate grouped terms in my numerical
solution.

\begin{figure}[!ht] \igq{CJFig11}
\caption[Thermal diffusion from one boundary]{Temperature distribution over time
  for a slab with a jump in one boundary.Ordinant is fraction of the temperature
  rise to final state; abscissa is fraction of the way into the slab. Curve
  values shown in the legend are normalized time: $\kappa t/ l^2$. See text.
\label{CJFig11} CJFig11.png  }
\end{figure} 
% how made:  qkrcsimp first section

If a KRC model is thick enough to attenuate the annual cycle (about 4 times the
annual skin depth, $l=4D_y$ and $D_y=\sqrt{\kappa \frac{P_y}{\pi}} $. Then, to
reach 90\% effect, need $T=1$ or $t=l^2/\kappa \Rightarrow t=16 P_y/\pi$, or
about 5 years.

A relatively quick way to approximate heat-flow models is to compute a KRC model
with no heat-flow, estimate the change in surface temperature and the thermal
gradient on the model soil stack using the thermal conductivity profile and add
these to the KRC no-heat-flow model.

\subsubsection{Issues ?? CHECK \label{issu}}  % ---------------------------------
\begin{enumerate}    % numbered items 
 \item 
Heat flow determined from KRC run output temperatures does approach the
lower-boundary input values. However, when IC2 invokes a second material, get
only about 80\% of the expected heat flow. Have not found the cause of this.
 \item 
Ts differences for the two T-dep cases are up to about 1K at the equator and 2K
for $\pm 60$\qd.

Convergence test for the lower material in Version 2 was against CONVG, so a
large value generated thick layers in the lower material; inappropriate.  V34
cannot reproduce the exact layer and time-doubling configuration of V23. Yet,
the remaining difference are much larger than expected. See the working document
\nf{V33issues.tex}

\end{enumerate}

\section{Far-field Detailed design \label{fffd}}

Enhancement: Option for sloped models to use temperatures from a far-field model
(first case, if it was run and stored) for the far ground.

Terminology:
\begin{description}  % labeled items   
\item [self mode] Radiation from the far field horizontal surface assumed to be
  at the same temperature at the sloped surface, and of the same albedo. The
  atmosphere temperature is calculated based on sloped-surface temperature. This
  was KRC's only mode until version 3.4
   
\item [far mode] Radiation from the far field surface is based on
  prior KRC run. Atmosphere down-going radiation also based on this prior run.

\item [far-field file == fff] A file that contains the temperatures from a model
  of the ``remote'' surface and atmosphere as needed by a sloped case. A fff
  contains the values computed on the last day, not extrapolated to the end of a
  season

\item [``flat'' case] A zero-slope case that matchs a sloped case in all
  parameters except slope and azimuth.

\end{description}

Sequence in the software set:

\begin{itemize}      % ticked items  

\item TCARD reads the file name. 

\item LOPN3=true indicates that a fff is available.

\item KRC checks the name length, and can call TFAR(1 to open the file.  If fff
  open fails, TFAR will write an error message and KRC will stop to prevent a
  possible long run with the wrong fff.

\item TSEAS calls TFAR*(2) before the first season to get all the size and date
  information. For each run season, it interpolates the far-field file to that
  season; if the nearest fff season is within 1\% of a season length, it will
  use that season only; else it calls TFAR twice to get the bounding
  seasons. Tsurf, and Tatm if needed, are interpolated to the desired date and
  stored in COMMON.

\item TLATS extracts the right latitude, and will do cubic interpolation in hour
  to the N2 times if needed. TLATS converts fff surface temperature to radiance at each
  time-step and places in COMMON as FARAD. If KRC is using an atmosphere, TLATS
  interpolates fff Tatm temperature for each time-step and places in COMMON as
  HARTA.

\item TADY uses LOPN3 as the flag to indicate the fff is being used. When using
  fff, the atmosphere temperatures and radiation come from fff.  Frost is still
  calculated.

 \end{itemize}

\subsection{Far-field for sloped surfaces } %----- =============--

Atmosphere and far-field to be based on a no-slope run with all other conditions
normally the same.

No additional changes are made for pits. 

User should not invoke global frost integration (KPREF=2) for sloped runs.

\vspace{3 mm}

For onePoint mode; more complicated: \textbf{not yet implimented}
\qi Must specify the name of an appropriate fff; must read it successfully
\qii fff seasons must cover the range needed by the onePoint master
\qii fff latitudes must cover the range in the onePoint lines
\qi Print a stern warning that a single  fff is being used.
\qi For each onePoint line, need to interpolate: 
\qii in season; same as for normal run
\qii in latitude; requires additional code
\qi Limitation: only allow a single fff for a onePoint run.

\subsection{New Inputs} %.................................................
Control is by the presence of the file name defined by change line : 8 3 x name
\qi  
\subsection{New Outputs} %...............................................
 No plan for \np{TDISK} to be able to read type 52

Size of fff is set by firm-code sizes, TSF(MAXNH,MAXN4) for each season =
96*37*8=28416 bytes for each temperature for each season. The first record in
file contains KRCCOM; season start at record 2. Thus, the file size in bytes is
28416*(seasons+1)*[1,2 or 3]
 
\subsection{New Common Items} %.........................................

Must set the maximum number of time-steps that can be stored in a fff. To simplify writing records in TDISK,  write the existing arrays, TSF etc., these are (MAXNH,MAXN4). Because direct-access files are not open to write for more than one case at a time, it would be possible to use smaller record sizes and write 
\qi ((TSF(I,J),J=1,NLAT),I=1,N24) as long as record can hold KRCCOM.
\qii FORTRAN executes the outer loop first
\qi However, tests indicate that must write complete arrays to direct-access records

Must also set the maximum number of time-steps that can be interpolated from an fff temperature set.  This couild be smaller than MAXN2=384*4*256=393216
\qi


.
\\ FILCOM/  CHARACTER*80 FFAR     ! far-field temperatures input
\\ HATCOM/        INTEGER MAXFF
      PARAMETER (MAXFF=384*4*4=6144) ! dimension of far-field times of day
REAL*8  SALB                 ! spherical albedo of the soil
\qi REAL*8 FARTS(MAXNH,MAXN4,2) ! far-field Tsurf/Tatm for current season
\qi ``    FARAD(MAXFF) ! far-field radiance for every time-step at current latitude
\qi  ``  HARTA(MAXFF) ! far-field Tatm for every time-step at current latitude
\qi  ``  SOLDIF(MAXN2) ! Solar diffuse (with bounce) insolation at each time W/m$^2$
\\ UNITS/ INTEGER  MINT ! the number of temperature sets in file being handled by IOD3 

Type -n files as 1,2 or 3 arrays each Real*8 (MAXNH,MAXN4)
\subsubsection{Array size limits} 
TFAR8: FTS ,FTP,FTA (MAXNH,MAXN4) for one season

TSEAS: same 3
\qi FARTS(MAXNH,MAXN4,2) ! far-field Tsurf/Tatm for current season

TLATS:  REAL*8 WORK(MAXFP=MAXNH+3)        ! to hold extended hours 
 CUBUTERP8 outputs into FARAD=KIM*NY=  must be less than MAXFF

KRC runs requires N2 le MAXN2 =384*4*256=393216; runs using fff will limit N2 to MAXFF=384*4*4 = 6144
\subsection{Implementation} %.........................................

\textbf{ALERT} The use of the output file flag K4OUT has been changed for
version 3.4 . It now controls only the direct access type being written. Actual
reading and writing of data file is controlled by the presence and length of
three file names; all 3 names default to 'no'.  Sees \S \ref{help}.



 \subsection{ What arrays are available in earlier versions}
In \np{TLATS} [or \np{TDAY}]:
\\ for the last day computed, not extrapolated
\qi TSF(I,J4)=TSFH(I) (hour,lat)
\qi TPF(I,J4)=TPFH(I) (hour,lat)
\qi TAF(IH,J4)=TATMJ (hour,lat) saved in \np{TDAY}
\\ Extrapolated to the end of a season
 \qi TTS4(J4)=xof TTS , TTB4(J4)=xof TTB  : surface and bottom diurnal average 
 \qi TTA4(J4), midnight Atm
 \qi FROST4(J4)=EFP   ! frost amount
\qi TMN4(I,J4) , predicted TT1(layer,lat) temperature at midnight

\qsc{Self-heating versus Far-field}  %. . . . . . . . . . . . .

Minimize code changes; aim at needing only Tsurf and Tatm from the far-field model.
 Assume normally will use same atmosphere parameters, although could use different!

from KRC paper: [68]...  \small
\\ The surface condition for a frost-free level surface is :
\qbn W=(1.-A)S_{(t)}'  + \Omega \epsilon R_{\Downarrow}
+k \frac{\partial T}{ \partial z}_{(z=0)} - \overbrace{ \Omega \epsilon\sigma}^{FAC5} T^4  \ \ (jgr 13) \qen

where $W$ is the heat flow into the surface, $A$ is the current surface albedo,
$S_{(t)}'$ is the total solar radiation onto the surface as in Eq. (1), $
R_{\Downarrow}$ is the down-welling thermal radiation (assumed isotropic), $T$
is the kinetic temperature of the surface, $k$ is the thermal conductivity of
the top layer. $\Omega$ is the visible fraction of the sky, $\epsilon$ is the
surface emissivity and $\sigma$ the Stefan-Boltzmann constant.  In the absence
of frost, the boundary condition is satisfied when $W=0$. \normalsize


from KRC paper: [70]... \small
\\ The collimated incident
beam is treated rigorously, intensities of the diffuse solar and thermal fields
are modified by the fraction of sky visible, and the average reflectance and
emittance of the surrounding surface (absent in the level case) are approximated
as: the brightness of level terrain with the same albedo, and material having
the same temperature as the target surface, respectively; this last
approximation accentuates the diurnal surface temperature variation with
increasing slope. Then 
 \qbn S_{(t)}' = S_M \left[ \underbrace{F_\parallel \cos i_2}_{direct} 
+ \underbrace{ \Omega F_\ominus^\downarrow }_{diffuse} 
+ \underbrace{ \alpha A (G_1 \cos i F_\parallel 
+ \Omega F_\ominus^\downarrow)}_{bounce} \right]   \ \ (jgr 14) \qen

$F_\parallel$ =\nv{COLL} is the collimated beam in the Delta-Eddington model and
$F_\ominus^\downarrow$=\nv{BOTDOWN} is the down-going diffuse beam. 
$\Omega$=\nv{SKYFAC}: $\Omega \equiv 1 - \alpha $ here and in
Eq. (13). $G_1$=\nv{G1} is the fraction of the visible surrounding surface which
is illuminated. Within the brackets in Eq. (jgr 14),
\qi the first term is the direct collimated beam, \texttt{DIRECT}
\qi the second is the diffuse skylight directly onto the target surface, \texttt{DIFFUSE}
\qii $ F_\ominus^\downarrow)$ does not depended upon slope.
\qi the third term is light that has scattered once off the
surrounding surface, \texttt{BOUNCE}

For a sloped surface, $G_1$ is taken as unity. As a first approximation, for
depressions $G_1=$ $(90-i)/s \ < 1)$ where $s$ is the slope to the lip of the
depression (the apparent horizon). For the flat-bottom of a depression, $i_2 =
i_0$ when the sun is above this slope, and $ \cos i_2 =0$ when below. \normalsize


ALERT:  $\cos i$ factor in the bounce $F_\parallel$ term is missing in JGR paper.
It is in the tlats8.f code back to at least 2011aug.

For far-field, need to expands the thermal radiation balance term, and (jgr 13) becomes  

\qbn \underbrace{W}_{POWER}=\underbrace{\overbrace{(1.-A)}^{FAC3} S_{(t)}'  
+ \overbrace{\Omega \epsilon}^{FAC6} R_{\Downarrow}^0 }_{ABRAD}
+ \underbrace{k \frac{\partial T}{ \partial z}_{(z=0)}}_{SHEATF}  
-  \overbrace{\epsilon \sigma}^{FAC5} T^4  
+ \underbrace{\overbrace{(1-\Omega) \epsilon \sigma \epsilon_x }^{FAC5X} T_x^4}_{FARAD}   \qen

where $R_{\Downarrow}^0$ is for the equivalent no-slope case and $T_x$ is the
far-field  surface temperature and $\epsilon_x$ its emissivity; $T_x \equiv
T $ if self-heating.  The next-to-last term is surface emission into a
hemisphere and the last term is thermal radiation from the far surface. If the
sloped surface has frost, $T$ becomes fixed at the frost temperature but the
equation remains the same.

With the ability of albedo to depend upon incidence angle, need to expand $ S_{(t)}'$ and (jgr 13) becomes 

\qb \underbrace{W}_{power}=  S_M \left[(1.-A_{h(i_2)}) \underbrace{F_\parallel \cos i_2}_{direct}
+  (1-A_s ) \left( \underbrace{ \Omega F_\ominus^\downarrow }_{diffuse} 
+  \underbrace{ \alpha A_s (G_1 \cos i F_\parallel 
+ \Omega F_\ominus^\downarrow)}_{bounce} \right)  \right] 
\qe

\qbn
+ \underbrace{ \Omega \epsilon R_{\Downarrow}^0}_{atm \ IR}
+ \underbrace{k \frac{\partial T}{ \partial z}_{(z=0)}}_{conduction}  
- \underbrace{\epsilon \sigma T^4}_{emission}  
+ \underbrace{(1-\Omega) \epsilon \sigma \epsilon_x T_x^4}_{back \ radiation} \qen 
 where all the terms within the square brackets are normalized (are unitless).

Reformulate, under- and overbrace terms indicate FORTRAN variable names 
\qb \underbrace{W}_{POWER}=  \underbrace{(1.- \overbrace{A_{h(i_2)}}^{ALBJ} )}_{FAC3}
 \overbrace{ S_M  F_\parallel \cos i_2}^{ASOL}
+  \underbrace{(1-\overbrace{A_s}^{SALB} )}_{FAC3S} \underbrace{ S_M 
  \left( \overbrace{ \Omega F_\ominus^\downarrow  }^{DIFFUSE}
+ \overbrace{ \alpha A_s (G_1 \cos i F_\parallel
+ \Omega F_\ominus^\downarrow ) }^{BOUNCE}  \right)  }_{SOLDIF} 
\qe

\qbn
+ \underbrace{\Omega \epsilon}_{FAC6} \underbrace{ R_{\Downarrow}^0}_{ATMRAD}
+ \underbrace{k \frac{\partial T}{ \partial z}_{(z=0)}}_{SHEATF}  
- \underbrace{\epsilon \sigma}_{FAC5} T^4  
+ \overbrace{\underbrace{(1-\Omega) \epsilon \sigma \epsilon_x }_{FAC5X} T_x^4}^{FARAD}   \ql{wb} 
where the overbrace items are computed in TLATS and
transfered in COMMON. All terms up to and including ATMRAD make up the total
absorbed radiation ABRAD.  When frost is present, its albedo replaces $A_h$ and
$A_s$ on a time-step basis except the $A_s$ in SOLDIF (from TLATS) is on a
season basis; however, the $A_S$ term includes the far-ground fraction $\alpha$
which is small except for steep slopes.

Assumes that normal albedo is the same for the sloped and the flat surfaces.

The fraction of solar flux reflected ALBJ$\equiv A_h =$ALB*AHF is composed of
two factors, ALB$\equiv A_0$ and AHF$=A_h(i)/A_h(0)$, a hemispherical
reflectance function.  Likewise, the spherical albedo is $A_s=$ ALB*PUS where
the second factor is $P_s$.

The floor of a ``pit'' does not see the flat terrain, but rather the same slope
at all azimuths, and therefor different temperatures. The most practical
assumption is that the average radiation temperature of the pit walls is the
same as flat terrain. This will be an under-approximation. In a later version of
KRC with more input parameters, a radiation scale factor could be included; if
practical, code to include a constant factor, initally unity for v 3.4.

Because \nv{FARAD} is not dependent upon the calculation of $T$, it can
pre-computed for a given day. $T_x$ is interpolated to the proper season in
\np{TSEAS}; \np{TLATS} selects the proper latitude, multiplies by \nv{FAC5X} for
each of its stored hours, and interpolates to each time-step to form
\nv{FARAD}$_t$ transfered to \nv{TDAY}. However, to then accomodate variable
frost emission, need to multiply by $\epsilon_f/\epsilon$ for the frost case
(relatively rare).

\vspace{0.2cm}
Because frost temperature changes only with pressure, it does not need to change with Hour.
\pagebreak
\subsubsection{Equilibrium temperature  \label{eqT} }
The equilibrium temperature $T_e$ is that value of $T$ that would make the diurnal average of $W$  in \qr{wb} zero. Or: 

\qbn FAC5 \ast T_e^4 = \langle \overbrace{FAC3 \ast ASOL  + FAC2S \ast SOLDIF}^{\Delta AVEI} \rangle  + \ FAC6 \ast \langle ATMRAD \rangle  + H_g  + \langle FARAD \rangle \ql{Te}

where $ \langle \  \rangle $ represents the diurnal average.

To reach the equivalent of JRG Eq. (12), need to modify JGR Eq. (11) by allowing angle[time]-variable albedo $A$ and adding the geothermal heat-flow term $H_g$ to become 
\qbn \epsilon \sigma  \langle T_s^4 \rangle =\langle (1.-A) S_{(t)}' \rangle  
+ H_g + \epsilon \sigma  \beta_e \langle T_a^4 \rangle \ql{sbal} 

JRG Eq. (12) then becomes:

 \qbn \langle T_a^4 \rangle = \frac{ \overbrace{\langle H_V \rangle / \beta_e}^{QS} + \overbrace{\langle (1-A) S_{(t)}' \rangle}^{AVEI}  +\overbrace{ H_g}^{GHF} } 
{ \sigma (2- \epsilon \beta_e) }  \ql{Ta4} 

 Then ATMRAD (= FAC9*TATMJ**4) $= \sigma \beta_e  \langle T_a^4 \rangle $  

JGR eq. (2) remains the same:

\qbn \overbrace{H_V}^{HUV} = \overbrace{S_M}^{SOLR}\overbrace{ \left( \mu_0 - F_\ominus^\uparrow(0)-(1-A_h(t)) 
\left[ \mu_0  \ F_\parallel + F_\ominus^\downarrow(\tau_v) \right] \right) }^{ATMHEAT}\ql{aheat} 


Atmosphere IR heating is the average of $H_R$ in JGR Eq. (5): 
$ \langle H_R \rangle = \sigma \beta_e \left( \epsilon  \langle T_s^4 \rangle - 2 \langle T_a^4 \rangle \right) $

\subsubsection{code in TLATS}
Snippits of code in TLATS for radiation values placed in COMMON; omitting all the logical tests.  Minor edits for clarity.

.
\qi  SOLR=SOLCON/(DAU*DAU)     ! solar flux at this heliocentric range
\qi  call DEDING28 (omega,g0,avea,COSI,opacity, bond,COLL,deri)
\qi  DIRECT=COS2*COLL     ! slope is in sunlight  or =0
 \\ ASOL(JJ)=QI=DIRECT*SOLR         ! collimated solar onto slope surface

.
\qi  AHF= (1.D0+COS2*DLOG(COS2/(1.D0+COS2)))/2. ! Lommel-Seeliger, e.g.
\qi  HALB=ALB*AHF/AH0     ! normalized hemispherical albedo
\\ ALBJ(JJ)=MIN(MAX(HALB,0.D0),1.D0) ! current hemispheric albedo

.
\\ SKYFAC = (1.D0+ DCOS(SLOPE/RADC))/2.D0 ! effective isotropic radiation.
\qii   call DEDING28 (omega,g0,avea,COS3,opacity, bond,COL3,deri)
\qii   BOTDOWN=PIVAL*(DERI(1,2)+F23*DERI(2,2))  or 0 ! diffuse down at surf
\qi  DIFFUSE=SKYFAC*BOTDOWN ! diffuse flux onto surface
\qii   PUS=1.3333333    ! e.g. Lommel-Seeliger  $P_S$
\qii   SALB=PUS*ALB              ! spherical albedo, for diffuse irradiance
\qii   G1=DMIN1 (1.D0,(90.D0-AINC)/SLOPE) ! (90-i)/slope   or 1.
\qii   DIRFLAT=COSI*COLL ! or COSI if no atm. collimated onto regional flat plane
\qi   BOUNCE=(1.D0-SKYFAC)*SALB*(G1*DIRFLAT+DIFFUSE)   
\qi         QI=DIRECT*SOLR         ! collimated solar onto slope surface
\\ SOLDIF(JJ)=(DIFFUSE+BOUNCE)*SOLR ! all diffuse, = but the direct.
\\ AVEI=AVEI+(1.d0-ALBJ(JJ))*QI+(1.-SALB)*SOLDIF(JJ) !
.
\qi  ATMHEAT=COSI-TOPUP-(1.-AVEA)*(BOTDOWN+COSI*COLL) ! atm. heating
\\ ADGR(JJ)=QA=ATMHEAT*SOLR        ! solar flux available for heating of atm.

.
\qi in TFAR, extract from fff: FELP(8)= F3FD(2)        ! surface emissivity
\qi FAC5X=(1.-SKYFAC)*EMIS*SIGSB*DELP(8) ! last is fff surface emissivity
\qi Extract fff surface temperatures into WORK for the proper latitude
\qii add midnight wrap to both ends.
\qii raise to 4'th power and multiply by FAC5X
\\ CALL CUBUTERP8 (2,WORK,NHF,TENS,N2,FARAD) ! cubic interpolation to timesteps

\subsubsection{Code in TDAY}

Snippits of code in TDAY for calculating surface temperature; omitting all the logical tests and the convergence loops.  Minor edits for clarity.  
When there is no atmosphere:

.
\qii   FAC3S = 1.D0-SALB         ! spherical absorption
\qii   FAC3  = 1.D0-ALBJ(JJ) ! hemispherical absorption
\qi   ABRAD = FAC3*ASOL(JJ)+FAC3S*SOLDIF(JJ) ! surface absorbed radiation
\qii   FAC5 = SKYFAC*EMIS*SIGSB ! if self-heating 
\qii   FAC5 = EMIS*SIGSB ! if fff 
\qii   FAC7 = KTT(2)/XCEN(2)    ! current redone if T-dep conductivity
\qii   TS3 = TSUR**3         ! bare ground
\qi   SHEATF = FAC7*(TTJ(2)-TSUR) ! upward heat flow to surface
\\ POWER = ABRAD +SHEATF - FAC5*TSUR*TS3 ! unbalanced flux
\\ POWER = POWER+FARAD(JJ) ! only if fff


ERROR: of 10 deg slope, fff 30 K hotter than self. Not physical
\qi And  hour 23 and 23.5 are much colder.
\qi daytime rise with fff= flat:NoA greater than with fff= flat:tinyA
\\ Suggests: far view factor much too big

\subsubsection{find T for W=0} %........................
Need to modify JGR Eq. 29 : ``
 From Eq. (13), find 
\qbn \frac{\partial W}{\partial T} = \overbrace{-k / X_2}^{FAC7} - \overbrace{4 \Omega \epsilon \sigma}^{FAC45} T^3  \ \ (jgr29) \qen 

 where $X_2$ is the depth to the center of the first soil layer. `` becomes \qbn
 \frac{\partial W}{\partial T} = -\overbrace{ k / X_2}^{FAC7} -\overbrace{4
   \epsilon \sigma}^{FAC45} T^3 \mc{coded \ as} \overbrace{\Delta T}^{DELT}=
 \frac{W}{k / X_2 + 4 \epsilon \sigma T^3} \qen i.e., $\Omega$ becomes 1, so
 omit this from \nv{FAC45}
 
Thus, the fff must contain at least  $R_{\Downarrow}$ [or Tatm] and $T_x$. 
Probably desireable that it contain KRCCOM for insurance.

To handle many seasons, could write direct access files.

\qsd{Interpolation of the fff}

Generally expect that the far-field case will be run with exactly the same grid in
season, latitude and time as a sloped case. Because the stored hours are less
dense than time-steps, will need interpolation in at least that dimension.

Accomodate linear interpolation in season, but if the seasons are within
DELSEAS=1\% of a season-step, use that season without interpolation.

To avoid many complications in interpolation, require that fff contain a
latitude within DLATEST=0.1 degree of those in the sloped case, and use that
without interpolation.

Will interpolate smoothly (cubic spline, replicated across midnight) in hour
from the stored N24 points to the N2 time-steps. Firm-code the maximum number of
times steps for using a fff to a generous MAXFF=384*4*4=6144.

However, this requires that N2 for the slope run be an integral multiple, 2 or
more, of N24 for the fff run.


 

\subsubsection{Getting Tatm from TPlan and Tsurf. Not used.}
in \np{TLATS}:  tauir is in krcc8m.f;
 \qi It does vary with elevation (due to PRES) , which can be a function of latitude
\qi  it can vary with season if PZREF varies due to KPREF ne 0
\vspace{-3.mm} 
\begin{verbatim}
in TLATS:
        TAUIR=(CABR+TAUVIS*TAURAT)*(PRES/PTOTAL)+TAUICE ! thermal opacity, zenith
        QA=AMIN1(0.0168455D0,AMAX1(TAUIR,62.4353D0)) ! limits 1. < FACTOR < 2.
        FACTOR= 1.50307D0 -0.121687D0*DLOG(QA) ! from fit to hemisphere integrals
        TAUEFF=FACTOR*TAUIR     ! effective hemispheric opacity
        BETA=1.-DEXP(-TAUEFF)   ! hemispheric thermal absorption of atmosphere


in TDAY:  IF (LATM) ....
      EMTIR = DEXP(-TAUIR)       ! Zenith Infrared transmission of atm
      FAC82=1.-EMTIR            !  " absorption "
      FAC9=SIGSB*BETA           ! factor for downwelling hemispheric flux
SIGSB is a constant and BETA
      IF (EFROST.GT.0.) THEN
        FAC8=EMTIR*FEMIS        ! ground effective emissivity through atmosphere
      ELSE                      ! bare ground
        FAC8=EMTIR*EMIS
      ENDIF

            ATMRAD= FAC9*TATMJ**4 ! hemispheric downwelling  IR flux
              TPFH(IH)=(FAC8*TSUR4+FAC82*TATM4)**0.25 ! planetary  
              TAF(IH,J4)=TATMJ  ! save Atm Temp.
              DOWNIR(IH,J4)=ATMRAD ! save downward IR flux
\end{verbatim}
This is a mess, as seasonal PRES is not in type -1 files.
and EFROST is only in the single KRCCOM in the first record.

 Alternate solution is to store Tatm in traditional type -1 files.

To get the right Tplan for output for sloped surfaces, need to have Tatm for the flat case, so would have to include that also in the type -1 files.

Could avoid worsening the large size of type -1 files by making them R*4, convert to/from R*8 in \np{TDISK}

Not much more work to define a new type similar to -1 but which includes TATM. This could be read/write simultaneously with type 52.

Could store the any new flags and arrays  in hatcom to avoid impacting any other commons. hatcom is already included in \np{TLATS} and \np{TDAY}.


Better solution might be to put only  DOWNIR and TATM  is a separate type -2 file, 
 and perhaps be able to write -1,-2 and 52 all at the same time?

 This would take additional input parameters to set up, or could set the required logical flags (in \np{TCARD}) based on setting the file names

\subsubsection{File handling in version 3.4 \label{mint}}
 To utilize far-field  temperatures, need to have an input data file
 open simultaneous with at least one output file. Version 3.4 can handle 0 to 3
 data files open at once. To deal efficiently with fff for both with and without
 atmophere cases, two additional types of binary files have been defined, and
 the prior type -1 has becomes type -2.

\subsubsection{ Handling 3 data files at once. \label{help}}
Because type 52 is written after all cases done, it should be possible to have direct-access file open at the same time with little conflict. Will need more complex control logic.  Note: type - requires open/close for each case.

Need to have multiple file names active, and \np{TCARD} must distinguish when to open/close direct access (each case) versus bin5 files (may be multi-case) 

May have zero to 3 data files active at one time, determined by the second field in a change card starting '8':
\begin{description}  % labeled items   
\item [ 5] A ``bin'' Type 5x (52) bin5-format to be written. Name is \nv{FDISK}.
  PIO (c-level I/O) system determines the unit.  LOPN4 is true when active. All interface is through \np{TDISK}, which calls \np{BINF5} to open or close.

\item [21] A direct-access file to be written. Name is \nv{FDIRA}.  Uses IOD2,
  LOPN2 is true when active. Five types are available; they are distinguished by
  the value of K4OUT. By convention, the file extension should indicate the file
  type, but the KRC system makes no decisions based on this extension.  All interface is through \np{TDISK}.
\item [ 3] A fff direct access file to be read. Name is
  \nv{FFAR}, it must be type -N; -1 is adequate for air-less bodies and -3 is
  required for atmospheres. Uses IOD3, LOPN3 is true when active.  All interface is through \np{TFAR}, with open and close initiated by calls from \np{KRC}  

BEWARE: seasons in this file must cover all that will be computed in later KRC
run, including the spin-up. It is best to save a full year, with no wrap.

 \end{description}
Each  direct access file is opened after a new name of length four or more characters is read into FILCOM by \np{TCARD}
\\ Each  direct access file is closed from \np{TCARD} when a new name of any length is listed or from \np{KRC} when the run ends.
\qii IOD2 is closed at the end of a case, as direct-access files as implimented by KRC can only hold one case.
\\ An open file is always active!
\qi e.g., a sloped case will use FFAR if that is open, else it will self-heat
\\ Seasons-records are written or read from \np{TSEAS} by calls to \np{TDISK} or read by calls to \np{TFAR}.  
 Sloped cases are required in abundance to address thermal beaming, which may be
 especially relevant for airless bodies. For these, need only the surface
 tempertures, so include the capability of writing direct access files with only Tsurf.

\subsection{Lab notes on tests}
Start with master34.inp, fewer latitudes and shorter spinup. Generate fff tm3 for flat and for fake steep self-heating slope.

Run case with epsilon slope to compare with flat, and a case with slope and azimuth identical to the fake steep case, expecting same temperatures. Output to /work/work1/krc/beam/BeamBa*
\vspace{-3.mm} 
\begin{verbatim}

2016 May 25 19:07:35
Edit candi.inp for 321 and 341, run both
kv3@115, then parf[[5,6,0,1]]=['/work2/KRC/321/run/out/',    'candi' $
                              ,'/home/hkieffer/krc/tes/out/','candi341']
 
Doing -------------->     550
Num lat*seas*case with NDJ4 same/diff=         632         128


test341.inp: 40 seasons, 2 year spinup, 3 latitudes
   6 cases exercising zone table, photometric function, heatflow, constant KofT

----------  consistency between file types --------
test341a.inp: Double run: output in: /work/work1/krc/test/

1) 670 seasons, soly, no spinup, 5 latitudes
   6 cases exercising zone table, photometric function, heatflow, constant KofT
   54271232 May 23 23:11 v341aTest.t52
   57201408 May 23 23:11 v341aFlat.tm3

2) 40 seasons, 2 year spinup, 19 latitudes
 1 case, variable frost albedo and temperature
    1047488 May 23 23:11 v341aTest2.t52
    3495168 May 23 23:11 v341aFlat2.tm3

kv3@ 147 for both pairs of files confirms that t52 and tm3 temperatures 
are identical.

--------- comparison of 341 to older versions ------

321//VerTest.inp
670 seasons soly, no spinup 5 latitudes
 6 cases test KofT with/without atmosphere



@2 pari 7=3 8=11 17=2

@45  Case 1-Case 3
Item in ttt Mean     Std    mean_ABS_std
    Tsurf  -0.417   2.814   0.480   2.804
    Tplan   1.633   4.516   3.654   3.115
     Tatm   0.000   0.548   0.450   0.313
  DownVIS   0.030   0.380   0.030   0.380
   DownIR   0.000   0.319   0.244   0.205

 dt=reform(t1[*,0,jlat,*])
clot,dt   shows that last hour are all near -.8, read within +-.2

@45  Case 2-Case 4
Item in ttt Mean     Std    mean_ABS_std
    Tsurf  -0.241   2.141   0.375   2.122
    Tplan   1.818   4.046   3.539   2.675
     Tatm   0.000   0.543   0.440   0.318
  DownVIS   0.020   0.363   0.020   0.363
   DownIR   0.000   0.318   0.240   0.209

2016 May 24 08:03:02
KRCINDIFF: test for changes. Input limits:       64     120     220
 11 10   ABRPHA      27.955     -0.0000      27.955
 22 21ARC3/Safe     0.80010     -0.0000     0.80010
 32 31     fd32      3182.5      0.0000      3182.5
 76 75    TATMJ      184.59      184.61   -0.015201
117 16    K4OUT      -3      52     -55
118 17    JBARE    9999       0    9999
134 33      KKK       4       8      -4


.rnew kv3
@114  4  232 342
@11 1=test342f
@111 123
@12  7=0 20=3  23=
@402 Clot DOWNVIS and Tsurf for all cases     WRONG? slope greater max
   WRONG? Tsurf far is 36K greater than self at noon
@45   stats on case deltas
@46   plot case delta for DOWNVIS and Tsurf

@12  0=34 23=-3
 stem='v342Flatf'   read the type -n for 1st case 
@51    yields FOUT=dblArray[5, 2], TTOU= dblarr[48, 5, 3, 40]
t4=ttt[*,0:2,*,*,0] & t4=transpose(t4,[0,2,1,3])
HISTFAST,ttou-t4 ; all 0
  stem='v342Flata' & pari[23]=-1     read the type -n for 2nd case 
@51    yields  TTOU= dblarr[48, 5, 1, 40]
t4=ttt[*,0,*,*,1] & t4=transpose(t4,[0,2,1,3])
HISTFAST,ttou-t4 ; all 0

\end{verbatim}
Far-field heating has a smaller effect than self-heating; see Figure
\ref{tslope}.  An artifact of KRC is that atmospheric pressure can be constant
even when frost forms so that the amount of frost is not limited; this limits
night temperatures and can delay the dawn temperature rise, as seen in the first
case in the Figure.
\begin{figure}[!ht] \igq{tslope}
\caption[Effect of far heating]{Test with slope of 30\qd~ dipping to the West
  for a nominal asteroid surface in Mars orbit; diurnal surface kinetic
  temperatures at the equator at $L_S=134$. The 5 cases in the legend are:
  top=white: Flat with surface pressure ($P_T$) 1.1 Pa and clear with frost
  temperature 146; 2=blue: flat with no atmosphere; 3=green: sloped with no
  atmosphere and self-heating; 4=yellow: sloped with no atmosphere and far-field
  from first case, which is artificial; bottom=red: sloped with no atmosphere
  and far-field from 2nd case, which is realistic. The dashed line shows the
  difference self-heating minus far-flat-heating magnified by a factor of ten
  and offset from 200; it reaches 9K near dawn.
\label{tslope}  tslope.png }
\end{figure} 
% how made: kv3: 114 11 1=t342b 111 402 : t342b slope=30, azi=90
%t342a slope=10, azi=80



\subsubsection{fff types}
All lunar-like with Mars orbit, 5 lats, 40 seasons.  Cases:
\qi  1: tiny atmosphere, save /work1/krc/test/v342Flatf.tm3
\qi  2: no atm, save v342Flata.tm1
\qi  3: 10 degree slope, save v342Flatb.tm1
\qi  4: same slope, use fff, save v342Flatc.tm1


\subsubsection{temporary}

 
  tday 686             TAF(IH,J4)=TATMJ  ! save Atm Temp.
 Done only on the saved hour, so would need to either use as is or interpolate to each time step.
C0uld do the interpolation in \np{TLATS} time-step loop


\subsubsection{Reminder, FORTRAN file types}  % ---------------------
http://www.fortran.com/fortran/F77_std/rjcnf.html is the F77 definition
OPEN arguments that control the nature of the file.

 ACCESS: 
\qi 'SEQUENTIAL'=default  
\qi 'APPEND' 
\qi 'DIRECT'
\qii RECL must also be given, since all I/O transfers are done in multiples of fixed-size records.
\qii UNFORMATTED is the default

  FORM:
\qi 'FORMATTED'=default for sequential. Each record is terminated with a newline character; that is, each record actually has one extra character.
\qi 'UNFORMATTED' the size of each transfer depends upon the data transferred.
\qii Each record is preceded and terminated with an INTEGER*4 count, making each record 8 characters longer than normal. This convention is not shared with other languages, so it is useful only for communicating between FORTRAN programs.
\qi 'PRINT' 

RECL=rl: required if ACCESS='DIRECT' and ignored otherwise.
\qi rl is an integer expression for the length in characters of each record of a file. rl must be positive.
\qii If -xl[d] is set, rl is number of words, and record length is rl*4. 
\qiii else, rl is number of characters [bytes], and record length is rl.
\qii -xl does not occur in the KRC Makefile, as of 2016may11. Excerpt from \np{TDISK}:
\vspace{-3.mm} 
\begin{verbatim}
     IF (K4OUT.LT.0) THEN  !  K4OUT is negative  .tm1
        NWTOT=2*MAXNH*MAXN4
     NRECL=8*NWTOT    ! bytes: or  NRECL=NWTOT  ! depends upon compiler <<<< 
     OPEN (UNIT=IOD2,FILE=FDISK,ACCESS='DIRECT',STATUS=CSTAT,RECL=NRECL,...
\end{verbatim}

\subsubsection{Early plan and timing tests}
Although dicussed in this early email, changing files to Real*4 has not been impliments in V3.4.2
\vspace{-3.mm} 
\begin{verbatim}
Robin:

With regard to the need for additional stored information to enable use of a
 flat far field (fff) for sloped surfaces in KRC. 

I have reached a compromise solution that optionally adds atmospheric
 temperature (Ta) as a third array after Surface kinetic temperature (Ts) and
 top-of-atmosphere nadir brightness temperature (Tp, planetary temperature) in
 the type -1 file. This is called type -3; the file extension would be .tm3  

An added advantage of type -3 is that it allows relatively easy calculation
 (estimation) of the brightness temperature for off-nadir viewing: e.g.,
B=(emis*Ts^4 - Tp^4)/( Ta^4-emis*Ts^4)  B is the transmission of the atmosphere
tau = -ln(B)                    is the atmosphere column opacity
C=exp(-tau/cos e)               where e is the off-nadir (emittance) angle
To^4=(1-C)emis*Ts^4 + C*Ta^4    To is the off-nadir brightness temperature

Use of fff requires an additional TDISK-like routine to read-only a type -3
 file, called TDIF3. This results from needing to have two KRC direst-access
 files open simultaneously.  TDISK has been modified to allow writing a type 52
 simultaneously with either -1 or -3.

Changes required to read a -3 file to get Ts and Tp in the fashion of a -1 file:
 FORTRAN: Change the RECL argument in the OPEN statement. 
   (Beware, its meaning depends upon the compiler setting of -xl[d])

Note that type -1 and -3 file records are fixed size set by the MAXN4=37 and
 MAXNH=96 and the word type (currently REAL*8) so -3 will be MAXNH*MAXN4*3=10656
 words or 85248 bytes each. The first record contains KRCCOM, currently 1704
 bytes. A typical KRC model of 40 seasons is 3.4 Mbytes

-----

I am considering changeing the type of the Ts, Tp and Ta arrays
 written to type -3 files from REAL*8 to REAL*4 simply to keep the size
 down. This would be done transparently to users of TDISK. KRCCOM would still be
 REAL*8, but there is no need for double precision in the final temperatures.  I
 believe the time to do REAL*4 <-> REAL*8 conversion is trivial.

Changes required to accomodate REAL*4 in the files for other readers:
If user really wants REAL*8 temperatures, 
  READ arrays as REAL*4, also define REAL*8 arrays 
  Loop over hour and latitude with 
   TP8(I,J)=TP4(I,J) and similar for Ts and Ta
  Loop limits could be MAXNH and MAXN4 or the actual N24 and N4

I have tested timing to read type -3 files and to convert between R*4 and R*8.
  For a typical file with 40 seasons (does not matter how many hours or
 latitudes as the array sizes are fixed at the maximum allowed). For type -3,
 reading takes 0.6 to 0.9 ms, but this may be highly dependent upon caching.
  Conversion of the maximum set of hours and latitudes for all seasons takes 1.6
 ms (array was filled with random temperatures).

\end{verbatim} 
  %<<<<<<<<<<<<<<<<<<<<<<<<<<<<<<<<<<<<<<<<<<<<<<<<<<

\section{Plans for next release \label{plans}} %_____________________________

Current concept for thermal beaming is as a post-run process and as such has no
impact on KRC.

``Moon-ready'' KRC will require changes to several of the KRC commons, and will
require an additional loop to handle longitude. This will be a major
restructuring.


\section{ Error codes}
 The code-set has progressed toward a consistent usage: IRET is the return code argument in a routine,
 IRL is the name in a call to a lower routine.

\np{TCARD}: 1 = normal start    2 = restarted from disk record
\qi   3 = continue from current conditions  4 = Switch to "one-point" mode
\qi   5 = END of data (no more change cards) in input file
\qi   6= Error reading internal buffer  7= End while reading internal buffer

\np{TDAY}:   1=normal return   \np{TDAY}(2, 2=numerical blowup 
\qi  \np{TDAY}(1,  3=Some layer unstable  4=Too many Layers generated by zone table

\np{TLATS}: 
\qi  1=normal   2,3,4= error of same number in \np{TDAY} 
\qi  C  5=no matching latitude in fff

\np{TSEAS}: returns from: 
\qi  \np{TCARD}(2: if $>5$, then +10   4=switch to onePoint  5=End of input file
\qi  \np{TDAY}(1   if $\ne 1$, then +20 
\qi  \np{TLATS}    no change
\qi  \np{TFAR}     41: Tatm needed but not in the fff.

\np{KRC}: 
4=PARAMETER ERROR IN TDAY(1)
\section{Test runs  \label{t1} } 

Because of the extensive testing done with KRC version 321, see "KRC version 2
and 3: Thin/deep layers and long runs", that is considered the base for testing of
later versions.

\subsubsection{ A: Obsolete} %.......................
Test Run A is primarily a body in Mars orbit with no atmosphere; it used a
2-year spin-up followed by 1 year that is saved. It had 15 cases defined in
\nf{krc33.inp}, produced \nf{M33A.prt} and \nf{M33A.t52}; the differences in
diurnal temperatures from case 0 for the last day of the last season are shown
in Figure \ref{kv782M33A}.  Case 0 had: No atmosphere, was homogeneous with
depth, used 44 layers and Lambert albedo. Differences from this case are listed
below:
\qi 1: Normal Mars with atmosphere and Lambertian soil. Two materials
\qii TUN radiance output to fort.77, renamed to \nf{M33A.77}
\qi 2: Lommel-Seeliger albedo
\qi 3: Minnaert=0.7 albedo 
\qi 4:  1 W/m2 heat-flow
\qi 5:  2nd material at IC2=9 
\qi 6:  2nd material at IC2=9,  1 W/m2 heat-flow
\qi 7:  30 layers
\qi 8:  30 layers, 1 W/m2 heat-flow
\qi 9:  30 layers , KofT
\qi 10: 30 layers , KofT, 1 W/m2 heat-flow
\qi 11: less time doubling
\qi 12: no time doubling
\qi 13: zone table: zoneX (similar to 2 materials)
\qi 14: zone table: zoneY (exercise most table options)

\begin{figure}[!ht] \igq{kv782M33A}
\caption[Test run A]{Diurnal surface temperature in Version 3.3 test run A at
  the last season (Ls=0) at latitude 0 relative to the first case. Difference in
  cases described in the text.
\label{kv782M33A} kv782M33A.png }
\end{figure} 
% how made: input krc33.inp, jlat=0, kv3 139,123

Test Run B is basically the Version 3.3 standard for 3 latitudes: 0, -30 and
-45; with the 2nd material starting with IC2=7, used 44 layers and Lambert
albedo, had a 2 year spin-up and ran for a total of 20 years. It had 12 cases
defined in \nf{krc33B.inp}, and produced \nf{M33B.prt} and \nf{M33B.t52}.
%; the differences in diurnal temperatures from case 0 for the last day of the last season are shown in Figure \ref{kv782M33B}.  
Case 1,7: are the version 3.3 standard Mars homogeneous with depth; the second
case number is with 100 milli-Watt/m$^2$ geothermal head-flow. Case differences
from this are listed below: (1-based index)
\qi 2,8: Two materials with IC2=7
\qi 3,9: use T-dep materials
\qi 4,10: Two materials with IC2=7 and use T-dep materials
\qi 5,11: zone table: zoneX (similar to 2 materials)
\qi 6,12: zone table: \nf{Grott07.tab}, which follows \qcite{Grott07} with their higher conductivity

Heat flow of 100 milli-Watt/m$^2$ is about a factor of 3 higher than expected for
Mars, but chosen to make the effects of heat-flow easier to see in plots.

The effect of heat flow is shown in Figure \ref{kv7847}. The effect of
20mW/m$^2$ is 1K at about index 36, KRC layer 38, depth 2.7m .

\begin{figure}[!ht] \igq{kv7847}
\caption[InSight with heat-flow]{Temperature versus depth for a model similar to
  the higher conductivity model of \qcite{Grott07}, but at latitude 30S which
  has larger seasonal variation than the planned InSight landing area.  Abscissa
  is 0-based physical layer index, which is roughly proportional to the log of
  depth. Ordinate is temperature. First 4 curves are without heat-flow and the
  second 4 with heat-flow of 20 mW/m$^2$; the sets virtually overlay for index
  less than 25. Of each set, the first 2 are at the cold season, Ls=93\qd;
  second 2 at the hot season, Ls= 240\qd. Of each pair, the first is the minimum
  diurnal temperature and the second is the maximum. Colors lower in the legend
  can overwrite those above.
\label{kv7847} kv7847.png  }
\end{figure} 
% how made: krc33B.inp M33B.t52, last two cases. kv3.pro @ 139,123 then 7847

\subsubsection{nill Atmosphere} %.......................

A test of the progression toward low atmosphere pressures was run, 342c.  One
must be careful to ensure that the Photometry parameter ARC2/PHT is 0 for the
nill-Pressure case to ensure that it is Lambertian, as the atmosphere cases are
forced to be Lambertian in the TLATS code.

Results are shown in Figure \ref{622c} 
\begin{figure}[!ht] \igq{622c}
\caption [Effect of an atmosphere]{Surface temperatures as a function of
    atmospheric pressure; Mars seasonal variation of the equator at 13 hours
    after a t2-year spin-up. Solid lines are V342, long dash are V321. White is
    standard mars conditions with constant pressure, blue is pressure following
    the Viking lander annual variation. Green, yellow and red, are constant
    pressures of 1/10, 1/100 and 1/500 Mars, reducing the dust opacity and the
    \qcc background opacity by the same fraction.  Purple is also .1/500 Mars
    (1.1 mBar) but with dust \qcc background opacity both set to zero; orange is
    no atmosphere.
\label{622c}  622c.png }
\end{figure} 
% how made: 


\subsubsection{Version tests} %.......................
Standard comparison of version 321 and 342 was with a set of 5 latitudes for
every sol with 15 layers and no spin-up, Figure  \ref{342soly-321}; runs had 
the same six cases:
\qi 1: base=Mars std atmosphere, surface with T-constant properties
\qi 2: as case 1, but with T-dependent properties
\qi 3: as case 1, with T-con. but with variable frost temperature and albedo
\qi 4: no atmosphere, Lambertian, T-constant properties
\qi 5: as case 4, but T-dep. properties
\qi 6: as case 5, but the T-dependence set to zero

A second standard comparison used 20 layers for 19 latitudes for 40 seasons after a two-year spin-up but only case 1; Figure  \ref{342vt2-321}).

For both of these runs,  CONVF was 2 for V321, yielding lower layer of time doubling  2    4    6    8   10   12   14   15or20;  and CONVF was 3 for V342, yielding  4  5  7  9 11 13 15 [20].

A run of 5 latitudes with 20 layers for 40 seasons after a two-year spin-up but 
CONVF=3 for both versions, yielding lower layer of time doubling  4    5    7    9   11   13   15   20 for both versions, had smaller differences: Figure \ref{kv57}

 
\begin{figure}[!ht] \igq{342soly-321}
\caption[V342-V321 for each sol]{Difference of surface temperature for V342-V321 for each sol for five latitudes as shown in the legend. See text.
\label{342soly-321} 342soly-321.png }
\end{figure} 
% how made: kv3: @114 2 321 342, @115, @123, @56 t 0, @561, @57

 
\begin{figure}[!ht] \igq{342vt2-321}
\caption[V342-V321 for 19 latitudes]{Difference of surface temperature for V342-V321 for 19 latitudes for a runs with 20 layers for 40 seasons after a two-year spin-up. 
\label{342vt2-321}  342vt2-321.png }
\end{figure} 
% how made: kv3: @114 3 321 342, @115, @123, @56 t 0, @561, @57


Matching with version 232, 321 and 342 were done using for 5 latitudes (-60,
-30, 0, 30, 60), 40 seasons with a 2-year spinup, 20 layers with similar time
doubling ( Lower layer of time doubling: 4 5 7 9 11 13 15 20 for V321 and V342; 4
6 8 10 12 20 for V232) were done for representative Martian physical properties; 
I=200 over ice begining at 6 cm. Each run had the same 8 cases:
\qi 1: base=Mars std atmosphere, surface with T-constant properties
\qi 2: as case 1, but with T-dep. properties
\qi 3: as case 1, but with variable frost temperature and albedo
\qi 4: as case 1, but 30 \qd~ slope to 90\qd~ azimuth (West) 
\qi 5: no atmosphere, Lambertian, T-constant properties
\qi 6: as case 5, but T-dep. properties
\qi 7: as case 6, but the T-dependence set to zero
\qi 8: as case 5, but 30 \qd~ slope to 90\qd~ azimuth (West) 

Run times: 232= 2.445 sec,  321= 3.036  341= 3.0445

Differences in Tsurf are shown in Figures \ref{kv57}, \ref{kv571} and
\ref{kv571b}. The difference for T-dep cases results from small changes in the
layer thicknesses as discussed in the Alert in \S \ref{need}.

\begin{figure}[!ht] \igq{kv57}
\caption[V342-V321]{V342-V321 for 8 cases. Abscissa is hours * season *
  case. Ordinant is Tsurf of 342 - 321.  Five cases have differences less than
  about 0.1 K, apart from winter frost edges. Those with T-dep properties have differences are up to 1.5 K, apart from frost edge.
\label{kv57}  kv51.png }
\end{figure} 

\begin{figure}[!ht] \igq{kv571}
\caption[V342-V321, attenuated]{Same as Fig. \ref{kv57}. Cases with large excursions are shown attenuated by factors listed in the figure.
\label{kv571}  kv571.png }
\end{figure} 
% how made:  kv3 114, 4 321 342, 115, 123, 13 10=0, 56 t 0, 561, 57, 571
 
\begin{figure}[!ht] \igq{kv571b}
\caption[V232-V321]{V232-V321 for 6 cases. See caption for Fig.  \ref{kv571}
  Abscissa is hours * season * case. Ordinate is Tsurf of 232 - 321.  Four
  cases with no atmosphere have differences less than about 0.1 K, those cases
  with atmosphere are shown attenuated by a factor listed in the figure; the
  differences are up to 3.2K, apart from seasonal cap edge.
\label{kv571b}  kv571b.png }
\end{figure} 
% how made:  kv3 114 4 321 342 115 123 570 571

DownVis: 342-321 has deltas up to 8 for the atm. cases.  Unexplained.
\qii looks like revised amplitude.  but ratio largest at noon, and varies with season
\qi 232-321 has deltas up to 15 that resemble the delta Ts.
\qi no-atm. cases identically zero


DownIR: 232-321: delta up to 8 for the atm cases.

 @570 342b-321 lat=0, last season:
\qi Ts:  case 2 and 5 ( T-dep) 0.4K cool in morning, 0.8K hot after dusk
 \qi Insolation low by up to 80 at noon
\qiii 342b-321, deltas at roundoff.
 downir  deltas largest for case 2, and 3:5 all the sma.

 342c: downir values for 341 seem to be the same for all cases! Fixed
\qii downir is ATMRAD

 
\vspace{-3.mm} 
\begin{verbatim}

KRCINDIFF: test for changes. Input limits:       64     120     220
                    V342       V321
out  i    Label     Arg1       Arg2       Arg1-Arg2
 34 33     FLAY     0.21600     0.18000    0.036000 < but yields same layers
 65 64     HUGE 1.0000e+308  3.3000e+38 1.0000e+308
 66 65     TINY 1.0000e-307  2.0000e-38 -2.0000e-38
 67 66   EXPMIN      700.00      86.800      613.20
 71 70   TATMIN      143.39      143.39 -1.4859e-05
 72 71     PRES      912.60      912.60  -0.0020989
 73 72  OPACITY     0.50143     0.50143 -1.1532e-06
 74 73    TAUIR     0.30921     0.30921 -7.1116e-07
 75 74   TAUEFF     0.61843     0.61843 -1.4223e-06
 76 75    TATMJ      165.99      165.99 -0.00011073
 82 81   TEQUIL      154.52      194.78     -40.258  < ??
 85 84   SCALEH      8.2048      8.2047  3.7947e-05
 86 85     BETA     0.46121     0.46121 -7.6633e-07
117 16    K4OUT      -2      52     -54
118 17    JBARE    9999       0    9999

Using Aug 30 13:35 tlats; 
 82 81   TEQUIL      192.24      194.78     -2.5368


321----------
      AVEA=ALB ! surface albedo; will be frost if frosty at end of prior day
      AVEI=AMAX1((1.-AVEA)*AVEI/DFLOAT(N2),0.) ! average absorbed insolation
      IF (LATM) THEN            !v-v-v-v-v  with atmosphere
        TSEQ4=BETA*TAEQ4+AVEI/(SIGSB*AVEE) ! equilib T_s^4
        TEQUIL = AMAX1( TSEQ4,1.D4)**0.25D0 ! equilib T_s, min of 10.
      ELSE                  ! no atmosphere
        TEQUIL = ((1.D0-AVEA)*AVEI/(SIGSB*AVEE))**0.25D0 ! equilib T_s

342----------
      AVEA=ALB ! surface albedo; will be frost if frosty at end of prior day

      IF (LATM) THEN            !v-v-v-v-v  with atmosphere
        PHOG=0.                 ! force to be Lambert
        KOP=1                   ! Lambert flag
      ELSE                  ! +-+-+-+-+  no atm. may use photometric functions
        PHOG=ARC2               ! reassigned to be the photometric value

             AVEA=ALB*AHF   where AHF is photometric factor, =1 for Lambert
      AVEI=AMAX1((1.-AVEA)*AVEI/DFLOAT(N2),0.) ! average absorbed insolation
      IF (LATM) THEN
        TSEQ4=BETA*TAEQ4+(AVEI+GHF)/(SIGSB*AVEE) ! equilib T_s^4
        TEQUIL = AMAX1( TSEQ4,1.D4)**0.25D0 ! equilib T_s, min of 10.
      ELSE                  ! no atmosphere
        TEQUIL = (((1.D0-AVEA)*AVEI+GHF)/(SIGSB*AVEE))**0.25D0 ! equilib T_s


from 232 to 321 to 342, Tlats grew from 480 to 506 to 679 lines
 tday:  569 to 487 (deletion of many debug) to  823


2016 Sep 2 18:45:40 check processing of  232 .t52 versus .tm1
a) kv3 @114 4 232 321 @11 6=V232test2 17=.tm1   @201 202 207 @252 
    @12 0=23 23=-1 @50  @51 -> 0   @53
b)  kv3 @114 4 342 342  @11 17=.tm2  @201 202 207 @252
    @12 0=43 23=-2 @50  @51 -> 0   @53
All identical.

         ___THICKNESS____    __CENTER_DEPTH__    Total  Converg.
 LAYER    scale     meter     scale     meter   kg/m^2  factor
    2    0.2160    0.0070    0.1080    0.0035    11.224   2.851  
   20   39.3144    1.2768  199.4641    6.4780  6639.141  15.790

         ___THICKNESS____    __CENTER_DEPTH__  Conductiv. Density Sp.Heat     Total Converg.
 LAYER  D_scale     meter   D_scale     meter      W/m-K   kg/m^3   J/kg     kg/m^2  factor

    2    0.2160    0.0070    0.1080    0.0035 0.3864E-01  1600.00  647.00    11.224   2.851
   20    5.7506    1.2768   30.5485    6.4780  2.770       928.00 1711.00  6639.141  15.790

    2    0.2160    0.0076    0.1080    0.0038 0.3951E-01  1600.00  568.17    12.112   2.851
   20    5.7506    1.4720   30.5485    7.4642  3.226       928.00 1499.21  7647.643  15.790

    
T-global= 190.09
\end{verbatim} 

\subsubsection{IDL notes \label{t2}} %.........................................
\begin{verbatim} 
.rnew kv3 
@114: 4, 321 342 ,  111 123 
@65  
@66:  46 48 55 (respond many 1) -2  to prepare for HALB comparison
@ 661
@ 651 662 

@402   CLOT DownVis then Tsur for all cases
@56  t 0    Select array and item
@561        Prepare the difference
@562        Stats versus latitude
@563        QUILT3 and make dd

CLOT,reform(qy[*,0,0,*])

CLOT,reform(ttt[*,0,2,3,*]),caset,locc=1  DIurnal T, equator, one season, all cases

@12 7=0 8=1  check effect of TAUD and CABR when no atm, 
@ 56 561 ...result == 0
@12 7=-1 8=0 compare to lambert
@13 9=0 not absolute
@ 56 561 
 clot,reform(ttt[*,0,2,3,*]),caset,locc=1
 clot,reform(ttt[*,3,2,3,*]),caset,locc=1
\end{verbatim} 

%\clearpage

\bibliography{heat,moon,mars}   %>>>> bibliography data
\bibliographystyle{plain}   % alpha  abbrev 

\appendix %=-=-=-=-=-=-=-=-=-=-=-=-=-=-=-=-=-=-=-=-=-=-=-=-=-=-=-=-=-=-=-=-=-=

\section{Lunar albedoes \label{LA} }

\subsection{Email from Sylvain}
\vspace{-3.mm} 
\begin{verbatim}
05/05/2016 04:18 PM

As I am diving in the new KRC capabilities (photometric functions,
temperature-dependent properties at depth, etc.) as well as what's potentially
missing for airless bodies, there is this one other thing that bugs me:

KRC can now use a choice of surface photometric functions (Minnaert, etc.) but
an integration is necessary to turn each one of the chosen function into
hemispherical reflectance = surface albedo that depends on solar incidence. I
started to think more about these things after seeing a presentation for a
student of David Paige that compared hemispherically integrated photometric
functions from lab measurements vs. theoretical formulations and existing
formulation (Keihm 1984 and others).

I can't figure out where in KRC this integration is done, but more importantly,
why not using a Kheim1984-like formulation which actually directly gives the
hemispherically integrated albedo as a function of solar incidence (i.e. the
integration is already done for us, without having KRC doing it) as opposed to
-as of now- the photometric function that must be integrated by a black box (to
most users unless they look at the KRC codes) in KRC?

This way, the user would directly provide a and b (from Keihm 1984 Eq. A5) as
inputs, which would directly lead to the surface albedo, without relying on KRC
for the intermediate hemispherical integration. Again, a and b are from Equation
A5 in the Keihm paper we discussed a few months ago (attached with this email)
and would be the 2 input provided by the user. Is there a reason not to do that?

And while I am looking at the Keihm paper and some of my old notes from a
telecon last year with you and Phil, I wrote that KRC could integrate a
temperature-dependent surface emissivity (Keihm, Eq. A4). Do you still believe
that an upcoming version of KRC could integrate that capability?

05/06/2016 09:25 AM

I think what is throwing me off is the opposite trends between the photometric
functions found in the general optical literature and the functions described by
the lunar folks.

The Minnaert/Lommel-Seeliger/other functions have the bond albedo decreasing
when the Sun gets closer to the horizon; the bond albedo derived from lunar
observations and laboratory work increase when the Sun gets closer to the
horizon. Vasavada et al. refined the Keihm 1984 formulation with new values for
a and b to best fit the Diviner data (See Vasavada paper Eq.(1) vs Keihm
A5). Also see a presentation by Emilie Foote and David Paige compiling
laboratory measurements (specifically the last 2 slides) also confirming the
opposite trends.

With a negative exponent, the Minnaert function can get somewhat close to the
Kheim or Vasavada formulation (except near the limb as you mention in the
documentation where the bond albedo > 1) but the current input system in KRC
cannot understand negative exponents (ARC2 < 0 => Lommel-Seeliger).

\end{verbatim} 

\subsection{Hemispheric Albedo}

In \nf{/work2/Reprints/} I have \qcite{Keihm84} as \nf{Lunar/Keihm84.pdf}, 
\qcite{Vasavada12} as  \nf{ Photom/Vasavada12Lunar.pdf} 
and the Foote and Paige presentation as \nf{Photom/Foote13DivinerOxford.pdf} .

 \qcite{Keihm84} has (equation A5): 
\qbn A(\theta)= 0.12+0.03 \left( \theta/45\right)^3 + 0.14 \left( \theta/90\right)^8 \qen

Extracts from \qcite{Vasavada12}

 \S 3.1 [25]: \ `` Within this constrained data set, we find that for darker
 surfaces, the dependence of reflectance on phase angle can be removed by
 dividing by$\mu_0^{1.3}$ . This is a slightly stronger dependence than for a
 Lambertian surface (i.e., dividing by $\mu_0$ ). It is difficult to assess its
 appropriateness for brighter surfaces due to (unresolved) surface slopes that
 cause higher levels of scatter in the data.''

 \S 5.2  [44]: \
`` Because
the Diviner solar reflectance data used in section 3.1 are
measured normal to the surface, they cannot be used to
define the full bidirectional reflectance of the surface. But
daytime temperatures, being close to radiative equilibrium
with the instantaneous insolation, can be used to infer the
angular dependence of albedo. We find that the Apollo-derived 
formulation of Keihm [1984] reproduces the observations well, where

 \qbn A(\theta)= A_0 + a\left( \theta/45\right)^3 + b\left( \theta/90\right)^8, 
\ \ \ (1) \qen 

and $A_0$  is our Diviner normal albedo at each longitude. We
derive a best fit value of $a$ = 0.045 (modified from 0.03) and
keep Keihm’s value of $b$ = 0.14 .''

Converting to a normalized form:

 Can represent Keihm as  \qbn A(\theta)= 0.12\left[ 1.+0.25 \left(\theta/45\right)^3 + 1.17 \left( \theta/90\right)^8 \right] \qen


Vasavada would be, at least for $A_0=.12$,   \qbn A(\theta)=  A_0\left[ 1.+0.375 \left(\theta/45\right)^3 
+ 1.17 \left( \theta/90\right)^8  \right] \qen

 In TLATS, code as factor=$1+ f_1 \theta^3+f_2\theta^8$ where $ \theta$ is in radians and $f_1=x/(\pi/4)^3$ where $x$ is derived from the input parameter and $f_2=1.17/(\pi/2)^8$

Both are encoded into KRC 3.4 by over-using the single available photometry parameter: -x: $0<x<1$  


In some later version of KRC, isolate several input parameters to handle
hemispheric albedo. Possible form:

\qbn A(\theta)=c_0 \left[ 1.+ c_1 \left( \theta/45 \right)^f + c_2 \left( \theta/90 \right)^g \right] \qen

\subsubsection{Foote}

We have developed a simplified BRDF function that takes the following form:
\qb REFF (i,e,g) = \frac{2X}{\mu_0 + \mu + Y} \left( 1 + B(g) \right) p(g) \qe
Where X and Y are constants.
 
Where \qbn B(g)=\frac{b_0}{1+\frac{\tan(g/2)}{h}} 
\mc{and} p(g)=\frac{(1-c)(1-b^2)}{\left( 1+2b\cos(g)+b^2\right)^{3/2}} +\frac{c(1-b^2)}{\left( 1-2b\cos(g)+b^2\right)^{3/2}} \qen

Adjustable parameters: $b, \ b_0, \ c, h, \ Y, \ X$ . 


\subsubsection{Birkebak Apollo samples}

Birkebak measured hemispherical albedo (reciprocal method) of several Apollo
soil samples; his term is ``directional reflectance'', \qcite{Birkebak70} and
\qcite{Birkebak74} .

Spectral observations were made at 15,30,45 and 60\qd, weighted with the solar
irradiance and fit with polynomials in $i$ up to 6'th degree; \qcite{Birkebak74}
Tables IV and V list the coefficients.  However, the analytic expressions were
forced to go thorough unity at 90 degrees.  The analytic fits are shown in
Figure \ref{Birk3}

His Table III lists ``solar albedo'' at six discrete incidence angles from
earlier work; these are shown in Figure \ref{Birk2} and Figure \ref{Birk3}

\begin{figure}[!ht] \igq{Birk2}
\caption[Lunar albedo]{Solar albedo for specific incidence angle, from Birkebak table III. Lines omitted across missing data.
\label{Birk2} Birk2.png  }
\end{figure} 
% how made: hemialb @21
\begin{figure}[!ht] \igq{Birk3}
\caption[Lunar albedo, normalized]{Solar albedo for specific incidence angle,
  Data same as Figure \ref{Birk2}, but normalized to values at $i$=30\qd. Apollo
  12 with density 1600 seems wayward.
\label{Birk3} Birk3.png }
\end{figure} 
% how made: hemialb @ 22

 Some of the published fits pass through all the data points, others have up to
 about .01 residuals, but most have too much curvature to look realistic.


\subsubsection{My try at fits. FUTURE} %.

Not many data points, so can't use many coefficients

$A_h=A_0(1.+ c_0 (\theta/90.) ^{c_1}) $, one non-linear term, can use BRENTX
See what the family of curves looks like.
Begin coding: hemialb.pro @71

\section{Effect of photometric function}

Results for each type of photometric function in KRC v3.4.x are shown in Figures
\ref{kv402v} and \ref{kv402}. Some normalization is missing for
Lommel-Seeliger. The Vasavada relation appears unrealistic.
 
\begin{figure}[!ht] \igq{kv402v}
\caption[DownVis for photometric functions]{Downward solar flux at the surface
  for different photometric functions, no atmosphere. First case (in legend,
  from the top) is Lambertian (2nd is same, testing no atm), third is
  Lommel-Seeliger, 4th and 5th are Minnaert with k= 0.7and 0.3, 6th is Keihm,
  7th is Vasavada.
\label{kv402v}  kv402v.png }
\end{figure} 
% how made: run test342k, kv3 @111 402
 
\begin{figure}[!ht] \igq{kv402}
\caption[Tsurf for photometric functions]{Surface temperature for a lunar
  material (I=50, no atmosphere, Mars orbit and length of day) for different
  photometric functions. Different curves as identified in Fig. \ref{kv402v}.
\label{kv402}  kv402.png }
\end{figure} 
% how made: run test342k, kv3 @111 402

\section{Tuning} 

Using a nominal Mars case for I=300 with a year of 40 seasons after a 2-year
spin-up; latitudes 0, -30 and -45\qd ~as the southern hemisphere as the more
extreme seasons. Tried various geometric ratios RLAY and number of layers N1;
FLAY was adjusted so that the bottom depths are all the same.

%Input: mtime.inp, to mtimeD.prt.

Ran a case with the maximum number of layers, RLAY= 1.12, and 8-fold increase in
times steps as the reference, which takes about 8 times as long to run. Case
inputs and results for 30\qd S are in Table \ref{tuneTab}; Temperature through
the day of the last season are shown in Figures \ref{kv782Dm30} and performance
is in Figure \ref{kv782bDm30}.

\begin{table} 
\caption[Tuning cases]{Cases for the tuning run. 'Deep' is depth to the bottom
  in units of D; 'Sconv' is the average convergence safety factor; 'secs' is the
  execution time for the case (all lats and seasons); MAR is the mean absolute
  residual of surface temperature for 30\qd S; 'Tdel' is the temperature
  difference at 7 Hours. The first 3 cases are identical except for CONVG, and
  are cases 3:5 3. Case 6 has 1/2 the time steps. Case 7 is similar to case 2,
  but must use a larger first layer and smaller RLAY because of the larger
  time-step. Last are 3 sets of constant RLAY, each with increasing number of
  layers. }  \label{tuneTab} 
\begin{verbatim}
 i  RLAY  FLAY CONVF  N1    N2   Deep  Sconv   secs    MAR   Tdel
 0 1.120 0.063  3.00  50 12288 134.36   7.69  6.388  0.000  0.000
 1 1.120 0.063  2.00  50 12288 134.36   7.69  6.088  0.007 -0.006
 2 1.120 0.063  1.00  50 12288 134.36   7.69  5.696  0.028 -0.004
 3 1.150 0.100  1.00  39  1536 134.36   2.44  0.791  0.084 -0.106
 4 1.150 0.100  3.00  39  1536 134.36   2.44  0.921  0.054 -0.133
 5 1.150 0.100  2.00  39  1536 134.36   2.44  0.884  0.053 -0.143
 6 1.150 0.100  2.00  39   768 134.36   1.22  0.487  0.129 -0.359
 7 1.100 0.127  2.00  50  1536 134.36   3.95  0.927  0.038 -0.073
 8 1.120 0.124  2.00  44  1536 134.36   3.78  0.884  0.038 -0.075
 9 1.120 0.099  2.00  46  1536 134.36   2.39  0.954  0.061 -0.150
10 1.120 0.079  2.00  48  1536 134.36   1.52  1.016  0.078 -0.198
11 1.150 0.115  2.00  38  1536 134.36   3.24  0.852  0.043 -0.098
12 1.150 0.100  2.00  39  1536 134.36   2.44  0.884  0.053 -0.143
13 1.150 0.076  2.00  41  1536 134.36   1.39  0.951  0.070 -0.205
14 1.200 0.114  2.00  31  1536 134.36   3.16  0.781  0.053 -0.092
15 1.200 0.095  2.00  32  1536 134.36   2.19  0.813  0.062 -0.155
16 1.200 0.079  2.00  33  1536 134.36   1.52  0.844  0.069 -0.197
\end{verbatim}
\vspace{-3.0mm}
\hrulefill \end{table}  
\begin{figure}[!ht] \igq{kv782Dm30}
\caption[Tuning for accuracy]{Results for I=300 on Mars with 40 seasons after a
  2-year spin-up. Run mtimeD; 17 cases, most with 1536 times per sol, referenced
  to 12288 times per sol with CONVG=3. Legend columns are: 0-based case index,
  RLAY, FLAY, CONVG, N1, N2, bottom depth in units of D.
\label{kv782Dm30} kv782Dm30.png  }
\end{figure} 
% how made: kv3.pro, 138, 123

\begin{figure}[!ht] \igq{kv782bDm30}
\caption[Accuracy versus time]{Efficiency results for similar homogeneous cases;
  abscissa is log of execution time for the case (all latitudes and seasons),
  ordinate is the Mean Absolute Residual (MAR) of surface temperature for 30\qd
  S relative to the reference case.
\label{kv782bDm30} kv782bDm30.png  }
\end{figure} 
% how made: as above

 Efficiency is basically displacement down perpendicular to the dotted line in
 Fig. \ref{kv782bDm30}. Cases 7,11 and 14 are similar, with case 8 being the
 best (cases 5 and 12 identical).  The fine time-step cases show that reducing
 the convergence safety factor in this case saves about 0.30 sec (5\%) for each
 reduction of 1, with small MAR, 7 mK, for the first and another 21 mK for the
 second. Cases 3,4 and 5 tested the same CONVG changes, also indicating that the
 major performance improvement is going from CONVG of 1 to 2.

Although the MAR for all 3 latitudes are similar, the generality of these
results has not been explored.



\end{document} %===============================================================

2016 Aug 26 16:08:21 

  0 VerA=new DIR    200 = /home/hkieffer/krc/tes/out/
  1  " case file   202  = V342test1
  2  " multi stem  203  = V331test2
  3  " OnePoint [.prt]  = V331Mone
  4  " DIR for prt      = /home/hkieffer/krc/tes/
  5 VerB=prior DIR 201  = /work2/KRC/232c/run/out/
  6  " case file   202  = V232test1.t52
@115 123 570  large diffs.

2016 Aug 26 17:01:33 Generate new 342v3t.inp, Total time [s]=   2.6575961
  Lower layer of time doubling:   4  5  7  9 11 13 15 20
 --v3tb after many changes in tlats8   [s]=   2.6375990 


also  run v321  Total time [s]=   2.5516119
  Bottom layers for time doubling:     4    5    7    9   11   13   15   20
 
also run 232    Total time [s]=    t4=2.1016800
  Bottom layers for time doubling:      2    4    6    8   10   20
     -t4  with CONVF=4,                 4    6    8   10   12   20

321 342: Num lat*seas*case with NDJ4 same/diff=     1118          82   


kv3  @114: 4 321 342  @11 0=/home/hkieffer/krc/tes/out/
@115 123
@12 6=1 17=29  -30 at largest insolation




321 - 232
Num lat*seas*case with NDJ4 same/diff=        1110          90

diff 232:321 .prt shows nothing unexpected!

56,561 for difference of any arry, lats, seasons

570 difference 2 runs


@12 pari: data ranges
@13 parj: QUILT ranges and 
@14 paru: tolerances on KRCCOM item deltas
@140 park: list of cases @785
@15 parp: 4 sets of plot ranges   157=VEC2CODE
@16 parr: 3 sets of plot ranges   167=VEC2CODE
@17 parg: KRC input values
%======================================
Figure \ref{}  
\begin{figure}[!ht] \igq{}
\caption[]{
\label{}  .png }
\end{figure} 
% how made:


\begin{table} \caption[]{}  \label{}
\begin{verbatim}
---
\end{verbatim}
\vspace{-3.0mm}
\hrulefill \end{table}  


