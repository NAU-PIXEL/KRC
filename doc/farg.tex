

Prior to version 3, KRC with a sloped surface assumes that the far ``ground''
was at the same temperature as the slope, a model called in the
literature``self-heating''. This assumption becomes increasingly non-realistic
with increasing slope, and with slopes oriented East or West. Version 3.4 allows
the far-field ground temperature to come from any prior model for which an
appropriate direct-access file was saved, normally a zero-slope model of the
same thermal properties; this is termed the ``far-field file'' or fff.

To create a type -3 model, set K4OUT=-3 and include the input line:
\vspace{-3.mm}
\begin{verbatim}
8 21 0 '<file>.tm3' / Direct-access output file for far-field
\end{verbatim}
This may be the first case of a run. This can be in addition to saving a type 52
file. If there is no atmosphere (PTOTAL less than 1.), may use type -2 or -1,
with the corresponding changes to K4OUT and the file extension.

To invoke use of a ``far-flat'' model, include the input line:
\vspace{-3.mm}
\begin{verbatim}
8 3 0  '<file>.tm3' / Direct-access file to read for far-field
\end{verbatim}
Because the type of file (-1,-2 or -3) is stored in the file (as K4OUT in
KRCCOM), the current value of K4OUT is ignored. Thus, it is possible to save a
different type of direct-access by setting K4OUT as desired for output.


To revert to ``self-heating'', include an input line with full path name less than 4 characters; e.g. 'off'
\vspace{-3.mm}
\begin{verbatim}
8 3 0  'off' / Revert to self-heating for sloped cases
\end{verbatim}

Note: For change type 8, the third argument, 0 in the above examples, is ignored.

The latitudes in a fff must be within 0.1\qd of those of the slope run. The
seasons of fff must include the range of the slope run. The number of hours,
N24, need not match, TLATS will do cubic interpolation if needed.


\subsubsection{How things work:  short file names as 'off'}
Zone table: TCARD detects name length, sets LZONE true if length is 4 characters or more, else sets it false.

 Far-field file: TCARD does not look at the name length. Any write should be closed after each case.
\qi  KRC calls TFAR(1  inside the case loop if the name length is $>3$
\qi  If Far-field invoked, N2 will be limited to the values of MAXFF (firm-code as 6144)

Although TFAR can handle type -1 to -3 files, the far-field algorithm requires
type -3 if there is an atmosphere; TSEAS will check for this, and an error will
return code 41.

To terminate writing a type 52 file or reading a fff, define a new name of less
than 4 characters lengths, such as 'off'; to change to a new file, enter its
name (4 characters or more).

fff's are written as type -1,-2, or -3 by TDISK; they are read by
TFAR. TDISK can read type -n files, but it is not used for that.

KRC will close any direct-access file open for write at the end of a case
and turn off further writes to such files until a new direct-access file name is
read.

A fff may be left open for read for multiple cases

All type -n files contain KRCCOM with the values when the file being written was
closed. Because such files are closed at the end of each case, the value of
K4OUT must be the same as when the file was opened for write, and thus proper
for the file.

Development details are in \S \ref{fffd}.


