\documentclass{article} 
\usepackage{underscore} % accepts  _ in text mode
\usepackage{iftex} % detects if processing is by pdflatex
\usepackage{../newcom}  % Hughs conventions
% \newcommand{\qj}{\\ \hspace*{-2.em}}      % outdent 1
\newcommand{\erfc}{\mathrm{erfc}}  % error function inside math
% Use only one of the following two
\newcommand{\ql}[1]{\label{eq:#1} \hspace{1cm} \mathrm{eq:#1} \end{equation}}
%\newcommand{\ql}[1]{\label{eq:#1} \end{equation} } % for final
\newcommand{\eqq}{ \ < \! \! < \! > \ } %  end quote within equation

\title{An essay on slope distributions and visibility}
\author{Hugh H. Kieffer  \ \ File=-/krc/Doc/DV3/slopes.tex  2016oct}
\begin{document} %==========================================================
\maketitle
\tableofcontents
\listoffigures
%\listoftables\hrulefill .\hrulefill
% \pagebreak
\begin{abstract}
 Rough slopes and their effect on photometry and temperatures have been largely
 studied from a statisical approach, yielding results from modest to great
 mathmatical complexity; pragmatic simplifications are often involved. Yet,
 results that might represent a range of geologic morphologies are difficult to
 find. The goal here is to develop relatively simple mathmatical representations
 of slope distributions and ``hiding'' and visibiliity that can span [most] of
 the range of surfaces likely to be encountered on asteriods and that could
 be applied to thermal models without the invocation of radiosity.
\end{abstract}

The symbols \bq and \eq are used here to bound direct quotes from
articles. Ellipsis, ``...'' indicate omitted material.

``p nnnn.f a'' indicates the beginning location of a quote where nnnn is the
page number, f is the decimal fraction of the way down the page, and a/b
indicates the left/right column if on a two-column page.

Equation numbers mildly right of an equation are from the cited work; equation
numbers at the right margin are sequential in this work.

\section{History}

Be alert that the rough-surface shadow function $S$ commonly mentioned in the literature and here is the probability of NOT being in shadow.

\subsubsection{Gaussian Distribution}

Gaussian distribution: mean $\mu$ and stdDev $\sigma$
\qb G_{( \sigma,\mu )}=\frac{1}{\sigma \sqrt{2 \pi}}e^{-\frac{(x-\mu)^2}{2 \sigma^2}} \qe


Considering the solid angle at each slope, using a mean of zero, 
weighting by $\sin \theta$ 
so $\overline{\theta}=\int_{x=0}^{\pi/2} \frac{1}{\sigma \sqrt{2 \pi}}e^{-\frac{x^2}{2 \sigma^2}} \sin x dx $  ?? where did this come from ??


\subsection{Smith  1967} %-------------------------------------

Derives analytically the shadowing probability for a surface having a Gaussian
distribution of elevations. [these old (slow computer era) guys were good at
  math!] His results are used by Bandfield, Davidsson and others? .

p 4059.8a \bq The density of surface height deviations $(\xi)$ from the mean
plane in the $z$ direction is described by a continuous probability function
$P_1(\xi)$, of zero mean, chosen to be Gaussian for computational ease, where
the probability of finding a height deviation within the range $\Delta \xi$
about $\xi$ is 
\qb P_1(\xi) \Delta \xi = \frac{1}{(2 \pi) ^{1/2} \sigma}
e^{-\xi^2/2 \sigma ^2} \cdot \Delta \xi \qeq (1) \qe 
where $\sigma$ is the root-mean-square height deviation. \eq

p 4060.1a \bq The problem is the following: what is the probability $ S( \xi_0,
p_0, q_0, \theta)$ that a point $F$ on a random rough surface, of given height
$\xi_0$ above the mean plane and with local slopes $p_0, q_0$ will not lie in
shadow when the surface is illuminated with a parallel beam of radiation at an
angle of incidence $\theta$ to the mean plane? \eq

%p 4061.4b  
\qb 2 \Lambda(\mu)=\left( \left(\frac{2}{\pi}\right)^{1/2} \cdot \frac{w}{\mu}
e^{-\mu^2 / 2w^2} -\erfc \left(  \mu / \sqrt{2} w \right) \right)  \qeq (21) \qe

p 4059.7, after Eq. 3:  \bq where $w^2$, [is] the mean square surface slope
% , is $[ -p''(0)]$, the primes denoting double differentiation with respect to $r$. 
\eq

p 4060.5b, after Eq. 9: \bq where $h$ is the unit step function and $\mu = \cot
\theta $ \ \eq
 
\qb S(\theta) = \frac{ \left[ 1- \frac{1}{2} \erfc \left( \mu / \sqrt{2} w
    \right) \right]} {\left[ \Lambda(\mu) +1 \right] } \qeq (24) \qe

p 4062.7a: \bq ... the probability that a point on the surface with local slopes
$p, q$ will be illuminated by a beam of incidence angle $\theta$ is
\qb S(p, q, \theta) = \frac{h(\mu-q)}{\left[ \Lambda(\mu)+1 \right]} =
S(q,\theta) \qeq (25) \qe (independent of $p$) where the subscripts have been
dropped from $p$ and $q$. \eq

The basic function $S(\theta)$ of \qcite{Smith67} is shown in Fig.
\ref{hids}
\begin{figure}[!ht] \igq{img/hids}
\caption[Shadow function]{The basic shadow relation of \qcite{Smith67} as a
  function of $x\equiv \mu/w$ or cotangent of the incidence angle divided by the
  RMS roughness angle, plotted as logrithm of 1-$S(\theta)$ , For small values
  of x $S$ asymptotically approaches 0 and $S \rightarrow 1$ at large values of
  $x$. $S$ difference from 1 is beyond the limit of single precision for $x >
  5$.
\label{hids}  hids.png  }
\end{figure} 
% how made: hiding.pro  parh=[1.,33.,-0.5,45.,90.,35.,2.]

The terms with $S(\theta)$ are shown in Fig.
\ref{hidp}
\begin{figure}[!ht] \igq{img/hidp}
\caption[Terms within the shadow function]{Terms within the basic shadow hiding
  function. cef (white) is the $ \erfc \left( \mu / \sqrt{2} w \right)$; d1
  (blue) is $\frac{1}{2} \sqrt{2/\pi} \cdot \frac{w}{\mu}
  e^{-(\mu^2/2w^2)}$, the first term in the denominator and is $>1$
  for $x\equiv \mu/w < 0.37$; 1.-sth (green) is 1-$S(\theta)$ .
\label{hidp}  hidp.png  }
\end{figure} 
% how made: 

In beaming.pro @48 I reproduce Fig. 2 of \qcite{Smith67} by calling
\np{hiding.pro} with parameter[2]=-3, shown in Fig.  \ref{SmithFig2}
\begin{figure}[!ht] \igq{img/SmithFig2}
\caption[Smith Fig. 2]{Reproduction of the theory part of \qcite{Smith67}
  Fig. 2. Abscissa is the ``grazing angle'', or 90\qd~ minus the incidence
  angle; the ordinate is the fraction of the surface that is illuminated. Curves
  are for specific RMS slope of roughness; see legend. \label{SmithFig2}
  SmithFig2.png }
\end{figure} 
% how made: beaming.pro @481 parh=[0.,55.,15.,45.,177.,75.,1.]

 I should extract his results for both illumination and viewing shadows,
 although they are only for in the principal plane.

\subsection{Hansen  1977} %-----------------------------------------
Uses spherical crater sections and normalizes wavelength to $\lambda_{max} =2900
$ \um \qd K $/0.905 T_{max}$ where the .905 is empirical based on earlier
thermal observations of asteriods.

Fig 6 has his beaming functions.  His appendix contains the crater model, which
is solved numerically.

\subsection{Hapke 1984} %-------------------------------------

p43.4a \bq (4) It is assumed that the surface is made up of facets tilted at a
variety of angles which have no preferred direction in azimuth but can be
described by a Gaussian distribution in zenith angle. \eq 

p43.0b \bq No other assumptions about the geometry of the roughness are
made. The present treatment attempts to keep the morphology as general as
possible.

The general procedure of the derivation will be as follows. First, expressions
which are mathematically rigorous will be derived and the parameters necessary
for their evaluation defined. Since the effects are maximum at grazing
illumination and viewing, these expressions will be evaluated exactly under
these conditions. To obtain useful approximations the equations will also be
evaluated for vertical viewing and illumination. The resulting expressions will
be connected by analytic extrapolation to give a solution for intermediate
angles.  \eq

p43.6b \bq The normals to the facets are tilted at various angles $\theta$ with
respect to the local vertical ...  Let $a(\theta)$ be the function which
describes the distribution of tilts.  ...  Let the slope distribution function
be normalized so that 
\qb \int_0^{\pi/2} a(\theta) \ d \theta \ = \ 1 \qeq (4) \qe
 and characterized a mean slope angle $\overline{\theta}$  defined by 
\qb  \tan \overline{\theta} =(2/\pi) \int_0^{\pi/2} \tan \vartheta \ a(\vartheta) \ d \vartheta \ \qeq (5, 12.5) \qeq  \eqq   \qe 

Where two equation numbers are given, the first is for \qcite{Hapke84} and the
second for \qcite{Hapke93}.

p50.3: \bq Define the average value of any function $ F(\theta)$ as $ \langle F(\theta) \rangle = \int _0^{\pi/2} F(\theta ) a(\theta ) d\theta $ . \eq

p50.9a \bq Since (43) contains only averages of trigonometric functions weighted
by $a(\theta)$ the equations are insensitive to the exact form of $a(\theta)$
used. Two distribution functions, Gaussian and exponential, which have been
widely used in planetary applications (Saunders, 1967; Hagfors, 1968) were
investigated. The numerical differences between the two turned out to be quite
minor; hence, the Gaussian was chosen. \eq

p51.5a:  ... the distribution of slopes on any vertical cut with arbitrary azimuth
through the surface \bq is assumed to be of the form
$ a_1(\theta ) \ d \theta  = A \exp \left[ - \tan^2 \theta /  B \tan^2 \overline{\theta} \right] \ d \ ( \tan \theta ) $, 
where $A$ and $B$ are constants. In general, if $ a_1(\theta ) \ d \theta$ is the 
one-dimensional distribution, then the corresponding two-dimensional, 
azimuth-independent slope distribution function is  
$ a_1(\theta )  \sin \theta d \theta d \zeta$  (Hagfors, 1968). 
Thus $ a(\theta )$ is taken as 
\qb a( \theta) = A \exp \left[ - \tan^2 \theta \ / \ B \tan^2 \overline{\theta} \right] \sec^2 \theta   \sin \theta . \qeq(44) \qe

Assumption (2), that the mean slope $ \overline{\theta}$ is not too large, will
now be used for the first time.  ... it is found that to second order in $ \overline{\theta}$,  
\qi $A=2/\pi \tan^2 \overline{\theta} $
\qi $B=\pi$  \ \  \eq

 And he proceeds to use approximations to second order in
 $\overline{\theta}$. For the remainder of Hapke roughness development I use his
 book as the formulation is clearer.

\subsection{Jakosky 1990} %---------------------
\qcite{Jakosky90} Measures brightness temperatures of natural and artificial
surfaces at several angles near simultaneously. Rougher surfaces showed the
strongest incidence and azimuth angle effects.

\subsection{Hapke 1993} %-------------------------------------
  \qcite{Hapke93} is a book compiling his work on scattering. 

Distinguishes ``tilt shadow'', when a surface normal is more than 90\qd~ from
the illumination or observation direction, from ``projected shadow'' when
another part of the surface obstructs the ray.  KRC 3.4 fully accounts for
tilt-shadow but does not treat projected shadow.

 \qcite{Hapke93} chapter 12 is mild rewrite of \qcite{Hapke84}.

p326.6 \ \bq Then, in general (Hagfors, 1968) if $a'(\vartheta)$ is the
one-dimensional function that describes the distribution of slopes on any
vertical cut through the surface made at an arbitrary azimuth angle, the
corresponding two-dimensional, azimuth-independent distribution function can be
written in the form 
\qb a(\vartheta) \ d \vartheta \ d \zeta 
= a'(\vartheta) \sin \vartheta \ d \vartheta \ d \zeta. \qeq (12.1) \qe
It will be assumed that $a'(\vartheta)$ can be described by a Gaussian
 distribution of the form
\qb a'(\vartheta ) \ d \vartheta  = \mathcal{A} \exp \left[ - \mathcal{B} \tan^2 \vartheta \right] \ d (\tan \vartheta), \qeq (12.2) \qe
 where $\mathcal{A}$ and $\mathcal{B}$ are constants to be determined. Then

\qb a(\vartheta ) =\mathcal{A} \exp \left[ - \mathcal{B} \tan^2 \vartheta \right]
\sec^2 \vartheta  \sin \vartheta \qeq (12.3) \eqq \qe

Normalizing so that the total distribution over 0:$\pi/2$ is unity, he finds 
 \qb \mathcal{A}=2/\pi \tan^2 \overline{\theta} \mc{and} \mathcal{B}=1/\pi \tan^2 \overline{\theta} \qeq (12.6a,b) \qe 

Using (12.6) yields the same relation as in \qcite{Hapke84} near Eq. 44.; a
relation similar to that shown in \qcite{Shepard95} Eq. 13 and
\qcite{Bandfield15} Eq. 1; the latter is repeated reformed here for convenience
\qb P(\theta)=\frac{\tan \theta}{\tan^2 \theta_0} \exp \left[ -\frac{\tan^2 \theta}{2 \tan^2 \theta_0} \right]  \qe

Note:  substitution of (12.6) into (12.3), yields 
\qbn a(\theta ) = 
 \frac{2}{\pi} \cdot  \frac{1}{\tan^2 \overline{\theta} } \exp \left[ -\frac{\tan^2 \vartheta}{\pi \tan^2 \overline{\theta}} \right] \cdot \sec^2 \theta   \sin \theta \qen
identical to 1984 work with substitution of $A$ and $B$ into (44). Substituting this result into (12.5), I get

\qbn \tan \overline{\theta}  = \frac{2}{\pi} \int_0^{\pi/2} \frac{2}{\pi} \cdot 
\underbrace{ \frac{\tan \vartheta}{\tan^2 \overline{\theta} } \exp \left[ 
- \overbrace{ \frac{\tan^2 \vartheta}{\pi \tan^2 \overline{\theta}} }^{Z} \right]}_{\sim P(\theta)} \sin \theta \cdot \underbrace{ \sec^2 \theta \ d \vartheta}_{d \ \tan \theta} \ql{mytb}
This has the same form but different constants, including in the exponent, as
the above two citations. It produces a normalized distribution distinct from
them because of the $\pi$ rather than 2 in the exponent. $Z$ is a term that differs between \qcite{Hapke93} and \qcite{Bandfield15} and which must be constrained to avoid underflow.

% if and only if those are weighted with $\Delta \tan \theta$ rather than $\Delta \theta$ .

\vspace{3.mm}

P 331.8 \ \bq ... we will try to write the rough-surface bidirectional
reflectance $r_R(i,e,g)$ as the product of a shadowing function $S(i,e,g)$ and
the bidirectional reflectance $r(i_e,e_e,g) $ of a smooth surface of effective
area $A_e$ tilted so as to have effective angle of incidence $i_e$ and angle of
emergence $e_e$, and with the same phase angle $g$. That is, we will seek
expressions for $i_e(i,e,g)$, $e_e(i,e,g)$, and $S(i,e,g)$ that will make the
following equation true: 
\qb r_R(i,e,g)= r(i_e,e_e,g)S(i,e,g) \qeq (10,12.10) \eqq \qe 

 In chapter 12, Photometric effects of large-scale roughness, at the start of 
his derivation, \qcite{Hapke93} states:   
p 333.7 \\
\bq We shall follow Sanders (1967) and assume that any facet that is not in a
tilt shadow has a statistical probability of being in a projected shadow that is
independent of the slope or azimuth of its tilt. \eq

 Eq. (12.29): relative azimuth dependence $f(\Psi)=\exp
  \left( -2 \tan \frac{\Psi}{2} \right)$, which is a pragmatic function that
  \bq decreases linearly from a value $f(0)=1$ to
  $f(1)=1-\frac{2}{3}=\frac{1}{3}$ and then decreases to zero as $\Psi
  \rightarrow \pi$ \eq.

  Finds solutions for two conditions: first \textbf{when} $\mathbf{i \leq e}$

 \qb S(i,e,\psi) = \frac{\mu_e}{\mu_e(0)} \frac{\mu_0}{\mu_{0e}(0)} 
\frac{\chi(\overline{\theta})} {1-f(\psi) + f(\psi) \chi(\overline{\theta})Z} \qeq (12.50) \qe
 where my $Z$ is his $[ \mu_0 / \mu_{0e}(0)] $

\qb f(\psi)=\exp (-2 \tan \frac{\psi}{2}) \qeq (12.51) \qe

\qb \chi(\overline{\theta}) = \langle \cos \vartheta \rangle = 1/\sqrt{1 + \pi \tan^2 \overline{\theta}  } \qeq (12.45a) \qe obtained by assuming $(\overline{\theta}$ is small and expanding to second order, as discussed before Eq (12.41a)

\qb E_1(x)=\exp \left( -\frac{2}{\pi} \cot  \overline{\theta} \cot x \right) \qeq (12.45b) \qe

\qb E_2(x)=\exp \left( -\frac{1}{\pi} \cot^2  \overline{\theta} \cot^2 x \right) \qeq (12.45c) \qe

\qb \mu_{0e}(\psi)=\chi(\overline{\theta}) \left[ \cos i +\sin i \tan \overline{\theta}
\overbrace{\frac{\cos \psi E_2(e) + \sin^2(\psi/2)E_2(i)}{2-E_1(e) -(\psi/\pi)E_1(i)} }^{r46}\right]   \qeq (46,12.46) \qe

\qb \mu_e(\psi)=\chi(\overline{\theta}) \left[ \cos e +\sin e \tan \overline{\theta}
\overbrace{\frac{E_2(e) - \sin^2(\psi/2)E_2(i)}{2-E_1(e) -(\psi/\pi)E_1(i)} }^{r47} \right]   \qeq (47,12.47) \qe

In the above two equations I have added the $(\psi)$ to the left side for clarity

\qb \mu_{0e}(0)=\chi(\overline{\theta}) \left[ \cos i +\sin i \tan \overline{\theta}
\frac{ E_2(e) }{2-E_1(e)} \right]   \qeq (48,12.49) \qe

\qb \mu_e(0)=\chi(\overline{\theta}) \left[ \cos e +\sin e \tan \overline{\theta}
\frac{ E_2(e) }{2-E_1(e)} \right]   \qeq (49,12.48) \qe

 \textbf{When} $\mathbf{ e \leq i}$:
 
\begin{itemize}      % ticked items    
\item 
 (12.46) is modified by replacing the term r46 with r47 and exchanging $i \Leftrightarrow e$ therein to make  (12.52 )
\item    (12.47) is modified by replacing the term r47 with r46 and exchanging $i \Leftrightarrow e$ therein to make  (12.53)    
\item my $Z$ in (12.50) becomes $[ \mu / \mu_e(0)] $ to make (12.54)

\end{itemize}
These equations were implemented in \np{hapkes.pro}; results covering a wide
range of $i$,$e$, and $\psi$ are shown in Figures \ref{hapkes} and
\ref{hapkes2}. The function generally decreases with increaseing incidence angle
and the variation alnog azimuth increases with increaseing emergence angle.

The increase toward large azimuths for high emergence angles seems non-physical.

\begin{figure}[!ht] \igq{img/hapkes}
\caption[Hapke shadow function]{Hapke shadow function for incidence and
  emergence angle of 5 to 75\qd~ and relative azimuths 5 to 175\qd~ for a
  roughness mean slope of 30\qd. Each color (see left legend) is for a single
  incidence angle; emergence angle is indicated by line type (see right legend)
\label{hapkes}  hapkes.png  }
\end{figure} 
% how made:

\begin{figure}[!ht] \igq{img/hapkes2}
\caption[Hapke shadow function]{Hapke shadow function, see caption for
  Fig. \ref{hapkes}. Each curve (see legend) is for a single incidence angle;
  azimuth increase radiply to the right repeatedly, each section is for an
  emergence angle indicated by labels near the bottom of the chart.
\label{hapkes2}  hapkes2.png  }
\end{figure} 
% how made:

The azimuth function is shown in Fig.  \ref{hida}
\begin{figure}[!ht] \igq{img/hida}
\caption[Azimuth functions]{The azimuth function $f(\Psi)$ of \qcite{Hapke93},
  Eq. (12.29).  Below 0.1, the solid line shows the approach to 0 in the forward
  scatter direction on a linear scale and the dashed line on a logrithmic scale
  (20 orders of magnitude; scale at the left margin. The $-\cdots$ line shows
  the simplistic function with continuous derivatives $\frac{1+ \cos \psi}{2}$
  and the dotted line is a compromise function $\frac{1+ \cos (\psi (1+.3 \sin
    \psi)) }{2}$ closer to $f(\Psi)$. \label{hida} hida.png }
\end{figure} 
% how made: hiding.pro parr[2]=-2

\subsection{Shepard 1995} %-------------------------------------

Based on the concept of fractal sufaces.  
Fractal surface statistics based on uniform separation of points in the
horizontal plane, so that a $\cos \theta $ term is not needed to in finding the
mean angle or slope.

 In \S 2.1 defines several terms: \bq

5. Variance $( \sigma^2 )$. The variance of all points on a surface or profile aboutn the mean value: 
\qb \sigma^2= \langle [ z- \overline{z} ]^2 \rangle \qeq (1) \qe


9. Slope histogram $(P(s) = P( \tan \theta) )$ . A function describing frequency
or number slopes within any angular interval.  Note that s is the slope or
tangent of slope angle $ \theta) $ 0.

10. Slope density function or slope distribution $(p(s) = p( \tan \theta) )$.  A
function describing frequency or number of slopes within any solid-angle
interval. The slope density function is essentially the slope histogram
normalized by solid angle. The difference will become apparent when
unidirectional and adirectional slope distributions are discussed. \eq 


In \S 3.3 discusses three slope distributions of fractal surfaces:
\begin{description}  % labeled items    
\item [Unidirectional]  
p11714.3 b \bq Efford [1990] notes that synthetic fractal surfaces have
unidirectional slope histograms that are accurately represented by Gaussian
functions. \eq

p11714.5 b \bq Therefore, the unidirectional slope distribution of a fractal
surface is Gaussian, with (slope) variance $<(\Delta z / \Delta x)^2>$ .  The
rms slope is just the square root of this variance and is typically converted
to degrees by taking the inverse tangent. \eq

p11714.7 b \bq Although seemingly a trivial distinction, the unidirectional
slope angle distribution of a fractal surface will not be Gaussian, except in
the limit as $\theta _\mathrm{rms} \rightarrow 0 $. This is because the inverse
tangent is increasingly nonlinear at higher slopes. \eq

 \item [Adirectional] Closely follows a Rayleigh distribution: 
\bq ... defined by \qb P(s)=\frac{s}{s_0^2}e^{-(s^2 / 2s_0^2)} \qeq (13) \qe
where $P$ is the slope histogram function, $s$ is the slope of the surface
normal polar angle, and $s_o$ is the mode of the distribution, equal to the
unidirectional rms slope. \eq

 \item [Effective] \bq We define the effective surface slope following Campbell
and Garvin [1993] as:
\qb s_\mathrm{EFF}=\frac{\sqrt{2} \sigma}{l} \qeq (18) \qe
where $l$ ... is the autocorrelation length \eq

\end{description}

Refers to:  Simpson, A., and G. L. Tyler, Radar scattering laws for the lunar surface. IEEE Trans. Antennas Propagat., AP-30, 438-448, 1982


\subsection{Bandfield 2008} %-------------------------------------
\S 2.4, \P 2 \bq We chose to use the $\theta$ -bar parameter described by Hapke
(1984) as a model for surface slopes (Fig. 2). This model is based on a Gaussian
distribution of slopes along a surface cross-section that is expanded to a full
three dimensional surface assuming that the azimuths of slopes are random. This
model is independent of length scales and can be described using a single
parameter. While this simplicity is well suited for the derivation of surface
slope distributions from the data, it is not adequate for geological
interpretation of surfaces as it does not account for the scale at which the
roughness occurs. \eq

\subsection{Bandfield 2015} %-------------------------------------
\qcite{Bandfield15} uses the relations of \qcite{Smith67} for shadowing at nadir
view. The relative azimuth dependence for off-nadir illumination and viewing comes from \qcite{Hapke84}.

p 358.9b: \S 2.1 \\ 
\bq Our thermal model assumes radiative equilibrium to predict surface
temperatures. Sloped surfaces include a downwelling radiative component
proportional to the fraction of the hemisphere filled by adjacent surfaces
rather than space. To maintain computational efficiency, the solar and thermal
infrared downwelling radiance is calculated assuming it is coming from a
horizontal surface at radiative equilibrium. \eq

p 359.4a \S 2.2. Surface roughness model \\ 
\bq We use a simple Gaussian roughness model
that is similar to that described by Helfenstein and Shepard (1999) and has been
used for comparison to martian TIR datasets (Bandfield and Edwards, 2008;
Bandfield, 2009). Surface temperatures are predicted for slopes of 0–90\qd~ at
2\qd~ intervals and azimuth orientations of 0–360\qd~ at 20\qd~ intervals. The
radiance of each slope/azimuth combination is calculated, and its contribution
to the total modeled radiance is weighted by the statistical probability of its
occurrence. This reduces the surface slopes/roughness to a single parameter (RMS
slope) that is independent of length scales while maintaining reasonable
fidelity to natural surfaces.

The probability distribution $P$ for a given slope angle, $\theta$, is described
by the following (derived from Eq. (13) of Shepard et al. (1995)):

\qb P(\theta)=\frac{\tan \theta}{\tan^2 \theta_0} \cdot e^{-\frac{\tan^2 \theta}{2 \tan^2 \theta_0} }  \qeq (1) \qe

where $\theta_0$ is the tangent of the RMS slope angle. This describes the
adirectional distribution of slopes, which closely approximates a
Gaussian distribution of unidirectional slopes for a RMS slope angle
of $\theta_0$ (Shepard et al., 1995). For our purpose, the azimuth direction
for the slope of any given surface has no preferential orientation
and our measurements are sensitive to an adirectional distribution
of slopes with random azimuths rather than the slope distribution
along any particular transect or orientation. \eq

The above statement `` $\theta_0$ is the tangent of the RMS slope angle'' seems to be in error as his Eq. 1 would then be taking the tangent of a tangent.

p 359.9a  \\ \bq 
We use a shadowing approximation methodology modified from that developed by
Smith (1967) and Hapke (1984) . This model assumes that any slope/azimuth
surface that would otherwise be illuminated has the same statistical chance of
being within a shadow cast by another surface. This is a relatively simple
approach for nadir observations described by the following (derived from
Eqs. (21) and (24) of Smith (1967)):

\qb S(\theta)=\frac{1-\frac{1}{2}  \erfc \left( \mu / \sqrt{2} \omega \right) }
{\frac{1}{2} \left( \sqrt{2/\pi} \cdot \frac{\omega}{\mu} e^{-(\mu^2/2\omega^2)}
 - \erfc \left( \mu / \sqrt{2} \omega \right) \right) +1 } \qeq (2) \qe

where $S(\theta)$ is the fraction of the shadowed surface, $\mu$ is the
cotangent of the solar incidence angle, $\omega$ is the RMS surface slope
distribution, and $\erfc(x)$ is the complementary error function (the integral
between 0 and x of a Gaussian distribution function).

This equation can only be applied as is to nadir observations where no surface
is hidden from the spacecraft view (assuming no overhanging surfaces). \eq



p 359.4b \\ 
 To account for viewing orientation effects, we use an approach similar to that
 described by Hapke (1984). Where the viewing emission angle is greater than
 the solar incidence angle, the shaded fraction observed at nadir,
$ S_{nadir} (\theta)$, is modified by the following:
\qb S_{view} (\theta) = S_{nadir} \cdot \left( 1 - e^{-2\tan \left[ \frac{1}{2 }\left( \phi_{sun}-\phi_{obs} \right) \right] } \right) , \qeq (3) \qe

where $S_{view} (\theta)$ is the fraction of shadowing that is present within
the field of view, $\phi_{sun}$ is the azimuth angle of the Sun with respect to
the surface, and $\phi_{obs}$ is the azimuth angle of the spacecraft with
respect to the surface. Where the viewing emission angle is less than the solar
incidence angle, the following applies:

\qb S_{view} (\theta) = S_{nadir} -S_{obs} \cdot e^{-2\tan\left[
    \frac{1}{2}\left( \phi_{sun}-\phi_{obs} \right) \right] } \qeq (4) \qe 

where $S_{obs}$) is the shadowing function described in Eq.  (1) , except as
applied to the viewing incidence angle of the spacecraft rather than the Sun.
 \eq

\S 3.2.3, p360.6b \\ \bq
Modeling indicates that surface roughness has little effect where solar
incidence is less than $\sim 30$\qd, and these observations can be considered
free of its interfering effects (Davidsson et al., submitted for publication) \eq

Fig 6 has some effects of slopes of 10,15,20,15 \qd.

\S 4.2.1. Day, p 363.9a \\ \bq 
The six daytime EPF observations acquired with moderate angles of
solar incidence (Table 2 and Fig. 7) all show a distinct asymmetry in
brightness temperature about the nadir observation.  In each case, the
measurement pointing equatorward has lower brightness temperatures than the
equivalent emission angle observation that points poleward. \eq

p 363.4b \\ \bq 
As expected, both the model and Diviner measurements show a trend of increasing
temperature differences at higher opposing emission angles. Temperature
differences reach $\sim 63$ K for the EPF observation near 98\qd E, 52\qd S at
opposing 7\qd~ up- and down-track emission angle observations. The modeled
temperature differences also predict increasing temperature differences for
higher RMS surface slope distributions. The six daytime EPF sequences compare
most closely with modeled surface slope distributions of $\sim 25 - 35 $ \qd
(Fig. 9). \eq

\section{Outline of pragmatic formulations}
Divided into two major sections
\qi Distribution of slopes; i.e. the physical surface independant of its temperature 
\qi Temperatures of the seen surface, and effect upon that of the far field it sees

Distributions should cover the complete range of physical likelyhood. E.g. dips
(slope relative to horizontal) from 0 to 90\qd, will ignore overhangs; all
azimuths, both abolute ( e.g., from North) and relative ( from the plane of
incidence). Functions must be continuous, and it is desireable that at least the
first dervative be continuous. Yet, the presence of ``ponded deposits'' of zero
slope would require a discontinuous first derivative in slope abundance


\subsection{Devils enumeration of approximations} %----------------------------
\begin{description}    % numbered items  
\item [UniAlb] Material albedo is the same for all slopes. \ \   Unknown.
\item [UniPhot] Material reflectance function is the same for all slopes. \ \   Unknown.
\item [UniLay] Material subsurface properties are the same for all slopes. \ \   Unlikely but magnitude unknown.   
\item [SlopeElev] Slope probablility is not related to elevation.  \ \ Depends upon geology.
\item [ffT] Far-field temperature is the same at all azimuths. \ \ Poor; address. 
\item [ShadeTilt] Temperature in the shade independent of tilt. \ \  Poor; address.
\item [ShadeTime] Temperature in the shade independent of hour.  \ \ Poor.
\item [EmissLamb] Thermal emission of a facet is Lambertian. \ \  Reasonable, accept. 
\item [AbsAzi] Absolute azimuth does not matter. \ \  Thought to be excellent.
\item [KofT] Conductivity not a function of temperature. \ \ Addressed by KRC.
\item [Homog] Material is homogeneous with depth.  \ \  Addressed by KRC.
\item [Sphere] Object is spherical. \ \ Knowable and addressable; harder if concave. 
\item [Geography] Object is the same at all locations (lat/long) / / OK initially. Addressable by mapping models. 

\end{description}

\subsection{Distribution of elevations and slopes} %----------------------------

I generate the IDL routine \np{hiding.pro} to compute shadow and hiding functions;  negative incidence angles generate various plots.

 
 [concave / convex] As a multiplier of the theoretical distribution

\subsection{RMS slope of a sphere} %-------------------------------------
Using \qcite{Hapke84} definition of mean slope angle (his Eq. 5), the mean slope of the visible half of a sphere (or a hemispherical crater) is  
\qbn \overline{\theta} \equiv \int_0^{\pi/2} \tan \theta  \cdot \sin \theta  \ \cos \theta  \ d \theta 
\Rightarrow  \int_0^{\pi/2} \sin^2 \theta \ d \theta 
\Rightarrow \bigg[_0^{\pi/2} \frac{\cos^3 \theta }{3}- \cos \theta 
= \frac{2}{3} \mc{and} \theta=33.7^\circ \qen

\subsection{Probability of hiding} %----------------------------

\subsection{Aside on Hiding}  %-------------------------------------

A hiding function can be used on the incidence angle $i$ to calculate the
likelyhood that a tile segment in shadow; but existing functions say nothing
about how long has it been in shadow.  A hiding function can be used on the
emergence angle $e$ to calculate the population of tilt facets that the viewer
sees. The hiding function should be symmetric with respect to $i$ and $e$.

In the morning, a facet it has probably been in shadow through the prior night,
so using the pre-dawn temperature would be appropriate. In the afternoon, would
need to estimate when that tilt might have entered shadow; this would require
integrating the probability that it was in shadow from noon to the current
time.

\subsubsection{Implication of geology}
Hapke shadowing formalism is used by several authors. I have a problem with the
p333.7 assumption as my experience is that in rugged terrain steeper slopes have
a lower probability of being shadowed; this is partially because steep slopes
are commonly higher in the terrain where there is less terrain above them to
generate debris that would bury them to make shallower slopes! This results when
surface degradation rates are relatively uniform over area, as for example by thermal fracturing.

One inportant exception on Earth includes ``hill and valley'' terrain, where
erosion is dominated by rainfall and fluid flow, a strongly non-linear process,
and surfaces become convex upward apart from the negative cusp of the
stream/river itself.

Since this is an effect of gravity, it might be less important on asteroids; but
the modest set of high-resolution images from small bodies (e.g., the Rosetta
mission, 433 Eros imaging by NEAR) suggest that gravity has some effect on them
as well. Diurnal thermal cycling will tend to disaggregate blocks larger than a
modest fraction of the diurnal skin depoth. Disaggregation and motion due to
seismic waves from impacts is discussed by \qcite{Richardson05}.

This leads me to wonder if a ``hiding'' function weighting visibility as a
function of facet slope would be appropriate.  

For dark objects, where multiple scattering is unimportant, visibility
increasing with slope would lead to limb brightening relative to a Lambertian
sphere.
 
\subsubsection{Discussion}

If radiative equilibrium is assumed, then all solar shadows are at 0K; modelers
usually assume some realistic fixed value, \qcite{Bandfield15} uses 100K.
 
 To improve on this would require treating what tilt is in the shadow and how
 long it had been is shadow.

 A limiting realistic case is to set the temperature of a tilt that is in shadow
 to the minimum diurnal temperature for that site and conditions. This is not
 impractical with the KRC formalism used here.
 
\subsection{Distribution of tilt within visibility} %---------------------------
Visibility $V$ is here defined as simply 1.-hiding, or $V\equiv 1-H$

Allow a modification of hiding related to tilt under the requirement that the
total hiding or visibility does not change.

$H$=$H_0 F(x)$ where $x$ is the dip angle of a facet, and require
$\int_0^{\pi/2} F(x)p(x) =1$ where $p(x)$ is the population function of dips and
$\int_0^{\pi/2} p(x) =1$.

A simple implemenation is $F$ as a Chebyshev polynomial of the first kind in
$x=2p/p_\mathrm{max} -1$; i.e., $Y=c_0+c_1x+c2(2x^2-1) + ...$, $F=
Y/\int_0^{\pi/2} Y(x)p(x)\ dx $ and degree 2 should be enough. Without loss of
generality, can let $c_0=1$ and this will be convenient.  $F$ must not be
negative anywhere.

\subsection{Effect of azimuth} %----------------------------

The azimuth function derived by \qcite{Hapke93}, Fig. \ref{hida}, has a
discontinuous slope crossing the backscatter plane; which might be physical only
for zero phase angle.  A simplistic function that as continuous derivatives and
the same limits as his Eq (12.51) is $\frac{1+ \cos \psi}{2}$.
 
 To account for the tendency of a slope in rough terrain to be facing slopes of
 the opposite azimuth, e.g., a west-facing slope near dusk is likely to view
 warm east-facing slopes in the far field, generate KRC models wherein each tilt
 facet uses a far-field with a tilt near the RMS slope value and an azimuth
 different by 180\qd.

\subsection{Nature of the far field} %----------------------------
\begin{description}  % labeled items   
\item [None] The far field (below the horizon) has zero radiance. This is
  non-physical but is simple to do and has been done in some asteroid papers to
  estimate the magnitude of the far-field effect. KRC has not had this
  capability, but version 3.4 could emulate it by using a far-field file with a
  negligable temperature.
  
\item [Self] The far field is the same temperature as the tilted surface. This
  was done in KRC before version 3

\item [Flat] The far field has the temperature of a flat (zero dip) surface of
  the same material. KRC 3+ has the option for the far field surface temperature
  to be that of any prior run, and normally this would be a run with identical
  materials with zero slope.

\item [Opposite] Assign the far field the temperature of the RMS slope at same
  time of day but 180\qd~ difference in azimuth from the subject facet. This
  could be closely approximated by running a set of KRC models first flat, then
  with the RMS slope using flat as the far field at all azimuths, then a full
  set of tilts with each azimuth using the opposite-azimuth RMS slope model as
  the far-field file.

\end{description}

\section{numerical results}
Slope distributions were calculated several ways, using slope angles every 0.1\qd~ from 0.1 to 89.9. For each distribution, the terms were separated into elements variable with angle, whose product is $F_i$ and all those constant with angle. The 


$d \tan \theta$ was implimented both numerical based on mid-point between each angle $\theta$  and $\sec^2 \theta d \theta$; the two differ by less than one ppt below 88.3\qd/

 \begin{description}  % labeled items  
\item [Hapke]  Equation \qr{mytb}. Constant= $(2. \cdot  2.)/( \pi \cdot \pi \cdot$ $d. \ tan \theta$ numerical based on mid-point between each angle $\theta$ 
\item [Bandfield] Similar to Equation \qr{mytb} except the $\pi$ in exponent
  section replaced with a 2. All else the same as above, His $\theta_0$ is taken
  to be the RMS slope angle
\item [literal]   Similar to Bandfield version, except Bandfield $\theta_0$ is taken literally to be tangent of the tangent of the RMS slope angle. 

\end{description}

\subsubsection{Averaging}
 In each case, first compute a distribution that totals to unity over the set of uniform angles used. All except ``Hapke'' below use this distribution $F$
 
\begin{description}  % labeled items   
\item [Hapke]  As shown in Eq. \qr{mytb}, then convert to angle and to degrees
\item [MnAng]  Mean angle: $ m=\sum_{i=1}^N p_i F_i $
\item [RMSan]  $ r=\sqrt{ \sum_{i=1}^N p^2_i F^2_i }$
\item [RMSsl] 
\item [RMStd] 
\end{description}





\begin{table}
\caption[Mean Roughness]{Mean roughness for three distribtions averaged five ways. See text for explanations   }
\label{tab:ruf}
\begin{center}
\begin{tabular}{ | r r r r r r r r r r | } \hline
i & Theta-bar in & 1.000 & 2.000 & 5.000 & 10.000 & 15.000 & 20.000 & 25.000 & 30.000 \\ \hline
0 & Hapke:Hapke & 0.999 & 1.994 & 4.914 & 9.385 & 13.211 & 16.400 & 19.043 & 21.247 \\
1 & Hapke:MnAng & 1.998 & 3.988 & 9.817 & 18.732 & 26.436 & 33.030 & 38.730 & 43.743 \\
2 & Hapke:RMSan & 2.169 & 4.326 & 10.617 & 20.093 & 28.103 & 34.827 & 40.548 & 45.517 \\
3 & Hapke:RMSsl & 2.169 & 4.331 & 10.694 & 20.573 & 29.270 & 36.791 & 43.288 & 48.941 \\
4 & Hapke:RMStd & 2.504 & 4.992 & 12.214 & 22.930 & 31.782 & 39.051 & 45.115 & 50.286 \\
5 & 2angl:Hapke & 0.508 & 1.014 & 2.516 & 4.904 & 7.086 & 9.037 & 10.769 & 12.306 \\
6 & 2angl:MnAng & 1.595 & 3.186 & 7.891 & 15.322 & 22.065 & 28.104 & 33.526 & 38.446 \\
7 & 2angl:RMSan & 1.731 & 3.456 & 8.544 & 16.493 & 23.579 & 29.815 & 35.328 & 40.265 \\
8 & 2angl:RMSsl & 1.731 & 3.459 & 8.586 & 16.770 & 24.318 & 31.168 & 37.358 & 42.964 \\
9 & 2angl:RMStd & 1.999 & 3.990 & 9.843 & 18.888 & 26.807 & 33.646 & 39.587 & 44.817 \\
10 & 2slop:Hapke & 0.507 & 1.011 & 2.461 & 4.517 & 5.991 & 6.908 & 7.370 & 7.485 \\
11 & 2slop:MnAng & 1.595 & 3.182 & 7.835 & 14.950 & 21.083 & 26.308 & 30.805 & 34.746 \\
12 & 2slop:RMSan & 1.731 & 3.452 & 8.484 & 16.103 & 22.573 & 28.014 & 32.647 & 36.674 \\
13 & 2slop:RMSsl & 1.731 & 3.455 & 8.524 & 16.363 & 23.243 & 29.217 & 34.446 & 39.098 \\
14 & 2slop:RMStd & 1.998 & 3.985 & 9.774 & 18.457 & 25.737 & 31.798 & 36.925 & 41.364 \\
 \hline
\end{tabular} \end{center} \end{table}
%  \multicolumn{#cols}{lcr}{text}


\ref{rough}
\begin{figure}[!ht] \igq{img/rough}
\caption[Roughness models]{Mean angle of roughness for three kinds of slope distributions and 5 measures of the mean. data from Table \ref{tab:ruf}
\label{rough} rough.png  }
\end{figure} 
% how made: beaming, @45, bbb=[1.,2.,5.,10.,15.,20.,25.,30.] @451,454

\bibliography{heat,moon,mars}   %>>>> bibliography data
\bibliographystyle{plain}   % alpha  abbrev 
\end{document} %===============================================================

\ref{}
\begin{figure}[!ht] \igq{img/}
\caption[]{
\label{}  .png  }
\end{figure} 
% how made: 

% \pagebreak

\appendix %====================================================================

\subsection{bits}
For a fractal surface Gaussian Uni-directional ...
