\vspace{-3.mm} 
\begin{verbatim}
Type 52 is a "bin5" file; this has an ASCII header followed by a N-dimensional
binary array whose dimensions and word-type are defined in the header; for
type 52, the number of dimensions is 5 and the type is REAL*4 or REAL*8. The 4th
dimension is increased to allow room for a "prefix" to be embedded in the binary
array for each case.

The array is written by the tdisk[8] routine, which stores values for each 
season for each case in the large buffer FFF.   

Type 52 = (N24 hours, 7 items, N4 latitudes, NDX+seasons, cases)
The 7 items are:  
1)=TSF      Final hourly surface temperature
2)=TPF      Final hourly planetary temperature
3)=TAF      Final hourly atmosphere temperature, not predicted
4)=DOWNVIS  Hourly net downward solar flux [W/m^2]
5)=DOWNIR   Hourly net downward thermal flux [W/m^2]
6) packed with: 
  NDJ4     Number of days to compute solution
  DTM4     RMS temperature change on last day
  TTA4     Predicted final atmosphere temperature   
  TIN(2:n) Minimum hourly layer temperature, starting with first real layer
        n is the smaller of: N1    (the number KRC computed  
                             N24-2 (limit of what fits in this file)   
7) packed with: 
  FROST4   Predicted frost amount, [kg/m^2]
  AFRO4    Frost albedo (at the last time step)
  HEATMM   Mean upward heat flow into soil surface on last day, [W/m^2]
             This would have contributed to sublime frost-cap if it were present
  TAX(2:n) Maximum hourly layer temperature. Parallel to TIN

Type 52 allows multiple cases, each with a "prefix" for each case stored in the
NDX leading extra seasons. This region contains:
  4 integers (converted to Real) that define sizes
   (1)=FLOAT(NWKRC)   Number of 4-byte words in KRCCOM, currently 255
   (2)=FLOAT(IDX)     1-based index of the dimension with extra values
   (3)=FLOAT(NDX)     Number of those extra
   (4)=FLOAT(NSOUT)   Number of output seasons  (Not used; could be redefined)
 followed by KRCCOM, defined in krccom.inc or krccom8.f
 followed by a sub-array (seasons,5)     (0-based index)
    0]=DJU5   Current Julian date (offset from J2000.0)
    1]=SUBS   Seasonal longitude of Sun, in degrees
    2]=PZREF  Current surface pressure at 0 elevation, [Pascal] 
    3]=TAUD   Mean visible opacity of dust, solar wavelengths
       If a climate model is used, value if for the last latitude.
    4]=SUMF   Global average columnar mass of frost [kg /m^2]  (If computed)

 Thus the prefix requires NPREF = [255 or 426]+4 +5*nseas  words;
 where nseas is the number of seasons output: NJ5-JDISK+1

Each season contains N24*7*N4 words, the number of leading pseudo-seasons is 
  NDX = Ceil ( NPREF / (N24*7*N4) )

For Type 52, the size of a case is set by the first case.  The number of cases
allowed is set by this size and printed as MASE at the end of the first case in
the print output.

KRC input items that would change any of the bin5 dimensions are not allowed to
increase between cases; i.e., N24, N4 and nseas=N5-JDISK. An invalid change of
these will be detected in tdisk.f; a note will go to the print file and the
error file, the output file will be written with any cases completed up to this
point and the file closed. All remaining cases will be computed but not saved.

The number of cases that can be stored is dynamic and fairly liberal; recent
versions of KRC reserve 10 M words for the bin5 array. So, for example, with
N24=24, 19 latitudes and 50 stored seasons, up to 61 cases can be saved in one
run.

In the IDL readkrc52 routine, these are expanded into 5 arrays and 
a structure. The dimensions here are typical; produced in krcvtest @188

TTT             FLOAT     = Array[24, 5, 3, 120, 8]
(hour,item,latitude,season,case)
itemt =  Tsurf, Tplan, Tatm, DownVIS, DownIR

DDD             FLOAT     = Array[21, 2, 3, 120, 8]
(layer,item,latitude,season,case)
itemd =  Tmin, Tmax

GGG             FLOAT     = Array[6, 3, 120, 8]
(item,latitude,season,case)
itemg =  NDJ4, DTM4, TTA4, FROST4, AFRO4,, HEATMM

UUU             FLOAT     = Array[3, 2, 8]
(nlat,item,case)       Values often the same for each case
itemu =  Lat., elev

VVV             FLOAT     = Array[120, 5, 8]
(season,item,case)   First 2 item values often the same for each case
itemv =  DJU5, SUBS, PZREF, TAUD, SUMF

KRCCOM is in kcom:  readkrc52 returns values for the first case
** Structure <cdcab8>, 7 tags, length=1020, data length=1020, refs=1:
   FD              FLOAT     Array[96]
   ID              LONG      Array[40]
   LD              LONG      Array[20]
   TIT             BYTE      Array[80]
   DAYTIM          BYTE      Array[20]
   ALAT            FLOAT     Array[37]
   ELEV            FLOAT     Array[37]

For the REAL*8 version, the order is changed
   FD              REAL*8    Array[96]
   ALAT            REAL*8    Array[37]
   ELEV            REAL*8    Array[37] 
   ID              LONG      Array[40]
   LD              LONG      Array[20]
   TIT             BYTE      Array[80]
   DAYTIM          BYTE      Array[20]
\end{verbatim}
