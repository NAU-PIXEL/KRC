%\title{Description of the bin5 file system -i}
\begin{verbatim}
        Description of the "bin5" and "bin8" file system  
               Hugh Kieffer 2020jan16

The bin5 file system is designed to allow transfer of binary arrays between
different languages, operating systems and hardware types. Each bin5/8 file
contains a leading 512 byte (or multiple thereof) region that is ASCII text
and describes the array which follows it. .bin5 files are written/read by 
bin5.pro (single binary array) and .bin8 by bin8.pro (up to 3 binary arrays); 
bin8.pro can read/write .bin5 files by not using its default extension.

The first few words are integers separated by spaces, these define 
the array sizes and type using the IDL 'size' convention:
  The first integer is the number of dimensions to the array, N.
  The next N integers are the sizes of each dimension, in the order of 
    most-rapidly varying index first, 
Then comes an integer that defines the word type; this follow the IDL
convention. Word types greater than 5 are not supported in Fortran.
       1	Byte	
       2	Integer	
       3	Longword integer	
       4	Floating point	
       5	Double-precision floating	
       6	Complex floating	
       7	String	
       8	Structure	
       9	Double-precision complex	
      10	Pointer	
      11	Object reference	
      12	Unsigned Integer	
      13	Unsigned Longword Integer	
      14	64-bit Integer	
      15        Unsigned 64-bit Integer

The next integer is the number of elements in the array, this is redundant with
the product of the dimension sizes.

Files written after 1999march should have the next integer that indicates the
number of leading ASCII bytes in the file; this is always a multiple of 512.
Bin8 files can then have 1 or 2 more sets of size definitions for the 2nd and 
3rd binary arrays.

Next is some text bounded by << .. >>; the first part will usually be
"<<IDL_SIZE + headlen " , the latter part should be the creation date of the
file, e.g.," Tue Jul 9 05:08:18 2002 >>"

This is followed by the "header", which is free text describing the file. Files
 may contain Keyword=value sections and/or embedded small arrays. Embedded
 arrays use a separator,such as |,#, or ^, between each element, the separator
 is doubled to indicate the limits of the embedded array.  E.g.,
 Geom=||7.|0.|7.|0.|0.|1.|| Such arrays are created/read by strum.pro

Keyword=value: Keywords must be unique in all the header text. Values are 
defined as starting at the first non-blank after the next "=" and ending just 
before the next non-blank  or  matching delimiter.
If the first non-blank character after '=" is a valid starting delimiter, values 
separated by commas constitute a value-array. Valid delimiters are: 
single quotes, double quotes, matching left/right parens or brackets.

The last 5 bytes of the ASCII section should always be "C_END"; these should be
immediately preceeded by a 5- or 8-byte indication of the source hardware
architecture, e.g. "x86 ".

It may be convenient to define an alias to look at the ASCII section:
  alias bhead  'dd count=1 if=\!*'  # display first 512  bytes of one binary file
  alias bhead2 'dd count=2 if=\!*'  # display first 1024 bytes of one binary file

Versions are available in  IDL, Fortran (uses calls to C routines)

Example: 

 2 32 2 4 64 512 <<IDL_SIZE + headlen Tue Jul  9 05:08:18 2002 >>synmoonspec: 
Multiply factor for ROLO irrad model r311f and Solar model= solar_bb. Synthetic 
spectrum: Match mean and StdDev of Apollo with Breccia fraction: 0.05. Geom=
||7.|0.|7.|0.|0.|1.||                                                          
                                                                               
                                                                               
                             x86_64  C_END
 
This example defines a single-precision floating-point array sized (32,2)


BIN8,R: If there are more arguments than arrays in the file, 
            those args are returned as -1
  If there are more arrays in the file than arguments, those arrays are ignored. 
bin8.pro is coded in a fashion that would allow simple extension to more arrays.

\end{verbatim}
