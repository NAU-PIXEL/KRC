
\section{Far-field Detailed design \label{fffd}}

Enhancement: Option for sloped models to use temperatures from a far-field model
(first case, if it was run and stored) for the far ground.

Terminology:
\begin{description}  % labeled items   
\item [self mode] Radiation from the far field horizontal surface assumed to be
  at the same temperature at the sloped surface, and of the same albedo. The
  atmosphere temperature is calculated based on sloped-surface temperature. This
  was KRC's only mode until version 3.4
   
\item [far mode] Radiation from the far field surface is based on
  prior KRC run. Atmosphere down-going radiation also based on this prior run.

\item [far-field file == fff] A file that contains the temperatures from a model
  of the ``remote'' surface and atmosphere as needed by a sloped case. A fff
  contains the values computed on the last day, not extrapolated to the end of a
  season

\item [``flat'' case] A zero-slope case that matchs a sloped case in all
  parameters except slope and azimuth.

\end{description}

Sequence in the software set:

\begin{itemize}      % ticked items  

\item TCARD reads the file name. 

\item LOPN3=true indicates that a fff is available.

\item KRC checks the name length, and can call TFAR(1 to open the file.  If fff
  open fails, TFAR will write an error message and KRC will stop to prevent a
  possible long run with the wrong fff.

\item TSEAS calls TFAR*(2) before the first season to get all the size and date
  information. For each run season, it interpolates the far-field file to that
  season; if the nearest fff season is within 1\% of a season length, it will
  use that season only; else it calls TFAR twice to get the bounding
  seasons. Tsurf, and Tatm if needed, are interpolated to the desired date and
  stored in COMMON.

\item TLATS extracts the right latitude, and will do cubic interpolation in hour
  to the N2 times if needed. TLATS converts fff surface temperature to radiance at each
  time-step and places in COMMON as FARAD. If KRC is using an atmosphere, TLATS
  interpolates fff Tatm temperature for each time-step and places in COMMON as
  HARTA.

\item TADY uses LOPN3 as the flag to indicate the fff is being used. When using
  fff, the atmosphere temperatures and radiation come from fff.  Frost is still
  calculated.

 \end{itemize}

\subsection{Far-field for sloped surfaces } %----- =============--

Atmosphere and far-field to be based on a no-slope run with all other conditions
normally the same.

No additional changes are made for pits. 

User should not invoke global frost integration (KPREF=2) for sloped runs.

\vspace{3 mm}

For onePoint mode; more complicated: \textbf{not yet implimented}
\qi Must specify the name of an appropriate fff; must read it successfully
\qii fff seasons must cover the range needed by the onePoint master
\qii fff latitudes must cover the range in the onePoint lines
\qi Print a stern warning that a single  fff is being used.
\qi For each onePoint line, need to interpolate: 
\qii in season; same as for normal run
\qii in latitude; requires additional code
\qi Limitation: only allow a single fff for a onePoint run.

\subsection{New Inputs} %.................................................
Control is by the presence of the file name defined by change line : 8 3 x name
\qi  
\subsection{New Outputs} %...............................................
 No plan for \np{TDISK} to be able to read type 52

Size of fff is set by firm-code sizes, TSF(MAXNH,MAXN4) for each season =
96*37*8=28416 bytes for each temperature for each season. The first record in
file contains KRCCOM; season start at record 2. Thus, the file size in bytes is
28416*(seasons+1)*[1,2 or 3]
 
\subsection{New Common Items} %.........................................

Must set the maximum number of time-steps that can be stored in a fff. To simplify writing records in TDISK,  write the existing arrays, TSF etc., these are (MAXNH,MAXN4). Because direct-access files are not open to write for more than one case at a time, it would be possible to use smaller record sizes and write 
\qi ((TSF(I,J),J=1,NLAT),I=1,N24) as long as record can hold KRCCOM.
\qii FORTRAN executes the outer loop first
\qi However, tests indicate that must write complete arrays to direct-access records

Must also set the maximum number of time-steps that can be interpolated from an fff temperature set.  This couild be smaller than MAXN2=384*4*256=393216
\qi


.
\\ FILCOM/  CHARACTER*80 FFAR     ! far-field temperatures input
\\ HATCOM/        INTEGER MAXFF
      PARAMETER (MAXFF=384*4*4=6144) ! dimension of far-field times of day
REAL*8  SALB                 ! spherical albedo of the soil
\qi REAL*8 FARTS(MAXNH,MAXN4,2) ! far-field Tsurf/Tatm for current season
\qi ``    FARAD(MAXFF) ! far-field radiance for every time-step at current latitude
\qi  ``  HARTA(MAXFF) ! far-field Tatm for every time-step at current latitude
\qi  ``  SOLDIF(MAXN2) ! Solar diffuse (with bounce) insolation at each time W/m$^2$
\\ UNITS/ INTEGER  MINT ! the number of temperature sets in file being handled by IOD3 

Type -n files as 1,2 or 3 arrays each Real*8 (MAXNH,MAXN4)
\subsubsection{Array size limits} 
TFAR8: FTS ,FTP,FTA (MAXNH,MAXN4) for one season

TSEAS: same 3
\qi FARTS(MAXNH,MAXN4,2) ! far-field Tsurf/Tatm for current season

TLATS:  REAL*8 WORK(MAXFP=MAXNH+3)        ! to hold extended hours 
 CUBUTERP8 outputs into FARAD=KIM*NY=  must be less than MAXFF

KRC runs requires N2 le MAXN2 =384*4*256=393216; runs using fff will limit N2 to MAXFF=384*4*4 = 6144
\subsection{Implementation} %.........................................

\textbf{ALERT} The use of the output file flag K4OUT has been changed for
version 3.4 . It now controls only the direct access type being written. Actual
reading and writing of data file is controlled by the presence and length of
three file names; all 3 names default to 'no'.  Sees \S \ref{help}.



 \subsection{ What arrays are available in earlier versions}
In \np{TLATS} [or \np{TDAY}]:
\\ for the last day computed, not extrapolated
\qi TSF(I,J4)=TSFH(I) (hour,lat)
\qi TPF(I,J4)=TPFH(I) (hour,lat)
\qi TAF(IH,J4)=TATMJ (hour,lat) saved in \np{TDAY}
\\ Extrapolated to the end of a season
 \qi TTS4(J4)=xof TTS , TTB4(J4)=xof TTB  : surface and bottom diurnal average 
 \qi TTA4(J4), midnight Atm
 \qi FROST4(J4)=EFP   ! frost amount
\qi TMN4(I,J4) , predicted TT1(layer,lat) temperature at midnight

\qsc{Self-heating versus Far-field}  %. . . . . . . . . . . . .

Minimize code changes; aim at needing only Tsurf and Tatm from the far-field model.
 Assume normally will use same atmosphere parameters, although could use different!

from KRC paper: [68]...  \small
\\ The surface condition for a frost-free level surface is :
\qbn W=(1.-A)S_{(t)}'  + \Omega \epsilon R_{\Downarrow}
+k \frac{\partial T}{ \partial z}_{(z=0)} - \overbrace{ \Omega \epsilon\sigma}^{FAC5} T^4  \ \ (jgr 13) \qen

where $W$ is the heat flow into the surface, $A$ is the current surface albedo,
$S_{(t)}'$ is the total solar radiation onto the surface as in Eq. (1), $
R_{\Downarrow}$ is the down-welling thermal radiation (assumed isotropic), $T$
is the kinetic temperature of the surface, $k$ is the thermal conductivity of
the top layer. $\Omega$ is the visible fraction of the sky, $\epsilon$ is the
surface emissivity and $\sigma$ the Stefan-Boltzmann constant.  In the absence
of frost, the boundary condition is satisfied when $W=0$. \normalsize


from KRC paper: [70]... \small
\\ The collimated incident
beam is treated rigorously, intensities of the diffuse solar and thermal fields
are modified by the fraction of sky visible, and the average reflectance and
emittance of the surrounding surface (absent in the level case) are approximated
as: the brightness of level terrain with the same albedo, and material having
the same temperature as the target surface, respectively; this last
approximation accentuates the diurnal surface temperature variation with
increasing slope. Then 
 \qbn S_{(t)}' = S_M \left[ \underbrace{F_\parallel \cos i_2}_{direct} 
+ \underbrace{ \Omega F_\ominus^\downarrow }_{diffuse} 
+ \underbrace{ \alpha A (G_1 \cos i F_\parallel 
+ \Omega F_\ominus^\downarrow)}_{bounce} \right]   \ \ (jgr 14) \qen

$F_\parallel$ =\nv{COLL} is the collimated beam in the Delta-Eddington model and
$F_\ominus^\downarrow$=\nv{BOTDOWN} is the down-going diffuse beam. 
$\Omega$=\nv{SKYFAC}: $\Omega \equiv 1 - \alpha $ here and in
Eq. (13). $G_1$=\nv{G1} is the fraction of the visible surrounding surface which
is illuminated. Within the brackets in Eq. (jgr 14),
\qi the first term is the direct collimated beam, \texttt{DIRECT}
\qi the second is the diffuse skylight directly onto the target surface, \texttt{DIFFUSE}
\qii $ F_\ominus^\downarrow)$ does not depended upon slope.
\qi the third term is light that has scattered once off the
surrounding surface, \texttt{BOUNCE}

For a sloped surface, $G_1$ is taken as unity. As a first approximation, for
depressions $G_1=$ $(90-i)/s \ < 1)$ where $s$ is the slope to the lip of the
depression (the apparent horizon). For the flat-bottom of a depression, $i_2 =
i_0$ when the sun is above this slope, and $ \cos i_2 =0$ when below. \normalsize


ALERT:  $\cos i$ factor in the bounce $F_\parallel$ term is missing in JGR paper.
It is in the tlats8.f code back to at least 2011aug.

For far-field, need to expands the thermal radiation balance term, and (jgr 13) becomes  

\qbn \underbrace{W}_{POWER}=\underbrace{\overbrace{(1.-A)}^{FAC3} S_{(t)}'  
+ \overbrace{\Omega \epsilon}^{FAC6} R_{\Downarrow}^0 }_{ABRAD}
+ \underbrace{k \frac{\partial T}{ \partial z}_{(z=0)}}_{SHEATF}  
-  \overbrace{\epsilon \sigma}^{FAC5} T^4  
+ \underbrace{\overbrace{(1-\Omega) \epsilon \sigma \epsilon_x }^{FAC5X} T_x^4}_{FARAD}   \qen

where $R_{\Downarrow}^0$ is for the equivalent no-slope case and $T_x$ is the
far-field  surface temperature and $\epsilon_x$ its emissivity; $T_x \equiv
T $ if self-heating.  The next-to-last term is surface emission into a
hemisphere and the last term is thermal radiation from the far surface. If the
sloped surface has frost, $T$ becomes fixed at the frost temperature but the
equation remains the same.

With the ability of albedo to depend upon incidence angle, need to expand $ S_{(t)}'$ and (jgr 13) becomes 

\qb \underbrace{W}_{power}=  S_M \left[(1.-A_{h(i_2)}) \underbrace{F_\parallel \cos i_2}_{direct}
+  (1-A_s ) \left( \underbrace{ \Omega F_\ominus^\downarrow }_{diffuse} 
+  \underbrace{ \alpha A_s (G_1 \cos i F_\parallel 
+ \Omega F_\ominus^\downarrow)}_{bounce} \right)  \right] 
\qe

\qbn
+ \underbrace{ \Omega \epsilon R_{\Downarrow}^0}_{atm \ IR}
+ \underbrace{k \frac{\partial T}{ \partial z}_{(z=0)}}_{conduction}  
- \underbrace{\epsilon \sigma T^4}_{emission}  
+ \underbrace{(1-\Omega) \epsilon \sigma \epsilon_x T_x^4}_{back \ radiation} \qen 
 where all the terms within the square brackets are normalized (are unitless).

Reformulate, under- and overbrace terms indicate FORTRAN variable names 
\qb \underbrace{W}_{POWER}=  \underbrace{(1.- \overbrace{A_{h(i_2)}}^{ALBJ} )}_{FAC3}
 \overbrace{ S_M  F_\parallel \cos i_2}^{ASOL}
+  \underbrace{(1-\overbrace{A_s}^{SALB} )}_{FAC3S} \underbrace{ S_M 
  \left( \overbrace{ \Omega F_\ominus^\downarrow  }^{DIFFUSE}
+ \overbrace{ \alpha A_s (G_1 \cos i F_\parallel
+ \Omega F_\ominus^\downarrow ) }^{BOUNCE}  \right)  }_{SOLDIF} 
\qe

\qbn
+ \underbrace{\Omega \epsilon}_{FAC6} \underbrace{ R_{\Downarrow}^0}_{ATMRAD}
+ \underbrace{k \frac{\partial T}{ \partial z}_{(z=0)}}_{SHEATF}  
- \underbrace{\epsilon \sigma}_{FAC5} T^4  
+ \overbrace{\underbrace{(1-\Omega) \epsilon \sigma \epsilon_x }_{FAC5X} T_x^4}^{FARAD}   \ql{wb} 
where the overbrace items are computed in TLATS and
transfered in COMMON. All terms up to and including ATMRAD make up the total
absorbed radiation ABRAD.  When frost is present, its albedo replaces $A_h$ and
$A_s$ on a time-step basis except the $A_s$ in SOLDIF (from TLATS) is on a
season basis; however, the $A_S$ term includes the far-ground fraction $\alpha$
which is small except for steep slopes.

Assumes that normal albedo is the same for the sloped and the flat surfaces.

The fraction of solar flux reflected ALBJ$\equiv A_h =$ALB*AHF is composed of
two factors, ALB$\equiv A_0$ and AHF$=A_h(i)/A_h(0)$, a hemispherical
reflectance function.  Likewise, the spherical albedo is $A_s=$ ALB*PUS where
the second factor is $P_s$.

The floor of a ``pit'' does not see the flat terrain, but rather the same slope
at all azimuths, and therefor different temperatures. The most practical
assumption is that the average radiation temperature of the pit walls is the
same as flat terrain. This will be an under-approximation. In a later version of
KRC with more input parameters, a radiation scale factor could be included; if
practical, code to include a constant factor, initally unity for v 3.4.

Because \nv{FARAD} is not dependent upon the calculation of $T$, it can
pre-computed for a given day. $T_x$ is interpolated to the proper season in
\np{TSEAS}; \np{TLATS} selects the proper latitude, multiplies by \nv{FAC5X} for
each of its stored hours, and interpolates to each time-step to form
\nv{FARAD}$_t$ transfered to \nv{TDAY}. However, to then accomodate variable
frost emission, need to multiply by $\epsilon_f/\epsilon$ for the frost case
(relatively rare).

\vspace{0.2cm}
Because frost temperature changes only with pressure, it does not need to change with Hour.
\pagebreak
\subsubsection{Equilibrium temperature  \label{eqT} }
The equilibrium temperature $T_e$ is that value of $T$ that would make the diurnal average of $W$  in \qr{wb} zero. Or: 

\qbn FAC5 \ast T_e^4 = \langle \overbrace{FAC3 \ast ASOL  + FAC2S \ast SOLDIF}^{\Delta AVEI} \rangle  + \ FAC6 \ast \langle ATMRAD \rangle  + H_g  + \langle FARAD \rangle \ql{Te}

where $ \langle \  \rangle $ represents the diurnal average.

To reach the equivalent of JRG Eq. (12), need to modify JGR Eq. (11) by allowing angle[time]-variable albedo $A$ and adding the geothermal heat-flow term $H_g$ to become 
\qbn \epsilon \sigma  \langle T_s^4 \rangle =\langle (1.-A) S_{(t)}' \rangle  
+ H_g + \epsilon \sigma  \beta_e \langle T_a^4 \rangle \ql{sbal} 

JRG Eq. (12) then becomes:

 \qbn \langle T_a^4 \rangle = \frac{ \overbrace{\langle H_V \rangle / \beta_e}^{QS} + \overbrace{\langle (1-A) S_{(t)}' \rangle}^{AVEI}  +\overbrace{ H_g}^{GHF} } 
{ \sigma (2- \epsilon \beta_e) }  \ql{Ta4} 

 Then ATMRAD (= FAC9*TATMJ**4) $= \sigma \beta_e  \langle T_a^4 \rangle $  

JGR eq. (2) remains the same:

\qbn \overbrace{H_V}^{HUV} = \overbrace{S_M}^{SOLR}\overbrace{ \left( \mu_0 - F_\ominus^\uparrow(0)-(1-A_h(t)) 
\left[ \mu_0  \ F_\parallel + F_\ominus^\downarrow(\tau_v) \right] \right) }^{ATMHEAT}\ql{aheat} 


Atmosphere IR heating is the average of $H_R$ in JGR Eq. (5): 
$ \langle H_R \rangle = \sigma \beta_e \left( \epsilon  \langle T_s^4 \rangle - 2 \langle T_a^4 \rangle \right) $

\subsubsection{code in TLATS}
Snippits of code in TLATS for radiation values placed in COMMON; omitting all the logical tests.  Minor edits for clarity.

.
\qi  SOLR=SOLCON/(DAU*DAU)     ! solar flux at this heliocentric range
\qi  call DEDING28 (omega,g0,avea,COSI,opacity, bond,COLL,deri)
\qi  DIRECT=COS2*COLL     ! slope is in sunlight  or =0
 \\ ASOL(JJ)=QI=DIRECT*SOLR         ! collimated solar onto slope surface

.
\qi  AHF= (1.D0+COS2*DLOG(COS2/(1.D0+COS2)))/2. ! Lommel-Seeliger, e.g.
\qi  HALB=ALB*AHF/AH0     ! normalized hemispherical albedo
\\ ALBJ(JJ)=MIN(MAX(HALB,0.D0),1.D0) ! current hemispheric albedo

.
\\ SKYFAC = (1.D0+ DCOS(SLOPE/RADC))/2.D0 ! effective isotropic radiation.
\qii   call DEDING28 (omega,g0,avea,COS3,opacity, bond,COL3,deri)
\qii   BOTDOWN=PIVAL*(DERI(1,2)+F23*DERI(2,2))  or 0 ! diffuse down at surf
\qi  DIFFUSE=SKYFAC*BOTDOWN ! diffuse flux onto surface
\qii   PUS=1.3333333    ! e.g. Lommel-Seeliger  $P_S$
\qii   SALB=PUS*ALB              ! spherical albedo, for diffuse irradiance
\qii   G1=DMIN1 (1.D0,(90.D0-AINC)/SLOPE) ! (90-i)/slope   or 1.
\qii   DIRFLAT=COSI*COLL ! or COSI if no atm. collimated onto regional flat plane
\qi   BOUNCE=(1.D0-SKYFAC)*SALB*(G1*DIRFLAT+DIFFUSE)   
\qi         QI=DIRECT*SOLR         ! collimated solar onto slope surface
\\ SOLDIF(JJ)=(DIFFUSE+BOUNCE)*SOLR ! all diffuse, = but the direct.
\\ AVEI=AVEI+(1.d0-ALBJ(JJ))*QI+(1.-SALB)*SOLDIF(JJ) !
.
\qi  ATMHEAT=COSI-TOPUP-(1.-AVEA)*(BOTDOWN+COSI*COLL) ! atm. heating
\\ ADGR(JJ)=QA=ATMHEAT*SOLR        ! solar flux available for heating of atm.

.
\qi in TFAR, extract from fff: FELP(8)= F3FD(2)        ! surface emissivity
\qi FAC5X=(1.-SKYFAC)*EMIS*SIGSB*DELP(8) ! last is fff surface emissivity
\qi Extract fff surface temperatures into WORK for the proper latitude
\qii add midnight wrap to both ends.
\qii raise to 4'th power and multiply by FAC5X
\\ CALL CUBUTERP8 (2,WORK,NHF,TENS,N2,FARAD) ! cubic interpolation to timesteps

\subsubsection{Code in TDAY}

Snippits of code in TDAY for calculating surface temperature; omitting all the logical tests and the convergence loops.  Minor edits for clarity.  
When there is no atmosphere:

.
\qii   FAC3S = 1.D0-SALB         ! spherical absorption
\qii   FAC3  = 1.D0-ALBJ(JJ) ! hemispherical absorption
\qi   ABRAD = FAC3*ASOL(JJ)+FAC3S*SOLDIF(JJ) ! surface absorbed radiation
\qii   FAC5 = SKYFAC*EMIS*SIGSB ! if self-heating 
\qii   FAC5 = EMIS*SIGSB ! if fff 
\qii   FAC7 = KTT(2)/XCEN(2)    ! current redone if T-dep conductivity
\qii   TS3 = TSUR**3         ! bare ground
\qi   SHEATF = FAC7*(TTJ(2)-TSUR) ! upward heat flow to surface
\\ POWER = ABRAD +SHEATF - FAC5*TSUR*TS3 ! unbalanced flux
\\ POWER = POWER+FARAD(JJ) ! only if fff


ERROR: of 10 deg slope, fff 30 K hotter than self. Not physical
\qi And  hour 23 and 23.5 are much colder.
\qi daytime rise with fff= flat:NoA greater than with fff= flat:tinyA
\\ Suggests: far view factor much too big

\subsubsection{find T for W=0} %........................
Need to modify JGR Eq. 29 : ``
 From Eq. (13), find 
\qbn \frac{\partial W}{\partial T} = \overbrace{-k / X_2}^{FAC7} - \overbrace{4 \Omega \epsilon \sigma}^{FAC45} T^3  \ \ (jgr29) \qen 

 where $X_2$ is the depth to the center of the first soil layer. `` becomes \qbn
 \frac{\partial W}{\partial T} = -\overbrace{ k / X_2}^{FAC7} -\overbrace{4
   \epsilon \sigma}^{FAC45} T^3 \mc{coded \ as} \overbrace{\Delta T}^{DELT}=
 \frac{W}{k / X_2 + 4 \epsilon \sigma T^3} \qen i.e., $\Omega$ becomes 1, so
 omit this from \nv{FAC45}
 
Thus, the fff must contain at least  $R_{\Downarrow}$ [or Tatm] and $T_x$. 
Probably desireable that it contain KRCCOM for insurance.

To handle many seasons, could write direct access files.

\qsd{Interpolation of the fff}

Generally expect that the far-field case will be run with exactly the same grid in
season, latitude and time as a sloped case. Because the stored hours are less
dense than time-steps, will need interpolation in at least that dimension.

Accomodate linear interpolation in season, but if the seasons are within
DELSEAS=1\% of a season-step, use that season without interpolation.

To avoid many complications in interpolation, require that fff contain a
latitude within DLATEST=0.1 degree of those in the sloped case, and use that
without interpolation.

Will interpolate smoothly (cubic spline, replicated across midnight) in hour
from the stored N24 points to the N2 time-steps. Firm-code the maximum number of
times steps for using a fff to a generous MAXFF=384*4*4=6144.

However, this requires that N2 for the slope run be an integral multiple, 2 or
more, of N24 for the fff run.


 

\subsubsection{Getting Tatm from TPlan and Tsurf. Not used.}
in \np{TLATS}:  tauir is in krcc8m.f;
 \qi It does vary with elevation (due to PRES) , which can be a function of latitude
\qi  it can vary with season if PZREF varies due to KPREF ne 0
\vspace{-3.mm} 
\begin{verbatim}
in TLATS:
        TAUIR=(CABR+TAUVIS*TAURAT)*(PRES/PTOTAL)+TAUICE ! thermal opacity, zenith
        QA=AMIN1(0.0168455D0,AMAX1(TAUIR,62.4353D0)) ! limits 1. < FACTOR < 2.
        FACTOR= 1.50307D0 -0.121687D0*DLOG(QA) ! from fit to hemisphere integrals
        TAUEFF=FACTOR*TAUIR     ! effective hemispheric opacity
        BETA=1.-DEXP(-TAUEFF)   ! hemispheric thermal absorption of atmosphere


in TDAY:  IF (LATM) ....
      EMTIR = DEXP(-TAUIR)       ! Zenith Infrared transmission of atm
      FAC82=1.-EMTIR            !  " absorption "
      FAC9=SIGSB*BETA           ! factor for downwelling hemispheric flux
SIGSB is a constant and BETA
      IF (EFROST.GT.0.) THEN
        FAC8=EMTIR*FEMIS        ! ground effective emissivity through atmosphere
      ELSE                      ! bare ground
        FAC8=EMTIR*EMIS
      ENDIF

            ATMRAD= FAC9*TATMJ**4 ! hemispheric downwelling  IR flux
              TPFH(IH)=(FAC8*TSUR4+FAC82*TATM4)**0.25 ! planetary  
              TAF(IH,J4)=TATMJ  ! save Atm Temp.
              DOWNIR(IH,J4)=ATMRAD ! save downward IR flux
\end{verbatim}
This is a mess, as seasonal PRES is not in type -1 files.
and EFROST is only in the single KRCCOM in the first record.

 Alternate solution is to store Tatm in traditional type -1 files.

To get the right Tplan for output for sloped surfaces, need to have Tatm for the flat case, so would have to include that also in the type -1 files.

Could avoid worsening the large size of type -1 files by making them R*4, convert to/from R*8 in \np{TDISK}

Not much more work to define a new type similar to -1 but which includes TATM. This could be read/write simultaneously with type 52.

Could store the any new flags and arrays  in hatcom to avoid impacting any other commons. hatcom is already included in \np{TLATS} and \np{TDAY}.


Better solution might be to put only  DOWNIR and TATM  is a separate type -2 file, 
 and perhaps be able to write -1,-2 and 52 all at the same time?

 This would take additional input parameters to set up, or could set the required logical flags (in \np{TCARD}) based on setting the file names

\subsubsection{File handling in version 3.4 \label{mint}}
 To utilize far-field  temperatures, need to have an input data file
 open simultaneous with at least one output file. Version 3.4 can handle 0 to 3
 data files open at once. To deal efficiently with fff for both with and without
 atmophere cases, two additional types of binary files have been defined, and
 the prior type -1 has becomes type -2.

\subsubsection{ Handling 3 data files at once. \label{help}}
Because type 52 is written after all cases done, it should be possible to have direct-access file open at the same time with little conflict. Will need more complex control logic.  Note: type - requires open/close for each case.

Need to have multiple file names active, and \np{TCARD} must distinguish when to open/close direct access (each case) versus bin5 files (may be multi-case) 

May have zero to 3 data files active at one time, determined by the second field in a change card starting '8':
\begin{description}  % labeled items   
\item [ 5] A ``bin'' Type 5x (52) bin5-format to be written. Name is \nv{FDISK}.
  PIO (c-level I/O) system determines the unit.  LOPN4 is true when active. All interface is through \np{TDISK}, which calls \np{BINF5} to open or close.

\item [21] A direct-access file to be written. Name is \nv{FDIRA}.  Uses IOD2,
  LOPN2 is true when active. Five types are available; they are distinguished by
  the value of K4OUT. By convention, the file extension should indicate the file
  type, but the KRC system makes no decisions based on this extension.  All interface is through \np{TDISK}.
\item [ 3] A fff direct access file to be read. Name is
  \nv{FFAR}, it must be type -N; -1 is adequate for air-less bodies and -3 is
  required for atmospheres. Uses IOD3, LOPN3 is true when active.  All interface is through \np{TFAR}, with open and close initiated by calls from \np{KRC}  

BEWARE: seasons in this file must cover all that will be computed in later KRC
run, including the spin-up. It is best to save a full year, with no wrap.

 \end{description}
Each  direct access file is opened after a new name of length four or more characters is read into FILCOM by \np{TCARD}
\\ Each  direct access file is closed from \np{TCARD} when a new name of any length is listed or from \np{KRC} when the run ends.
\qii IOD2 is closed at the end of a case, as direct-access files as implimented by KRC can only hold one case.
\\ An open file is always active!
\qi e.g., a sloped case will use FFAR if that is open, else it will self-heat
\\ Seasons-records are written or read from \np{TSEAS} by calls to \np{TDISK} or read by calls to \np{TFAR}.  
 Sloped cases are required in abundance to address thermal beaming, which may be
 especially relevant for airless bodies. For these, need only the surface
 tempertures, so include the capability of writing direct access files with only Tsurf.

\subsection{Lab notes on tests}
Start with master34.inp, fewer latitudes and shorter spinup. Generate fff tm3 for flat and for fake steep self-heating slope.

Run case with epsilon slope to compare with flat, and a case with slope and azimuth identical to the fake steep case, expecting same temperatures. Output to /work/work1/krc/beam/BeamBa*
\vspace{-3.mm} 
\begin{verbatim}

2016 May 25 19:07:35
Edit candi.inp for 321 and 341, run both
kv3@115, then parf[[5,6,0,1]]=['/work2/KRC/321/run/out/',    'candi' $
                              ,'/home/hkieffer/krc/tes/out/','candi341']
 
Doing -------------->     550
Num lat*seas*case with NDJ4 same/diff=         632         128


test341.inp: 40 seasons, 2 year spinup, 3 latitudes
   6 cases exercising zone table, photometric function, heatflow, constant KofT

----------  consistency between file types --------
test341a.inp: Double run: output in: /work/work1/krc/test/

1) 670 seasons, soly, no spinup, 5 latitudes
   6 cases exercising zone table, photometric function, heatflow, constant KofT
   54271232 May 23 23:11 v341aTest.t52
   57201408 May 23 23:11 v341aFlat.tm3

2) 40 seasons, 2 year spinup, 19 latitudes
 1 case, variable frost albedo and temperature
    1047488 May 23 23:11 v341aTest2.t52
    3495168 May 23 23:11 v341aFlat2.tm3

kv3@ 147 for both pairs of files confirms that t52 and tm3 temperatures 
are identical.

--------- comparison of 341 to older versions ------

321//VerTest.inp
670 seasons soly, no spinup 5 latitudes
 6 cases test KofT with/without atmosphere



@2 pari 7=3 8=11 17=2

@45  Case 1-Case 3
Item in ttt Mean     Std    mean_ABS_std
    Tsurf  -0.417   2.814   0.480   2.804
    Tplan   1.633   4.516   3.654   3.115
     Tatm   0.000   0.548   0.450   0.313
  DownVIS   0.030   0.380   0.030   0.380
   DownIR   0.000   0.319   0.244   0.205

 dt=reform(t1[*,0,jlat,*])
clot,dt   shows that last hour are all near -.8, read within +-.2

@45  Case 2-Case 4
Item in ttt Mean     Std    mean_ABS_std
    Tsurf  -0.241   2.141   0.375   2.122
    Tplan   1.818   4.046   3.539   2.675
     Tatm   0.000   0.543   0.440   0.318
  DownVIS   0.020   0.363   0.020   0.363
   DownIR   0.000   0.318   0.240   0.209

2016 May 24 08:03:02
KRCINDIFF: test for changes. Input limits:       64     120     220
 11 10   ABRPHA      27.955     -0.0000      27.955
 22 21ARC3/Safe     0.80010     -0.0000     0.80010
 32 31     fd32      3182.5      0.0000      3182.5
 76 75    TATMJ      184.59      184.61   -0.015201
117 16    K4OUT      -3      52     -55
118 17    JBARE    9999       0    9999
134 33      KKK       4       8      -4


.rnew kv3
@114  4  232 342
@11 1=test342f
@111 123
@12  7=0 20=3  23=
@402 Clot DOWNVIS and Tsurf for all cases     WRONG? slope greater max
   WRONG? Tsurf far is 36K greater than self at noon
@45   stats on case deltas
@46   plot case delta for DOWNVIS and Tsurf

@12  0=34 23=-3
 stem='v342Flatf'   read the type -n for 1st case 
@51    yields FOUT=dblArray[5, 2], TTOU= dblarr[48, 5, 3, 40]
t4=ttt[*,0:2,*,*,0] & t4=transpose(t4,[0,2,1,3])
HISTFAST,ttou-t4 ; all 0
  stem='v342Flata' & pari[23]=-1     read the type -n for 2nd case 
@51    yields  TTOU= dblarr[48, 5, 1, 40]
t4=ttt[*,0,*,*,1] & t4=transpose(t4,[0,2,1,3])
HISTFAST,ttou-t4 ; all 0

\end{verbatim}
Far-field heating has a smaller effect than self-heating; see Figure
\ref{tslope}.  An artifact of KRC is that atmospheric pressure can be constant
even when frost forms so that the amount of frost is not limited; this limits
night temperatures and can delay the dawn temperature rise, as seen in the first
case in the Figure.
\begin{figure}[!ht] \igq{tslope}
\caption[Effect of far heating]{Test with slope of 30\qd~ dipping to the West
  for a nominal asteroid surface in Mars orbit; diurnal surface kinetic
  temperatures at the equator at $L_S=134$. The 5 cases in the legend are:
  top=white: Flat with surface pressure ($P_T$) 1.1 Pa and clear with frost
  temperature 146; 2=blue: flat with no atmosphere; 3=green: sloped with no
  atmosphere and self-heating; 4=yellow: sloped with no atmosphere and far-field
  from first case, which is artificial; bottom=red: sloped with no atmosphere
  and far-field from 2nd case, which is realistic. The dashed line shows the
  difference self-heating minus far-flat-heating magnified by a factor of ten
  and offset from 200; it reaches 9K near dawn.
\label{tslope}  tslope.png }
\end{figure} 
% how made: kv3: 114 11 1=t342b 111 402 : t342b slope=30, azi=90
%t342a slope=10, azi=80



\subsubsection{fff types}
All lunar-like with Mars orbit, 5 lats, 40 seasons.  Cases:
\qi  1: tiny atmosphere, save /work1/krc/test/v342Flatf.tm3
\qi  2: no atm, save v342Flata.tm1
\qi  3: 10 degree slope, save v342Flatb.tm1
\qi  4: same slope, use fff, save v342Flatc.tm1


\subsubsection{temporary}

 
  tday 686             TAF(IH,J4)=TATMJ  ! save Atm Temp.
 Done only on the saved hour, so would need to either use as is or interpolate to each time step.
C0uld do the interpolation in \np{TLATS} time-step loop


\subsubsection{Reminder, FORTRAN file types}  % ---------------------
http://www.fortran.com/fortran/F77_std/rjcnf.html is the F77 definition
OPEN arguments that control the nature of the file.

 ACCESS: 
\qi 'SEQUENTIAL'=default  
\qi 'APPEND' 
\qi 'DIRECT'
\qii RECL must also be given, since all I/O transfers are done in multiples of fixed-size records.
\qii UNFORMATTED is the default

  FORM:
\qi 'FORMATTED'=default for sequential. Each record is terminated with a newline character; that is, each record actually has one extra character.
\qi 'UNFORMATTED' the size of each transfer depends upon the data transferred.
\qii Each record is preceded and terminated with an INTEGER*4 count, making each record 8 characters longer than normal. This convention is not shared with other languages, so it is useful only for communicating between FORTRAN programs.
\qi 'PRINT' 

RECL=rl: required if ACCESS='DIRECT' and ignored otherwise.
\qi rl is an integer expression for the length in characters of each record of a file. rl must be positive.
\qii If -xl[d] is set, rl is number of words, and record length is rl*4. 
\qiii else, rl is number of characters [bytes], and record length is rl.
\qii -xl does not occur in the KRC Makefile, as of 2016may11. Excerpt from \np{TDISK}:
\vspace{-3.mm} 
\begin{verbatim}
     IF (K4OUT.LT.0) THEN  !  K4OUT is negative  .tm1
        NWTOT=2*MAXNH*MAXN4
     NRECL=8*NWTOT    ! bytes: or  NRECL=NWTOT  ! depends upon compiler <<<< 
     OPEN (UNIT=IOD2,FILE=FDISK,ACCESS='DIRECT',STATUS=CSTAT,RECL=NRECL,...
\end{verbatim}

\subsubsection{Early plan and timing tests}
Although dicussed in this early email, changing files to Real*4 has not been impliments in V3.4.2
\vspace{-3.mm} 
\begin{verbatim}
Robin:

With regard to the need for additional stored information to enable use of a
 flat far field (fff) for sloped surfaces in KRC. 

I have reached a compromise solution that optionally adds atmospheric
 temperature (Ta) as a third array after Surface kinetic temperature (Ts) and
 top-of-atmosphere nadir brightness temperature (Tp, planetary temperature) in
 the type -1 file. This is called type -3; the file extension would be .tm3  

An added advantage of type -3 is that it allows relatively easy calculation
 (estimation) of the brightness temperature for off-nadir viewing: e.g.,
B=(emis*Ts^4 - Tp^4)/( Ta^4-emis*Ts^4)  B is the transmission of the atmosphere
tau = -ln(B)                    is the atmosphere column opacity
C=exp(-tau/cos e)               where e is the off-nadir (emittance) angle
To^4=(1-C)emis*Ts^4 + C*Ta^4    To is the off-nadir brightness temperature

Use of fff requires an additional TDISK-like routine to read-only a type -3
 file, called TDIF3. This results from needing to have two KRC direst-access
 files open simultaneously.  TDISK has been modified to allow writing a type 52
 simultaneously with either -1 or -3.

Changes required to read a -3 file to get Ts and Tp in the fashion of a -1 file:
 FORTRAN: Change the RECL argument in the OPEN statement. 
   (Beware, its meaning depends upon the compiler setting of -xl[d])

Note that type -1 and -3 file records are fixed size set by the MAXN4=37 and
 MAXNH=96 and the word type (currently REAL*8) so -3 will be MAXNH*MAXN4*3=10656
 words or 85248 bytes each. The first record contains KRCCOM, currently 1704
 bytes. A typical KRC model of 40 seasons is 3.4 Mbytes

-----

I am considering changeing the type of the Ts, Tp and Ta arrays
 written to type -3 files from REAL*8 to REAL*4 simply to keep the size
 down. This would be done transparently to users of TDISK. KRCCOM would still be
 REAL*8, but there is no need for double precision in the final temperatures.  I
 believe the time to do REAL*4 <-> REAL*8 conversion is trivial.

Changes required to accomodate REAL*4 in the files for other readers:
If user really wants REAL*8 temperatures, 
  READ arrays as REAL*4, also define REAL*8 arrays 
  Loop over hour and latitude with 
   TP8(I,J)=TP4(I,J) and similar for Ts and Ta
  Loop limits could be MAXNH and MAXN4 or the actual N24 and N4

I have tested timing to read type -3 files and to convert between R*4 and R*8.
  For a typical file with 40 seasons (does not matter how many hours or
 latitudes as the array sizes are fixed at the maximum allowed). For type -3,
 reading takes 0.6 to 0.9 ms, but this may be highly dependent upon caching.
  Conversion of the maximum set of hours and latitudes for all seasons takes 1.6
 ms (array was filled with random temperatures).

\end{verbatim} 
